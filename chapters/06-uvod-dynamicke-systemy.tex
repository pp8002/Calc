% !TEX root = ../main.tex
\section{Úvod do dynamických systémů}
\label{sec:dynamicke-systemy}

\blocktitle{Cíl kapitoly}
Tato kapitola představuje diferenciální rovnice z pohledu teorie dynamických systémů,
zaměřuje se na dlouhodobé chování řešení, zavádí pojmy fázového prostoru, trajektorií, invariantních množin, atraktorů,
bifurkací a základy teorie chaosu.

\spc

\subsection{Základní pojmy a definice}
\label{sec:zakladni-pojmy-ds}

\begin{definition}[Dynamický systém]
Dynamický systém na metrickém prostoru $X$ je grupové působení $\phi:\R\times X\to X$ (spojité) nebo $\phi:\Z\times X\to X$, kde:
\begin{itemize}
\item $\phi(0,x)=x$,
\item $\phi(t+s,x)=\phi(t,\phi(s,x))$.
\end{itemize}
Pro autonomní ODE $y'=f(y)$ je $\phi(t,y_0)$ řešení s počáteční podmínkou $y(0)=y_0$.
\end{definition}

\begin{definition}[Fázový prostor a trajektorie]
\begin{itemize}
\item Fázový prostor = množina všech stavů $X$,
\item Trajektorie bodu $x$: $\{\phi(t,x):t\in\R\}$,
\item Pozitivní polotrajektorie: $t\ge0$, negativní: $t\le0$.
\end{itemize}
\end{definition}

\begin{example}[Fázové portréty jednoduchých systémů]
\begin{itemize}
\item $\dot{x}=y$, $\dot{y}=-x$: kružnice (harmonický oscilátor),
\item $\dot{x}=-x$, $\dot{y}=-2y$: trajektorie ke středu (uzel),
\item $\dot{x}=x$, $\dot{y}=-y$: hyperbolické trajektorie (sedlo).
\end{itemize}
\end{example}

\spc

\subsection{Invariantní množiny a limita množina}
\label{sec:invariantni-mnoziny}

\begin{definition}[Invariantní množina]
$M\subset X$ je
\begin{itemize}
\item invariantní: $\phi(t,M)=M$ pro všechna $t$,
\item pozitivně invariantní: $\phi(t,M)\subset M$ pro $t\ge0$,
\item negativně invariantní: $\phi(t,M)\subset M$ pro $t\le0$.
\end{itemize}
\end{definition}

\begin{definition}[ω-limita množina]
Pro $x\in X$:
\[
\omega(x)=\{y:\exists t_n\to+\infty,\;\phi(t_n,x)\to y\}.
\]
Analogicky $\alpha(x)$ pro $t_n\to-\infty$.
\end{definition}

\begin{theorem}[Vlastnosti $\omega(x)$]
Pokud $X$ je kompaktní:
\begin{romanenum}
\item $\omega(x)$ je neprázdná, kompaktní, spojitá,
\item $\omega(x)$ je invariantní,
\item Pokud $M$ je pozitivně invariantní a kompaktní, pak $\omega(x)\subset M$ pro $x\in M$.
\end{romanenum}
\end{theorem}

\begin{example}[Limita množiny]
\begin{itemize}
\item Stabilní uzel: $\omega(x)=\{0\}$,
\item Limitní cyklus: $\omega(x)$ = cyklus,
\item Sedlo: $\omega(x)=\{0\}$ jen pro body na stabilní varietě.
\end{itemize}
\end{example}

\spc

\subsection{Atraktory a basiny přitažlivosti}
\label{sec:atraktory}

\begin{definition}[Atraktor]
$A\subset X$ je atraktor, pokud:
\begin{romanenum}
\item $A$ je invariantní,
\item $\exists U: \phi(t,U)\subset U$ pro $t>0$ a $\bigcap_{t\ge0}\phi(t,U)=A$,
\item $A$ je minimální s touto vlastností.
\end{romanenum}
\end{definition}

\begin{definition}[Basin přitažlivosti]
\[
B(A)=\{x:\omega(x)\subset A\}.
\]
\end{definition}

\begin{example}[Typy atraktorů]
\begin{itemize}
\item Bodový atraktor: stabilní rovnovážný bod,
\item Limitní cyklus,
\item Podivný atraktor (Lorenz).
\end{itemize}
\end{example}

\spc

\subsection{Hartmanova–Grobmanova věta}
\label{sec:veta-linearizace}

\begin{theorem}[Hartman–Grobman]
Pokud $f\in C^1$, $f(0)=0$ a Jacobiho matice $A=Df(0)$ má všechna $\Re\lambda\ne0$, pak existuje homeomorfismus $h$, který zobrazuje trajektorie $\dot{x}=f(x)$ v okolí $0$ na trajektorie $\dot{y}=Ay$.
\end{theorem}

\begin{remark}
Hyperbolický bod má lokální fázový portrét topologicky ekvivalentní linearizovanému systému.
\end{remark}

\begin{example}[Sedlo]
$\dot{x}=x+x^2$, $\dot{y}=-y+y^2$: počátek je sedlo, lokálně ekvivalentní $\dot{x}=x$, $\dot{y}=-y$.
\end{example}

\spc

\subsection{Úvod do bifurkací}
\label{sec:uvod-bifurkace}

\begin{definition}[Bifurkace]
Kvalitativní změna fázového portrétu při změně parametru.
\end{definition}

\begin{theorem}[Sedlo–uzel bifurkace]
Pokud $\dot{x}=f(x,\mu)$ splňuje $f(0,0)=0$, $f_x(0,0)=0$, $f_{xx}(0,0)\ne0$, $f_\mu(0,0)\ne0$, nastává sedlo–uzel bifurkace.
\end{theorem}

\begin{example}[Sedlo–uzel]
$\dot{x}=\mu-x^2$: pro $\mu>0$ dva body, pro $\mu=0$ jeden, pro $\mu<0$ žádný.
\end{example}

\begin{definition}[Vidličková bifurkace]
$\dot{x}=\mu x-x^3$:
\begin{itemize}
\item $\mu\le0$: stabilní $x=0$,
\item $\mu>0$: $x=0$ nestabilní, $x=\pm\sqrt{\mu}$ stabilní.
\end{itemize}
\end{definition}

\spc

\subsection{Základy teorie chaosu}
\label{sec:zaklady-chaosu}

\begin{definition}[Chaotický systém (Devaney)]
Je chaotický, pokud:
\begin{romanenum}
\item má citlivou závislost na počátečních podmínkách,
\item je topologicky transitivní,
\item má hustou množinu periodických bodů.
\end{romanenum}
\end{definition}

\begin{example}[Logistické zobrazení]
$f(x)=rx(1-x)$, pro $r\gtrsim3.57$ dochází ke chaosu.
\end{example}

\begin{example}[Lorenzův systém]
\[
\dot{x}=\sigma(y-x),\quad \dot{y}=rx-y-xz,\quad \dot{z}=xy-bz
\]
pro $\sigma=10$, $r=28$, $b=8/3$.
\end{example}

\spc

\subsection{Aplikace v biologii a ekologii}
\label{sec:aplikace-bio}

\begin{example}[Populační dynamika]
\begin{itemize}
\item Diskrétní logistické zobrazení,
\item Lotka–Volterra: oscilace,
\item Konkurence druhů: bifurkace.
\end{itemize}
\end{example}

\begin{example}[Neuronové sítě]
FitzHugh–Nagumo model: excitační chování, bifurkace.
\end{example}

\spc

\subsection*{Shrnutí kapitoly}
\begin{itemize}
\item Dynamické systémy studují dlouhodobé chování řešení.
\item Klíčové pojmy: trajektorie, invariantní množiny, atraktory.
\item Hartman–Grobman: lokální ekvivalence nelineárního a lineárního systému.
\item Bifurkace: mechanismus kvalitativních změn.
\item Chaos: citlivá závislost, transitivita, hustota periodických bodů.
\end{itemize}

\spc

\subsection*{Cvičení}
\begin{enumerate}
\item Najděte invariantní množiny pro $\dot{x}=-x$, $\dot{y}=-2y$.
\item Ukažte, že kružnice je atraktor pro $\dot{r}=r(1-r)$, $\dot{\theta}=1$.
\item Analyzujte bifurkaci v $\dot{x}=\mu x-x^3$.
\item Numericky najděte periodu 4 pro $f(x)=3.5x(1-x)$.
\item Dokažte, že lineární $\dot{x}=Ax$ s $A$ hyperbolickou není chaotický.
\item Analyzujte Van der Polův oscilátor: $\dot{x}=y$, $\dot{y}=-x+\mu y-x^2y$.
\end{enumerate}
