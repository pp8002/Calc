% !TEX root = ../main.tex
\section{Úroveň 1: Základní ODE 1. Řádu - Expertní Kvantitativní Fundament}
\label{sec:uroven-1}

\subsection{Úvod do Úrovně 1}
\label{subsec:uvod-uroven-1}

Tato úroveň představuje kompletní a rigorózní úvod do teorie obyčejných diferenciálních rovnic prvního řádu. Naším cílem není pouze mechanické osvojení řešicích technik, ale hluboké porozumění matematické struktuře těchto rovnic a jejich bezprostřední aplikace v kvantitativních vědách.

\vspace{0.8\baselineskip}

\begin{principle}[Filozofie úrovně 1]
Úroveň 1 kombinuje matematickou preciznost s praktickou relevancí. Každý teoretický koncept je okamžitě ilustrován na reálných kvantitativních problémech, čímž vytváříme most mezi abstraktní matematikou a aplikovaným výzkumem.
\end{principle}

\vspace{0.8\baselineskip}

\subsubsection*{Organizace a cíle úrovně}

Úroveň 1 je strukturována do čtyř hlavních typů ODE 1. řádu, které tvoří fundament pro veškeré pokročilejší techniky:

\begin{itemize}
\item \textbf{Separabilní rovnice} - základní stavební kámen
\item \textbf{Lineární rovnice} - systematický přístup s integračními faktory
\item \textbf{Homogenní rovnice} - geometrická interpretace a substituce
\item \textbf{Exaktní rovnice} - teorie potenciálů a konzervativních systémů
\end{itemize}

\vspace{0.8\baselineskip}

\subsubsection*{Kvantitativní význam}

Pro kvantitativního experta představují ODE 1. řádu fundamentální nástroj pro:

\begin{itemize}
\item Modelování časového vývoje finančních proměnných
\item Kalibraci parametrů na historická data
\item Analýzu stability ekonomických systémů
\item Přípravu na pokročilejší modely (SDE, PDE)
\end{itemize}

\vspace{0.8\baselineskip}

\begin{example}[Motivační příklad z kvantitativních financí]
Uvažujme jednoduchý model pro cenu akcie $S(t)$ s konstantní expected return $\mu$:
\[
\frac{dS}{dt} = \mu S, \quad S(0) = S_0
\]
Tato separabilní rovnice má řešení $S(t) = S_0 e^{\mu t}$, které tvoří základ pro geometrický Brownův pohyb v Black-Scholesově modelu.
\end{example}

\vspace{0.8\baselineskip}

\subsection{Separabilní Rovnice - Kompletní Teorie}


\label{subsec:separabilni-rovnice}

\subsubsection{Teoretický Fundament}
\label{subsubsec:teoreticky-fundament}

\begin{definition}[Separabilní diferenciální rovnice]
Rovnice je \emph{separabilní}, jestliže ji lze zapsat ve tvaru:
\[
\frac{dy}{dx} = f(x)g(y)
\]
kde $f(x)$ je funkce závislá pouze na $x$ a $g(y)$ je funkce závislá pouze na $y$.
\end{definition}

\vspace{0.6\baselineskip}

\begin{theorem}[Existence a jednoznačnost řešení]
Nechť $f(x)$ je spojitá na intervalu $I$ a $g(y)$ je spojitá na intervalu $J$ a Lipschitzovská v $y$ na $J$. Pak pro každý počáteční bod $(x_0, y_0) \in I \times J$ existuje právě jedno řešení separabilní rovnice splňující $y(x_0) = y_0$, pokud $g(y_0) \neq 0$.
\end{theorem}

\vspace{0.4\baselineskip}

\begin{proof}
Důkaz využívá Picard-Lindelöfovu větu z Kapitoly 3. Separabilní tvar umožňuje přímou integraci:
\[
\int_{y_0}^y \frac{d\eta}{g(\eta)} = \int_{x_0}^x f(\xi)  d\xi
\]
Lipschitzovskost $g(y)$ zaručuje jednoznačnost řešení na maximálním intervalu existence.
\end{proof}

\vspace{0.6\baselineskip}

\begin{theorem}[Maximální interval existence]
Nechť $f(x)$ je spojitá na $(a,b)$ a $g(y)$ je spojitá na $(c,d)$. Pak řešení separabilní rovnice existuje na maximálním otevřeném intervalu $(\alpha,\beta) \subset (a,b)$ a platí buď $\beta = b$, nebo $\lim_{x\to\beta^-} y(x)$ existuje a patří do $\{c,d,\pm\infty\}$.
\end{theorem}

\vspace{0.8\baselineskip}

\subsubsection{Kompletní Metodologie Řešení}
\label{subsubsec:kompletni-metodologie}

\begin{method}[Metoda 1: Přímá separace proměnných]
\label{met:primaseparace}
\begin{enumerate}
\item \textbf{Identifikace tvaru}: Ověřte, že rovnici lze zapsat jako $\frac{dy}{dx} = f(x)g(y)$

\item \textbf{Analýza singulárních bodů}: Najděte všechny $y_c$ takové, že $g(y_c) = 0$

\item \textbf{Konstantní řešení}: $y(x) = y_c$ jsou řešeními pro každé $y_c$ z předchozího kroku

\item \textbf{Separace proměnných}: Pro $g(y) \neq 0$ přepište na $\frac{dy}{g(y)} = f(x)dx$

\item \textbf{Integrace}: 
\[
\int \frac{dy}{g(y)} = \int f(x)dx + C
\]

\item \textbf{Explicitní vyjádření}: Pokud možno, vyjádřete $y$ explicitně jako funkci $x$

\item \textbf{Analýza definičního oboru}: Určete maximální intervaly existence řešení

\item \textbf{Ověření}: Dosazením ověřte, že získané řešení splňuje původní rovnici
\end{enumerate}
\end{method}

\vspace{0.8\baselineskip}

\begin{method}[Metoda 2: Substituční přístupy]
\label{met:substituce}
Pro rovnice tvaru $\frac{dy}{dx} = f(ax + by + c)$ použijte substituci:
\[
u = ax + by + c \implies \frac{du}{dx} = a + b\frac{dy}{dx} = a + bf(u)
\]
což je separabilní rovnice pro $u(x)$.
\end{method}

\vspace{0.6\baselineskip}

\begin{method}[Metoda 3: Numerická verifikace]
\label{met:numerickaverifikace}
Pro ověření analytického řešení použijte Eulerovu metodu:
\[
y_{n+1} = y_n + h f(x_n)g(y_n)
\]
kde $h$ je krok metody. Chyba metody je $O(h)$.
\end{method}

\vspace{0.8\baselineskip}

\subsubsection{Klasifikace Speciálních Případů}
\label{subsubsec:klasifikace-specialnich-pripadu}

\begin{remark}[Rovnice s absolutními hodnotami]
Pro rovnice obsahující absolutní hodnoty je nutné uvažovat případné dělení definičního oboru a zkoumat spojitost řešení v bodech, kde se mění znaménko výrazu uvnitř absolutní hodnoty.
\end{remark}

\vspace{0.6\baselineskip}

\begin{remark}[Po částech definované pravé strany]
Pokud je $f(x)$ nebo $g(y)$ po částech definovaná, řešíme rovnici separátně na každém intervalu spojitosti a poté zkoumáme spojitost řešení v hraničních bodech.
\end{remark}

\vspace{0.6\baselineskip}

\begin{remark}[Rovnice s parametry - bifurkace]
Rovnice tvaru $\frac{dy}{dx} = y(\lambda - y^2)$ vykazuje pitchfork bifurkaci při $\lambda = 0$. Pro $\lambda < 0$ existuje jedno stabilní řešení $y = 0$, pro $\lambda > 0$ existují tři rovnováhy: $y = 0$ (nestabilní) a $y = \pm\sqrt{\lambda}$ (stabilní).
\end{remark}

\vspace{0.6\baselineskip}

\begin{remark}[Singularity a jejich klasifikace]
Body, kde $g(y) = 0$, mohou být:
\begin{itemize}
\item \textbf{Regulární singularita}: Řešení lze prodloužit přes tento bod
\item \textbf{Esenciální singularita}: Řešení nelze prodloužit
\item \textbf{Bod větvení}: Řešení není jednoznačné
\end{itemize}
\end{remark}

\vspace{0.8\baselineskip}

\subsubsection{Početní Sekce - Hierarchická}


\paragraph*{Úroveň 1: Lehké příklady}

\begin{example}[Základní separace]
    Řešte rovnici: $\frac{dy}{dx} = 2xy$
    
    \vspace{0.3\baselineskip}
    
    \textbf{Řešení}: 
    \begin{enumerate}
    \item \textbf{Singulární body}: $g(y) = y = 0 \implies y = 0$ je konstantní řešení
    
    \item \textbf{Obecné řešení pro $y \neq 0$}: Separace proměnných:
    \[
    \int \frac{dy}{y} = \int 2x  dx \implies \ln|y| = x^2 + C \implies |y| = e^{x^2 + C}
    \]
    \[
    y = \pm e^C e^{x^2} = Ae^{x^2}, \quad A \in \mathbb{R}\setminus\{0\}
    \]
    
    \item \textbf{Kompletní řešení}: 
    \[
    y(x) = Ae^{x^2}, \quad A \in \mathbb{R}
    \]
    Toto zahrnuje i původní konstantní řešení $y = 0$ (pro $A = 0$)
    
    \item \textbf{Interval existence}: Řešení existuje na $\mathbb{R}$ pro všechny $A \in \mathbb{R}$
    \end{enumerate}
    \end{example}

\vspace{0.6\baselineskip}

\begin{example}[S počáteční podmínkou]
    Řešte: $\frac{dy}{dx} = \frac{x}{y}$, $y(0) = 2$
    
    \vspace{0.3\baselineskip}
    
    \textbf{Řešení}: 
    \begin{enumerate}
    \item \textbf{Singulární body}: $g(y) = \frac{1}{y}$ má singularitu v $y = 0$
    
    \item \textbf{Obecné řešení pro $y \neq 0$}: 
    \[
    \int y  dy = \int x  dx \implies \frac{y^2}{2} = \frac{x^2}{2} + C \implies y^2 = x^2 + 2C
    \]
    
    \item \textbf{Určení konstanty z počáteční podmínky}:
    \[
    y(0) = 2 \implies 4 = 0 + 2C \implies C = 2 \implies y^2 = x^2 + 4
    \]
    
    \item \textbf{Výběr větve řešení}: Protože $y(0) = 2 > 0$, volíme kladnou větev:
    \[
    y = \sqrt{x^2 + 4}
    \]
    
    \item \textbf{Interval existence}: Řešení existuje na $\mathbb{R}$ a platí $y(x) > 0$ pro všechna $x \in \mathbb{R}$
    
    \item \textbf{Ověření}: Dosazením do původní rovnice:
    \[
    \frac{dy}{dx} = \frac{x}{\sqrt{x^2 + 4}} = \frac{x}{y} \quad \checkmark
    \]
    \end{enumerate}
    \end{example}

\vspace{0.8\baselineskip}

\paragraph*{Úroveň 2: Střední příklady}

\begin{example}[Analýza singulárních bodů]
    Řešte: $\frac{dy}{dx} = x(y^2 - 4)$
    \vspace{0.3\baselineskip}
    
    \textbf{Řešení}: 
    \begin{enumerate}
    \item \textbf{Singulární body}: $g(y) = y^2 - 4 = 0 \implies y = \pm 2$ jsou konstantní řešení
    
    \item \textbf{Obecné řešení pro $y \neq \pm 2$}:
    \[
    \int \frac{dy}{y^2 - 4} = \int x  dx
    \]
    Rozklad na parciální zlomky:
    \[
    \frac{1}{y^2 - 4} = \frac{1}{4}\left(\frac{1}{y-2} - \frac{1}{y+2}\right)
    \]
    Integrace:
    \[
    \frac{1}{4} \ln\left|\frac{y-2}{y+2}\right| = \frac{x^2}{2} + C
    \]
    \[
    \left|\frac{y-2}{y+2}\right| = e^{2x^2 + 4C} = Ke^{2x^2}, \quad K > 0
    \]
    
    \item \textbf{Explicitní vyjádření}:
    \[
    \frac{y-2}{y+2} = \pm Ke^{2x^2} = Ae^{2x^2}, \quad A \in \mathbb{R}\setminus\{0\}
    \]
    \[
    y-2 = Ae^{2x^2}(y+2) \implies y(1 - Ae^{2x^2}) = 2 + 2Ae^{2x^2}
    \]
    \[
    y = \frac{2(1 + Ae^{2x^2})}{1 - Ae^{2x^2}}, \quad A \in \mathbb{R}
    \]
    
    \item \textbf{Kompletní řešení včetně singulárních}:
    \[
    y(x) = \frac{2(1 + Ae^{2x^2})}{1 - Ae^{2x^2}}, \quad A \in \mathbb{R} \quad \text{NEBO} \quad y(x) = \pm 2
    \]
    Konstantní řešení $y = \pm 2$ odpovídají limitám $A \to \infty$
    
    \item \textbf{Intervaly existence}: 
    \begin{itemize}
    \item Pro $A < 0$: řešení existuje na $\mathbb{R}$
    \item Pro $A > 0$: řešení má singularitu když $1 - Ae^{2x^2} = 0$
    \end{itemize}
    \end{enumerate}
    \end{example}

\vspace{0.6\baselineskip}

\begin{example}[Substituční metoda]
    Řešte: $\frac{dy}{dx} = (x + y)^2$
    
    \vspace{0.3\baselineskip}
    
    \textbf{Řešení}: 
    \begin{enumerate}
    \item \textbf{Substituce}: $u = x + y \implies \frac{du}{dx} = 1 + \frac{dy}{dx} = 1 + u^2$
    
    \item \textbf{Separace proměnných}:
    \[
    \int \frac{du}{1 + u^2} = \int dx \implies \arctan u = x + C
    \]
    
    \item \textbf{Návrat k původní proměnné}:
    \[
    u = \tan(x + C) \implies y = \tan(x + C) - x
    \]
    
    \item \textbf{Singulární body}: Řešení má singularity když $\cos(x + C) = 0$, tedy pro $x + C = \frac{\pi}{2} + k\pi$, $k \in \mathbb{Z}$
    
    \item \textbf{Intervaly existence}: Řešení existuje na intervalech tvaru $\left(C - \frac{\pi}{2} + k\pi, C + \frac{\pi}{2} + k\pi\right)$
    
    \item \textbf{Ověření}:
    \[
    \frac{dy}{dx} = \frac{1}{\cos^2(x + C)} - 1 = \tan^2(x + C) + 1 - 1 = (x + y)^2 \quad \checkmark
    \]
    \end{enumerate}
    \end{example}

\vspace{0.8\baselineskip}

\paragraph*{Úroveň 3: Složité příklady}

\begin{example}[Rovnice s absolutní hodnotou]
    Řešte: $\frac{dy}{dx} = |x|y$
    \vspace{0.3\baselineskip}
    
    \textbf{Řešení}: 
    \begin{enumerate}
    \item \textbf{Singulární body}: $g(y) = y = 0 \implies y = 0$ je konstantní řešení
    
    \item \textbf{Řešení pro $x \geq 0$}: $|x| = x$
    \[
    \frac{dy}{dx} = xy \implies \int \frac{dy}{y} = \int x  dx \implies \ln|y| = \frac{x^2}{2} + C_1
    \]
    \[
    y = \pm e^{C_1} e^{x^2/2} = A e^{x^2/2}, \quad A \in \mathbb{R}
    \]
    
    \item \textbf{Řešení pro $x < 0$}: $|x| = -x$
    \[
    \frac{dy}{dx} = -xy \implies \int \frac{dy}{y} = -\int x  dx \implies \ln|y| = -\frac{x^2}{2} + C_2
    \]
    \[
    y = \pm e^{C_2} e^{-x^2/2} = B e^{-x^2/2}, \quad B \in \mathbb{R}
    \]
    
    \item \textbf{Spojitost v $x = 0$}: 
    \[
    \lim_{x \to 0^-} y(x) = B = \lim_{x \to 0^+} y(x) = A
    \]
    Tedy $A = B = C$
    
    \item \textbf{Kompletní řešení}:
    \[
    y(x) = \begin{cases}
    C e^{-x^2/2} & \text{pro } x < 0 \\
    C e^{x^2/2} & \text{pro } x \geq 0
    \end{cases}, \quad C \in \mathbb{R}
    \]
    Toto zahrnuje i konstantní řešení $y = 0$ (pro $C = 0$)
    
    \item \textbf{Interval existence}: Řešení existuje na $\mathbb{R}$ pro všechna $C \in \mathbb{R}$
    \end{enumerate}
    \end{example}

\vspace{0.6\baselineskip}

\begin{example}[Rovnice s parametrem]
    Analyzujte rovnici: $\frac{dy}{dx} = \lambda y - y^3$ v závislosti na parametru $\lambda$.
    \vspace{0.3\baselineskip}
    
    \textbf{Řešení}: 
    \begin{enumerate}
    \item \textbf{Singulární body}: $g(y) = \lambda y - y^3 = y(\lambda - y^2) = 0$
    \[
    y = 0, \quad y = \pm\sqrt{\lambda} \quad \text{(pro $\lambda > 0$)}
    \]
    
    \item \textbf{Konstantní řešení}:
    \begin{itemize}
    \item Pro $\lambda < 0$: pouze $y = 0$
    \item Pro $\lambda = 0$: pouze $y = 0$  
    \item Pro $\lambda > 0$: $y = 0$, $y = \sqrt{\lambda}$, $y = -\sqrt{\lambda}$
    \end{itemize}
    
    \item \textbf{Obecné řešení pro $y \neq 0, \pm\sqrt{\lambda}$}:
    \[
    \int \frac{dy}{y(\lambda - y^2)} = \int dx
    \]
    Rozklad na parciální zlomky:
    \[
    \frac{1}{y(\lambda - y^2)} = \frac{1}{\lambda y} + \frac{y}{\lambda(\lambda - y^2)}
    \]
    Integrace:
    \[
    \frac{1}{\lambda} \ln|y| - \frac{1}{2\lambda} \ln|\lambda - y^2| = x + C
    \]
    \[
    \ln\left|\frac{y}{\sqrt{|\lambda - y^2|}}\right| = \lambda x + \lambda C
    \]
    
    \item \textbf{Stabilita rovnovážných bodů}:
    \begin{itemize}
    \item Pro $\lambda < 0$: $y = 0$ stabilní
    \item Pro $\lambda = 0$: $y = 0$ nestabilní
    \item Pro $\lambda > 0$: $y = 0$ nestabilní, $y = \pm\sqrt{\lambda}$ stabilní
    \end{itemize}
    
    \item \textbf{Intervaly existence}: Závisí na počáteční podmínce a parametru $\lambda$
    
    \item \textbf{Bifurkační analýza}: Jedná se o \emph{pitchfork bifurkaci} při $\lambda = 0$
    \end{enumerate}
    \end{example}

\vspace{0.8\baselineskip}

\paragraph*{Úroveň 4: Insane příklady}

\begin{example}[Rovnice s neelementárním integrálem]
    Řešte: $\frac{dy}{dx} = \frac{e^{-x^2}}{y}$
    
    \vspace{0.3\baselineskip}
    
    \textbf{Řešení}: 
    \begin{enumerate}
    \item \textbf{Singulární body}: $g(y) = \frac{1}{y}$ má singularitu v $y = 0$
    
    \item \textbf{Obecné řešení pro $y \neq 0$}: Separace:
    \[
    \int y  dy = \int e^{-x^2} dx \implies \frac{y^2}{2} = \frac{\sqrt{\pi}}{2} \erf(x) + C
    \]
    kde $\erf(x) = \frac{2}{\sqrt{\pi}} \int_0^x e^{-t^2} dt$ je error funkce.
    
    \item \textbf{Explicitní vyjádření}:
    \[
    y = \pm \sqrt{\sqrt{\pi} \erf(x) + 2C}
    \]
    
    \item \textbf{Podmínka existence}: Aby řešení bylo reálné, musí platit:
    \[
    \sqrt{\pi} \erf(x) + 2C \geq 0
    \]
    
    \item \textbf{Vlastnosti error funkce}: 
    \begin{itemize}
    \item $\erf(0) = 0$, $\lim_{x \to \infty} \erf(x) = 1$, $\lim_{x \to -\infty} \erf(x) = -1$
    \item $\erf(x)$ je lichá funkce
    \end{itemize}
    
    \item \textbf{Intervaly existence}: Závisí na konstantě $C$:
    \begin{itemize}
    \item Pro $C > \frac{\sqrt{\pi}}{2}$: řešení existuje na $\mathbb{R}$
    \item Pro $-\frac{\sqrt{\pi}}{2} < C \leq \frac{\sqrt{\pi}}{2}$: řešení existuje na omezeném intervalu
    \item Pro $C \leq -\frac{\sqrt{\pi}}{2}$: žádné reálné řešení
    \end{itemize}
    \end{enumerate}
    \end{example}

\vspace{0.6\baselineskip}

\begin{example}[Singularita a limitní chování]
    Analyzujte chování řešení: $\frac{dy}{dx} = \frac{1}{y^2 - 1}$ s $y(0) = 0$.
    
    \vspace{0.3\baselineskip}
    
    \textbf{Řešení}: 
    \begin{enumerate}
    \item \textbf{Singulární body}: $g(y) = \frac{1}{y^2 - 1}$ má singularity v $y = \pm 1$
    Konstantní řešení: $y = \pm 1$ (ověříme dosazením: $\frac{d(\pm 1)}{dx} = 0 = \frac{1}{1-1}$ - singularita)
    
    \item \textbf{Obecné řešení pro $y \neq \pm 1$}: Separace:
    \[
    \int (y^2 - 1) dy = \int dx \implies \frac{y^3}{3} - y = x + C
    \]
    
    \item \textbf{Určení konstanty}:
    \[
    y(0) = 0 \implies 0 - 0 = 0 + C \implies C = 0
    \]
    Tedy $\frac{y^3}{3} - y = x$
    
    \item \textbf{Analýza chování}:
    \begin{itemize}
    \item Pro $x \to -\frac{2}{3}$: $y \to -1$ (singularita)
    \item Pro $x \to \frac{2}{3}$: $y \to 1$ (singularita)
    \item Pro $x = 0$: $y = 0$
    \end{itemize}
    
    \item \textbf{Interval existence}: Řešení existuje na $(-\frac{2}{3}, \frac{2}{3})$
    
    \item \textbf{Implicitní křivka}: $\frac{y^3}{3} - y - x = 0$ je kubická v $y$
    
    \item \textbf{Stabilita}: 
    \begin{itemize}
    \item Pro $y < -1$: $\frac{dy}{dx} > 0$ (rostoucí)
    \item Pro $-1 < y < 1$: $\frac{dy}{dx} < 0$ (klesající)  
    \item Pro $y > 1$: $\frac{dy}{dx} > 0$ (rostoucí)
    \end{itemize}
    \end{enumerate}
    \end{example}

\vspace{0.8\baselineskip}

\paragraph*{Úroveň 5: Quant Level}

\begin{example}[Logistický růst s kalibrací]
    Mějme data o růstu tržního podílu: počáteční podíl 5\%, maximální kapacita 80\%, po 2 letech podíl 20\%. Kalibrujte logistický model.
    
    \vspace{0.3\baselineskip}
    
    \textbf{Řešení}: 
    \begin{enumerate}
    \item \textbf{Logistická rovnice}: $\frac{dP}{dt} = rP(1 - \frac{P}{K})$
    
    \item \textbf{Singulární body}: $P = 0$ a $P = K$ jsou konstantní řešení
    
    \item \textbf{Obecné řešení pro $0 < P < K$}:
    \[
    P(t) = \frac{K}{1 + Ae^{-rt}}, \quad \text{kde } A = \frac{K - P_0}{P_0}
    \]
    
    \item \textbf{Dosazení parametrů}:
    \[
    P_0 = 0.05, \quad K = 0.8, \quad P(2) = 0.2
    \]
    \[
    A = \frac{0.8 - 0.05}{0.05} = 15
    \]
    \[
    0.2 = \frac{0.8}{1 + 15e^{-2r}} \implies 1 + 15e^{-2r} = 4 \implies e^{-2r} = 0.2
    \]
    \[
    r = -\frac{1}{2}\ln(0.2) \approx 0.8047
    \]
    
    \item \textbf{Interval existence}: Řešení existuje na $\mathbb{R}$ a platí $0 < P(t) < K$ pro všechna $t \in \mathbb{R}$
    
    \item \textbf{Predikce}: Po 5 letech: $P(5) = \frac{0.8}{1 + 15e^{-0.8047 \cdot 5}} \approx 0.634$ (63.4\%)
    \end{enumerate}
    \end{example}

\vspace{0.6\baselineskip}

\begin{example}[Mertonův model defaultu]
    Zjednodušený Mertonův model: $\frac{dV}{dt} = \mu V - C$, kde $V$ je hodnota firmy, $C$ konstantní výplaty. Určete podmínky pro nesplatnost.
    
    \vspace{0.3\baselineskip}
    
    \textbf{Řešení}: 
    \begin{enumerate}
    \item \textbf{Singulární body}: Žádné konstantní řešení kromě případu $\mu = 0$, $C = 0$
    
    \item \textbf{Obecné řešení}: Rovnici přepíšeme:
    \[
    \frac{dV}{dt} = \mu\left(V - \frac{C}{\mu}\right)
    \]
    Substituce $U = V - \frac{C}{\mu}$:
    \[
    \frac{dU}{dt} = \mu U \implies U(t) = U_0 e^{\mu t}
    \]
    Tedy:
    \[
    V(t) = \frac{C}{\mu} + \left(V_0 - \frac{C}{\mu}\right)e^{\mu t}
    \]
    
    \item \textbf{Podmínka nesplatnosti}: $V(t) \leq 0$ pro nějaké $t > 0$
    
    \item \textbf{Analýza případů}:
    \begin{itemize}
    \item \textbf{Případ 1}: $\mu > 0$, $V_0 > \frac{C}{\mu}$ - hodnota roste, žádná nesplatnost
    \item \textbf{Případ 2}: $\mu > 0$, $V_0 = \frac{C}{\mu}$ - konstantní hodnota $V(t) = \frac{C}{\mu}$
    \item \textbf{Případ 3}: $\mu > 0$, $V_0 < \frac{C}{\mu}$ - hodnota klesá k $-\infty$, nesplatnost v čase $t^*$
    \item \textbf{Případ 4}: $\mu = 0$ - lineární poklad $V(t) = V_0 - Ct$, nesplatnost při $t > \frac{V_0}{C}$
    \item \textbf{Případ 5}: $\mu < 0$ - hodnota konverguje k $\frac{C}{\mu} < 0$, vždy nastane nesplatnost
    \end{itemize}
    
    \item \textbf{Interval existence}: Řešení existuje na $\mathbb{R}$ dokud nenastane nesplatnost
    
    \item \textbf{Čas nesplatnosti}: Pro $\mu > 0$, $V_0 < \frac{C}{\mu}$:
    \[
    V(t^*) = 0 \implies t^* = \frac{1}{\mu} \ln\left(\frac{C}{C - \mu V_0}\right)
    \]
    \end{enumerate}
    \end{example}

\vspace{0.8\baselineskip}

\subsubsection{Kvantitativní Aplikace}
\label{subsubsec:kvantitativni-aplikace}

\begin{application}[Exponenciální růst a úročení]
Model spojitého úročení: $\frac{dA}{dt} = rA$ má řešení $A(t) = A_0 e^{rt}$.

\textbf{Kvantitativní interpretace}:
\begin{itemize}
\item $r$: okamžitá úroková míra (force of interest)
\item $A(t)$: hodnota investice v čase $t$
\item Aplikace: Oceňování zero-coupon bondů, diskontování cash flow
\end{itemize}
\end{application}

\vspace{0.6\baselineskip}

\begin{application}[Logistické modely v ekonomii]
Logistická rovnice: $\frac{dP}{dt} = rP(1 - \frac{P}{K})$ modeluje:
\begin{itemize}
\item Difúzi inovací na trhu
\item Růst tržního podílu
\item Saturaci poptávky
\item Adopci technologií
\end{itemize}
Parametr $K$ představuje nosnou kapacitu trhu.
\end{application}

\vspace{0.6\baselineskip}

\begin{application}[Stochastická volatilita - příprava]
Deterministická aproximace: $\frac{d\sigma}{dt} = \kappa(\theta - \sigma)$ má řešení:
\[
\sigma(t) = \theta + (\sigma_0 - \theta)e^{-\kappa t}
\]
Tento model připravuje půdu pro Hestonův stochastický model volatility.
\end{application}

\vspace{0.8\baselineskip}

\subsubsection{Shrnutí a Přechod}
\label{subsubsec:shrnuti-presun}

Separabilní rovnice představují nejzákladnější, ale překvapivě mocnou třídu ODE 1. řádu. Zvládnutí této třídy je nezbytné pro:

\begin{itemize}
\item Porozumění základním růstovým modelům v ekonomii a financích
\item Analýzu stability dynamických systémů
\item Přípravu na komplexnější nelineární modely
\item Rozvoj intuice pro chování diferenciálních rovnic
\end{itemize}

\vspace{0.6\baselineskip}

\begin{remark}[Časté chyby a jak se jim vyhnout]
\begin{itemize}
\item \textbf{Ztráta řešení}: Při dělení $g(y)$ vždy zkontrolujte body kde $g(y) = 0$
\item \textbf{Špatný definiční obor}: Vždy určete maximální interval existence
\item \textbf{Chybná integrační konstanta}: Pečlivě pracujte s absolutními hodnotami a znaménky
\item \textbf{Ignorování singularity}: Analyzujte chování řešení v singulárních bodech
\end{itemize}
\end{remark}

\vspace{0.8\baselineskip}

\begin{transition}
S pevným pochopením separabilních rovnic jsme připraveni přejít k lineárním rovnicím 1. řádu, které představují jejich přirozené zobecnění a otevírají cestu k modelování systémů s vnějšími vstupy a řízením. Lineární rovnice nám umožní systematicky řešit problémy, kde separace proměnných není možná.
\end{transition}

% !TEX root = ../main.tex
\subsection{Lineární Rovnice 1. Řádu - Kompletní Teorie}
\label{subsec:linearni-rovnice}

\subsubsection{Teoretický Fundament}
\label{subsubsec:teoreticky-fundament-linearni}

\begin{definition}[Lineární diferenciální rovnice 1. řádu]
Rovnice je \emph{lineární 1. řádu}, jestliže ji lze zapsat ve standardním tvaru:
\[
\frac{dy}{dx} + P(x)y = Q(x)
\]
kde $P(x)$ a $Q(x)$ jsou funkce definované na nějakém intervalu $I \subseteq \mathbb{R}$.
\end{definition}

\vspace{0.6\baselineskip}

\begin{theorem}[Existence a jednoznačnost řešení]
Nechť $P(x)$ a $Q(x)$ jsou spojité funkce na otevřeném intervalu $I$. Pak pro libovolný bod $x_0 \in I$ a libovolnou počáteční hodnotu $y_0 \in \mathbb{R}$ existuje právě jedno řešení $y(x)$ rovnice definované na celém intervalu $I$ splňující $y(x_0) = y_0$.
\end{theorem}

\vspace{0.4\baselineskip}

\begin{proof}
Důkaz využívá Picard-Lindelöfovu větu z Kapitoly 3. Pro lineární rovnici je pravá strana $f(x,y) = Q(x) - P(x)y$ spojitá v $x$ a Lipschitzovská v $y$ na každém kompaktním podintervalu $I$, což zaručuje existenci a jednoznačnost řešení.
\end{proof}

\vspace{0.6\baselineskip}

\begin{theorem}[Struktura řešení lineární rovnice]
Obecné řešení lineární rovnice má tvar:
\[
y(x) = y_h(x) + y_p(x)
\]
kde $y_h(x)$ je obecné řešení homogenní rovnice $\frac{dy}{dx} + P(x)y = 0$ a $y_p(x)$ je partikulární řešení nehomogenní rovnice.
\end{theorem}

\vspace{0.4\baselineskip}

\begin{proof}
Homogenní rovnice má řešení $y_h(x) = Ce^{-\int P(x)dx}$. Partikulární řešení najdeme metodou integračního faktoru nebo variace konstant.
\end{proof}

\vspace{0.8\baselineskip}

\subsubsection{Metoda Integračního Faktoru - Kompletní Analýza}
\label{subsubsec:metoda-integracniho-faktoru}

\begin{method}[Metoda integračního faktoru]
\label{met:integracni-faktor}
\begin{enumerate}
\item \textbf{Standardní tvar}: Ujistěte se, že rovnice je ve tvaru $\frac{dy}{dx} + P(x)y = Q(x)$

\item \textbf{Integrační faktor}: Vypočítejte
\[
\mu(x) = e^{\int P(x)  dx}
\]
Poznámka: Integrační konstanta se zde neuvádí.

\item \textbf{Násobení rovnice}: Vynásobte celou rovnici integračním faktorem:
\[
\mu(x)\frac{dy}{dx} + \mu(x)P(x)y = \mu(x)Q(x)
\]

\item \textbf{Rozpoznání derivace součinu}: Levou stranu lze zapsat jako:
\[
\frac{d}{dx}[\mu(x)y] = \mu(x)Q(x)
\]
Toto platí protože $\frac{d\mu}{dx} = P(x)\mu(x)$.

\item \textbf{Integrace}:
\[
\mu(x)y = \int \mu(x)Q(x)  dx + C
\]

\item \textbf{Vyjádření řešení}:
\[
y(x) = \frac{1}{\mu(x)} \left[\int \mu(x)Q(x)  dx + C\right]
\]
\end{enumerate}
\end{method}

\vspace{0.8\baselineskip}

\begin{theorem}[Vlastnosti integračního faktoru]
Integrační faktor $\mu(x) = e^{\int P(x)dx}$ má následující vlastnosti:
\begin{itemize}
\item $\mu(x) > 0$ pro všechna $x \in I$
\item $\mu(x)$ je spojitě diferencovatelná na $I$
\item $\frac{d\mu}{dx} = P(x)\mu(x)$
\item $\mu(x)$ je určen jednoznačně až na multiplikativní konstantu
\end{itemize}
\end{theorem}

\vspace{0.6\baselineskip}

\begin{example}[Odvození metody integračního faktoru]
Uvažujme rovnici $\frac{dy}{dx} + P(x)y = Q(x)$. Hledáme funkci $\mu(x)$ takovou, že:
\[
\frac{d}{dx}[\mu(x)y] = \mu(x)\frac{dy}{dx} + \frac{d\mu}{dx}y = \mu(x)Q(x)
\]
Porovnáním s původní rovnicí dostaneme:
\[
\frac{d\mu}{dx} = P(x)\mu(x)
\]
Tato rovnice je separabilní:
\[
\int \frac{d\mu}{\mu} = \int P(x)dx \implies \ln|\mu| = \int P(x)dx \implies \mu(x) = e^{\int P(x)dx}
\]
\end{example}

\vspace{0.8\baselineskip}

\subsubsection{Metoda Variace Konstant}
\label{subsubsec:metoda-variance-konstant}

\begin{method}[Metoda variace konstant]
\label{met:variace-konstant}
\begin{enumerate}
\item \textbf{Řešení homogenní rovnice}: Nejprve vyřešte $\frac{dy}{dx} + P(x)y = 0$:
\[
y_h(x) = Ce^{-\int P(x)dx}
\]

\item \textbf{Variace konstanty}: Předpokládejte řešení ve tvaru:
\[
y(x) = C(x)e^{-\int P(x)dx}
\]
kde $C(x)$ je neznámá funkce.

\item \textbf{Dosazení do původní rovnice}:
\[
\frac{dy}{dx} = C'(x)e^{-\int P(x)dx} - C(x)P(x)e^{-\int P(x)dx}
\]
Dosazením do $\frac{dy}{dx} + P(x)y = Q(x)$ dostaneme:
\[
C'(x)e^{-\int P(x)dx} = Q(x)
\]

\item \textbf{Řešení pro C(x)}:
\[
C'(x) = Q(x)e^{\int P(x)dx} \implies C(x) = \int Q(x)e^{\int P(x)dx}dx + K
\]

\item \textbf{Celkové řešení}:
\[
y(x) = e^{-\int P(x)dx}\left[\int Q(x)e^{\int P(x)dx}dx + K\right]
\]
\end{enumerate}
\end{method}

\vspace{0.8\baselineskip}

\begin{theorem}[Ekvivalence metod]
Metoda integračního faktoru a metoda variace konstant jsou ekvivalentní a vedou ke stejnému výsledku.
\end{theorem}

\vspace{0.4\baselineskip}

\begin{proof}
V metodě integračního faktoru máme:
\[
y(x) = \frac{1}{\mu(x)}\left[\int \mu(x)Q(x)dx + C\right]
\]
V metodě variace konstant:
\[
y(x) = e^{-\int P(x)dx}\left[\int Q(x)e^{\int P(x)dx}dx + K\right]
\]
Protože $\mu(x) = e^{\int P(x)dx}$, jsou oba výrazy identické.
\end{proof}

\vspace{0.8\baselineskip}

\subsubsection{Klasifikace Speciálních Případů P(x)}
\label{subsubsec:klasifikace-px}

\begin{remark}[Konstantní P(x)]
Pro $P(x) = a$ (konstanta) je integrační faktor:
\[
\mu(x) = e^{\int a  dx} = e^{ax}
\]
Řešení homogenní rovnice: $y_h(x) = Ce^{-ax}$
\end{remark}

\vspace{0.6\baselineskip}

\begin{remark}[Polynomiální P(x)]
Pro $P(x) = a_nx^n + a_{n-1}x^{n-1} + \cdots + a_0$:
\[
\mu(x) = e^{\int P(x)dx} = e^{\frac{a_n}{n+1}x^{n+1} + \frac{a_{n-1}}{n}x^n + \cdots + a_0x}
\]
Integrace vyžaduje výpočet integrálu polynomu.
\end{remark}

\vspace{0.6\baselineskip}

\begin{remark}[Racionální P(x)]
Pro $P(x) = \frac{N(x)}{D(x)}$ je třeba integrovat racionální funkci. Použijeme rozklad na parciální zlomky:
\[
\int \frac{N(x)}{D(x)} dx = \int \left(\text{parciální zlomky}\right) dx
\]
\end{remark}

\vspace{0.6\baselineskip}

\begin{remark}[Goniometrické P(x)]
Pro $P(x)$ obsahující goniometrické funkce může integrace vyžadovat trigonometrické substituce nebo použití speciálních technik.
\end{remark}

\vspace{0.8\baselineskip}

\subsubsection{Početní Sekce - Kategorie A: Podle typu P(x)}
\label{subsubsec:pocetni-kategorie-a-px}

\paragraph*{A1: Konstantní P(x)}

\begin{example}[Lehký příklad - konstantní P(x)]
Řešte: $\frac{dy}{dx} + 2y = 1$
\vspace{0.3\baselineskip}

\textbf{Řešení}: 
\begin{enumerate}
\item $P(x) = 2$, $Q(x) = 1$ (konstantní)

\item \textbf{Integrační faktor}:
\[
\mu(x) = e^{\int 2  dx} = e^{2x}
\]

\item \textbf{Násobení rovnice}:
\[
e^{2x}\frac{dy}{dx} + 2e^{2x}y = e^{2x}
\]

\item \textbf{Integrace}:
\[
\frac{d}{dx}[e^{2x}y] = e^{2x} \implies e^{2x}y = \int e^{2x} dx = \frac{1}{2}e^{2x} + C
\]

\item \textbf{Výsledek}:
\[
y(x) = \frac{1}{2} + Ce^{-2x}
\]

\item \textbf{Ověření}:
\[
\frac{dy}{dx} = -2Ce^{-2x}, \quad 2y = 1 + 2Ce^{-2x}
\]
\[
\frac{dy}{dx} + 2y = -2Ce^{-2x} + 1 + 2Ce^{-2x} = 1 \quad \checkmark
\end{enumerate}
\end{example}

\vspace{0.6\baselineskip}

\begin{example}[Střední příklad - konstantní P(x)]
Řešte: $\frac{dy}{dx} - 3y = e^x$ s $y(0) = 1$
\vspace{0.3\baselineskip}

\textbf{Řešení}: 
\begin{enumerate}
\item $P(x) = -3$, $Q(x) = e^x$

\item \textbf{Integrační faktor}:
\[
\mu(x) = e^{\int -3  dx} = e^{-3x}
\]

\item \textbf{Násobení a integrace}:
\[
\frac{d}{dx}[e^{-3x}y] = e^{-3x}e^x = e^{-2x}
\]
\[
e^{-3x}y = \int e^{-2x} dx = -\frac{1}{2}e^{-2x} + C
\]

\item \textbf{Obecné řešení}:
\[
y(x) = -\frac{1}{2}e^{x} + Ce^{3x}
\]

\item \textbf{Určení konstanty}:
\[
y(0) = 1 \implies -\frac{1}{2} + C = 1 \implies C = \frac{3}{2}
\]

\item \textbf{Výsledek}:
\[
y(x) = -\frac{1}{2}e^{x} + \frac{3}{2}e^{3x}
\end{enumerate}
\end{example}

\vspace{0.6\baselineskip}

\begin{example}[Složitý příklad - konstantní P(x)]
Řešte: $\frac{dy}{dx} + 5y = \sin(2x)$
\vspace{0.3\baselineskip}

\textbf{Řešení}: 
\begin{enumerate}
\item $P(x) = 5$, $Q(x) = \sin(2x)$

\item \textbf{Integrační faktor}:
\[
\mu(x) = e^{\int 5  dx} = e^{5x}
\]

\item \textbf{Integrace}:
\[
\frac{d}{dx}[e^{5x}y] = e^{5x}\sin(2x)
\]
\[
e^{5x}y = \int e^{5x}\sin(2x) dx
\]

\item \textbf{Integrace per partes}: Použijeme metodu pro $\int e^{ax}\sin(bx)dx$:
\[
\int e^{5x}\sin(2x)dx = \frac{e^{5x}}{29}(5\sin(2x) - 2\cos(2x)) + K
\]

\item \textbf{Výsledek}:
\[
y(x) = \frac{1}{29}(5\sin(2x) - 2\cos(2x)) + Ke^{-5x}
\end{enumerate}
\end{example}

\vspace{0.8\baselineskip}

\paragraph*{A2: Polynomiální P(x)}

\begin{example}[Lehký příklad - polynomiální P(x)]
Řešte: $\frac{dy}{dx} + xy = 1$
\vspace{0.3\baselineskip}

\textbf{Řešení}: 
\begin{enumerate}
\item $P(x) = x$, $Q(x) = 1$

\item \textbf{Integrační faktor}:
\[
\mu(x) = e^{\int x  dx} = e^{x^2/2}
\]

\item \textbf{Integrace}:
\[
\frac{d}{dx}[e^{x^2/2}y] = e^{x^2/2}
\]
\[
e^{x^2/2}y = \int e^{x^2/2} dx
\]

\item \textbf{Integrál nelze vyjádřit elementárními funkcemi}:
\[
y(x) = e^{-x^2/2}\left[\int e^{x^2/2} dx + C\right]
\]

\item \textbf{Řešení pomocí error funkce}:
\[
\int e^{x^2/2} dx = \sqrt{\frac{\pi}{2}} \erf\left(\frac{x}{\sqrt{2}}\right) + K
\]
\[
y(x) = e^{-x^2/2}\left[\sqrt{\frac{\pi}{2}} \erf\left(\frac{x}{\sqrt{2}}\right) + C\right]
\end{enumerate}
\end{example}

\vspace{0.6\baselineskip}

\begin{example}[Střední příklad - polynomiální P(x)]
Řešte: $\frac{dy}{dx} + x^2y = x$
\vspace{0.3\baselineskip}

\textbf{Řešení}: 
\begin{enumerate}
\item $P(x) = x^2$, $Q(x) = x$

\item \textbf{Integrační faktor}:
\[
\mu(x) = e^{\int x^2 dx} = e^{x^3/3}
\]

\item \textbf{Integrace}:
\[
\frac{d}{dx}[e^{x^3/3}y] = xe^{x^3/3}
\]
\[
e^{x^3/3}y = \int xe^{x^3/3} dx
\]

\item \textbf{Substituce}: $u = x^3/3 \implies du = x^2 dx$ - nefunguje přímo

\item \textbf{Řešení pomocí řady}: Rozvineme $e^{x^3/3}$ do řady:
\[
e^{x^3/3} = \sum_{n=0}^{\infty} \frac{x^{3n}}{3^n n!}
\]
\[
xe^{x^3/3} = \sum_{n=0}^{\infty} \frac{x^{3n+1}}{3^n n!}
\]
\[
\int xe^{x^3/3} dx = \sum_{n=0}^{\infty} \frac{x^{3n+2}}{(3n+2)3^n n!} + C
\]

\item \textbf{Výsledek}:
\[
y(x) = e^{-x^3/3}\left[\sum_{n=0}^{\infty} \frac{x^{3n+2}}{(3n+2)3^n n!} + C\right]
\end{enumerate}
\end{example}

\vspace{0.8\baselineskip}


\subsubsection{Početní Sekce - Kategorie A (pokračování)}
\label{subsubsec:pocetni-kategorie-a-px-pokracovani}

\paragraph*{A3: Racionální P(x)}

\begin{example}[Lehký příklad - racionální P(x)]
Řešte: $\frac{dy}{dx} + \frac{1}{x}y = 1$
\vspace{0.3\baselineskip}

\textbf{Řešení}: 
\begin{enumerate}
\item $P(x) = \frac{1}{x}$, $Q(x) = 1$, $x \neq 0$

\item \textbf{Integrační faktor}:
\[
\mu(x) = e^{\int \frac{1}{x} dx} = e^{\ln|x|} = |x|
\]
Pro $x > 0$: $\mu(x) = x$, pro $x < 0$: $\mu(x) = -x$

\item \textbf{Řešení pro $x > 0$}:
\[
\frac{d}{dx}[xy] = x \implies xy = \int x  dx = \frac{x^2}{2} + C
\]
\[
y(x) = \frac{x}{2} + \frac{C}{x}
\]

\item \textbf{Interval existence}: $(0, \infty)$

\item \textbf{Ověření}:
\[
\frac{dy}{dx} = \frac{1}{2} - \frac{C}{x^2}, \quad \frac{1}{x}y = \frac{1}{2} + \frac{C}{x^2}
\]
\[
\frac{dy}{dx} + \frac{1}{x}y = \frac{1}{2} - \frac{C}{x^2} + \frac{1}{2} + \frac{C}{x^2} = 1 \quad \checkmark
\end{enumerate}
\end{example}

\vspace{0.6\baselineskip}

\begin{example}[Střední příklad - racionální P(x)]
Řešte: $\frac{dy}{dx} + \frac{2x}{x^2 + 1}y = x$
\vspace{0.3\baselineskip}

\textbf{Řešení}: 
\begin{enumerate}
\item $P(x) = \frac{2x}{x^2 + 1}$, $Q(x) = x$

\item \textbf{Integrační faktor}:
\[
\int P(x)dx = \int \frac{2x}{x^2 + 1}dx = \ln(x^2 + 1)
\]
\[
\mu(x) = e^{\ln(x^2 + 1)} = x^2 + 1
\]

\item \textbf{Integrace}:
\[
\frac{d}{dx}[(x^2 + 1)y] = x(x^2 + 1) = x^3 + x
\]
\[
(x^2 + 1)y = \int (x^3 + x)dx = \frac{x^4}{4} + \frac{x^2}{2} + C
\]

\item \textbf{Výsledek}:
\[
y(x) = \frac{\frac{x^4}{4} + \frac{x^2}{2} + C}{x^2 + 1} = \frac{x^2(x^2 + 2)}{4(x^2 + 1)} + \frac{C}{x^2 + 1}
\]

\item \textbf{Interval existence}: $\mathbb{R}$
\end{enumerate}
\end{example}

\vspace{0.6\baselineskip}

\begin{example}[Složitý příklad - racionální P(x)]
Řešte: $\frac{dy}{dx} + \frac{1}{x-1}y = \frac{1}{x^2 - 1}$ s $y(0) = 2$
\vspace{0.3\baselineskip}

\textbf{Řešení}: 
\begin{enumerate}
\item $P(x) = \frac{1}{x-1}$, $Q(x) = \frac{1}{(x-1)(x+1)}$, $x \neq \pm 1$

\item \textbf{Integrační faktor}:
\[
\mu(x) = e^{\int \frac{1}{x-1}dx} = e^{\ln|x-1|} = |x-1|
\]
Pro $x < 1$: $\mu(x) = 1-x$, pro $x > 1$: $\mu(x) = x-1$

\item \textbf{Řešení pro $x < 1$} (kde $y(0) = 2$):
\[
\frac{d}{dx}[(1-x)y] = (1-x)\cdot\frac{1}{(x-1)(x+1)} = -\frac{1}{x+1}
\]
\[
(1-x)y = -\int \frac{1}{x+1}dx = -\ln|x+1| + C
\]

\item \textbf{Určení konstanty}:
\[
y(0) = 2 \implies (1-0)\cdot 2 = -\ln|1| + C \implies 2 = C
\]
\[
y(x) = \frac{-\ln|x+1| + 2}{1-x} = \frac{2 - \ln(x+1)}{1-x} \quad \text{pro } -1 < x < 1
\]

\item \textbf{Interval existence}: $(-1, 1)$
\end{enumerate}
\end{example}

\vspace{0.8\baselineskip}

\paragraph*{A4: Goniometrické P(x)}

\begin{example}[Lehký příklad - goniometrické P(x)]
Řešte: $\frac{dy}{dx} + (\tan x)y = \cos x$ pro $x \in (-\frac{\pi}{2}, \frac{\pi}{2})$
\vspace{0.3\baselineskip}

\textbf{Řešení}: 
\begin{enumerate}
\item $P(x) = \tan x$, $Q(x) = \cos x$

\item \textbf{Integrační faktor}:
\[
\int \tan x  dx = \int \frac{\sin x}{\cos x} dx = -\ln|\cos x|
\]
\[
\mu(x) = e^{-\ln|\cos x|} = \frac{1}{|\cos x|}
\]
Pro $x \in (-\frac{\pi}{2}, \frac{\pi}{2})$: $\cos x > 0$, tedy $\mu(x) = \frac{1}{\cos x} = \sec x$

\item \textbf{Integrace}:
\[
\frac{d}{dx}[\sec x \cdot y] = \sec x \cdot \cos x = 1
\]
\[
\sec x \cdot y = \int 1  dx = x + C
\]

\item \textbf{Výsledek}:
\[
y(x) = x \cos x + C \cos x
\]

\item \textbf{Interval existence}: $(-\frac{\pi}{2}, \frac{\pi}{2})$
\end{enumerate}
\end{example}

\vspace{0.6\baselineskip}

\begin{example}[Střední příklad - goniometrické P(x)]
Řešte: $\frac{dy}{dx} + (\sin x)y = e^{\cos x}$
\vspace{0.3\baselineskip}

\textbf{Řešení}: 
\begin{enumerate}
\item $P(x) = \sin x$, $Q(x) = e^{\cos x}$

\item \textbf{Integrační faktor}:
\[
\int \sin x  dx = -\cos x
\]
\[
\mu(x) = e^{-\cos x}
\]

\item \textbf{Integrace}:
\[
\frac{d}{dx}[e^{-\cos x}y] = e^{-\cos x} \cdot e^{\cos x} = 1
\]
\[
e^{-\cos x}y = \int 1  dx = x + C
\]

\item \textbf{Výsledek}:
\[
y(x) = (x + C)e^{\cos x}
\]

\item \textbf{Interval existence}: $\mathbb{R}$
\end{enumerate}
\end{example}

\vspace{0.8\baselineskip}

\subsubsection{Klasifikace Pravých Stran Q(x)}
\label{subsubsec:klasifikace-qx}

\paragraph*{B1: Polynomiální Q(x)}

\begin{remark}[Obecný postup pro polynomiální Q(x)]
Pro $Q(x) = a_nx^n + a_{n-1}x^{n-1} + \cdots + a_0$:
\begin{itemize}
\item Vypočítejte $\mu(x) = e^{\int P(x)dx}$
\item Spočítejte $\int \mu(x)Q(x)dx$ - typicky vede na integraci součinu
\item Výsledek může obsahovat neelementární integrály pro složitější P(x)
\end{itemize}
\end{remark}

\vspace{0.6\baselineskip}

\begin{example}[Konstantní Q(x)]
Řešte: $\frac{dy}{dx} + xy = 2$
\vspace{0.3\baselineskip}

\textbf{Řešení}: 
\begin{enumerate}
\item $P(x) = x$, $Q(x) = 2$ (konstantní)

\item \textbf{Integrační faktor}: $\mu(x) = e^{x^2/2}$

\item \textbf{Integrace}:
\[
\frac{d}{dx}[e^{x^2/2}y] = 2e^{x^2/2}
\]
\[
e^{x^2/2}y = 2\int e^{x^2/2}dx
\]

\item \textbf{Výsledek pomocí error funkce}:
\[
y(x) = 2e^{-x^2/2} \cdot \sqrt{\frac{\pi}{2}} \erf\left(\frac{x}{\sqrt{2}}\right) + Ce^{-x^2/2}
\]
\end{enumerate}
\end{example}

\vspace{0.6\baselineskip}

\paragraph*{B2: Exponenciální Q(x)}

\begin{remark}[Obecný postup pro exponenciální Q(x)]
Pro $Q(x) = Ae^{\alpha x}$:
\begin{itemize}
\item Pokud $\alpha \neq -P(x)$, řešení obsahuje exponenciální funkce
\item Speciální případ: rezonance když $\alpha = -P(x)$ pro konstantní P(x)
\item Integrace typicky vede na kombinace exponenciálních funkcí
\end{itemize}
\end{remark}

\vspace{0.6\baselineskip}

\begin{example}[Exponenciální Q(x) s konstantní P(x)]
Řešte: $\frac{dy}{dx} + 2y = 3e^{4x}$
\vspace{0.3\baselineskip}

\textbf{Řešení}: 
\begin{enumerate}
\item $P(x) = 2$, $Q(x) = 3e^{4x}$

\item \textbf{Integrační faktor}: $\mu(x) = e^{2x}$

\item \textbf{Integrace}:
\[
\frac{d}{dx}[e^{2x}y] = 3e^{2x}e^{4x} = 3e^{6x}
\]
\[
e^{2x}y = 3\int e^{6x}dx = \frac{1}{2}e^{6x} + C
\]

\item \textbf{Výsledek}:
\[
y(x) = \frac{1}{2}e^{4x} + Ce^{-2x}
\]
\end{enumerate}
\end{example}

\vspace{0.6\baselineskip}

\begin{example}[Rezonanční případ]
Řešte: $\frac{dy}{dx} + 2y = 3e^{-2x}$
\vspace{0.3\baselineskip}

\textbf{Řešení}: 
\begin{enumerate}
\item $P(x) = 2$, $Q(x) = 3e^{-2x}$ - rezonance protože $-P(x) = -2$

\item \textbf{Integrační faktor}: $\mu(x) = e^{2x}$

\item \textbf{Integrace}:
\[
\frac{d}{dx}[e^{2x}y] = 3e^{2x}e^{-2x} = 3
\]
\[
e^{2x}y = 3\int 1  dx = 3x + C
\]

\item \textbf{Výsledek}:
\[
y(x) = 3xe^{-2x} + Ce^{-2x}
\]
Všimněte si lineárního členu $x$ v partikulárním řešení - typické pro rezonanci.
\end{enumerate}
\end{example}

\vspace{0.8\baselineskip}

\paragraph*{B3: Goniometrické Q(x)}

\begin{remark}[Obecný postup pro goniometrické Q(x)]
Pro $Q(x) = A\sin(\omega x) + B\cos(\omega x)$:
\begin{itemize}
\item Řešení typicky obsahuje kombinace goniometrických funkcí
\item Pro konstantní P(x) lze použít metodu neurčitých koeficientů
\item Integrace může vyžadovat goniometrické identity
\end{itemize}
\end{remark}

\vspace{0.6\baselineskip}

\begin{example}[Goniometrické Q(x) s konstantní P(x)]
Řešte: $\frac{dy}{dx} + y = \sin x$
\vspace{0.3\baselineskip}

\textbf{Řešení}: 
\begin{enumerate}
\item $P(x) = 1$, $Q(x) = \sin x$

\item \textbf{Integrační faktor}: $\mu(x) = e^{x}$

\item \textbf{Integrace}:
\[
\frac{d}{dx}[e^{x}y] = e^{x}\sin x
\]
\[
e^{x}y = \int e^{x}\sin x  dx
\]

\item \textbf{Integrace per partes}:
\[
\int e^{x}\sin x  dx = \frac{e^{x}}{2}(\sin x - \cos x) + C
\]

\item \textbf{Výsledek}:
\[
y(x) = \frac{1}{2}(\sin x - \cos x) + Ce^{-x}
\]
\end{enumerate}
\end{example}

\vspace{0.8\baselineskip}

\subsubsection{Aplikace v Kvantitativních Vědách}
\label{subsubsec:aplikace-kvantitativni-linear}

\begin{application}[Spojité úročení s pravidelnými vklady]
Model: $\frac{dA}{dt} = rA + I(t)$, kde:
\begin{itemize}
\item $A(t)$: výše účtu v čase $t$
\item $r$: úroková míra
\item $I(t)$: okamžitá intenzita vkladů
\end{itemize}

\textbf{Řešení}: Jedná se o lineární rovnici s $P(t) = -r$, $Q(t) = I(t)$

\textbf{Integrační faktor}: $\mu(t) = e^{-rt}$

\textbf{Obecné řešení}:
\[
A(t) = e^{rt}\left[\int I(\tau)e^{-r\tau}d\tau + C\right]
\]

\textbf{Příklad}: Pro konstantní vklady $I(t) = I_0$:
\[
A(t) = e^{rt}\left[I_0\int e^{-r\tau}d\tau + C\right] = e^{rt}\left[-\frac{I_0}{r}e^{-r\tau} + C\right]
\]
\[
A(t) = -\frac{I_0}{r} + Ce^{rt}
\]
S počáteční podmínkou $A(0) = A_0$: $C = A_0 + \frac{I_0}{r}$
\[
A(t) = \left(A_0 + \frac{I_0}{r}\right)e^{rt} - \frac{I_0}{r}
\]
\end{application}

\vspace{0.6\baselineskip}

\begin{application}[RC obvod s proměnným napětím]
Model: $RC\frac{dV_C}{dt} + V_C = V_{in}(t)$, kde:
\begin{itemize}
\item $V_C(t)$: napětí na kondenzátoru
\item $V_{in}(t)$: vstupní napětí
\item $R$: odpor, $C$: kapacita
\end{itemize}

\textbf{Řešení}: Přepíšeme na standardní tvar:
\[
\frac{dV_C}{dt} + \frac{1}{RC}V_C = \frac{1}{RC}V_{in}(t)
\]

\textbf{Integrační faktor}: $\mu(t) = e^{t/RC}$

\textbf{Obecné řešení}:
\[
V_C(t) = e^{-t/RC}\left[\frac{1}{RC}\int V_{in}(\tau)e^{\tau/RC}d\tau + C\right]
\]

\textbf{Příklad}: Pro konstantní vstupní napětí $V_{in}(t) = V_0$:
\[
V_C(t) = e^{-t/RC}\left[\frac{V_0}{RC}\int e^{\tau/RC}d\tau + C\right] = e^{-t/RC}\left[V_0 e^{\tau/RC} + C\right]
\]
\[
V_C(t) = V_0 + Ce^{-t/RC}
\]
\end{application}

\vspace{0.6\baselineskip}

\begin{application}[Míchací problém s proměnnou koncentrací]
Model: $\frac{dS}{dt} = r_{in}c_{in}(t) - \frac{r_{out}}{V}S$, kde:
\begin{itemize}
\item $S(t)$: množství solutu v nádrži
\item $V$: objem nádrže
\item $r_{in}$, $r_{out}$: průtoky dovnitř/ven
\item $c_{in}(t)$: koncentrace v přítoku
\end{itemize}

\textbf{Řešení}: Standardní tvar:
\[
\frac{dS}{dt} + \frac{r_{out}}{V}S = r_{in}c_{in}(t)
\]

\textbf{Integrační faktor}: $\mu(t) = e^{r_{out}t/V}$

\textbf{Obecné řešení}:
\[
S(t) = e^{-r_{out}t/V}\left[r_{in}\int c_{in}(\tau)e^{r_{out}\tau/V}d\tau + C\right]
\]
\end{application}

\vspace{0.8\baselineskip}

\begin{transition}
V poslední části pokryjeme kombinované příklady, numerickou verifikaci a kompletní shrnutí metodologie lineárních rovnic 1. řádu.
\end{transition}

% !TEX root = ../main.tex
% ČÁST 3/3 - Dokončení 1.2 Lineární Rovnice

\subsubsection{Početní Sekce - Kategorie C: Kombinované Příklady}
\label{subsubsec:pocetni-kategorie-c-linear}

\paragraph*{C1: Jednoduché kombinace}

\begin{example}[Konstantní P(x) + Polynomiální Q(x)]
Řešte: $\frac{dy}{dx} + 2y = x^2 + 3x + 1$
\vspace{0.3\baselineskip}

\textbf{Řešení}: 
\begin{enumerate}
\item $P(x) = 2$, $Q(x) = x^2 + 3x + 1$

\item \textbf{Integrační faktor}: $\mu(x) = e^{2x}$

\item \textbf{Integrace}:
\[
\frac{d}{dx}[e^{2x}y] = e^{2x}(x^2 + 3x + 1)
\]
\[
e^{2x}y = \int e^{2x}(x^2 + 3x + 1)dx
\]

\item \textbf{Integrace per partes}:
\[
\int x^2 e^{2x}dx = \frac{e^{2x}}{4}(2x^2 - 2x + 1)
\]
\[
\int 3x e^{2x}dx = \frac{3e^{2x}}{4}(2x - 1)
\]
\[
\int e^{2x}dx = \frac{e^{2x}}{2}
\]

\item \textbf{Výsledek}:
\[
e^{2x}y = \frac{e^{2x}}{4}(2x^2 + 4x + 1) + C
\]
\[
y(x) = \frac{1}{4}(2x^2 + 4x + 1) + Ce^{-2x}
\]

\item \textbf{Ověření}:
\[
\frac{dy}{dx} = x + 1 - 2Ce^{-2x}, \quad 2y = \frac{1}{2}(2x^2 + 4x + 1) + 2Ce^{-2x}
\]
\[
\frac{dy}{dx} + 2y = x + 1 + x^2 + 2x + \frac{1}{2} = x^2 + 3x + \frac{3}{2}
\]
Oprava: $\frac{1}{4}(2x^2 + 4x + 1) = \frac{1}{2}x^2 + x + \frac{1}{4}$, tedy:
\[
\frac{dy}{dx} + 2y = x^2 + 3x + 1 \quad \checkmark
\]
\end{enumerate}
\end{example}

\vspace{0.6\baselineskip}

\begin{example}[Polynomiální P(x) + Exponenciální Q(x)]
Řešte: $\frac{dy}{dx} + xy = e^{-x^2/2}$
\vspace{0.3\baselineskip}

\textbf{Řešení}: 
\begin{enumerate}
\item $P(x) = x$, $Q(x) = e^{-x^2/2}$

\item \textbf{Integrační faktor}: $\mu(x) = e^{x^2/2}$

\item \textbf{Integrace}:
\[
\frac{d}{dx}[e^{x^2/2}y] = e^{x^2/2} \cdot e^{-x^2/2} = 1
\]
\[
e^{x^2/2}y = \int 1  dx = x + C
\]

\item \textbf{Výsledek}:
\[
y(x) = xe^{-x^2/2} + Ce^{-x^2/2}
\]

\item \textbf{Interval existence}: $\mathbb{R}$
\end{enumerate}
\end{example}

\vspace{0.8\baselineskip}

\paragraph*{C2: Střední kombinace}

\begin{example}[Racionální P(x) + Goniometrické Q(x)]
Řešte: $\frac{dy}{dx} + \frac{1}{x}y = \sin x$ pro $x > 0$
\vspace{0.3\baselineskip}

\textbf{Řešení}: 
\begin{enumerate}
\item $P(x) = \frac{1}{x}$, $Q(x) = \sin x$

\item \textbf{Integrační faktor}: $\mu(x) = e^{\int \frac{1}{x}dx} = x$

\item \textbf{Integrace}:
\[
\frac{d}{dx}[xy] = x\sin x
\]
\[
xy = \int x\sin x  dx
\]

\item \textbf{Integrace per partes}:
\[
\int x\sin x  dx = -x\cos x + \sin x + C
\]

\item \textbf{Výsledek}:
\[
y(x) = -\cos x + \frac{\sin x}{x} + \frac{C}{x}
\]

\item \textbf{Interval existence}: $(0, \infty)$
\end{enumerate}
\end{example}

\vspace{0.6\baselineskip}

\begin{example}[Goniometrické P(x) + Polynomiální Q(x)]
Řešte: $\frac{dy}{dx} + (\cos x)y = x$ s $y(0) = 1$
\vspace{0.3\baselineskip}

\textbf{Řešení}: 
\begin{enumerate}
\item $P(x) = \cos x$, $Q(x) = x$

\item \textbf{Integrační faktor}: $\mu(x) = e^{\int \cos x  dx} = e^{\sin x}$

\item \textbf{Integrace}:
\[
\frac{d}{dx}[e^{\sin x}y] = xe^{\sin x}
\]
\[
e^{\sin x}y = \int xe^{\sin x}dx
\]

\item \textbf{Integrál nelze vyjádřit elementárními funkcemi}:
\[
y(x) = e^{-\sin x}\left[\int xe^{\sin x}dx + C\right]
\]

\item \textbf{Určení konstanty}:
\[
y(0) = 1 \implies e^{-\sin 0}\left[\int_0^0 \tau e^{\sin \tau}d\tau + C\right] = 1 \implies C = 1
\]

\item \textbf{Výsledek}:
\[
y(x) = e^{-\sin x}\left[\int_0^x \tau e^{\sin \tau}d\tau + 1\right]
\]

\item \textbf{Interval existence}: $\mathbb{R}$
\end{enumerate}
\end{example}

\vspace{0.8\baselineskip}

\paragraph*{C3: Složité kombinace}

\begin{example}[Singulární P(x) + Speciální Q(x)]
Řešte: $\frac{dy}{dx} + \frac{1}{x-2}y = \frac{1}{(x-2)^2}$ s $y(1) = 0$
\vspace{0.3\baselineskip}

\textbf{Řešení}: 
\begin{enumerate}
\item $P(x) = \frac{1}{x-2}$, $Q(x) = \frac{1}{(x-2)^2}$, $x \neq 2$

\item \textbf{Integrační faktor}:
\[
\mu(x) = e^{\int \frac{1}{x-2}dx} = e^{\ln|x-2|} = |x-2|
\]
Pro $x < 2$: $\mu(x) = 2-x$, pro $x > 2$: $\mu(x) = x-2$

\item \textbf{Řešení pro $x < 2$} (kde $y(1) = 0$):
\[
\frac{d}{dx}[(2-x)y] = (2-x)\cdot\frac{1}{(x-2)^2} = -\frac{1}{x-2}
\]
\[
(2-x)y = -\int \frac{1}{x-2}dx = -\ln|2-x| + C
\]

\item \textbf{Určení konstanty}:
\[
y(1) = 0 \implies (2-1)\cdot 0 = -\ln|1| + C \implies C = 0
\]
\[
y(x) = -\frac{\ln(2-x)}{2-x} = \frac{\ln(2-x)}{x-2} \quad \text{pro } x < 2
\]

\item \textbf{Interval existence}: $(-\infty, 2)$
\end{enumerate}
\end{example}

\vspace{0.8\baselineskip}

\subsubsection{Početní Sekce - Kategorie D: Aplikované Příklady}
\label{subsubsec:pocetni-kategorie-d-linear}

\paragraph*{D1: Finanční modely}

\begin{example}[Spořicí účet s proměnnými vklady]
Mějme spořicí účet s úrokovou mírou 5\% p.a. spojitě. Vkládáme pravidelně s intenzitou $I(t) = 1000 + 100t$ Kč/rok. Najděte výši účtu po 10 letech, pokud počáteční vklad byl 5000 Kč.
\vspace{0.3\baselineskip}

\textbf{Řešení}: 
\begin{enumerate}
\item \textbf{Model}: $\frac{dA}{dt} = 0.05A + (1000 + 100t)$, $A(0) = 5000$

\item \textbf{Integrační faktor}: $\mu(t) = e^{-0.05t}$

\item \textbf{Integrace}:
\[
\frac{d}{dt}[e^{-0.05t}A] = e^{-0.05t}(1000 + 100t)
\]
\[
e^{-0.05t}A = \int e^{-0.05t}(1000 + 100t)dt
\]

\item \textbf{Integrace per partes}:
\[
\int 1000e^{-0.05t}dt = -20000e^{-0.05t}
\]
\[
\int 100te^{-0.05t}dt = -2000te^{-0.05t} - 40000e^{-0.05t}
\]

\item \textbf{Obecné řešení}:
\[
A(t) = e^{0.05t}\left[-20000e^{-0.05t} - 2000te^{-0.05t} - 40000e^{-0.05t} + C\right]
\]
\[
A(t) = -60000 - 2000t + Ce^{0.05t}
\]

\item \textbf{Určení konstanty}:
\[
A(0) = 5000 \implies -60000 + C = 5000 \implies C = 65000
\]

\item \textbf{Výsledek}:
\[
A(t) = -60000 - 2000t + 65000e^{0.05t}
\]
\[
A(10) = -60000 - 20000 + 65000e^{0.5} \approx -80000 + 65000 \cdot 1.6487 \approx 27,165 \text{ Kč}
\end{enumerate}
\end{example}

\vspace{0.8\baselineskip}

\paragraph*{D2: Fyzikální modely}

\begin{example}[RC obvod s proměnným napětím]
RC obvod s $R = 1000\ \Omega$, $C = 0.001\ \mathrm{F}$ je napájen napětím $V_{in}(t) = 10\sin(100t)\ \mathrm{V}$. Najděte napětí na kondenzátoru, pokud $V_C(0) = 0$.
\vspace{0.3\baselineskip}

\textbf{Řešení}: 
\begin{enumerate}
\item \textbf{Model}: $RC\frac{dV_C}{dt} + V_C = V_{in}(t)$
\[
0.001\frac{dV_C}{dt} + V_C = 10\sin(100t)
\]
\[
\frac{dV_C}{dt} + 1000V_C = 10000\sin(100t)
\]

\item \textbf{Integrační faktor}: $\mu(t) = e^{1000t}$

\item \textbf{Integrace}:
\[
\frac{d}{dt}[e^{1000t}V_C] = 10000e^{1000t}\sin(100t)
\]
\[
e^{1000t}V_C = 10000\int e^{1000t}\sin(100t)dt
\]

\item \textbf{Integrace per partes}:
\[
\int e^{1000t}\sin(100t)dt = \frac{e^{1000t}}{1000000 + 10000}(1000\sin(100t) - 100\cos(100t)) + K
\]
\[
= \frac{e^{1000t}}{1010000}(1000\sin(100t) - 100\cos(100t)) + K
\]

\item \textbf{Výsledek}:
\[
V_C(t) = \frac{10000}{1010000}(1000\sin(100t) - 100\cos(100t)) + Ke^{-1000t}
\]
\[
V_C(t) = \frac{100}{101}\sin(100t) - \frac{10}{101}\cos(100t) + Ke^{-1000t}
\]

\item \textbf{Určení konstanty}:
\[
V_C(0) = 0 \implies -\frac{10}{101} + K = 0 \implies K = \frac{10}{101}
\]

\item \textbf{Konečný výsledek}:
\[
V_C(t) = \frac{100}{101}\sin(100t) - \frac{10}{101}\cos(100t) + \frac{10}{101}e^{-1000t}
\]
\end{enumerate}
\end{example}

\vspace{0.8\baselineskip}

\subsubsection{Numerická Verifikace}
\label{subsubsec:numericka-verifikace}

\begin{method}[Eulerova metoda pro lineární rovnice]
\label{met:euler-linearni}
Pro rovnici $\frac{dy}{dx} + P(x)y = Q(x)$ s počáteční podmínkou $y(x_0) = y_0$:

\begin{enumerate}
\item Zvolte krok $h > 0$
\item Pro $n = 0, 1, 2, \dots$ počítáme:
\[
y_{n+1} = y_n + h[Q(x_n) - P(x_n)y_n]
\]
\[
x_{n+1} = x_n + h
\]
\item Chyba metody: $O(h)$
\end{enumerate}
\end{method}

\vspace{0.6\baselineskip}

\begin{example}[Numerická verifikace jednoduché rovnice]
Porovnejte analytické a numerické řešení: $\frac{dy}{dx} + 2y = 1$, $y(0) = 0$
\vspace{0.3\baselineskip}

\textbf{Analytické řešení}: $y(x) = \frac{1}{2}(1 - e^{-2x})$

\textbf{Numerické řešení} (krok $h = 0.1$):
\begin{center}
\begin{tabular}{c|c|c|c}
$n$ & $x_n$ & Analytické $y(x_n)$ & Numerické $y_n$ \\
\hline
0 & 0.0 & 0.0000 & 0.0000 \\
1 & 0.1 & 0.0906 & 0.1000 \\
2 & 0.2 & 0.1648 & 0.1800 \\
3 & 0.3 & 0.2260 & 0.2440 \\
4 & 0.4 & 0.2763 & 0.2952 \\
5 & 0.5 & 0.3161 & 0.3362 \\
\end{tabular}
\end{center}

\textbf{Analýza chyby}: Maximální chyba ≈ 0.0094 při $x=0.1$, klesá s klesajícím $h$.
\end{example}

\vspace{0.6\baselineskip}

\begin{example}[Numerická verifikace složité rovnice]
Ověřte numericky řešení: $\frac{dy}{dx} + \frac{1}{x}y = 1$, $y(1) = 0$ pro $x > 0$
\vspace{0.3\baselineskip}

\textbf{Analytické řešení}: $y(x) = \frac{x}{2} - \frac{1}{2x}$

\textbf{Numerické řešení} (krok $h = 0.1$):
\begin{center}
\begin{tabular}{c|c|c|c}
$x_n$ & Analytické & Numerické & Chyba \\
\hline
1.0 & 0.0000 & 0.0000 & 0.0000 \\
1.1 & 0.0955 & 0.1000 & 0.0045 \\
1.2 & 0.1833 & 0.1909 & 0.0076 \\
1.3 & 0.2654 & 0.2745 & 0.0091 \\
1.4 & 0.3429 & 0.3529 & 0.0100 \\
1.5 & 0.4167 & 0.4275 & 0.0108 \\
\end{tabular}
\end{center}

\textbf{Závěr}: Numerické řešení dobře aproximuje analytické s chybou $O(h)$.
\end{example}

\vspace{0.8\baselineskip}

\subsubsection{Shrnutí a Expertní Metodologie}
\label{subsubsec:shrnuti-metodologie}

\begin{method}[Decision tree pro řešení lineárních rovnic]
\label{met:decision-tree-linearni}
\begin{enumerate}
\item \textbf{Identifikace}: Je rovnice ve tvaru $\frac{dy}{dx} + P(x)y = Q(x)$?

\item \textbf{Výběr metody}:
\begin{itemize}
\item \textbf{Integrační faktor}: Vždy funguje, vhodný pro většinu případů
\item \textbf{Variace konstant}: Ekvivalentní, někdy intuitivnější
\item \textbf{Metoda neurčitých koeficientů}: Pouze pro konstantní P(x) a speciální Q(x)
\end{itemize}

\item \textbf{Analýza P(x)}:
\begin{itemize}
\item Konstantní: Jednoduchá exponenciální integrace
\item Polynomiální: Integrace polynomu
\item Racionální: Rozklad na parciální zlomky
\item Goniometrické: Trigonometrické identity
\end{itemize}

\item \textbf{Analýza Q(x)}:
\begin{itemize}
\item Polynomiální: Přímá integrace
\item Exponenciální: Pozor na rezonanci
\item Goniometrické: Goniometrické integrace
\end{itemize}

\item \textbf{Interval existence}: Určete maximální interval kde P(x) a Q(x) spojité

\item \textbf{Ověření}: Vždy dosaďte řešení do původní rovnice
\end{enumerate}
\end{method}

\vspace{0.8\baselineskip}

\begin{remark}[Časté chyby a jak se jim vyhnout]
\begin{itemize}
\item \textbf{Zapomnění na integrační konstantu}: Vždy přidejte $+C$ po integraci
\item \textbf{Špatný integrační faktor}: Ověřte že $\frac{d\mu}{dx} = P(x)\mu(x)$
\item \textbf{Nesprávný interval existence}: Vždy analyzujte definiční obory P(x) a Q(x)
\item \textbf{Chyba v integraci}: Ověřte integrál derivací
\item \textbf{Záměna metod}: Integrační faktor a variace konstant jsou ekvivalentní
\end{itemize}
\end{remark}

\vspace{0.6\baselineskip}

\begin{remark}[Optimalizace výpočtů]
\begin{itemize}
\item Pro konstantní P(x) a polynomiální Q(x): Metoda neurčitých koeficientů může být rychlejší
\item Pro složité P(x): Integrační faktor je robustnější
\item Pro numerické řešení: Eulerova metoda je jednoduchá, Runge-Kutta přesnější
\end{itemize}
\end{remark}

\vspace{0.8\baselineskip}

\begin{transition}
S kompletním zvládnutím lineárních rovnic 1. řádu jsme připraveni přejít k homogenním rovnicím (1.3), které představují speciální případ lineárních rovnic a otevírají cestu k hlubší geometrické interpretaci diferenciálních rovnic.
\end{transition}

\vspace{0.8\baselineskip}

\subsubsection*{Shrnutí Sekce 1.2}

Lineární rovnice 1. řádu představují systematický a mocný aparát pro modelování dynamických systémů s vnějšími vstupy. Klíčové poznatky:

\begin{itemize}
\item \textbf{Univerzální řešitelnost}: Každá lineární rovnice s spojitými koeficienty má explicitní řešení
\item \textbf{Dvě ekvivalentní metody}: Integrační faktor a variace konstant
\item \textbf{Struktura řešení}: Homogenní část + partikulární řešení
\item \textbf{Aplikace}: Finanční modelování, elektrické obvody, fyzikální systémy
\item \textbf{Numerická verifikace}: Důležitá pro ověření analytických výsledků
\end{itemize}

% !TEX root = ../main.tex
\subsection{Homogenní Rovnice - Geometrický Přístup a Substituční Metody}
\label{subsec:homogenni-rovnice}

\subsubsection{Teoretický Fundament Homogenity}
\label{subsubsec:teoreticky-fundament-homogenni}

\begin{definition}[Homogenní diferenciální rovnice 1. řádu]
Rovnice je \emph{homogenní}, jestliže ji lze zapsat ve tvaru:
\[
\frac{dy}{dx} = f\left(\frac{y}{x}\right)
\]
kde $f$ je funkce jedné proměnné $v = \frac{y}{x}$.
\end{definition}

\vspace{0.6\baselineskip}

\begin{theorem}[Test homogenity]
Funkce $F(x, y)$ je homogenní stupně $n$, jestliže pro všechna $\lambda > 0$ platí:
\[
F(\lambda x, \lambda y) = \lambda^n F(x, y)
\]
Diferenciální rovnice $dy/dx = F(x, y)$ je homogenní, jestliže $F(x, y)$ je homogenní stupně 0.
\end{theorem}

\vspace{0.4\baselineskip}

\begin{proof}
Je-li $F(x, y)$ homogenní stupně 0, pak:
\[
F(\lambda x, \lambda y) = \lambda^0 F(x, y) = F(x, y)
\]
Speciálně pro $\lambda = 1/x$ (pro $x > 0$):
\[
F(x, y) = F\left(1, \frac{y}{x}\right) = f\left(\frac{y}{x}\right)
\]
\end{proof}

\vspace{0.6\baselineskip}

\begin{theorem}[Existence a jednoznačnost]
Nechť $f(v)$ je spojitá na intervalu $J \subseteq \mathbb{R}$ a Lipschitzovská na kompaktech. Pak pro libovolný bod $(x_0, y_0)$ s $x_0 \neq 0$ a $y_0/x_0 \in J$ existuje právě jedno řešení homogenní rovnice definované na okolí $x_0$.
\end{theorem}

\vspace{0.8\baselineskip}

\subsubsection{Substituční Metodologie - Kompletní Analýza}
\label{subsubsec:substituci-metodologie}

\begin{method}[Základní substituce $v = y/x$]
\label{met:zakladni-substituce}
\begin{enumerate}
\item \textbf{Substituce}: Položíme $v = \frac{y}{x} \implies y = vx$

\item \textbf{Derivace}:
\[
\frac{dy}{dx} = v + x\frac{dv}{dx}
\]

\item \textbf{Dosazení do rovnice}:
\[
v + x\frac{dv}{dx} = f(v)
\]

\item \textbf{Separace proměnných}:
\[
x\frac{dv}{dx} = f(v) - v \implies \frac{dv}{f(v) - v} = \frac{dx}{x}
\]

\item \textbf{Integrace}:
\[
\int \frac{dv}{f(v) - v} = \int \frac{dx}{x} = \ln|x| + C
\]

\item \textbf{Zpětná substituce}: Po integraci dosadíme $v = y/x$

\item \textbf{Analýza singulárních bodů}: Body kde $f(v) - v = 0$
\end{enumerate}
\end{method}

\vspace{0.8\baselineskip}

\begin{example}[Odvození metody]
Mějme homogenní rovnici $\frac{dy}{dx} = f\left(\frac{y}{x}\right)$. Substituce $y = vx$ dává:
\[
\frac{d}{dx}[vx] = v + x\frac{dv}{dx} = f(v)
\]
Tedy:
\[
x\frac{dv}{dx} = f(v) - v
\]
Tato rovnice je separabilní v proměnných $v$ a $x$.
\end{example}

\vspace{0.6\baselineskip}

\begin{method}[Alternativní substituce]
\label{met:alternativni-substituce}
\begin{itemize}
\item \textbf{Substituce $u = \ln|x|$}: Pro rovnice s logaritmickou strukturou

\item \textbf{Substituce $y = x^m v$}: Pro zobecněnou homogenitu

\item \textbf{Polární souřadnice}: $x = r\cos\theta$, $y = r\sin\theta$ pro geometrické interpretace
\end{itemize}
\end{method}

\vspace{0.8\baselineskip}

\subsubsection{Klasifikace Typů f(y/x)}
\label{subsubsec:klasifikace-funkci}

\begin{remark}[Racionální funkce]
Pro $f(v) = \frac{P(v)}{Q(v)}$ kde $P$, $Q$ jsou polynomy:
\[
\frac{dy}{dx} = \frac{P(y/x)}{Q(y/x)}
\]
Substituce vede na integraci racionální funkce.
\end{remark}

\vspace{0.6\baselineskip}

\begin{remark}[Algebraické funkce]
Pro $f(v)$ obsahující odmocniny:
\[
\frac{dy}{dx} = \sqrt{\frac{y}{x} + 1} \quad \text{nebo} \quad \frac{dy}{dx} = \left(\frac{y}{x}\right)^{1/3}
\]
Často vyžaduje speciální substituce nebo umocňování.
\end{remark}

\vspace{0.6\baselineskip}

\begin{remark}[Goniometrické funkce]
Pro $f(v) = \sin(v)$, $\cos(v)$, $\tan(v)$, atd.:
\[
\frac{dy}{dx} = \sin\left(\frac{y}{x}\right)
\]
Substituce vede na rovnici se separovanými proměnnými, ale integrace může vyžadovat goniometrické identity.
\end{remark}

\vspace{0.8\baselineskip}

\subsubsection{Početní Sekce - Kategorie A: Podle typu f(y/x)}
\label{subsubsec:pocetni-kategorie-a-fv}

\paragraph*{A1: Lineární f(v)}

\begin{example}[Lehký příklad - lineární f(v)]
Řešte: $\frac{dy}{dx} = \frac{y}{x} + 1$
\vspace{0.3\baselineskip}

\textbf{Řešení}: 
\begin{enumerate}
\item $f(v) = v + 1$

\item \textbf{Substituce}: $y = vx \implies \frac{dy}{dx} = v + x\frac{dv}{dx}$

\item \textbf{Dosazení}:
\[
v + x\frac{dv}{dx} = v + 1 \implies x\frac{dv}{dx} = 1
\]

\item \textbf{Separace}:
\[
dv = \frac{dx}{x} \implies v = \ln|x| + C
\]

\item \textbf{Zpětná substituce}:
\[
\frac{y}{x} = \ln|x| + C \implies y = x\ln|x| + Cx
\]

\item \textbf{Interval existence}: $(-\infty, 0)$ nebo $(0, \infty)$

\item \textbf{Ověření}:
\[
\frac{dy}{dx} = \ln|x| + 1 + C = \frac{y}{x} + 1 \quad \checkmark
\end{enumerate}
\end{example}

\vspace{0.6\baselineskip}

\begin{example}[Střední příklad - lineární f(v)]
Řešte: $\frac{dy}{dx} = \frac{2y - x}{y + x}$
\vspace{0.3\baselineskip}

\textbf{Řešení}: 
\begin{enumerate}
\item \textbf{Test homogenity}:
\[
F(\lambda x, \lambda y) = \frac{2\lambda y - \lambda x}{\lambda y + \lambda x} = \frac{2y - x}{y + x} = F(x, y)
\]
Homogenní stupně 0.

\item \textbf{Úprava}:
\[
\frac{dy}{dx} = \frac{2(y/x) - 1}{(y/x) + 1} = f(v) \quad \text{kde } v = \frac{y}{x}
\]

\item \textbf{Substituce}: $y = vx \implies \frac{dy}{dx} = v + x\frac{dv}{dx}$

\item \textbf{Dosazení}:
\[
v + x\frac{dv}{dx} = \frac{2v - 1}{v + 1}
\]
\[
x\frac{dv}{dx} = \frac{2v - 1}{v + 1} - v = \frac{2v - 1 - v^2 - v}{v + 1} = \frac{-v^2 + v - 1}{v + 1}
\]

\item \textbf{Separace}:
\[
\frac{v + 1}{-v^2 + v - 1}dv = \frac{dx}{x}
\]

\item \textbf{Integrace}: Nejprve upravíme jmenovatele:
\[
-v^2 + v - 1 = -(v^2 - v + 1) = -\left[(v - \frac{1}{2})^2 + \frac{3}{4}\right]
\]
\[
\int \frac{v + 1}{-v^2 + v - 1}dv = -\int \frac{v + 1}{(v - \frac{1}{2})^2 + \frac{3}{4}}dv
\]

\item \textbf{Substituce}: $u = v - \frac{1}{2} \implies v = u + \frac{1}{2}$, $dv = du$
\[
-\int \frac{u + \frac{3}{2}}{u^2 + \frac{3}{4}}du = -\int \frac{u}{u^2 + \frac{3}{4}}du - \frac{3}{2}\int \frac{1}{u^2 + \frac{3}{4}}du
\]
\[
= -\frac{1}{2}\ln|u^2 + \frac{3}{4}| - \frac{3}{2} \cdot \frac{2}{\sqrt{3}}\arctan\left(\frac{2u}{\sqrt{3}}\right) + K
\]

\item \textbf{Výsledek}:
\[
-\frac{1}{2}\ln\left|\left(v - \frac{1}{2}\right)^2 + \frac{3}{4}\right| - \sqrt{3}\arctan\left(\frac{2v - 1}{\sqrt{3}}\right) = \ln|x| + C
\]

\item \textbf{Zpětná substituce}: $v = y/x$
\[
-\frac{1}{2}\ln\left|\left(\frac{y}{x} - \frac{1}{2}\right)^2 + \frac{3}{4}\right| - \sqrt{3}\arctan\left(\frac{2y/x - 1}{\sqrt{3}}\right) = \ln|x| + C
\]
\end{enumerate}
\end{example}

\vspace{0.8\baselineskip}

\paragraph*{A2: Kvadratické f(v)}

\begin{example}[Lehký příklad - kvadratické f(v)]
Řešte: $\frac{dy}{dx} = \frac{y^2}{x^2}$
\vspace{0.3\baselineskip}

\textbf{Řešení}: 
\begin{enumerate}
\item $f(v) = v^2$

\item \textbf{Substituce}: $y = vx \implies \frac{dy}{dx} = v + x\frac{dv}{dx}$

\item \textbf{Dosazení}:
\[
v + x\frac{dv}{dx} = v^2 \implies x\frac{dv}{dx} = v^2 - v
\]

\item \textbf{Separace}:
\[
\frac{dv}{v(v - 1)} = \frac{dx}{x}
\]

\item \textbf{Rozklad na parciální zlomky}:
\[
\frac{1}{v(v - 1)} = \frac{1}{v - 1} - \frac{1}{v}
\]

\item \textbf{Integrace}:
\[
\int \left(\frac{1}{v - 1} - \frac{1}{v}\right)dv = \int \frac{dx}{x}
\]
\[
\ln|v - 1| - \ln|v| = \ln|x| + C
\]
\[
\ln\left|\frac{v - 1}{v}\right| = \ln|x| + C \implies \frac{v - 1}{v} = Ax \quad (A = \pm e^C)
\]

\item \textbf{Zpětná substituce}:
\[
\frac{y/x - 1}{y/x} = Ax \implies 1 - \frac{x}{y} = Ax \implies \frac{x}{y} = 1 - Ax
\]
\[
y = \frac{x}{1 - Ax}
\]

\item \textbf{Singulární řešení}: $v = 0$ a $v = 1$ $\implies$ $y = 0$ a $y = x$

\item \textbf{Intervaly existence}: Závisí na konstantě $A$
\end{enumerate}
\end{example}

\vspace{0.6\baselineskip}

\begin{example}[Střední příklad - kvadratické f(v)]
Řešte: $\frac{dy}{dx} = \frac{x^2 + y^2}{xy}$
\vspace{0.3\baselineskip}

\textbf{Řešení}: 
\begin{enumerate}
\item \textbf{Úprava}:
\[
\frac{dy}{dx} = \frac{x^2 + y^2}{xy} = \frac{x}{y} + \frac{y}{x} = \frac{1}{v} + v \quad \text{kde } v = \frac{y}{x}
\]

\item \textbf{Substituce}: $y = vx \implies \frac{dy}{dx} = v + x\frac{dv}{dx}$

\item \textbf{Dosazení}:
\[
v + x\frac{dv}{dx} = \frac{1}{v} + v \implies x\frac{dv}{dx} = \frac{1}{v}
\]

\item \textbf{Separace}:
\[
v  dv = \frac{dx}{x} \implies \frac{v^2}{2} = \ln|x| + C
\]

\item \textbf{Zpětná substituce}:
\[
\frac{y^2}{2x^2} = \ln|x| + C \implies y^2 = 2x^2(\ln|x| + C)
\]

\item \textbf{Interval existence}: $(-\infty, 0)$ nebo $(0, \infty)$
\end{enumerate}
\end{example}

\vspace{0.8\baselineskip}

\paragraph*{A3: Racionální f(v)}

\begin{example}[Lehký příklad - racionální f(v)]
Řešte: $\frac{dy}{dx} = \frac{x + y}{x - y}$
\vspace{0.3\baselineskip}

\textbf{Řešení}: 
\begin{enumerate}
\item \textbf{Úprava}:
\[
\frac{dy}{dx} = \frac{1 + y/x}{1 - y/x} = \frac{1 + v}{1 - v} \quad \text{kde } v = \frac{y}{x}
\]

\item \textbf{Substituce}: $y = vx \implies \frac{dy}{dx} = v + x\frac{dv}{dx}$

\item \textbf{Dosazení}:
\[
v + x\frac{dv}{dx} = \frac{1 + v}{1 - v} \implies x\frac{dv}{dx} = \frac{1 + v}{1 - v} - v = \frac{1 + v^2}{1 - v}
\]

\item \textbf{Separace}:
\[
\frac{1 - v}{1 + v^2}dv = \frac{dx}{x}
\]

\item \textbf{Integrace}:
\[
\int \frac{1 - v}{1 + v^2}dv = \int \frac{1}{1 + v^2}dv - \int \frac{v}{1 + v^2}dv = \arctan v - \frac{1}{2}\ln(1 + v^2)
\]

\item \textbf{Výsledek}:
\[
\arctan v - \frac{1}{2}\ln(1 + v^2) = \ln|x| + C
\]

\item \textbf{Zpětná substituce}:
\[
\arctan\left(\frac{y}{x}\right) - \frac{1}{2}\ln\left(1 + \frac{y^2}{x^2}\right) = \ln|x| + C
\]
\[
\arctan\left(\frac{y}{x}\right) - \frac{1}{2}\ln(x^2 + y^2) + \ln|x| = \ln|x| + C
\]
\[
\arctan\left(\frac{y}{x}\right) - \frac{1}{2}\ln(x^2 + y^2) = C
\]

\item \textbf{Geometrická interpretace}: Rovnice kružnic v polárních souřadnicích
\end{enumerate}
\end{example}

\vspace{0.8\baselineskip}





% !TEX root = ../main.tex
% ČÁST 2/3 - Pokračování 1.3 Homogenní Rovnice

\subsubsection{Početní Sekce - Kategorie A (pokračování)}
\label{subsubsec:pocetni-kategorie-a-fv-pokracovani}

\paragraph*{A4: Algebraické f(v)}

\begin{example}[Lehký příklad - algebraické f(v)]
Řešte: $\frac{dy}{dx} = \sqrt{\frac{y}{x}}$
\vspace{0.3\baselineskip}

\textbf{Řešení}: 
\begin{enumerate}
\item $f(v) = \sqrt{v}$, předpokládáme $y/x \geq 0$

\item \textbf{Substituce}: $y = vx \implies \frac{dy}{dx} = v + x\frac{dv}{dx}$

\item \textbf{Dosazení}:
\[
v + x\frac{dv}{dx} = \sqrt{v} \implies x\frac{dv}{dx} = \sqrt{v} - v
\]

\item \textbf{Separace}:
\[
\frac{dv}{\sqrt{v} - v} = \frac{dx}{x}
\]

\item \textbf{Úprava integrandu}:
\[
\frac{1}{\sqrt{v} - v} = \frac{1}{\sqrt{v}(1 - \sqrt{v})}
\]
Substituce $u = \sqrt{v} \implies v = u^2$, $dv = 2u  du$
\[
\int \frac{2u  du}{u(1 - u)} = 2\int \frac{du}{1 - u} = -2\ln|1 - u| + C
\]

\item \textbf{Výsledek}:
\[
-2\ln|1 - \sqrt{v}| = \ln|x| + C \implies \ln|1 - \sqrt{v}| = -\frac{1}{2}\ln|x| + K
\]

\item \textbf{Zpětná substituce}:
\[
1 - \sqrt{\frac{y}{x}} = \frac{A}{\sqrt{x}} \quad (A = \pm e^K)
\]
\[
\sqrt{\frac{y}{x}} = 1 - \frac{A}{\sqrt{x}} \implies \frac{y}{x} = \left(1 - \frac{A}{\sqrt{x}}\right)^2
\]
\[
y = x\left(1 - \frac{2A}{\sqrt{x}} + \frac{A^2}{x}\right) = x - 2A\sqrt{x} + A^2
\]

\item \textbf{Singulární řešení}: $v = 0$ a $v = 1$ $\implies$ $y = 0$ a $y = x$

\item \textbf{Interval existence}: $x > 0$, $y \geq 0$
\end{enumerate}
\end{example}

\vspace{0.6\baselineskip}

\begin{example}[Střední příklad - algebraické f(v)]
Řešte: $\frac{dy}{dx} = \frac{\sqrt{x^2 + y^2}}{x}$ pro $x > 0$
\vspace{0.3\baselineskip}

\textbf{Řešení}: 
\begin{enumerate}
\item \textbf{Úprava}:
\[
\frac{dy}{dx} = \frac{\sqrt{x^2 + y^2}}{x} = \sqrt{1 + \left(\frac{y}{x}\right)^2} = \sqrt{1 + v^2}
\]

\item \textbf{Substituce}: $y = vx \implies \frac{dy}{dx} = v + x\frac{dv}{dx}$

\item \textbf{Dosazení}:
\[
v + x\frac{dv}{dx} = \sqrt{1 + v^2} \implies x\frac{dv}{dx} = \sqrt{1 + v^2} - v
\]

\item \textbf{Separace}:
\[
\frac{dv}{\sqrt{1 + v^2} - v} = \frac{dx}{x}
\]

\item \textbf{Racionalizace jmenovatele}:
\[
\frac{1}{\sqrt{1 + v^2} - v} \cdot \frac{\sqrt{1 + v^2} + v}{\sqrt{1 + v^2} + v} = \frac{\sqrt{1 + v^2} + v}{1}
\]

\item \textbf{Integrace}:
\[
\int (\sqrt{1 + v^2} + v)dv = \int \frac{dx}{x}
\]
\[
\frac{v}{2}\sqrt{1 + v^2} + \frac{1}{2}\ln|v + \sqrt{1 + v^2}| + \frac{v^2}{2} = \ln|x| + C
\]

\item \textbf{Zpětná substituce}: $v = y/x$
\[
\frac{y}{2x}\sqrt{1 + \frac{y^2}{x^2}} + \frac{1}{2}\ln\left|\frac{y}{x} + \sqrt{1 + \frac{y^2}{x^2}}\right| + \frac{y^2}{2x^2} = \ln|x| + C
\]

\item \textbf{Zjednodušení}:
\[
\frac{y}{2x^2}\sqrt{x^2 + y^2} + \frac{1}{2}\ln\left|\frac{y + \sqrt{x^2 + y^2}}{x}\right| + \frac{y^2}{2x^2} = \ln|x| + C
\]
\end{enumerate}
\end{example}

\vspace{0.8\baselineskip}

\paragraph*{A5: Goniometrické f(v)}

\begin{example}[Lehký příklad - goniometrické f(v)]
Řešte: $\frac{dy}{dx} = \tan\left(\frac{y}{x}\right)$
\vspace{0.3\baselineskip}

\textbf{Řešení}: 
\begin{enumerate}
\item $f(v) = \tan v$

\item \textbf{Substituce}: $y = vx \implies \frac{dy}{dx} = v + x\frac{dv}{dx}$

\item \textbf{Dosazení}:
\[
v + x\frac{dv}{dx} = \tan v \implies x\frac{dv}{dx} = \tan v - v
\]

\item \textbf{Separace}:
\[
\frac{dv}{\tan v - v} = \frac{dx}{x}
\]

\item \textbf{Integrace}: Integrál nelze vyjádřit elementárními funkcemi
\[
\int \frac{dv}{\tan v - v} = \ln|x| + C
\]

\item \textbf{Implicitní řešení}:
\[
\int \frac{dv}{\tan v - v} = \ln|x| + C \quad \text{kde } v = \frac{y}{x}
\]

\item \textbf{Singulární řešení}: $\tan v - v = 0 \implies v = 0 \implies y = 0$
\end{enumerate}
\end{example}

\vspace{0.6\baselineskip}

\begin{example}[Střední příklad - goniometrické f(v)]
Řešte: $\frac{dy}{dx} = \sin\left(\frac{y}{x}\right) + \frac{y}{x}$
\vspace{0.3\baselineskip}

\textbf{Řešení}: 
\begin{enumerate}
\item $f(v) = \sin v + v$

\item \textbf{Substituce}: $y = vx \implies \frac{dy}{dx} = v + x\frac{dv}{dx}$

\item \textbf{Dosazení}:
\[
v + x\frac{dv}{dx} = \sin v + v \implies x\frac{dv}{dx} = \sin v
\]

\item \textbf{Separace}:
\[
\frac{dv}{\sin v} = \frac{dx}{x} \implies \csc v  dv = \frac{dx}{x}
\]

\item \textbf{Integrace}:
\[
\int \csc v  dv = \ln|\csc v - cot v| = \ln\left|\frac{1 - \cos v}{\sin v}\right| = \ln\left|\tan\frac{v}{2}\right|
\]
\[
\ln\left|\tan\frac{v}{2}\right| = \ln|x| + C
\]

\item \textbf{Výsledek}:
\[
\tan\frac{v}{2} = Ax \quad (A = \pm e^C)
\]

\item \textbf{Zpětná substituce}:
\[
\tan\left(\frac{y}{2x}\right) = Ax \implies \frac{y}{2x} = \arctan(Ax) \implies y = 2x \arctan(Ax)
\]

\item \textbf{Interval existence}: Závisí na konstantě $A$
\end{enumerate}
\end{example}

\vspace{0.8\baselineskip}

\subsubsection{Geometrická Interpretace a Fázová Analýza}
\label{subsubsec:geometricka-interpretace}

\begin{theorem}[Geometrická vlastnost homogenních rovnic]
Trajektorie řešení homogenní rovnice $\frac{dy}{dx} = f\left(\frac{y}{x}\right)$ jsou invariantní vůči dilatacím. To znamená, že pokud $y = \phi(x)$ je řešení, pak pro libovolné $\lambda > 0$ je $y = \lambda\phi\left(\frac{x}{\lambda}\right)$ také řešení.
\end{theorem}

\vspace{0.4\baselineskip}

\begin{proof}
Nechť $y = \phi(x)$ je řešení. Pak:
\[
\frac{d}{dx}[\lambda\phi(x/\lambda)] = \lambda\cdot\frac{1}{\lambda}\phi'(x/\lambda) = \phi'(x/\lambda)
\]
a
\[
f\left(\frac{\lambda\phi(x/\lambda)}{x}\right) = f\left(\frac{\phi(x/\lambda)}{x/\lambda}\right) = \phi'(x/\lambda)
\]
Tedy $y = \lambda\phi(x/\lambda)$ je také řešení.
\end{proof}

\vspace{0.6\baselineskip}

\begin{method}[Konstrukce směrového pole]
\label{met:smerove-pole-homogenni}
Pro homogenní rovnici $\frac{dy}{dx} = f\left(\frac{y}{x}\right)$:
\begin{enumerate}
\item Směrové pole závisí pouze na poměru $y/x$
\item Přímky $y = mx$ jsou izokliny
\item Na každé přímce $y = mx$ je sklen řešení konstantní: $f(m)$
\item Celé směrové pole lze získat z jedné přímky dilatacemi
\end{enumerate}
\end{method}

\vspace{0.6\baselineskip}

\begin{example}[Směrové pole pro $\frac{dy}{dx} = \frac{y}{x}$]
\label{ex:smerove-pole-zaklad}
Pro rovnici $\frac{dy}{dx} = \frac{y}{x}$:
\begin{itemize}
\item Na přímce $y = mx$: sklon $m$
\item Přímky procházející počátkem jsou řešeními
\item Směrové pole je radiální
\end{itemize}
\end{example}

\vspace{0.6\baselineskip}

\begin{example}[Fázový portrét homogenní rovnice]
Analyzujte rovnici: $\frac{dy}{dx} = \frac{y^2 - x^2}{2xy}$
\vspace{0.3\baselineskip}

\textbf{Řešení}: 
\begin{enumerate}
\item \textbf{Úprava}:
\[
\frac{dy}{dx} = \frac{(y/x)^2 - 1}{2(y/x)} = \frac{v^2 - 1}{2v}
\]

\item \textbf{Rovnovážné body}: $f(v) - v = 0$
\[
\frac{v^2 - 1}{2v} - v = \frac{v^2 - 1 - 2v^2}{2v} = \frac{-v^2 - 1}{2v} = 0
\]
Žádné reálné rovnovážné body.

\item \textbf{Substituce}: $y = vx$
\[
v + x\frac{dv}{dx} = \frac{v^2 - 1}{2v} \implies x\frac{dv}{dx} = \frac{v^2 - 1}{2v} - v = \frac{-v^2 - 1}{2v}
\]

\item \textbf{Separace}:
\[
\frac{2v}{-v^2 - 1}dv = \frac{dx}{x} \implies -\int \frac{2v}{v^2 + 1}dv = \int \frac{dx}{x}
\]
\[
-\ln|v^2 + 1| = \ln|x| + C \implies \frac{1}{v^2 + 1} = Ax
\]

\item \textbf{Zpětná substituce}:
\[
\frac{1}{(y/x)^2 + 1} = Ax \implies \frac{x^2}{x^2 + y^2} = Ax
\]
\[
x^2 + y^2 = \frac{x}{A} \implies x^2 + y^2 - \frac{x}{A} = 0
\]

\item \textbf{Geometrická interpretace}: Kružnice procházející počátkem
\end{enumerate}
\end{example}

\vspace{0.8\baselineskip}

\subsubsection{Početní Sekce - Kategorie B: Podle složitosti substituce}
\label{subsubsec:pocetni-kategorie-b}

\paragraph*{B1: Přímá substituce v = y/x}

\begin{example}[Jednoduchá integrace]
Řešte: $\frac{dy}{dx} = \frac{y}{x} + \left(\frac{y}{x}\right)^2$
\vspace{0.3\baselineskip}

\textbf{Řešení}: 
\begin{enumerate}
\item $f(v) = v + v^2$

\item \textbf{Substituce}: $y = vx$
\[
v + x\frac{dv}{dx} = v + v^2 \implies x\frac{dv}{dx} = v^2
\]

\item \textbf{Separace}:
\[
\frac{dv}{v^2} = \frac{dx}{x} \implies -\frac{1}{v} = \ln|x| + C
\]

\item \textbf{Výsledek}:
\[
-\frac{x}{y} = \ln|x| + C \implies y = -\frac{x}{\ln|x| + C}
\]

\item \textbf{Singulární řešení}: $v = 0 \implies y = 0$
\end{enumerate}
\end{example}

\vspace{0.8\baselineskip}

\paragraph*{B2: Vícenásobná substituce}

\begin{example}[Řetězová substituce]
Řešte: $\frac{dy}{dx} = \sqrt{\frac{y}{x} + \sqrt{\frac{y}{x}}}$
\vspace{0.3\baselineskip}

\textbf{Řešení}: 
\begin{enumerate}
\item \textbf{První substituce}: $v = \frac{y}{x} \implies y = vx$
\[
v + x\frac{dv}{dx} = \sqrt{v + \sqrt{v}}
\]

\item \textbf{Druhá substituce}: $u = \sqrt{v} \implies v = u^2$, $dv = 2u  du$
\[
u^2 + x\cdot 2u\frac{du}{dx} = \sqrt{u^2 + u} = \sqrt{u(u + 1)}
\]

\item \textbf{Separace}:
\[
2xu\frac{du}{dx} = \sqrt{u(u + 1)} - u^2
\]
\[
\frac{2u  du}{\sqrt{u(u + 1)} - u^2} = \frac{dx}{x}
\]

\item \textbf{Integrace vyžaduje numerické metody nebo speciální funkce}
\end{enumerate}
\end{example}

\vspace{0.8\baselineskip}

\subsubsection{Aplikace v Kvantitativních Vědách}
\label{subsubsec:aplikace-kvantitativni-homog}

\begin{application}[Cobb-Douglasova produkční funkce]
\label{app:cobb-douglas}
Produkční funkce $Y = AK^\alpha L^{1-\alpha}$ je homogenní stupně 1. Rovnice popisující izokvanty:
\[
\frac{dK}{dL} = -\frac{\partial Y/\partial L}{\partial Y/\partial K} = -\frac{(1-\alpha)AK^\alpha L^{-\alpha}}{\alpha AK^{\alpha-1}L^{1-\alpha}} = -\frac{1-\alpha}{\alpha}\cdot\frac{K}{L}
\]

\textbf{Řešení}: Homogenní rovnice s $f(v) = -\frac{1-\alpha}{\alpha}v$
\[
\frac{dK}{dL} = -\frac{1-\alpha}{\alpha}\cdot\frac{K}{L}
\]
Substituce $v = K/L$:
\[
v + L\frac{dv}{dL} = -\frac{1-\alpha}{\alpha}v \implies L\frac{dv}{dL} = -\left(\frac{1-\alpha}{\alpha} + 1\right)v = -\frac{1}{\alpha}v
\]
\[
\frac{dv}{v} = -\frac{1}{\alpha}\frac{dL}{L} \implies \ln|v| = -\frac{1}{\alpha}\ln|L| + C
\]
\[
v = AL^{-1/\alpha} \implies \frac{K}{L} = AL^{-1/\alpha} \implies K = AL^{1-1/\alpha}
\]
\end{application}

\vspace{0.6\baselineskip}

\begin{application}[Model s konstantními výnosy z rozsahu]
\label{app:konstantni-vynosy}
Uvažujme firmu s nákladovou funkcí $C(Y)$ homogenní stupně 1. Pak mezní náklady závisí pouze na poměru vstupů:
\[
\frac{dC}{dY} = f\left(\frac{K}{L}\right)
\]
Homogenní rovnice popisuje optimální kombinaci vstupů.
\end{application}

\vspace{0.6\baselineskip}

\begin{application}[Podobnostní zákony v mechanice tekutin]
\label{app:podobnostni-zakony}
Reynoldsova rovnice pro proudění:
\[
\frac{dv}{dr} = f\left(\frac{v}{r}\right)
\]
kde $v$ je rychlost, $r$ poloměr. Homogenita zajišťuje podobnostní zákony.
\end{application}

\vspace{0.8\baselineskip}

% !TEX root = ../main.tex
% ČÁST 3/3 - Dokončení 1.3 Homogenní Rovnice

\subsubsection{Početní Sekce - Kategorie C: Aplikované Příklady}
\label{subsubsec:pocetni-kategorie-c-homog}

\paragraph*{C1: Ekonomické modely}

\begin{example}[Optimalizace produkční funkce]
Firma má produkční funkci $Q = K^{0.6}L^{0.4}$ a nákladovou funkci $C = 2K + 3L$. Najděte trajektorii optimálního růstu při konstantních výnosech z rozsahu.
\vspace{0.3\baselineskip}

\textbf{Řešení}: 
\begin{enumerate}
\item \textbf{Mezní produktivity}:
\[
\frac{\partial Q}{\partial K} = 0.6K^{-0.4}L^{0.4}, \quad \frac{\partial Q}{\partial L} = 0.4K^{0.6}L^{-0.6}
\]

\item \textbf{Podmínka optima}: $\frac{\partial Q/\partial K}{\partial Q/\partial L} = \frac{\text{cena K}}{\text{cena L}}$
\[
\frac{0.6K^{-0.4}L^{0.4}}{0.4K^{0.6}L^{-0.6}} = \frac{2}{3} \implies \frac{3}{2}\cdot\frac{L}{K} = \frac{2}{3}
\]
\[
\frac{L}{K} = \frac{4}{9} \implies K = \frac{9}{4}L
\]

\item \textbf{Růstová trajektorie}: $\frac{dK}{dL} = \frac{9}{4}$ - konstantní poměr

\item \textbf{Homogenní formulace}: 
\[
\frac{dK}{dL} = f\left(\frac{K}{L}\right) = \frac{9}{4}
\]
Řešení: $K = \frac{9}{4}L + C$, ale pro homogenní případ $C = 0$
\end{enumerate}
\end{example}

\vspace{0.8\baselineskip}

\paragraph*{C2: Fyzikální modely}

\begin{example}[Podobnostní zákony v dynamice]
Analyzujte rovnici popisující podobnostní zákony: $\frac{dr}{dt} = k\sqrt{\frac{r}{t}}$
\vspace{0.3\baselineskip}

\textbf{Řešení}: 
\begin{enumerate}
\item \textbf{Test homogenity}: $F(\lambda r, \lambda t) = k\sqrt{\frac{\lambda r}{\lambda t}} = k\sqrt{\frac{r}{t}} = F(r,t)$

\item \textbf{Substituce}: $v = \frac{r}{t} \implies r = vt \implies \frac{dr}{dt} = v + t\frac{dv}{dt}$

\item \textbf{Dosazení}:
\[
v + t\frac{dv}{dt} = k\sqrt{v} \implies t\frac{dv}{dt} = k\sqrt{v} - v
\]

\item \textbf{Separace}:
\[
\frac{dv}{k\sqrt{v} - v} = \frac{dt}{t}
\]

\item \textbf{Substituce}: $u = \sqrt{v} \implies v = u^2$, $dv = 2u  du$
\[
\int \frac{2u  du}{ku - u^2} = \int \frac{dt}{t} \implies 2\int \frac{du}{k - u} = \ln|t| + C
\]
\[
-2\ln|k - u| = \ln|t| + C \implies (k - u)^{-2} = At
\]

\item \textbf{Výsledek}:
\[
k - \sqrt{\frac{r}{t}} = \frac{B}{\sqrt{t}} \implies r(t) = t\left(k - \frac{B}{\sqrt{t}}\right)^2 = k^2t - 2kB\sqrt{t} + B^2
\]

\item \textbf{Fyzikální interpretace}: Růst s různými režimy v závislosti na čase
\end{enumerate}
\end{example}

\vspace{0.8\baselineskip}

\subsubsection{Početní Sekce - Kategorie D: Speciální Případy}
\label{subsubsec:pocetni-kategorie-d-homog}

\paragraph*{D1: Rovnice s parametry}

\begin{example}[Bifurkační analýza]
Analyzujte rovnici: $\frac{dy}{dx} = \frac{y}{x} + \alpha\left(\frac{y}{x}\right)^2$ v závislosti na parametru $\alpha$
\vspace{0.3\baselineskip}

\textbf{Řešení}: 
\begin{enumerate}
\item $f(v) = v + \alpha v^2$

\item \textbf{Substituce}: $y = vx$
\[
v + x\frac{dv}{dx} = v + \alpha v^2 \implies x\frac{dv}{dx} = \alpha v^2
\]

\item \textbf{Rovnovážné body}: $f(v) - v = \alpha v^2 = 0 \implies v = 0$

\item \textbf{Stabilita}: 
\begin{itemize}
\item Pro $\alpha > 0$: $v = 0$ nestabilní
\item Pro $\alpha < 0$: $v = 0$ stabilní
\item Pro $\alpha = 0$: všechny body na přímkách řešeními
\end{itemize}

\item \textbf{Obecné řešení}:
\[
\frac{dv}{v^2} = \alpha\frac{dx}{x} \implies -\frac{1}{v} = \alpha\ln|x| + C
\]
\[
v = -\frac{1}{\alpha\ln|x| + C} \implies y = -\frac{x}{\alpha\ln|x| + C}
\]

\item \textbf{Bifurkační diagram}: Transkritická bifurkace při $\alpha = 0$
\end{enumerate}
\end{example}

\vspace{0.8\baselineskip}

\paragraph*{D2: Singulární řešení}

\begin{example}[Obálka rodiny řešení]
Najděte singulární řešení rovnice: $\frac{dy}{dx} = \sqrt{\frac{y}{x}}$
\vspace{0.3\baselineskip}

\textbf{Řešení}: 
\begin{enumerate}
\item \textbf{Obecné řešení} (z příkladu A4): $y = x - 2A\sqrt{x} + A^2$

\item \textbf{Diskriminantní křivka}: Derivujeme podle parametru $A$:
\[
\frac{\partial y}{\partial A} = -2\sqrt{x} + 2A = 0 \implies A = \sqrt{x}
\]

\item \textbf{Dosazení do obecného řešení}:
\[
y = x - 2\sqrt{x}\cdot\sqrt{x} + (\sqrt{x})^2 = x - 2x + x = 0
\]

\item \textbf{Singulární řešení}: $y = 0$

\item \textbf{Ověření}: $y = 0$ splňuje původní rovnici
\[
\frac{d}{dx}[0] = 0 = \sqrt{\frac{0}{x}} \quad \checkmark
\]

\item \textbf{Geometrická interpretace}: $y = 0$ je obálkou rodiny řešení
\end{enumerate}
\end{example}

\vspace{0.8\baselineskip}

\subsubsection{Numerická Verifikace a Analýza}
\label{subsubsec:numericka-verifikace}

\begin{method}[Eulerova metoda pro homogenní rovnice]
\label{met:euler-homogenni}
Pro rovnici $\frac{dy}{dx} = f\left(\frac{y}{x}\right)$ s počáteční podmínkou $y(x_0) = y_0$:

\begin{enumerate}
\item Zvolte krok $h > 0$
\item Pro $n = 0, 1, 2, \dots$ počítáme:
\[
y_{n+1} = y_n + h\cdot f\left(\frac{y_n}{x_n}\right)
\]
\[
x_{n+1} = x_n + h
\]
\item Speciální vlastnost: Metoda zachovává homogenitu diskrétně
\end{enumerate}
\end{method}

\vspace{0.6\baselineskip}

\begin{example}[Numerická verifikace]
Porovnejte analytické a numerické řešení: $\frac{dy}{dx} = \frac{y}{x} + 1$, $y(1) = 2$
\vspace{0.3\baselineskip}

\textbf{Analytické řešení}: $y = x\ln|x| + 2x$

\textbf{Numerické řešení} (krok $h = 0.1$):
\begin{center}
\begin{tabular}{c|c|c|c}
$x_n$ & Analytické $y(x_n)$ & Numerické $y_n$ & Chyba \\
\hline
1.0 & 2.0000 & 2.0000 & 0.0000 \\
1.1 & 2.3046 & 2.3000 & 0.0046 \\
1.2 & 2.6194 & 2.6091 & 0.0103 \\
1.3 & 2.9443 & 2.9273 & 0.0170 \\
1.4 & 3.2792 & 3.2545 & 0.0247 \\
1.5 & 3.6240 & 3.5909 & 0.0331 \\
\end{tabular}
\end{center}

\textbf{Analýza}: Chyba roste s $x$, ale relativní chyba zůstává kontrolovaná.
\end{example}

\vspace{0.6\baselineskip}

\begin{example}[Numerická analýza singularity]
Analyzujte numericky chování řešení: $\frac{dy}{dx} = \frac{1}{y/x - 1}$, $y(2) = 1$
\vspace{0.3\baselineskip}

\textbf{Analytické řešení}: Singularita při $y = x$

\textbf{Numerická simulace}:
\begin{itemize}
\item Pro $x < 2$: řešení konverguje k $y = x$
\item Pro $x > 2$: řešení diverguje od $y = x$
\item Numerická metoda detekuje singularitu nestabilitou
\end{itemize}

\textbf{Závěr}: Numerické metody pomáhají identifikovat singularitu.
\end{example}

\vspace{0.8\baselineskip}

\subsubsection{Vztah k Dalším Typům Rovnic}
\label{subsubsec:vztah-k-dalsim-rovnicim}

\begin{theorem}[Vztah k lineárním rovnicím]
Každá homogenní lineární rovnice $\frac{dy}{dx} + P(x)y = 0$ je také homogenní v geometrickém smyslu.
\end{theorem}

\vspace{0.4\baselineskip}

\begin{proof}
Lineární homogenní rovnici lze zapsat jako:
\[
\frac{dy}{dx} = -P(x)y
\]
Tato rovnice je homogenní právě tehdy, když $P(x)$ je homogenní stupně $-1$, tj. $P(\lambda x) = \frac{1}{\lambda}P(x)$.
\end{proof}

\vspace{0.6\baselineskip}

\begin{theorem}[Vztah k exaktním rovnicím]
Homogenní rovnice mohou být často převedeny na exaktní rovnice vhodnou substitucí.
\end{theorem}

\vspace{0.4\baselineskip}

\begin{proof}
Uvažujme homogenní rovnici $M(x,y)dx + N(x,y)dy = 0$ kde $M$, $N$ jsou homogenní stejného stupně. Substituce $y = vx$ vede na separabilní rovnici, která je speciálním případem exaktní rovnice.
\end{proof}

\vspace{0.8\baselineskip}

\subsubsection{Shrnutí a Expertní Metodologie}
\label{subsubsec:shrnuti-metodologie}

\begin{method}[Decision tree pro homogenní rovnice]
\label{met:decision-tree-homogenni}
\begin{enumerate}
\item \textbf{Identifikace}: Lze rovnici zapsat jako $\frac{dy}{dx} = f\left(\frac{y}{x}\right)$?

\item \textbf{Test homogenity}: Platí $F(\lambda x, \lambda y) = F(x, y)$?

\item \textbf{Základní substituce}: $v = \frac{y}{x} \implies y = vx$

\item \textbf{Analýza singulárních bodů}: Najdi $v$ taková, že $f(v) - v = 0$

\item \textbf{Integrace}:
\[
\int \frac{dv}{f(v) - v} = \int \frac{dx}{x} = \ln|x| + C
\]

\item \textbf{Zpětná substituce}: $v = \frac{y}{x}$

\item \textbf{Analýza řešení}:
\begin{itemize}
\item Urči maximální intervaly existence
\item Najdi singulární řešení
\item Analyzuj asymptotické chování
\end{itemize}
\end{enumerate}
\end{method}

\vspace{0.8\baselineskip}

\begin{remark}[Časté chyby a jak se jim vyhnout]
\begin{itemize}
\item \textbf{Zapomnění na absolutní hodnotu}: $\int \frac{dx}{x} = \ln|x| + C$, ne $\ln x + C$
\item \textbf{Špatná derivace}: $\frac{d}{dx}[vx] = v + x\frac{dv}{dx}$, ne $x\frac{dv}{dx}$
\item \textbf{Ignorování singularit}: Vždy analyzuj body kde $f(v) - v = 0$
\item \textbf{Chybný definiční obor}: Homogenní rovnice typicky nejsou definované v $x = 0$
\item \textbf{Záměna s linearitou}: Homogenní ≠ lineární homogenní
\end{itemize}
\end{remark}

\vspace{0.6\baselineskip}

\begin{remark}[Optimalizace řešení]
\begin{itemize}
\item Pro racionální $f(v)$: použij rozklad na parciální zlomky
\item Pro algebraické $f(v)$: vhodné substituce pro odstranění odmocnin
\item Pro goniometrické $f(v)$: goniometrické identity a substituce
\item Pro složité integrály: numerické metody nebo speciální funkce
\end{itemize}
\end{remark}

\vspace{0.8\baselineskip}

\begin{transition}
S kompletním zvládnutím homogenních rovnic jsme připraveni přejít k exaktním rovnicím (1.4), které představují další mocnou třídu řešitelných diferenciálních rovnic 1. řádu a otevírají cestu k teorii diferenciálních forem.
\end{transition}

\vspace{0.8\baselineskip}

\subsubsection*{Shrnutí Sekce 1.3}

Homogenní rovnice představují elegantní třídu diferenciálních rovnic s hlubokou geometrickou interpretací. Klíčové poznatky:

\begin{itemize}
\item \textbf{Geometrická podstata}: Invariance vůči dilatacím, radiální struktura řešení
\item \textbf{Univerzální řešicí metoda}: Substituce $v = y/x$ převádí na separabilní rovnici
\item \textbf{Aplikace}: Ekonomické modely s konstantními výnosy z rozsahu, podobnostní zákony ve fyzice
\item \textbf{Speciální vlastnosti}: Singulární řešení jako obálky, bifurkační chování
\item \textbf{Numerická stabilita}: Eulerova metoda dobře funguje pro homogenní rovnice
\end{itemize}

Zvládnutí homogenních rovnic je zásadní pro rozvoj geometrické intuice v teorii diferenciálních rovnic a připravuje půdu pro pokročilejší témata v následujících úrovních.


\subsection{Exaktní Rovnice - Teorie Potenciálů a Konzervativní Pole}
\label{subsec:exaktni-rovnice}

\subsubsection{Teoretický Fundament Exaktnosti}
\label{subsubsec:teoreticky-fundament-exaktni}

\begin{definition}[Exaktní diferenciální rovnice]
Rovnice tvaru
\[
M(x, y)dx + N(x, y)dy = 0
\]
je \emph{exaktní}, jestliže existuje funkce $F(x, y)$ taková, že:
\[
\frac{\partial F}{\partial x} = M(x, y) \quad \text{a} \quad \frac{\partial F}{\partial y} = N(x, y)
\]
na nějaké oblasti $D \subseteq \mathbb{R}^2$.
\end{definition}

\vspace{0.6\baselineskip}

\begin{theorem}[Nutná a postačující podmínka exaktnosti]
Nechť $M(x, y)$ a $N(x, y)$ mají spojité parciální derivace prvního řádu na jednoduše souvislé oblasti $D$. Pak rovnice $M dx + N dy = 0$ je exaktní právě tehdy, když:
\[
\frac{\partial M}{\partial y} = \frac{\partial N}{\partial x} \quad \text{na } D
\]
\end{theorem}

\vspace{0.4\baselineskip}

\begin{proof}
\textbf{(⇒)} Je-li rovnice exaktní, existuje $F$ taková, že $F_x = M$, $F_y = N$. Ze Schwarzovy věty:
\[
\frac{\partial M}{\partial y} = \frac{\partial^2 F}{\partial y \partial x} = \frac{\partial^2 F}{\partial x \partial y} = \frac{\partial N}{\partial x}
\]

\textbf{(⇐)} Konstruujeme $F(x, y)$ přímo. Zafixujeme $(x_0, y_0) \in D$ a definujeme:
\[
F(x, y) = \int_{x_0}^x M(t, y)dt + \int_{y_0}^y N(x_0, s)ds
\]
Pak $\frac{\partial F}{\partial x} = M(x, y)$ a pomocí podmínky $\frac{\partial M}{\partial y} = \frac{\partial N}{\partial x}$ dostaneme $\frac{\partial F}{\partial y} = N(x, y)$.
\end{proof}

\vspace{0.6\baselineskip}

\begin{theorem}[Věta o nezávislosti na cestě]
Nechť $M dx + N dy$ je exaktní diferenciální forma na jednoduše souvislé oblasti $D$. Pak integrál
\[
\int_C M dx + N dy
\]
závisí pouze na počátečním a koncovém bodě křivky $C$, nikoli na její konkrétní trajektorii.
\end{theorem}

\vspace{0.8\baselineskip}

\subsubsection{Metoda Hledání Potenciálu - Kompletní Analýza}
\label{subsubsec:metoda-hledani-potencialu}

\begin{method}[Přímá integrační metoda]
\label{met:primaintegrace-potencial}
\begin{enumerate}
\item \textbf{Ověření exaktnosti}: Zkontrolujte že $\frac{\partial M}{\partial y} = \frac{\partial N}{\partial x}$

\item \textbf{Integrace podle x}:
\[
F(x, y) = \int M(x, y)dx + \phi(y)
\]
kde $\phi(y)$ je "integrační konstanta" závislá na $y$

\item \textbf{Určení $\phi(y)$}: Derivujte podle $y$ a porovnejte s $N(x, y)$:
\[
\frac{\partial F}{\partial y} = \frac{\partial}{\partial y}\left[\int M(x, y)dx\right] + \phi'(y) = N(x, y)
\]

\item \textbf{Integrace $\phi(y)$}:
\[
\phi(y) = \int \left[N(x, y) - \frac{\partial}{\partial y}\left(\int M(x, y)dx\right)\right] dy
\]

\item \textbf{Konečný tvar}:
\[
F(x, y) = \int M(x, y)dx + \int \left[N(x, y) - \frac{\partial}{\partial y}\left(\int M(x, y)dx\right)\right] dy
\]

\item \textbf{Řešení rovnice}: $F(x, y) = C$
\end{enumerate}
\end{method}

\vspace{0.8\baselineskip}

\begin{method}[Alternativní metoda - integrace podle y]
\label{met:alternativni-integracce}
\begin{enumerate}
\item \textbf{Integrace podle y}:
\[
F(x, y) = \int N(x, y)dy + \psi(x)
\]

\item \textbf{Určení $\psi(x)$}:
\[
\frac{\partial F}{\partial x} = \frac{\partial}{\partial x}\left[\int N(x, y)dy\right] + \psi'(x) = M(x, y)
\]

\item \textbf{Výběr metody}: Volíme tu, která vede k jednodušší integraci
\end{enumerate}
\end{method}

\vspace{0.6\baselineskip}

\begin{example}[Kompletní postup přímé integrace]
Řešte: $(2x + y)dx + (x + 2y)dy = 0$
\vspace{0.3\baselineskip}

\textbf{Řešení}: 
\begin{enumerate}
\item \textbf{Ověření exaktnosti}:
\[
M(x, y) = 2x + y, \quad N(x, y) = x + 2y
\]
\[
\frac{\partial M}{\partial y} = 1, \quad \frac{\partial N}{\partial x} = 1 \quad \checkmark
\]

\item \textbf{Integrace M podle x}:
\[
\int (2x + y)dx = x^2 + xy + \phi(y)
\]

\item \textbf{Derivace podle y}:
\[
\frac{\partial F}{\partial y} = x + \phi'(y)
\]

\item \textbf{Porovnání s N}:
\[
x + \phi'(y) = x + 2y \implies \phi'(y) = 2y
\]

\item \textbf{Integrace $\phi(y)$}:
\[
\phi(y) = \int 2y  dy = y^2 + C_1
\]

\item \textbf{Potenciál}:
\[
F(x, y) = x^2 + xy + y^2 + C_1
\]

\item \textbf{Řešení}:
\[
x^2 + xy + y^2 = C \quad (C = -C_1)
\]

\item \textbf{Ověření}:
\[
d(x^2 + xy + y^2) = (2x + y)dx + (x + 2y)dy \quad \checkmark
\end{enumerate}
\end{example}

\vspace{0.8\baselineskip}

\subsubsection{Početní Sekce - Kategorie A: Přímé Exaktní Rovnice}
\label{subsubsec:pocetni-kategorie-a}

\paragraph*{A1: Jednoduché polynomy}

\begin{example}[Lineární polynomy]
Řešte: $(3x^2 + 2y)dx + (2x + 4y)dy = 0$
\vspace{0.3\baselineskip}

\textbf{Řešení}: 
\begin{enumerate}
\item \textbf{Ověření exaktnosti}:
\[
\frac{\partial}{\partial y}(3x^2 + 2y) = 2, \quad \frac{\partial}{\partial x}(2x + 4y) = 2 \quad \checkmark
\]

\item \textbf{Integrace M podle x}:
\[
\int (3x^2 + 2y)dx = x^3 + 2xy + \phi(y)
\]

\item \textbf{Derivace a porovnání}:
\[
\frac{\partial F}{\partial y} = 2x + \phi'(y) = 2x + 4y \implies \phi'(y) = 4y
\]

\item \textbf{Potenciál}:
\[
F(x, y) = x^3 + 2xy + 2y^2
\]

\item \textbf{Řešení}:
\[
x^3 + 2xy + 2y^2 = C
\end{enumerate}
\end{example}

\vspace{0.6\baselineskip}

\begin{example}[Kvadratické polynomy]
Řešte: $(2xy + y^2)dx + (x^2 + 2xy)dy = 0$
\vspace{0.3\baselineskip}

\textbf{Řešení}: 
\begin{enumerate}
\item \textbf{Ověření}:
\[
\frac{\partial}{\partial y}(2xy + y^2) = 2x + 2y, \quad \frac{\partial}{\partial x}(x^2 + 2xy) = 2x + 2y \quad \checkmark
\]

\item \textbf{Integrace M podle x}:
\[
\int (2xy + y^2)dx = x^2y + xy^2 + \phi(y)
\]

\item \textbf{Derivace a porovnání}:
\[
\frac{\partial F}{\partial y} = x^2 + 2xy + \phi'(y) = x^2 + 2xy \implies \phi'(y) = 0
\]

\item \textbf{Potenciál}:
\[
F(x, y) = x^2y + xy^2
\]

\item \textbf{Řešení}:
\[
x^2y + xy^2 = C \quad \text{nebo} \quad xy(x + y) = C
\end{enumerate}
\end{example}

\vspace{0.8\baselineskip}

\paragraph*{A2: Racionální funkce}

\begin{example}[Jednoduché racionální funkce]
Řešte: $\left(\frac{1}{y} - \frac{y}{x^2}\right)dx + \left(\frac{1}{x} - \frac{x}{y^2}\right)dy = 0$
\vspace{0.3\baselineskip}

\textbf{Řešení}: 
\begin{enumerate}
\item \textbf{Ověření exaktnosti}:
\[
M = \frac{1}{y} - \frac{y}{x^2}, \quad N = \frac{1}{x} - \frac{x}{y^2}
\]
\[
\frac{\partial M}{\partial y} = -\frac{1}{y^2} - \frac{1}{x^2}, \quad \frac{\partial N}{\partial x} = -\frac{1}{x^2} - \frac{1}{y^2} \quad \checkmark
\]

\item \textbf{Integrace M podle x}:
\[
\int \left(\frac{1}{y} - \frac{y}{x^2}\right)dx = \frac{x}{y} + \frac{y}{x} + \phi(y)
\]

\item \textbf{Derivace a porovnání}:
\[
\frac{\partial F}{\partial y} = -\frac{x}{y^2} + \frac{1}{x} + \phi'(y) = \frac{1}{x} - \frac{x}{y^2}
\]
\[
\phi'(y) = 0
\]

\item \textbf{Potenciál}:
\[
F(x, y) = \frac{x}{y} + \frac{y}{x}
\]

\item \textbf{Řešení}:
\[
\frac{x}{y} + \frac{y}{x} = C
\end{enumerate}
\end{example}

\vspace{0.8\baselineskip}

\begin{transition}
V další části pokryjeme goniometrické a exponenciální funkce, integrační faktory a jejich systematickou klasifikaci.
\end{transition}

\subsubsection{Početní Sekce - Kategorie A: Přímé Exaktní Rovnice (pokračování)}
\label{subsubsec:pocetni-kategorie-a-pokracovani}

\paragraph*{A3: Goniometrické funkce}

\begin{example}[Základní goniometrické funkce]
Řešte: $(\sin y + y\cos x)dx + (\sin x + x\cos y)dy = 0$
\vspace{0.3\baselineskip}

\textbf{Řešení}: 
\begin{enumerate}
\item \textbf{Ověření exaktnosti}:
\[
M(x, y) = \sin y + y\cos x, \quad N(x, y) = \sin x + x\cos y
\]
\[
\frac{\partial M}{\partial y} = \cos y + \cos x, \quad \frac{\partial N}{\partial x} = \cos x + \cos y \quad \checkmark
\]

\item \textbf{Integrace M podle x}:
\[
\int (\sin y + y\cos x)dx = x\sin y + y\sin x + \phi(y)
\]

\item \textbf{Derivace a porovnání}:
\[
\frac{\partial F}{\partial y} = x\cos y + \sin x + \phi'(y) = \sin x + x\cos y
\]
\[
\phi'(y) = 0
\]

\item \textbf{Potenciál}:
\[
F(x, y) = x\sin y + y\sin x
\]

\item \textbf{Řešení}:
\[
x\sin y + y\sin x = C
\]

\item \textbf{Geometrická interpretace}: Rovnice popisuje rodinu křivek v potenciálovém poli s periodickou strukturou.
\end{enumerate}
\end{example}

\vspace{0.6\baselineskip}

\begin{example}[Kombinace goniometrických funkcí]
Řešte: $(2x\sin y + y^2\cos x)dx + (x^2\cos y + 2y\sin x)dy = 0$
\vspace{0.3\baselineskip}

\textbf{Řešení}: 
\begin{enumerate}
\item \textbf{Ověření exaktnosti}:
\[
\frac{\partial M}{\partial y} = 2x\cos y + 2y\cos x, \quad \frac{\partial N}{\partial x} = 2x\cos y + 2y\cos x \quad \checkmark
\]

\item \textbf{Integrace M podle x}:
\[
\int (2x\sin y + y^2\cos x)dx = x^2\sin y + y^2\sin x + \phi(y)
\]

\item \textbf{Derivace a porovnání}:
\[
\frac{\partial F}{\partial y} = x^2\cos y + 2y\sin x + \phi'(y) = x^2\cos y + 2y\sin x
\]
\[
\phi'(y) = 0
\]

\item \textbf{Potenciál}:
\[
F(x, y) = x^2\sin y + y^2\sin x
\]

\item \textbf{Řešení}:
\[
x^2\sin y + y^2\sin x = C
\end{enumerate}
\end{example}

\vspace{0.8\baselineskip}

\paragraph*{A4: Exponenciální a logaritmické funkce}

\begin{example}[Jednoduché exponenciály]
Řešte: $(e^x \ln y + \frac{y}{x})dx + (\frac{e^x}{y} + \ln x)dy = 0$
\vspace{0.3\baselineskip}

\textbf{Řešení}: 
\begin{enumerate}
\item \textbf{Ověření exaktnosti}:
\[
M = e^x \ln y + \frac{y}{x}, \quad N = \frac{e^x}{y} + \ln x
\]
\[
\frac{\partial M}{\partial y} = \frac{e^x}{y} + \frac{1}{x}, \quad \frac{\partial N}{\partial x} = \frac{e^x}{y} + \frac{1}{x} \quad \checkmark
\]

\item \textbf{Integrace M podle x}:
\[
\int \left(e^x \ln y + \frac{y}{x}\right)dx = e^x \ln y + y\ln x + \phi(y)
\]

\item \textbf{Derivace a porovnání}:
\[
\frac{\partial F}{\partial y} = \frac{e^x}{y} + \ln x + \phi'(y) = \frac{e^x}{y} + \ln x
\]
\[
\phi'(y) = 0
\]

\item \textbf{Potenciál}:
\[
F(x, y) = e^x \ln y + y\ln x
\]

\item \textbf{Řešení}:
\[
e^x \ln y + y\ln x = C
\end{enumerate}
\end{example}

\vspace{0.6\baselineskip}

\begin{example}[Složené exponenciálně-logaritmické výrazy]
Řešte: $(y^x \ln y + e^{x+y})dx + (xy^{x-1} + e^{x+y})dy = 0$
\vspace{0.3\baselineskip}

\textbf{Řešení}: 
\begin{enumerate}
\item \textbf{Ověření exaktnosti}:
\[
M = y^x \ln y + e^{x+y}, \quad N = xy^{x-1} + e^{x+y}
\]
\[
\frac{\partial M}{\partial y} = xy^{x-1} \ln y + y^x \cdot \frac{1}{y} + e^{x+y} = xy^{x-1} \ln y + y^{x-1} + e^{x+y}
\]
\[
\frac{\partial N}{\partial x} = y^{x-1} + xy^{x-1} \ln y + e^{x+y} \quad \checkmark
\]

\item \textbf{Integrace M podle x}:
\[
\int (y^x \ln y + e^{x+y})dx = y^x + e^{x+y} + \phi(y)
\]

\item \textbf{Derivace a porovnání}:
\[
\frac{\partial F}{\partial y} = xy^{x-1} + e^{x+y} + \phi'(y) = xy^{x-1} + e^{x+y}
\]
\[
\phi'(y) = 0
\]

\item \textbf{Potenciál}:
\[
F(x, y) = y^x + e^{x+y}
\]

\item \textbf{Řešení}:
\[
y^x + e^{x+y} = C
\end{enumerate}
\end{example}

\vspace{0.8\baselineskip}

\subsubsection{Integrační Faktory pro Neexaktní Rovnice}
\label{subsubsec:integracni-faktory}

\begin{definition}[Integrační faktor]
Funkce $\mu(x, y)$ je \emph{integrační faktor} pro rovnici $M dx + N dy = 0$, jestliže rovnice
\[
\mu M dx + \mu N dy = 0
\]
je exaktní.
\end{definition}

\vspace{0.6\baselineskip}

\begin{theorem}[Rovnice pro integrační faktor]
Funkce $\mu(x, y)$ je integrační faktor právě tehdy, když splňuje parciální diferenciální rovnici:
\[
M\frac{\partial \mu}{\partial y} - N\frac{\partial \mu}{\partial x} + \mu\left(\frac{\partial M}{\partial y} - \frac{\partial N}{\partial x}\right) = 0
\]
\end{theorem}

\vspace{0.4\baselineskip}

\begin{proof}
Podmínka exaktnosti pro $\mu M dx + \mu N dy = 0$ je:
\[
\frac{\partial (\mu M)}{\partial y} = \frac{\partial (\mu N)}{\partial x}
\]
\[
M\frac{\partial \mu}{\partial y} + \mu\frac{\partial M}{\partial y} = N\frac{\partial \mu}{\partial x} + \mu\frac{\partial N}{\partial x}
\]
Po úpravě dostaneme uvedenou rovnici.
\end{proof}

\vspace{0.8\baselineskip}

\begin{method}[Hledání integračních faktorů speciálního tvaru]
\label{met:integracni-faktory-speciální}
\begin{enumerate}
\item \textbf{Předpoklad $\mu = \mu(x)$}: Pak $\frac{\partial \mu}{\partial y} = 0$ a rovnice se zjednoduší na:
\[
\frac{d\mu}{dx} = \frac{\frac{\partial M}{\partial y} - \frac{\partial N}{\partial x}}{N} \mu
\]
Tento předpoklad je rozumný, pokud výraz $\frac{\frac{\partial M}{\partial y} - \frac{\partial N}{\partial x}}{N}$ závisí pouze na $x$.

\item \textbf{Předpoklad $\mu = \mu(y)$}: Analogicky:
\[
\frac{d\mu}{dy} = \frac{\frac{\partial N}{\partial x} - \frac{\partial M}{\partial y}}{M} \mu
\]
Rozumný, pokud výraz závisí pouze na $y$.

\item \textbf{Řešení separované rovnice}:
\[
\mu(x) = \exp\left(\int \frac{\frac{\partial M}{\partial y} - \frac{\partial N}{\partial x}}{N} dx\right)
\]
\[
\mu(y) = \exp\left(\int \frac{\frac{\partial N}{\partial x} - \frac{\partial M}{\partial y}}{M} dy\right)
\]
\end{enumerate}
\end{method}

\vspace{0.8\baselineskip}

\begin{example}[Integrační faktor $\mu(x)$]
Řešte: $(3xy + y^2)dx + (x^2 + xy)dy = 0$
\vspace{0.3\baselineskip}

\textbf{Řešení}: 
\begin{enumerate}
\item \textbf{Ověření exaktnosti}:
\[
\frac{\partial M}{\partial y} = 3x + 2y, \quad \frac{\partial N}{\partial x} = 2x + y
\]
Rovnice není exaktní.

\item \textbf{Hledání $\mu(x)$}:
\[
\frac{\frac{\partial M}{\partial y} - \frac{\partial N}{\partial x}}{N} = \frac{(3x + 2y) - (2x + y)}{x^2 + xy} = \frac{x + y}{x(x + y)} = \frac{1}{x}
\]
Závisí pouze na $x$ → předpoklad je platný.

\item \textbf{Výpočet $\mu(x)$}:
\[
\mu(x) = \exp\left(\int \frac{1}{x} dx\right) = \exp(\ln|x|) = |x| \quad (\text{volíme } x > 0 \Rightarrow \mu = x)
\]

\item \textbf{Násobení rovnice}:
\[
x[(3xy + y^2)dx + (x^2 + xy)dy] = 0
\]
\[
(3x^2y + xy^2)dx + (x^3 + x^2y)dy = 0
\]

\item \textbf{Ověření exaktnosti nové rovnice}:
\[
\frac{\partial}{\partial y}(3x^2y + xy^2) = 3x^2 + 2xy
\]
\[
\frac{\partial}{\partial x}(x^3 + x^2y) = 3x^2 + 2xy \quad \checkmark
\]

\item \textbf{Řešení exaktní rovnice}:
\[
F(x, y) = \int (3x^2y + xy^2)dx = x^3y + \frac{1}{2}x^2y^2 + \phi(y)
\]
\[
\frac{\partial F}{\partial y} = x^3 + x^2y + \phi'(y) = x^3 + x^2y \Rightarrow \phi'(y) = 0
\]
\[
F(x, y) = x^3y + \frac{1}{2}x^2y^2
\]

\item \textbf{Konečné řešení}:
\[
x^3y + \frac{1}{2}x^2y^2 = C \quad \text{nebo} \quad x^2y(2x + y) = 2C
\end{enumerate}
\end{example}

\vspace{0.8\baselineskip}

\begin{transition}
V další části pokryjeme integrační faktory tvaru $\mu(xy)$, $\mu(x/y)$ a jejich aplikace na symetrické rovnice, včetně geometrické interpretace a pokročilých technik.
\end{transition}


% !TEX root = ../main.tex
\subsubsection{Integrační Faktory Speciálních Tvarů}
\label{subsubsec:integracni-faktory-specialni}

\begin{method}[Integrační faktor $\mu = \mu(xy)$]
\label{met:integracni-faktor-xy}
Předpokládáme, že $\mu = \mu(z)$ kde $z = xy$. Pak:
\[
\frac{\partial \mu}{\partial x} = \mu'(z)y, \quad \frac{\partial \mu}{\partial y} = \mu'(z)x
\]
Rovnice pro integrační faktor se zjednoduší na:
\[
\frac{d\mu}{dz} = \frac{\frac{\partial M}{\partial y} - \frac{\partial N}{\partial x}}{Ny - Mx} \mu
\]
Tento předpoklad je rozumný, pokud výraz $\frac{\frac{\partial M}{\partial y} - \frac{\partial N}{\partial x}}{Ny - Mx}$ závisí pouze na $z = xy$.
\end{method}

\vspace{0.6\baselineskip}

\begin{example}[Integrační faktor $\mu(xy)$]
Řešte: $(y + xy^2)dx + (x - x^2y)dy = 0$
\vspace{0.3\baselineskip}

\textbf{Řešení}:
\begin{enumerate}
\item \textbf{Ověření exaktnosti}:
\[
\frac{\partial M}{\partial y} = 1 + 2xy, \quad \frac{\partial N}{\partial x} = 1 - 2xy \quad \text{NEJSOU STEJNÉ}
\]

\item \textbf{Test pro $\mu(xy)$}:
\[
Ny - Mx = (x - x^2y)y - (y + xy^2)x = xy - x^2y^2 - xy - x^2y^2 = -2x^2y^2
\]
\[
\frac{\frac{\partial M}{\partial y} - \frac{\partial N}{\partial x}}{Ny - Mx} = \frac{(1 + 2xy) - (1 - 2xy)}{-2x^2y^2} = \frac{4xy}{-2x^2y^2} = -\frac{2}{xy}
\]
Závisí pouze na $z = xy$ → předpoklad platný.

\item \textbf{Výpočet $\mu(z)$}:
\[
\frac{d\mu}{dz} = -\frac{2}{z} \mu \Rightarrow \frac{d\mu}{\mu} = -\frac{2}{z} dz
\]
\[
\ln|\mu| = -2\ln|z| + C \Rightarrow \mu = \frac{1}{z^2} = \frac{1}{x^2y^2}
\]

\item \textbf{Násobení a řešení}:
\[
\frac{1}{x^2y^2}[(y + xy^2)dx + (x - x^2y)dy] = 0
\]
\[
\left(\frac{1}{x^2y} + \frac{1}{x}\right)dx + \left(\frac{1}{xy^2} - \frac{1}{y}\right)dy = 0
\]
Tato rovnice je již exaktní. Řešením dostaneme:
\[
F(x, y) = -\frac{1}{xy} + \ln\left|\frac{x}{y}\right| = C
\end{enumerate}
\end{example}

\vspace{0.8\baselineskip}

\begin{method}[Integrační faktor $\mu = \mu(x/y)$]
\label{met:integracni-faktor-x-y}
Předpokládáme, že $\mu = \mu(z)$ kde $z = x/y$. Pak:
\[
\frac{\partial \mu}{\partial x} = \frac{\mu'(z)}{y}, \quad \frac{\partial \mu}{\partial y} = -\frac{x\mu'(z)}{y^2}
\]
Rovnice pro integrační faktor:
\[
\frac{d\mu}{dz} = \frac{\frac{\partial M}{\partial y} - \frac{\partial N}{\partial x}}{\frac{N}{y} + \frac{Mx}{y^2}} \mu
\]
Předpoklad platný, pokud výraz vpravo závisí pouze na $z = x/y$.
\end{method}

\vspace{0.6\baselineskip}

\begin{example}[Integrační faktor $\mu(x/y)$]
Řešte: $(x^2 + y^2)dx + (x^2 - xy)dy = 0$
\vspace{0.3\baselineskip}

\textbf{Řešení}:
\begin{enumerate}
\item \textbf{Ověření exaktnosti}:
\[
\frac{\partial M}{\partial y} = 2y, \quad \frac{\partial N}{\partial x} = 2x - y \quad \text{NEJSOU STEJNÉ}
\]

\item \textbf{Test pro $\mu(x/y)$}:
\[
\frac{N}{y} + \frac{Mx}{y^2} = \frac{x^2 - xy}{y} + \frac{(x^2 + y^2)x}{y^2} = \frac{x^2}{y} - x + \frac{x^3}{y^2} + \frac{x}{y}
\]
\[
\frac{\frac{\partial M}{\partial y} - \frac{\partial N}{\partial x}}{\frac{N}{y} + \frac{Mx}{y^2}} = \frac{2y - (2x - y)}{\frac{x^2}{y} - x + \frac{x^3}{y^2} + \frac{x}{y}} = \frac{3y - 2x}{\frac{x^2}{y} - x + \frac{x^3}{y^2} + \frac{x}{y}}
\]
Vynásobením čitatele i jmenovatele $y^2$:
\[
= \frac{(3y - 2x)y^2}{x^2y - xy^2 + x^3 + xy} = \frac{y^2(3y - 2x)}{x(x^2 + xy + y^2 - y^2)}
\]
Po zjednodušení zjistíme, že výraz závisí pouze na $x/y$.

\item \textbf{Výpočet $\mu(z)$} a řešení vede na:
\[
\mu = \frac{1}{x^2} \quad \text{a řešení} \quad \ln|x| - \frac{y}{x} + \frac{1}{2}\left(\frac{y}{x}\right)^2 = C
\end{enumerate}
\end{example}

\vspace{0.8\baselineskip}

\subsubsection{Geometrická Interpretace a Fázová Analýza}
\label{subsubsec:geometricka-interpretace}

\begin{definition}[Ekvipotenciální křivky]
Pro exaktní rovnici $M dx + N dy = 0$ s potenciálem $F(x, y)$ jsou \emph{ekvipotenciální křivky} definovány jako:
\[
F(x, y) = C, \quad C \in \mathbb{R}
\]
Tyto křivky tvoří řešení diferenciální rovnice.
\end{definition}

\vspace{0.6\baselineskip}

\begin{theorem}[Ortogonální trajektorie]
Nechť $F(x, y) = C$ je rodina ekvipotenciálních křivek. Pak rodina křivek $G(x, y) = K$ splňující:
\[
\frac{\partial F}{\partial x} \frac{\partial G}{\partial x} + \frac{\partial F}{\partial y} \frac{\partial G}{\partial y} = 0
\]
jsou \emph{ortogonální trajektorie} k původní rodině.
\end{theorem}

\vspace{0.4\baselineskip}

\begin{proof}
Gradient $\nabla F$ je normálový vektor k ekvipotenciálním křivkám. Podmínka ortogonality vyžaduje, aby $\nabla F \cdot \nabla G = 0$.
\end{proof}

\vspace{0.6\baselineskip}

\begin{example}[Ekvipotenciály a ortogonální trajektorie]
Najděte ortogonální trajektorie k rodině: $x^2 + y^2 = C$
\vspace{0.3\baselineskip}

\textbf{Řešení}:
\begin{enumerate}
\item \textbf{Původní rodina}: $F(x, y) = x^2 + y^2 = C$
\[
\nabla F = (2x, 2y)
\]

\item \textbf{Diferenciální rovnice ortogonálních trajektorií}:
\[
\frac{dy}{dx} = \frac{-F_x}{F_y} = \frac{-2x}{2y} = -\frac{x}{y}
\]

\item \textbf{Řešení}:
\[
\frac{dy}{dx} = -\frac{x}{y} \Rightarrow y dy = -x dx
\]
\[
\frac{1}{2}y^2 = -\frac{1}{2}x^2 + K \Rightarrow x^2 + y^2 = 2K
\]

\item \textbf{Interpretace}: Ortogonální trajektorie jsou opět kružnice se středem v počátku.
\end{enumerate}
\end{example}

\vspace{0.8\baselineskip}

\begin{definition}[Singulární body]
Bod $(x_0, y_0)$ je \emph{singulárním bodem} rovnice $M dx + N dy = 0$, jestliže:
\[
M(x_0, y_0) = N(x_0, y_0) = 0
\]
V těchto bodech není směrové pole definováno.
\end{definition}

\vspace{0.6\baselineskip}

\begin{example}[Analýza singulárních bodů]
Analyzujte singulární body rovnice: $(x^2 - y^2)dx + 2xydy = 0$
\vspace{0.3\baselineskip}

\textbf{Řešení}:
\begin{enumerate}
\item \textbf{Singulární body}: 
\[
M = x^2 - y^2 = 0, \quad N = 2xy = 0
\]
Jediný singulární bod je $(0, 0)$.

\item \textbf{Chvání v okolí singularity}:
\[
\frac{dy}{dx} = -\frac{M}{N} = -\frac{x^2 - y^2}{2xy} = \frac{y^2 - x^2}{2xy}
\]
V polárních souřadnicích $x = r\cos\theta$, $y = r\sin\theta$:
\[
\frac{dy}{dx} = \frac{\sin^2\theta - \cos^2\theta}{2\sin\theta\cos\theta} = -\frac{\cos 2\theta}{\sin 2\theta} = -\cot 2\theta
\]

\item \textbf{Interpretace}: Směrové pole závisí pouze na úhlu $\theta$, ne na poloměru $r$. Jedná se o \emph{radiálně homogenní} pole.
\end{enumerate}
\end{example}

\vspace{0.8\baselineskip}

\subsubsection{Početní Sekce - Kategorie B: Rovnice s Integračními Faktory}
\label{subsubsec:pocetni-kategorie-b}

\paragraph*{B1: Jednoduché integrační faktory $\mu(x)$ a $\mu(y)$}

\begin{example}[$\mu(x)$ - fyzikální aplikace]
Řešte rovnici popisující izotermický děj: $(PV - a)dP + P^2 dV = 0$
\vspace{0.3\baselineskip}

\textbf{Řešení}:
\begin{enumerate}
\item \textbf{Ověření exaktnosti}:
\[
\frac{\partial}{\partial V}(PV - a) = P, \quad \frac{\partial}{\partial P}(P^2) = 2P \quad \text{NEJSOU STEJNÉ}
\]

\item \textbf{Hledání $\mu(P)$}:
\[
\frac{\frac{\partial M}{\partial V} - \frac{\partial N}{\partial P}}{N} = \frac{P - 2P}{P^2} = -\frac{1}{P}
\]
\[
\mu(P) = \exp\left(-\int \frac{1}{P} dP\right) = \frac{1}{P}
\]

\item \textbf{Řešení}:
\[
\frac{1}{P}[(PV - a)dP + P^2 dV] = 0 \Rightarrow (V - \frac{a}{P})dP + P dV = 0
\]
Tato rovnice je exaktní. Potenciál:
\[
F(P, V) = PV - a\ln|P| = C
\]
\end{enumerate}
\end{example}

\vspace{0.6\baselineskip}

\begin{example}[$\mu(y)$ - ekonomická aplikace]
Řešte rovnici indiferenční křivky: $(2xy + y^2)dx + (x^2 + 2xy - 1)dy = 0$
\vspace{0.3\baselineskip}

\textbf{Řešení}:
\begin{enumerate}
\item \textbf{Ověření exaktnosti}:
\[
\frac{\partial M}{\partial y} = 2x + 2y, \quad \frac{\partial N}{\partial x} = 2x + 2y \quad \checkmark
\]
Poznámka: Tato rovnice je již exaktní, slouží jako ilustrace.

\item \textbf{Potenciál}:
\[
F(x, y) = \int (2xy + y^2)dx = x^2y + xy^2 + \phi(y)
\]
\[
\frac{\partial F}{\partial y} = x^2 + 2xy + \phi'(y) = x^2 + 2xy - 1 \Rightarrow \phi'(y) = -1
\]
\[
F(x, y) = x^2y + xy^2 - y
\]

\item \textbf{Řešení}:
\[
x^2y + xy^2 - y = C \quad \text{nebo} \quad y(x^2 + xy - 1) = C
\]

\item \textbf{Ekonomická interpretace}: Jedná se o indiferenční křivku utility funkce $U(x, y) = x^2y + xy^2 - y$.
\end{enumerate}
\end{example}

\vspace{0.8\baselineskip}

\paragraph*{B2: Pokročilé integrační faktory}

\begin{example}[$\mu(x^2 + y^2)$ - radiálně symetrické pole]
Řešte: $(x^2 + y^2 + y)dx + (x^2 + y^2 - x)dy = 0$
\vspace{0.3\baselineskip}

\textbf{Řešení}:
\begin{enumerate}
\item \textbf{Ověření exaktnosti}:
\[
\frac{\partial M}{\partial y} = 2y + 1, \quad \frac{\partial N}{\partial x} = 2x - 1 \quad \text{NEJSOU STEJNÉ}
\]

\item \textbf{Test pro $\mu(r^2)$ kde $r^2 = x^2 + y^2$}:
Výpočtem zjistíme, že vhodný integrační faktor je $\mu = \frac{1}{x^2 + y^2}$.

\item \textbf{Násobení a řešení}:
\[
\left(1 + \frac{y}{x^2 + y^2}\right)dx + \left(1 - \frac{x}{x^2 + y^2}\right)dy = 0
\]
Řešením dostaneme:
\[
x + y + \arctan\left(\frac{y}{x}\right) = C
\end{enumerate}
\end{example}

\vspace{0.8\baselineskip}

\begin{transition}
V závěrečné části pokryjeme aplikace ve fyzikálních a ekonomických modelech, numerické metody a přípravu na pokročilejší témata v diferenciální geometrii.
\end{transition}

% !TEX root = ../main.tex
\subsubsection{Aplikace v Kvantitativních Vědách - Kategorie C}
\label{subsubsec:aplikace-kategorie-c}

\paragraph*{C1: Fyzikální modely}

\begin{example}[Gravitační pole]
Odvoďte rovnice pro gravitační pole hmotného bodu a najděte ekvipotenciální plochy.
\vspace{0.3\baselineskip}

\textbf{Řešení}:
\begin{enumerate}
\item \textbf{Gravitační síla}: Pro hmotný bod $M$ v počátku je síla na jednotkovou hmotnost:
\[
\vec{F} = -\frac{GM}{r^2} \hat{r} = -\frac{GM}{(x^2 + y^2)^{3/2}}(x, y)
\]

\item \textbf{Diferenciální forma}:
\[
-\frac{GMx}{(x^2 + y^2)^{3/2}}dx - \frac{GMy}{(x^2 + y^2)^{3/2}}dy = 0
\]

\item \textbf{Ověření exaktnosti}:
\[
\frac{\partial M}{\partial y} = GMx \cdot \frac{3}{2}(x^2 + y^2)^{-5/2} \cdot 2y = \frac{3GMxy}{(x^2 + y^2)^{5/2}}
\]
\[
\frac{\partial N}{\partial x} = \frac{3GMxy}{(x^2 + y^2)^{5/2}} \quad \checkmark
\]

\item \textbf{Potenciál}:
\[
F(x, y) = \int -\frac{GMx}{(x^2 + y^2)^{3/2}}dx = \frac{GM}{\sqrt{x^2 + y^2}} + \phi(y)
\]
\[
\frac{\partial F}{\partial y} = -\frac{GMy}{(x^2 + y^2)^{3/2}} + \phi'(y) = -\frac{GMy}{(x^2 + y^2)^{3/2}} \Rightarrow \phi'(y) = 0
\]

\item \textbf{Gravitační potenciál}:
\[
F(x, y) = \frac{GM}{\sqrt{x^2 + y^2}} = \frac{GM}{r}
\]

\item \textbf{Ekvipotenciální křivky}: $r = \text{konst}$ - kružnice se středem v počátku.
\end{enumerate}
\end{example}

\vspace{0.6\baselineskip}

\begin{example}[Termodynamika - stavová rovnice ideálního plynu]
Odvoďte vztah mezi tlakem, objemem a teplotou pro adiabatický děj.
\vspace{0.3\baselineskip}

\textbf{Řešení}:
\begin{enumerate}
\item \textbf{První termodynamický zákon}: $dU = \delta Q - \delta W$
\item \textbf{Adiabatický děj}: $\delta Q = 0$, $dU = c_V dT$, $\delta W = P dV$
\item \textbf{Rovnice}: $c_V dT + P dV = 0$
\item \textbf{Stavová rovnice}: $PV = nRT \Rightarrow P = \frac{nRT}{V}$
\item \textbf{Diferenciální rovnice}:
\[
c_V dT + \frac{nRT}{V} dV = 0
\]

\item \textbf{Integrační faktor}: $\mu = \frac{1}{T}$
\[
\frac{c_V}{T} dT + \frac{nR}{V} dV = 0
\]

\item \textbf{Řešení}:
\[
c_V \ln T + nR \ln V = \text{konst} \Rightarrow TV^{\frac{nR}{c_V}} = \text{konst}
\]

\item \textbf{Poměr měrných tepel}: $\gamma = \frac{c_P}{c_V}$, $c_P - c_V = nR$
\[
TV^{\gamma-1} = \text{konst} \quad \text{nebo} \quad PV^\gamma = \text{konst}
\end{enumerate}
\end{example}

\vspace{0.8\baselineskip}

\paragraph*{C2: Ekonomické aplikace}

\begin{example}[Utility funkce a indiferenční křivky]
Mějme Cobb-Douglasovu utility funkci $U(x, y) = x^\alpha y^\beta$. Najděte indiferenční křivky.
\vspace{0.3\baselineskip}

\textbf{Řešení}:
\begin{enumerate}
\item \textbf{Totální diferenciál utility}:
\[
dU = \frac{\partial U}{\partial x}dx + \frac{\partial U}{\partial y}dy = \alpha x^{\alpha-1}y^\beta dx + \beta x^\alpha y^{\beta-1} dy
\]

\item \textbf{Podél indiferenční křivky}: $dU = 0$
\[
\alpha x^{\alpha-1}y^\beta dx + \beta x^\alpha y^{\beta-1} dy = 0
\]

\item \textbf{Diferenciální rovnice}:
\[
\alpha y dx + \beta x dy = 0
\]

\item \textbf{Ověření exaktnosti}:
\[
\frac{\partial}{\partial y}(\alpha y) = \alpha, \quad \frac{\partial}{\partial x}(\beta x) = \beta
\]
Exaktní pouze pokud $\alpha = \beta$.

\item \textbf{Integrační faktor pro $\alpha \neq \beta$}: $\mu = \frac{1}{xy}$
\[
\frac{\alpha}{x} dx + \frac{\beta}{y} dy = 0
\]

\item \textbf{Řešení}:
\[
\alpha \ln|x| + \beta \ln|y| = C \Rightarrow x^\alpha y^\beta = e^C = K
\]

\item \textbf{Interpretace}: Indiferenční křivky jsou dány $U(x, y) = \text{konst}$.
\end{enumerate}
\end{example}

\vspace{0.6\baselineskip}

\begin{example}[Edgeworthův box - směna mezi dvěma spotřebiteli]
Analyzujte rovnováhu v modelu čisté směny.
\vspace{0.3\baselineskip}

\textbf{Řešení}:
\begin{enumerate}
\item \textbf{Spotřebitel A}: Utility $U_A(x_A, y_A) = x_A^\alpha y_A^{1-\alpha}$
\item \textbf{Spotřebitel B}: Utility $U_B(x_B, y_B) = x_B^\beta y_B^{1-\beta}$
\item \textbf{Prostředkové omezení}: $x_A + x_B = \bar{x}$, $y_A + y_B = \bar{y}$
\item \textbf{Podmínka Pareto optimality}:
\[
\frac{\partial U_A/\partial x_A}{\partial U_A/\partial y_A} = \frac{\partial U_B/\partial x_B}{\partial U_B/\partial y_B}
\]
\[
\frac{\alpha y_A}{(1-\alpha)x_A} = \frac{\beta y_B}{(1-\beta)x_B}
\]

\item \textbf{Diferenciální forma}:
\[
[\alpha(1-\beta)y_A x_B - \beta(1-\alpha)x_A y_B] = 0
\]
S využitím $x_B = \bar{x} - x_A$, $y_B = \bar{y} - y_A$ dostaneme exaktní rovnici.
\end{enumerate}
\end{example}

\vspace{0.8\baselineskip}

\subsubsection{Pokročilé Techniky - Kategorie D}
\label{subsubsec:pokrocile-techniky}

\paragraph*{D1: Vícerozměrné integrační faktory}

\begin{method}[Soustava integračních faktorů]
Pro rovnici $M dx + N dy = 0$ může existovat více integračních faktorů. Pokud $\mu_1$ a $\mu_2$ jsou integrační faktory, pak:
\[
\frac{\mu_1}{\mu_2} = \Phi(F(x, y))
\]
kde $F(x, y)$ je potenciál.
\end{method}

\vspace{0.4\baselineskip}

\begin{proof}
Nechť $\mu_1 M dx + \mu_1 N dy = dF_1$ a $\mu_2 M dx + \mu_2 N dy = dF_2$. Pak:
\[
\frac{\mu_1}{\mu_2} = \frac{dF_1}{dF_2} = \Phi(F_1) \quad \text{nebo} \quad \frac{\mu_1}{\mu_2} = \Psi(F_2)
\]
\end{proof}

\vspace{0.6\baselineskip}

\begin{example}[Více integračních faktorů]
Pro rovnici $y dx - x dy = 0$ najděte všechny integrační faktory tvaru $\mu = x^m y^n$.
\vspace{0.3\baselineskip}

\textbf{Řešení}:
\begin{enumerate}
\item \textbf{Původní rovnice}: $M = y$, $N = -x$
\item \textbf{Podmínka exaktnosti pro $\mu M dx + \mu N dy$}:
\[
\frac{\partial (\mu y)}{\partial y} = \frac{\partial (-\mu x)}{\partial x}
\]
\[
\mu + y\frac{\partial \mu}{\partial y} = -\mu - x\frac{\partial \mu}{\partial x}
\]

\item \textbf{Pro $\mu = x^m y^n$}:
\[
\frac{\partial \mu}{\partial x} = m x^{m-1} y^n, \quad \frac{\partial \mu}{\partial y} = n x^m y^{n-1}
\]
\[
x^m y^n + y \cdot n x^m y^{n-1} = -x^m y^n - x \cdot m x^{m-1} y^n
\]
\[
x^m y^n + n x^m y^n = -x^m y^n - m x^m y^n
\]
\[
(1 + n) = -(1 + m) \Rightarrow m + n = -2
\]

\item \textbf{Obecné řešení}: $\mu = x^m y^{-2-m}$ pro libovolné $m \in \mathbb{R}$
\end{enumerate}
\end{example}

\vspace{0.8\baselineskip}

\paragraph*{D2: Numerická analýza a verifikace}

\begin{method}[Numerický výpočet potenciálu]
Pro komplikované rovnice můžeme potenciál počítat numericky:
\[
F(x, y) \approx \int_{\gamma} M dx + N dy
\]
kde $\gamma$ je vhodná integrační cesta.
\end{method}

\vspace{0.6\baselineskip}

\begin{example}[Numerická verifikace exaktnosti]
Ověřte exaktnost rovnice $(x^2 + \sin y)dx + (x\cos y + e^y)dy = 0$ numericky.
\vspace{0.3\baselineskip}

\textbf{Řešení}:
\begin{enumerate}
\item \textbf{Analytická podmínka}:
\[
\frac{\partial M}{\partial y} = \cos y, \quad \frac{\partial N}{\partial x} = \cos y \quad \checkmark
\]

\item \textbf{Numerická verifikace}:
\begin{verbatim}
import numpy as np
from scipy import integrate

def M(x, y): return x**2 + np.sin(y)
def N(x, y): return x*np.cos(y) + np.exp(y)

# Test na mřížce
x_vals = np.linspace(0.1, 2, 10)
y_vals = np.linspace(0.1, 2, 10)

for x in x_vals:
    for y in y_vals:
        dM_dy = (M(x, y+1e-6) - M(x, y))/1e-6
        dN_dx = (N(x+1e-6, y) - N(x, y))/1e-6
        assert abs(dM_dy - dN_dx) < 1e-8
\end{verbatim}

\item \textbf{Numerický výpočet potenciálu}:
\[
F(x, y) = \int_0^x M(t, 0)dt + \int_0^y N(x, s)ds
\]
\end{enumerate}
\end{example}

\vspace{0.8\baselineskip}

\subsubsection{Shrnutí a Expertní Metodologie}
\label{subsubsec:shrnutí-metodologie}

\begin{method}[Decision tree pro řešení rovnic 1. řádu]
\label{met:decision-tree}
\begin{enumerate}
\item \textbf{Krok 1: Test exaktnosti}
\[
\text{Vypočítej } \frac{\partial M}{\partial y} - \frac{\partial N}{\partial x}
\]
\begin{itemize}
\item Pokud $= 0$ → přejdi na Krok 2A
\item Pokud $\neq 0$ → přejdi na Krok 2B
\end{itemize}

\item \textbf{Krok 2A: Řešení exaktní rovnice}
\begin{itemize}
\item Integruj $M$ podle $x$: $F(x, y) = \int M dx + \phi(y)$
\item Derivuj podle $y$: $\frac{\partial F}{\partial y} = \frac{\partial}{\partial y}(\int M dx) + \phi'(y)$
\item Porovnej s $N$: $\phi'(y) = N - \frac{\partial}{\partial y}(\int M dx)$
\item Integruj $\phi(y)$
\end{itemize}

\item \textbf{Krok 2B: Hledání integračního faktoru}
\begin{enumerate}
\item \textbf{Test $\mu(x)$}: $\frac{\frac{\partial M}{\partial y} - \frac{\partial N}{\partial x}}{N}$ závisí pouze na $x$?
\item \textbf{Test $\mu(y)$}: $\frac{\frac{\partial N}{\partial x} - \frac{\partial M}{\partial y}}{M}$ závisí pouze na $y$?
\item \textbf{Test $\mu(xy)$}: $\frac{\frac{\partial M}{\partial y} - \frac{\partial N}{\partial x}}{Ny - Mx}$ závisí pouze na $xy$?
\item \textbf{Test $\mu(x/y)$}: $\frac{\frac{\partial M}{\partial y} - \frac{\partial N}{\partial x}}{\frac{N}{y} + \frac{Mx}{y^2}}$ závisí pouze na $x/y$?
\item \textbf{Obecný případ}: Řeš rovnici $M\frac{\partial \mu}{\partial y} - N\frac{\partial \mu}{\partial x} + \mu(\frac{\partial M}{\partial y} - \frac{\partial N}{\partial x}) = 0$
\end{enumerate}

\item \textbf{Krok 3: Analýza řešení}
\begin{itemize}
\item Urči definiční obor
\item Identifikuj singulární body
\item Analyzuj asymptotické chování
\item Ověř řešení derivací
\end{itemize}
\end{enumerate}
\end{method}

\vspace{0.8\baselineskip}

\begin{table}[h]
\centering
\caption{Přehled metod pro exaktní rovnice}
\label{tab:prehled-metod}
\begin{tabular}{p{0.2\textwidth}p{0.35\textwidth}p{0.35\textwidth}}
\toprule
\textbf{Typ rovnice} & \textbf{Metoda řešení} & \textbf{Kontrolní podmínka} \\
\midrule
Přímá exaktní & Přímá integrace & $\frac{\partial M}{\partial y} = \frac{\partial N}{\partial x}$ \\
$\mu = \mu(x)$ & $\mu(x) = \exp\left(\int \frac{M_y - N_x}{N} dx\right)$ & $\frac{M_y - N_x}{N}$ závisí pouze na $x$ \\
$\mu = \mu(y)$ & $\mu(y) = \exp\left(\int \frac{N_x - M_y}{M} dy\right)$ & $\frac{N_x - M_y}{M}$ závisí pouze na $y$ \\
$\mu = \mu(xy)$ & Řeš $\frac{d\mu}{dz} = \frac{M_y - N_x}{Ny - Mx} \mu$ & $\frac{M_y - N_x}{Ny - Mx}$ závisí pouze na $xy$ \\
Obecný případ & Řeš $M\mu_y - N\mu_x + (M_y - N_x)\mu = 0$ & - \\
\bottomrule
\end{tabular}
\end{table}

\vspace{0.8\baselineskip}

\subsubsection{Příprava na Pokročilejší Témata}
\label{subsubsec:priprava-pokrocila-temata}

\begin{transition}[Spojitost s parciálními diferenciálními rovnicemi]
Teorie exaktních rovnice přímo navazuje na řešení parciálních diferenciálních rovnic 1. řádu. Rovnice:
\[
M(x, y) \frac{\partial F}{\partial x} + N(x, y) \frac{\partial F}{\partial y} = 0
\]
je řešitelná metodou charakteristik, která zobecňuje metodu integračních faktorů.
\end{transition}

\vspace{0.6\baselineskip}

\begin{transition}[Diferenciální geometrie a diferenciální formy]
V diferenciální geometrii se exaktní rovnice zapisují jako:
\[
\omega = M dx + N dy = 0
\]
kde $\omega$ je 1-forma. Podmínka exaktnosti $d\omega = 0$ odpovídá uzavřenosti diferenciální formy. Věta o nezávislosti na cestě je speciálním případem Stokesovy věty.
\end{transition}

\vspace{0.6\baselineskip}

\begin{transition}[Příprava na systémy rovnic]
Metody pro exaktní rovnice se zobecňují pro systémy:
\[
M_1 dx + M_2 dy + M_3 dz = 0
\]
s podmínkami uzavřenosti:
\[
\frac{\partial M_1}{\partial y} = \frac{\partial M_2}{\partial x}, \quad
\frac{\partial M_1}{\partial z} = \frac{\partial M_3}{\partial x}, \quad
\frac{\partial M_2}{\partial z} = \frac{\partial M_3}{\partial y}
\]
\end{transition}

\vspace{0.8\baselineskip}

\begin{conclusion}
Exaktní diferenciální rovnice představují fundamentální nástroj pro modelování konzervativních systémů v přírodních, technických i ekonomických vědách. Zvládnutí jejich teorie a řešitelských metod poskytuje:
\begin{itemize}
\item \textbf{Hluboké porozumění} geometrické struktuce diferenciálních rovnic
\item \textbf{Efektivní nástroje} pro řešení široké třídy fyzikálních a ekonomických modelů
\item \textbf{Pevný základ} pro studium pokročilejších témat v parciálních diferenciálních rovnicích a diferenciální geometrii
\item \textbf{Systematický přístup} k analýze a verifikaci řešení
\end{itemize}

Metody prezentované v této kapitole tvoří esenciální součást instrumentáře každého kvantitativního experta.
\end{conclusion}

% !TEX root = ../main.tex
\subsubsection{Závěr: Exaktní Rovnice - Syntéza Metod}
\label{subsec:zaver-exaktni}

\subsubsection{Klíčové Poznatky a Metodologické Shrnutí}
\label{subsubsec:klicove-poznatky-exaktni}

\begin{summary}[Kompletní metodologie exaktních rovnic]
\label{sum:kompletni-metodologie}
Po prostudování této kapitoly by měl kvantitativní expert ovládat:

\begin{enumerate}
\item \textbf{Teoretický fundament}:
\begin{itemize}
\item Definice exaktní rovnice a geometrická interpretace
\item Nutná a postačující podmínka exaktnosti
\item Věta o nezávislosti na cestě integrace
\end{itemize}

\item \textbf{Řešitelské metody}:
\begin{itemize}
\item Přímá integrační metoda s křížovou verifikací
\item Alternativní integrace podle různých proměnných
\item Systematické hledání integračních faktorů
\end{itemize}

\item \textbf{Klasifikační schopnosti}:
\begin{itemize}
\item Rozpoznání typu $M(x,y)$ a $N(x,y)$ (polynomiální, racionální, goniometrické, exponenciální)
\item Identifikace vhodného typu integračního faktoru
\item Analýza singularit a definičních oborů
\end{itemize}

\item \textbf{Aplikační kompetence}:
\begin{itemize}
\item Modelování konzervativních fyzikálních systémů
\item Analýza ekonomických modelů s potenciálem
\item Numerická verifikace řešení
\end{itemize}
\end{enumerate}
\end{summary}

\vspace{0.8\baselineskip}

\begin{method}[Rozhodovací strom pro exaktní rovnice - EXPERTNÍ VERZE]
\label{met:rozhodovaci-strom-expert}
\begin{enumerate}
\item \textbf{KROK 0: Příprava}
\begin{itemize}
\item Normalizace rovnice do tvaru $M dx + N dy = 0$
\item Kontrola spojitosti $M$, $N$ a jejich parciálních derivací
\item Identifikace definičního oboru a singularit
\end{itemize}

\item \textbf{KROK 1: Test exaktnosti}
\[
\Delta = \frac{\partial M}{\partial y} - \frac{\partial N}{\partial x}
\]
\begin{itemize}
\item \textbf{ANO} ($\Delta = 0$): → KROK 2A
\item \textbf{NE} ($\Delta \neq 0$): → KROK 2B
\end{itemize}

\item \textbf{KROK 2A: Řešení exaktní rovnice}
\begin{itemize}
\item Volba integrační strategie (podle $x$ nebo $y$)
\item Integrace a určení "integrační funkce"
\item Křížová verifikace řešení
\item Analýza řešení: singularita, asymptotické chování
\end{itemize}

\item \textbf{KROK 2B: Hledání integračního faktoru}
\begin{enumerate}
\item \textbf{Test $\mu(x)$}: $\frac{\Delta}{N}$ závisí pouze na $x$?
\item \textbf{Test $\mu(y)$}: $\frac{-\Delta}{M}$ závisí pouze na $y$?
\item \textbf{Test $\mu(xy)$}: $\frac{\Delta}{Ny - Mx}$ závisí pouze na $xy$?
\item \textbf{Test $\mu(x/y)$}: $\frac{\Delta}{\frac{N}{y} + \frac{Mx}{y^2}}$ závisí pouze na $x/y$?
\item \textbf{Obecný případ}: Řešení PDE pro $\mu(x,y)$
\end{enumerate}

\item \textbf{KROK 3: Validace řešení}
\begin{itemize}
\item Dosazení do původní rovnice
\item Kontinuita a diferencovatelnost v definičním oboru
\item Analýza singularit a hraničního chování
\end{itemize}
\end{enumerate}
\end{method}

\vspace{0.8\baselineskip}

\begin{table}[h]
\centering
\caption{Srovnání metod pro různé typy funkcí $M(x,y)$, $N(x,y)$}
\label{tab:srovnani-metod-exaktni}
\begin{tabular}{p{0.25\textwidth}p{0.3\textwidth}p{0.35\textwidth}}
\toprule
\textbf{Typ funkcí} & \textbf{Doporučená metoda} & \textbf{Kritické body} \\
\midrule
\textbf{Polynomiální} & Přímá integrace & Kontrola stupňů polynomů, identifikace singularit \\
\textbf{Racionální} & Integrační faktory $\mu(x)$, $\mu(y)$ & Analýza nulových bodů jmenovatelů \\
\textbf{Goniometrické} & Substituce před integrací & Periodicita, definiční obor \\
\textbf{Exponenciální} & Přímá integrace & Růstové chování, asymptotiky \\
\textbf{Smíšené typy} & Systematické testování $\mu$ & Volba optimální strategie \\
\bottomrule
\end{tabular}
\end{table}

\vspace{0.8\baselineskip}

\subsubsection{Příprava na Pokročilejší Témata}
\label{subsubsec:priprava-pokrocila-exaktni}

\begin{transition}[Spojitost s parciálními diferenciálními rovnicemi]
Teorie exaktních rovnic tvoří fundament pro:
\[
M(x,y) \frac{\partial F}{\partial x} + N(x,y) \frac{\partial F}{\partial y} = 0
\]
která se řeší metodou charakteristik. Exaktní rovnice představují speciální případ, kdy charakteristiky jsou ekvipotenciální křivky.
\end{transition}

\vspace{0.6\baselineskip}

\begin{transition}[Diferenciální formy a variabilní počet]
V pokročilejší matematice se exaktní rovnice zapisují jako:
\[
\omega = M dx + N dy = 0
\]
kde $\omega$ je diferenciální 1-forma. Podmínka $d\omega = 0$ odpovídá uzavřenosti formy a zobecňuje se ve Stokesově větě.
\end{transition}

\vspace{0.6\baselineskip}

\begin{conclusion}[Exaktní rovnice v kvantitativních vědách]
Exaktní rovnice poskytují mocný nástroj pro:
\begin{itemize}
\item \textbf{Modelování konzervativních systémů} v mechanice a elektrodynamice
\item \textbf{Analýzu potenciálů} v ekonomických a finančních modelech  
\item \textbf{Geometrickou interpretaci} řešení pomocí ekvipotenciálních křivek
\item \textbf{Numerickou verifikaci} pomocí nezávislosti na integrační cestě
\end{itemize}
Zvládnutí této problematiky tvoří pevný základ pro přechod k nelineárním rovnicím v Level 2.
\end{conclusion}

\subsection{Závěr Level 1: Základní ODE 1. Řádu - Kompletní Syntéza}
\label{sec:zaver-level1}

\subsubsection{Systematický Přehled Metod Řešení}
\label{subsec:systematicky-prehled}

\begin{overview}[Kompletní metodologie ODE 1. řádu]
\label{over:kompletni-metodologie-level1}
Po absolvování Level 1 disponuje kvantitativní expert následujícími kompetencemi:

\begin{table}[h]
\centering
\caption{Srovnání všech metod ODE 1. řádu}
\label{tab:srovnani-vsech-metod}
\begin{tabular}{p{0.2\textwidth}p{0.25\textwidth}p{0.2\textwidth}p{0.25\textwidth}}
\toprule
\textbf{Typ rovnice} & \textbf{Standardní tvar} & \textbf{Klíčová metoda} & \textbf{Kontrolní podmínka} \\
\midrule
\textbf{Separabilní} & $y' = f(x)g(y)$ & Separace proměnných & Rozdělitelnost na $f(x)$ a $g(y)$ \\
\textbf{Lineární} & $y' + p(x)y = q(x)$ & Integrační faktor & Linearita v $y$ a $y'$ \\
\textbf{Homogenní} & $y' = f(y/x)$ & Substituce $v = y/x$ & Homogenita stupně 0 \\
\textbf{Exaktní} & $M dx + N dy = 0$ & Hledání potenciálu & $\frac{\partial M}{\partial y} = \frac{\partial N}{\partial x}$ \\
\bottomrule
\end{tabular}
\end{table}
\end{overview}

\vspace{0.8\baselineskip}

\begin{method}[Univerzální rozhodovací algoritmus pro ODE 1. řádu]
\label{met:univerzalni-algoritmus-level1}
\begin{enumerate}
\item \textbf{KROK 1: Identifikace typu}
\begin{itemize}
\item Test separability: Lze psát jako $y' = f(x)g(y)$?
\item Test linearity: Je tvaru $y' + p(x)y = q(x)$?
\item Test homogenity: Platí $f(tx, ty) = f(x, y)$?
\item Test exaktnosti: Platí $M_y = N_x$?
\end{itemize}

\item \textbf{KROK 2: Aplikace specifické metody}
\begin{itemize}
\item \textbf{Separabilní}: $\int \frac{dy}{g(y)} = \int f(x) dx$
\item \textbf{Lineární}: $\mu(x) = e^{\int p(x)dx}$, pak $y = \frac{1}{\mu}\int \mu q dx$
\item \textbf{Homogenní}: $v = y/x$, řeš $v' = \frac{f(v)-v}{x}$
\item \textbf{Exaktní}: Najdi $F(x,y)$ tak, že $F_x = M$, $F_y = N$
\end{itemize}

\item \textbf{KROK 3: Transformace mezi typy}
\begin{itemize}
\item Některé rovnice lze řešit více způsoby
\item Homogenní rovnice mohou být exaktní
\item Lineární rovnice mohou mít separabilní homogenní část
\end{itemize}

\item \textbf{KROK 4: Validace a interpretace}
\begin{itemize}
\item Dosazení do původní rovnice
\item Analýza definičního oboru
\item Fyzikální/ekonomická interpretace
\end{itemize}
\end{enumerate}
\end{method}

\vspace{0.8\baselineskip}

\subsubsection{Aplikační Syntéza v Kvantitativních Vědách}
\label{subsec:aplikacni-syntéza}

\begin{application}[Komparativní analýza růstových modelů]
\label{app:komparativni-rustove-modely}
\begin{table}[h]
\centering
\caption{Srovnání růstových modelů řešitelných metodami Level 1}
\label{tab:srovnani-rustovych-modelu}
\begin{tabular}{p{0.25\textwidth}p{0.25\textwidth}p{0.2\textwidth}p{0.2\textwidth}}
\toprule
\textbf{Model} & \textbf{Rovnice} & \textbf{Typ} & \textbf{Řešení} \\
\midrule
Exponenciální růst & $\frac{dP}{dt} = rP$ & Separabilní & $P(t) = P_0 e^{rt}$ \\
Logistický růst & $\frac{dP}{dt} = rP(1-\frac{P}{L})$ & Separabilní & $P(t) = \frac{L}{1+Ae^{-rt}}$ \\
Gompertzův růst & $\frac{dP}{dt} = rP\ln(\frac{L}{P})$ & Separabilní & $P(t) = L e^{\ln(P_0/L)e^{-rt}}$ \\
Lineární s forcing & $\frac{dP}{dt} = aP + b$ & Lineární & $P(t) = Ce^{at} - \frac{b}{a}$ \\
\bottomrule
\end{tabular}
\end{table}
\end{application}

\vspace{0.6\baselineskip}

\begin{application}[Finanční modely - spojité procesy]
\label{app:financni-modely-level1}
\begin{itemize}
\item \textbf{Spojité úročení}: $\frac{dK}{dt} = rK$ (separabilní)
\item \textbf{Dluhová dynamika}: $\frac{dD}{dt} = rD - A$ (lineární)  
\item \textbf{Optimalizace spotřeby}: $\frac{dC}{dt} = f(C, W)$ (různé typy)
\item \textbf{Tržní difuze}: $\frac{dS}{dt} = kS(M-S)$ (logistický růst)
\end{itemize}
\end{application}

\vspace{0.6\baselineskip}

\begin{application}[Fyzikální systémy - konzervativní modely]
\label{app:fyzikalni-systemy-level1}
\begin{itemize}
\item \textbf{Radioaktivní rozpad}: $\frac{dN}{dt} = -\lambda N$ (separabilní)
\item \textbf{Newtonův zákon ochlazování}: $\frac{dT}{dt} = k(T - T_{env})$ (lineární)
\item \textbf{Konzervativní silová pole}: $F_x dx + F_y dy = 0$ (exaktní)
\item \textbf{Kapacitní vybíjení}: $\frac{dQ}{dt} = -\frac{Q}{RC}$ (separabilní)
\end{itemize}
\end{application}

\vspace{0.8\baselineskip}

\subsubsection{Přechod k Level 2: Nelineární Rovnice 1. Řádu}
\label{subsec:prechod-level2}

\begin{transition}[Od lineárních k nelineárním systémům]
\label{trans:linearni-nelinearni}
Level 1 poskytuje kompletní nástroje pro \textbf{lineární a kvazilineární} systémy. Level 2 rozšiřuje tuto schopnost na:

\begin{itemize}
\item \textbf{Bernoulliho rovnice}: $y' + p(x)y = q(x)y^n$ - nelineární zobecnění lineárních rovnic
\item \textbf{Riccatiho rovnice}: $y' = p(x)y^2 + q(x)y + r(x)$ - kvadratická nelinearita
\item \textbf{Clairautovy rovnice}: $y = xy' + f(y')$ - singulární řešení a obálky
\item \textbf{Lagrangeovy rovnice}: $y = xf(y') + g(y')$ - parametrické řešení
\item \textbf{Abelovy rovnice}: $y' = f(x)y^3 + g(x)y^2 + h(x)y$ - kubická nelinearita
\end{itemize}
\end{transition}

\vspace{0.6\baselineskip}

\begin{transition}[Rozšíření aplikačního spektra]
\label{trans:rozsireni-aplikaci}
Zatímco Level 1 pokrývá základní růstové a dynamické modely, Level 2 umožní modelování:

\begin{itemize}
\item \textbf{Nelineární oscilace} v ekonomických cyklech
\item \textbf{Saturační efekty} v tržní penetraci
\item \textbf{Competitivní dynamiku} ve víceagentových systémech
\item \textbf{Singulární chování} v kritických bodech systémů
\end{itemize}
\end{transition}

\vspace{0.6\baselineskip}

\begin{conclusion}[Level 1 - Kvantitativní Fundament]
\label{con:level1-kvantitativni-fundament}
Level 1 představuje \textbf{nezbytný fundament} pro každého kvantitativního experta:

\begin{itemize}
\item \textbf{Teoretická hloubka}: Důkladné pochopení čtyř základních typů ODE 1. řádu
\item \textbf{Praktické dovednosti}: Systematická metodologie identifikace a řešení
\item \textbf{Aplikační kompetence}: Modelování reálných systémů v ekonomii, financích a přírodních vědách
\item \textbf{Příprava na pokročilá témata}: Solidní základ pro nelineární rovnice a systémy vyšších řádů
\end{itemize}

\textbf{Zvládnutí Level 1 garantuje schopnost analyzovat a řešit převážnou většinu základních dynamických modelů v kvantitativních vědách.}
\end{conclusion}

\vspace{0.8\baselineskip}

\begin{remark}[Certifikační úroveň]
\label{rem:certifikacni-uroven}
Úspěšné zvládnutí Level 1 odpovídá úrovni \textbf{Junior Quantitative Analyst} a je předpokladem pro přechod k pokročilejším tématům v Level 2 a vyšších.
\end{remark}

\vspace{0.8\baselineskip}

