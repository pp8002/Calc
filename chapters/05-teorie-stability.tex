% !TEX root = ../main.tex
\section{Teorie stability}
\label{sec:teorie-stability}

\blocktitle{Cíl kapitoly}
Tato kapitola představuje systematický přístup k analýze stability řešení diferenciálních rovnic. 
Zavedeme základní typy stability, metodu linearizace a přímou Ljapunovovu metodu jako nástroj pro nelineární systémy. 
Ukážeme také stabilitu lineárních systémů, fázové portréty a pokročilé koncepty jako Ljapunovovy exponenty.

\spc

\subsection{Základní pojmy stability}
\label{sec:zakladni-pojmy-stability}

\begin{definition}[Stabilita podle Ljapunova]
\label{def:stabilita-ljapunov}
Pro řešení $y^*(t)$ systému $y' = f(t,y)$ platí:
\begin{itemize}
\item \emph{stabilní}, pokud malé perturbace v čase $t_0$ zůstávají malé i pro $t \geq t_0$,
\item \emph{asymptoticky stabilní}, pokud je stabilní a zároveň $\|y(t)-y^*(t)\|\to 0$,
\item \emph{exponenciálně stabilní}, pokud konvergence probíhá rychlostí $Me^{-\alpha(t-t_0)}$,
\item \emph{nestabilní}, pokud podmínky stability neplatí.
\end{itemize}
\end{definition}

\begin{definition}[Autonomní systém a stacionární body]
\label{def:autonomni-system}
Systém $y'=f(y)$ je autonomní, pokud $f$ nezávisí na $t$. Bod $y^*\in\R^n$ je stacionární, když $f(y^*)=0$.
\end{definition}

\begin{remark}[Redukce na stabilitu nulového řešení]
\label{rem:stabilita-nuloveho-reseni}
Stabilitu $y^*$ můžeme převést substitucí $z=y-y^*$ na stabilitu nulového řešení.
\end{remark}

\spc

\subsection{Linearizace a stabilita}
\label{sec:linearizace-stabilita}

\begin{theorem}[Ljapunovova věta o linearizaci]
\label{vet:ljapunov-linearizace}
Uvažujme $y'=f(y)$, $f\in C^1$, $f(0)=0$, $A=Df(0)$. Pak:
\begin{romanenum}
\item Pokud $\Re\lambda<0$ pro všechna vlastní čísla $\lambda(A)$, nulové řešení je asymptoticky stabilní.
\item Pokud $\exists\lambda$ s $\Re\lambda>0$, nulové řešení je nestabilní.
\end{romanenum}
\end{theorem}

\begin{example}[Stabilita kyvadla]
\label{ex:stabilita-kyvadla}
Tlumené kyvadlo $\theta''+c\theta'+\tfrac{g}{L}\sin\theta=0$ má:
\begin{itemize}
\item $\theta=0$: stabilní při $c>0$,
\item $\theta=\pi$: nestabilní.
\end{itemize}
\end{example}

\begin{remark}[Kritický případ]
\label{rem:kriticky-pripad}
Pokud $\Re\lambda\le 0$ a některé $\Re\lambda=0$, linearizace nestačí, nutno použít Ljapunovovy funkce.
\end{remark}

\spc

\subsection{Přímá metoda Ljapunova}
\label{sec:prima-ljapunov}

\begin{definition}[Ljapunovova funkce]
\label{def:ljapunovova-funkce}
$V:D\to\R$ je Ljapunovova funkce pro $y'=f(y)$, pokud $V(0)=0$, $V(y)>0$ pro $y\neq 0$ a $\dot V(y)=\nabla V\cdot f(y)\le 0$.
\end{definition}

\begin{theorem}[Ljapunovova věta o stabilitě]
\label{vet:ljapunov-stabilita}
Existuje-li $V$ splňující podmínky výše, nulové řešení je stabilní. Je-li $\dot V<0$ pro $y\neq0$, je asymptoticky stabilní.
\end{theorem}

\begin{example}[Konstrukce Ljapunovovy funkce]
\label{ex:ljapunov-example}
Pro systém $x'=-x+xy$, $y'=-y-x^2$ vezměme $V(x,y)=\tfrac12(x^2+y^2)$. Pak $\dot V=-x^2-y^2<0$, tedy nulové řešení je asymptoticky stabilní.
\end{example}

\spc

\subsection{Stabilita lineárních systémů}
\label{sec:stabilita-linearnich}

\begin{theorem}[Lineární systémy $y'=Ay$]
\label{vet:linearni-stabilita}
\begin{romanenum}
\item Nulové řešení je asymptoticky stabilní $\iff \Re\lambda<0$ pro všechna $\lambda(A)$.
\item Stabilní $\iff \Re\lambda\le 0$ a vlastní čísla s $\Re\lambda=0$ nejsou defektní.
\item Nestabilní, pokud $\exists \lambda$ s $\Re\lambda>0$.
\end{romanenum}
\end{theorem}

\begin{theorem}[Ljapunovova rovnice]
\label{vet:ljapunovova-rovnice}
$y'=Ay$ je asymptoticky stabilní $\iff$ pro libovolné $Q=Q^T>0$ existuje jediné $P=P^T>0$ s $A^TP+PA=-Q$.
\end{theorem}

\spc

\subsection{Fázové portréty v rovině}
\label{sec:fazove-portrety}

\begin{definition}[Klasifikace stacionárních bodů v $\R^2$]
\label{def:klasifikace-2d}
Pro $y'=Ay$ s $\det A\neq 0$: uzel, sedlo, ohnisko, střed — podle vlastních čísel.
\end{definition}

\begin{theorem}[Bendixson–Dulac]
\label{vet:bendixson}
Pokud $\div f$ nemění znaménko a není nulová na jednoduše souvislé oblasti $D$, systém $y'=f(y)$ nemá v $D$ periodická řešení.
\end{theorem}

\spc

\subsection{Pokročilé koncepty stability}
\label{sec:pokrocile-stabilita}

\begin{definition}[Ljapunovovy exponenty]
\label{def:lyap-exponenty}
Pro trajektorii $y(t)$ definujeme
\[
\lambda_i=\limsup_{t\to\infty}\frac1t\ln\sigma_i(\Phi(t)),
\]
kde $\sigma_i$ jsou singulární hodnoty fundamentální matice $\Phi(t)$.
\end{definition}

\begin{theorem}[Interpretace]
\label{vet:lyap-exponenty}
Pokud všechny $\lambda_i<0$, trajektorie je asymptoticky stabilní. Pokud $\exists\lambda_i>0$, nastává chaos.
\end{theorem}

\spc

\subsection{Aplikace v teorii řízení}
\label{sec:aplikace-rizeni}

\begin{example}[Stabilizace inverzního kyvadla]
\label{ex:invert-pendulum}
Inverzní kyvadlo $\theta''=\tfrac gL\sin\theta+u\cos\theta$, stabilizace horní polohy zpětnou vazbou $u=-k_1\theta-k_2\theta'$.
\end{example}

\begin{example}[LQR regulátor]
\label{ex:lqr}
Pro $\dot x=Ax+Bu$ minimalizujeme $\int_0^\infty(x^TQx+u^TRu)\,dt$. Optimální $u=-R^{-1}B^TPx$, $P$ řeší Riccatiho rovnici.
\end{example}

\spc

\subsection*{Shrnutí kapitoly}
\begin{itemize}
\item Typy stability: Ljapunovova, asymptotická, exponenciální.
\item Linearizace — základní nástroj lokální analýzy.
\item Ljapunovovy funkce umožňují globální přístup.
\item Lineární systémy: stabilita plně určena spektrem $A$.
\item Fázové portréty dávají geometrickou intuici.
\item Ljapunovovy exponenty odlišují stabilní a chaotické chování.
\end{itemize}

\spc

\subsection*{Cvičení}
\begin{enumerate}
\item Analyzujte stabilitu systému $\dot x=-x+y^2$, $\dot y=-y+x^2$.
\item Najděte Ljapunovovu funkci pro $\dot x=-x-2y^2$, $\dot y=-y+xy$.
\item Použijte Ljapunovovu rovnici pro $\dot x=-x+y$, $\dot y=-x-y$.
\item Dokažte asymptotickou stabilitu $\dot x=y$, $\dot y=-x-y^3$.
\item Použijte Bendixsonovu větu pro $\dot x=x(2-x-y)$, $\dot y=y(3-2x-y)$.
\item Nakreslete fázový portrét $\dot x=y$, $\dot y=-x+x^3$.
\item Najděte kvadratickou Ljapunovovu funkci pro $\dot x=-x+xy$, $\dot y=-2y+y^2$.
\end{enumerate}
