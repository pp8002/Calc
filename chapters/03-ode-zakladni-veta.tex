% !TEX root = ../main.tex
\section{Teorie obyčejných diferenciálních rovnic I — základní věta}
\label{sec:ODE-zakladni-veta}

\blocktitle{Cíl kapitoly}
Tato kapitola představuje první fundamentální výsledek teorie obyčejných diferenciálních rovnic — Picardovu–Lindel\"ofovu větu o existenci a jednoznačnosti řešení počáteční úlohy. 
Prozkoumáme Lipschitzovu podmínku, ekvivalenci diferenciální a integrální rovnice a pomocí Banachovy věty o pevném bodě zkonstruujeme řešení. 
Dále zavedeme Gr\"onwallovo lemma jako klíčový nástroj pro analýzu vlastností řešení.

\spc

\subsection{Počáteční úloha (Cauchyova úloha)}
\label{sec:pocatecni-uloha}

\begin{definition}[Počáteční úloha]
\label{def:pocatecni-uloha}
Počáteční úlohou pro ODE prvního řádu rozumíme problém nalezení funkce $y:I\to\R$ na intervalu $I\ni x_0$ takové, že
\[
\begin{cases}
y' = f(x,y),\\
y(x_0)=y_0,
\end{cases}
\]
kde $f:D\subset\R^2\to\R$ je daná funkce a $(x_0,y_0)\in D$.
\end{definition}

\begin{definition}[Lipschitzova podmínka]
\label{def:lipschitz}
Řekneme, že $f:D\subset\R^2\to\R$ splňuje Lipschitzovu podmínku vzhledem k $y$ na $D$, jestliže existuje $L>0$ takové, že
\[
|f(x,y_1)-f(x,y_2)| \le L\,|y_1-y_2| \quad \text{pro všechna } (x,y_1),(x,y_2)\in D.
\]
Konstanta $L$ se nazývá Lipschitzova konstanta.
\end{definition}

\begin{theorem}[Ekvivalence s integrální rovnicí]
\label{vet:ekvivalence-integralni}
Nechť $f$ je spojitá na obdélníku $R=[x_0-a,x_0+a]\times[y_0-b,y_0+b]$. 
Spojitá funkce $y:I\to\R$ je řešením počáteční úlohy
\[
y'=f(x,y),\qquad y(x_0)=y_0
\]
na intervalu $I\ni x_0$ právě tehdy, když splňuje integrální rovnici
\[
y(x)=y_0+\int_{x_0}^{x} f(t,y(t))\,\dd t \quad \text{pro všechna } x\in I.
\]
\end{theorem}

\begin{proof}
($\Rightarrow$) Integrací $y'=f(x,y)$ na $[x_0,x]$ dostaneme
\[
y(x)-y(x_0)=\int_{x_0}^{x} f(t,y(t))\,\dd t,
\]
což dává hledaný tvar. 
($\Leftarrow$) Derivací integrální rovnice získáme $y'(x)=f(x,y(x))$ a dosazením $x=x_0$ také $y(x_0)=y_0$.
\end{proof}

\spc

\subsection{Picardova–Lindelöfova věta}
\label{sec:picard-lindelof}

\begin{theorem}[Picardova–Lindelöfova věta]
\label{vet:picard-lindelof}
Nechť $R=[x_0-a,x_0+a]\times[y_0-b,y_0+b]\subset\R^2$ a platí:
\begin{romanenum}
\item $f\in C(R)$,
\item $f$ splňuje na $R$ Lipschitzovu podmínku vzhledem k $y$ s konstantou $L$.
\end{romanenum}
Označme $M=\max_{(x,y)\in R}|f(x,y)|$ a $h=\min\!\big(a,\, b/M\big)$. Pak na intervalu $I=[x_0-h,x_0+h]$ existuje právě jedno řešení počáteční úlohy
\[
y'=f(x,y),\qquad y(x_0)=y_0.
\]
\end{theorem}

\begin{proof}
\textit{Krok 1 (prostor).} Uvažujme úplný prostor $X=C(I)$ se supremovou normou $\norm{\cdot}_\infty$.

\textit{Krok 2 (uzavřená množina).} Definujme
\[
Mset=\{\,y\in C(I): \norm{y-y_0}_\infty\le b\,\}.
\]
$Mset$ je uzavřená podmnožina $C(I)$, tedy úplná.

\textit{Krok 3 (Picardův operátor).} Položme
\[
(\Phi y)(x)=y_0+\int_{x_0}^{x} f(t,y(t))\,\dd t .
\]

\textit{Krok 4 ($\Phi(Mset)\subset Mset$).} Pro $y\in Mset$ a $x\in I$:
\[
|(\Phi y)(x)-y_0| \le \int_{x_0}^{x} |f(t,y(t))|\,\dd t \le M|x-x_0|\le Mh\le b.
\]

\textit{Krok 5 (kontrakce).} Pro $y_1,y_2\in Mset$:
\[
|(\Phi y_1)(x)-(\Phi y_2)(x)| \le \int_{x_0}^{x} |f(t,y_1(t))-f(t,y_2(t))|\,\dd t 
\le L \int_{x_0}^{x} |y_1(t)-y_2(t)|\,\dd t,
\]
tedy $\norm{\Phi y_1-\Phi y_2}_\infty \le Lh\,\norm{y_1-y_2}_\infty$. 
Zvolíme-li $h<1/L$, je $\Phi$ kontrakce.

\textit{Krok 6 (Banachova věta).} $\Phi$ má v $Mset$ právě jeden pevný bod $y^\ast$, což je hledané řešení.
\end{proof}

\begin{remark}[Lokální povaha]
\label{rem:lokalni}
Výsledek je lokální: řešení je zaručeno pouze na $I=[x_0-h,x_0+h]$. Prodloužení vyžaduje dodatečnou analýzu.
\end{remark}

\spc

\subsection{Gr\"onwallovo lemma a jeho varianty}
\label{sec:gronwall}

\begin{lemma}[Gr\"onwall — diferenciální tvar]
\label{lem:gronwall-diff}
Nechť $u:[a,b]\to\R$ je diferencovatelná a splňuje
\[
u'(t)\le \beta(t)\,u(t) \quad \text{pro a.\,e.\ } t\in[a,b],
\]
kde $\beta\in L^1([a,b])$ je nezáporná. Pak
\[
u(t)\le u(a)\exp\!\Big(\int_a^{t}\beta(s)\,\dd s\Big) \quad \forall t\in[a,b].
\]
\end{lemma}

\begin{proof}
Nechť $v(t)=u(t)\exp\!\big(-\int_a^{t}\beta\big)$. Pak $v'(t)=[u'(t)-\beta(t)u(t)]\exp(\cdots)\le 0$, tedy $v$ je nerostoucí a $v(t)\le v(a)=u(a)$.
\end{proof}

\begin{lemma}[Gr\"onwall — integrální tvar]
\label{lem:gronwall-int}
Nechť $u:[a,b]\to\R$ je spojitá a
\[
u(t)\le \alpha(t)+\int_a^{t}\beta(s)\,u(s)\,\dd s,
\]
kde $\alpha$ je spojitá a $\beta\in L^1([a,b])$ nezáporná. Pak
\[
u(t)\le \alpha(t)+\int_a^{t}\alpha(s)\,\beta(s)\exp\!\Big(\int_s^{t}\beta(\tau)\,\dd\tau\Big)\,\dd s.
\]
Speciálně, je-li $\alpha$ konstantní, pak $u(t)\le \alpha\,\exp\!\big(\int_a^{t}\beta\big)$.
\end{lemma}

\begin{proof}
Položme $R(t)=\int_a^{t}\beta(s)u(s)\,\dd s$. Potom $R'(t)\le \beta(t)\alpha(t)+\beta(t)R(t)$. 
Násobením integračním faktorem a integrací dostaneme tvrzení.
\end{proof}

\spc

\subsection{Důsledky a diskuse Picardovy–Lindel\"ofovy věty}
\label{sec:dusledky-picard}

\begin{theorem}[Jednoznačnost]
\label{vet:jednoznacnost}
Za předpokladů Picardovy–Lindel\"ofovy věty je řešení počáteční úlohy jednoznačné na $I$.
\end{theorem}

\begin{proof}
Nechť $y_1,y_2$ jsou řešení. Pak
\[
|y_1(x)-y_2(x)| \le \int_{x_0}^{x} |f(t,y_1(t))-f(t,y_2(t))|\,\dd t 
\le L\int_{x_0}^{x} |y_1(t)-y_2(t)|\,\dd t.
\]
Použitím Gr\"onwallova lemmatu s $\alpha=0$ dostaneme $|y_1(x)-y_2(x)|\equiv0$.
\end{proof}

\begin{example}[Porušení Lipschitzovy podmínky]
\label{ex:poruseni-lipschitz}
Uvažujme $y'=\sqrt{|y|}$, $y(0)=0$. Funkce $f(y)=\sqrt{|y|}$ není lipschitzovská v nule. Úloha má nekonečně mnoho řešení, např. $y\equiv0$ a také
\[
y(x)=
\begin{cases}
0, & x\le c,\\[2pt]
\dfrac{(x-c)^2}{4}, & x>c,
\end{cases}\qquad c\ge0.
\]
\end{example}

\begin{theorem}[Peanova věta o existenci]
\label{vet:peano}
Je-li $f$ spojitá na obdélníku $R$, pak počáteční úloha $y'=f(x,y)$, $y(x_0)=y_0$ má alespoň jedno řešení na nějakém intervalu $I\ni x_0$.
\end{theorem}

\begin{proof}
Lze použít Eulerovy lomené čáry nebo Arzel\`a–Ascoliho větu na Picardovy iterace.
\end{proof}

\begin{remark}[Picardovy iterace]
\label{rem:picard-iterace}
Důkaz konstruuje aproximace
\[
y_0(x)=y_0,\qquad y_{n+1}(x)=y_0+\int_{x_0}^{x} f(t,y_n(t))\,\dd t,
\]
které konvergují stejnoměrně k řešení $y$ na $I$.
\end{remark}

\begin{example}[Ukázka Picardových iterací]
\label{ex:picard-iterace-priklad}
Pro $y'=y$, $y(0)=1$:
\[
y_0=1,\quad y_1=1+x,\quad y_2=1+x+\tfrac{x^2}{2},\quad y_3=1+x+\tfrac{x^2}{2}+\tfrac{x^3}{6},
\]
což je Taylorův rozvoj $e^x$.
\end{example}

\spc

\subsection*{Shrnutí kapitoly}
\begin{itemize}
\item Picardova–Lindel\"ofova věta zaručuje existenci a jednoznačnost řešení při spojitosti a Lipschitzovskosti pravé strany.
\item Ekvivalence s integrální rovnicí umožňuje použít metody pevných bodů.
\item Gr\"onwallovo lemma poskytuje ostré odhady a kritéria jednoznačnosti.
\item Picardovy iterace dávají konstruktivní aproximace řešení.
\end{itemize}

\spc

\subsection*{Cvičení}
\begin{enumerate}
\item Dokažte, že $f(x,y)=x^2+y^2$ splňuje Lipschitzovu podmínku vzhledem k $y$ na $[-a,a]\times[-b,b]$ a najděte Lipschitzovu konstantu.
\item Ukažte, že $f(y)=|y|$ je na $\R$ Lipschitzovská, ale není diferencovatelná v nule.
\item Pomocí Picardových iterací najděte aproximaci řešení $y'=x+y$, $y(0)=0$ do třetího kroku.
\item Pro $y'=1+y^2$, $y(0)=0$ určete maximální interval existence.
\item Použijte Gr\"onwallovo lemma k odhadu
\[
|y_1(x)-y_2(x)| \le |y_1(x_0)-y_2(x_0)|\, e^{L|x-x_0|}.
\]
\item Najděte všechna řešení $y'=3y^{2/3}$, $y(0)=0$ a diskutujte jednoznačnost.
\item Je-li $|f(x,y)|\le M$ na $R$, ukažte, že existuje řešení na $[x_0-h,x_0+h]$ s $h=\min(a,b/M)$.
\end{enumerate}
