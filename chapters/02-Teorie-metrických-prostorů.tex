% !TEX root = ../main.tex
\section{Teorie metrických prostorů a pokročilé věty o pevném bodě}
\label{chap:metrika-pevny-bod}

\blocktitle{Cíl kapitoly}
Tato kapitola rozvíjí abstraktní teorii metrických prostorů a představuje pokročilé nástroje pro důkazy existence řešení diferenciálních rovnic. 
Kromě klasické Banachovy věty o kontrakci zavedeme topologické věty o pevném bodě a variační principy, které umožňují pracovat s širší třídou operátorů 
a jsou klíčové pro moderní teorii nelineárních diferenciálních rovnic.

\spc

\subsection{Základy teorie metrických prostorů}
\label{sec:zaklady-metrika}

\begin{definition}[Metrický prostor]
\label{def:metricky-prostor}
Metrickým prostorem rozumíme dvojici $(X,d)$, kde $X$ je množina a $d: X \times X \to \mathbb{R}$ je zobrazení splňující:
\begin{romanenum}
\item $d(x,y) \geq 0$ a $d(x,y) = 0$ právě když $x = y$,
\item $d(x,y) = d(y,x)$ (symetrie),
\item $d(x,z) \leq d(x,y) + d(y,z)$ (trojúhelníková nerovnost).
\end{romanenum}
\end{definition}

\begin{example}[Metriky na prostorech funkcí]
\label{ex:metriky-funkce}
\begin{enumerate}
\item $C([a,b])$ s metrikou $d(f,g) = \sup_{x \in [a,b]} |f(x) - g(x)|$,
\item $L^p([a,b])$ s metrikou $d(f,g) = \left(\int_a^b |f(x)-g(x)|^p dx\right)^{1/p}$,
\item $C^1([a,b])$ s metrikou $d(f,g) = \sup_{x \in [a,b]} |f(x)-g(x)| + \sup_{x \in [a,b]} |f'(x)-g'(x)|$.
\end{enumerate}
\end{example}

\begin{definition}[Úplný metrický prostor]
\label{def:uplny-prostor}
Metrický prostor $(X,d)$ se nazývá úplný, jestliže každá Cauchyovská posloupnost v $X$ konverguje k nějakému prvku $X$.
\end{definition}

\begin{theorem}[Vztah mezi úplností a kompaktností]
\label{vet:uplnost-kompaktnost}
Každá kompaktní množina v metrickém prostoru je úplná. Opačné tvrzení neplatí.
\end{theorem}

\subsection{Princip kontrahujícího zobrazení (Banachova věta)}
\label{sec:banach-veta}

\begin{definition}[Kontrahující zobrazení]
\label{def:kontrakce}
Nechť $(X,d)$ je metrický prostor. Zobrazení $T: X \to X$ se nazývá kontrahující (kontrakce), jestliže existuje konstanta $q \in [0,1)$ taková, že
\[
d(Tx, Ty) \leq q \cdot d(x,y) \quad \text{pro všechna } x,y \in X.
\]
\end{definition}

\begin{theorem}[Banachova věta o pevném bodě]
\label{vet:banach}
Nechť $(X,d)$ je úplný metrický prostor a $T: X \to X$ je kontrakce. Pak existuje právě jeden pevný bod $x^* \in X$ a pro libovolné $x_0 \in X$ posloupnost $x_{n+1} = Tx_n$ konverguje k $x^*$.
\end{theorem}

\begin{proof}
Zvolme $x_0 \in X$, $x_{n+1} = Tx_n$. Potom
\[
d(x_{n+1}, x_n) \leq q^n d(x_1,x_0).
\]
Tedy
\[
d(x_m, x_n) \leq \frac{q^n}{1-q} d(x_1,x_0), \quad m>n,
\]
a $\{x_n\}$ je Cauchyovská. Z úplnosti plyne $x_n \to x^*$. Navíc $Tx^* = x^*$ a jednoznačnost plyne z $d(x^*,y^*) \leq q d(x^*,y^*)$.
\end{proof}

\begin{corollary}[Odhad rychlosti konvergence]
\label{dusl:rychlost-konvergence}
Za předpokladů Banachovy věty platí:
\[
d(x_n, x^*) \leq \frac{q^n}{1-q} d(x_1, x_0).
\]
\end{corollary}

\subsection{Věta o implicitní funkci}
\label{sec:implicitni-funkce}

\begin{theorem}[Věta o implicitní funkci]
\label{vet:implicitni}
Nechť $F: U \subset \mathbb{R}^2 \to \mathbb{R}$ je třídy $C^1$ na otevřené množině $U$ a $(x_0, y_0) \in U$ splňuje $F(x_0,y_0)=0$, $\partial_y F(x_0,y_0)\neq 0$. Pak existuje okolí $V$ bodu $x_0$ a funkce $f: V \to \mathbb{R}$ třídy $C^1$ taková, že $f(x_0)=y_0$, $F(x,f(x))=0$ a
\[
f'(x) = -\frac{\partial_x F(x,f(x))}{\partial_y F(x,f(x))}.
\]
\end{theorem}

\begin{example}[Aplikace na diferenciální rovnice]
\label{ex:implicitni-dre}
Věta o implicitní funkci umožňuje dokázat spojitou a hladkou závislost řešení ODR na parametrech. Např. pro
\[
y' = f(x,y,\lambda), \quad y(x_0)=y_0,
\]
je řešení $y(x;\lambda)$ hladce závislé na $\lambda$.
\end{example}

\subsection{Variační principy a Ekelandův princip}
\label{sec:variacni-principy}

\begin{definition}[Spodní polospojitost]
\label{def:spodni-polospojitost}
Funkce $f: X \to \mathbb{R}\cup\{+\infty\}$ je spodně polospojitá, jestliže $f(x) \leq \liminf f(x_n)$ vždy, když $x_n \to x$.
\end{definition}

\begin{theorem}[Ekelandův princip]
\label{vet:ekeland}
Je-li $(X,d)$ úplný metrický prostor a $f: X \to \mathbb{R}\cup\{+\infty\}$ vlastní, spodně polospojitá a omezená zdola, pak pro každé $\epsilon>0$ existuje $x_\epsilon\in X$ takové, že $f(x_\epsilon)\leq f(x_0)-\epsilon d(x_0,x_\epsilon)$ a $f(x)>f(x_\epsilon)-\epsilon d(x,x_\epsilon)$ pro všechna $x\neq x_\epsilon$.
\end{theorem}

\begin{remark}
Ekelandův princip zaručuje existenci „téměř minima“ izolovaného vůči malým perturbacím.
\end{remark}

\begin{example}[Aplikace na existenční teorém]
\label{ex:ekeland-dre}
Ekelandův princip umožňuje dokazovat existenci řešení Euler–Lagrangeových rovnic v Hilbertových prostorech pomocí variační metody.
\end{example}

\subsection{Topologické věty o pevném bodě}
\label{sec:topologicke-vety}

\begin{theorem}[Brouwerova věta]
\label{vet:brouwer}
Je-li $K\subset\mathbb{R}^n$ kompaktní a konvexní, pak každé spojité $f:K\to K$ má pevný bod.
\end{theorem}

\begin{theorem}[Schauderova věta]
\label{vet:schauder}
Je-li $K\subset X$ neprázdná uzavřená konvexní množina v Banachově prostoru a $T:K\to K$ kompaktní operátor, pak $T$ má pevný bod.
\end{theorem}

\begin{theorem}[Leray–Schauderův princip]
\label{vet:leray-schauder}
Nechť $T:X\times[0,1]\to X$ je kompaktní a všechna řešení $x=T(x,\lambda)$ jsou ohraničená, pak $x=T(x,1)$ má řešení.
\end{theorem}

\begin{example}[Aplikace na nelineární ODR]
\label{ex:schauder-dre}
Pro $y''=f(x,y,y')$, $y(0)=y(1)=0$, lze integrální formulací a Schauderovou větou dokázat existenci řešení.
\end{example}

\spc

\subsection*{Shrnutí kapitoly}
\begin{itemize}
\item Banachova věta dává konstruktivní metodu existence řešení.
\item Implicitní funkce ukazuje závislost na parametrech.
\item Ekelandův princip slouží k hledání téměř minim funkcionálů.
\item Brouwerova a Schauderova věta umožňují dokazovat existenci i bez kontraktivity.
\end{itemize}

\subsection*{Cvičení}
\begin{enumerate}
\item Dokažte, že každá kontrakce je stejnoměrně spojitá.
\item Najděte pevný bod zobrazení $T(x)=\tfrac{1}{2}x$ na $\mathbb{R}$.
\item Aplikujte větu o implicitní funkci na $x^3+y^3-3xy=0$.
\item Použijte Ekelandův princip na funkcionál $J(y)=\int_0^1(y'^2+y^2)\,dx$.
\item Dokažte existenci pevného bodu operátoru $T(f)(x)=\tfrac12\int_0^1 \tfrac{f(t)}{1+x+t}\,dt$.
\end{enumerate}
