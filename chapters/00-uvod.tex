% !TEX root = ../main.tex
\section{Úvod do diferenciálních rovnic}
\label{sec:uvod-diffeq}

\blocktitle{Cíl kapitoly}
Tato kapitola uvede čtenáře do problematiky diferenciálních rovnic, vysvětlí jejich základní význam a naznačí, proč jsou tak zásadním nástrojem pro modelování reálného světa. 
Důraz je kladen na pochopení toho, že diferenciální rovnice popisují \emph{změny} — dynamiku procesů.

\spc

\subsection{Co je to diferenciální rovnice?}
\begin{definition}[Diferenciální rovnice]
Diferenciální rovnice je vztah mezi neznámou funkcí a jejími derivacemi. 
Pokud funkce závisí pouze na jedné proměnné a v rovnici vystupují pouze obyčejné derivace, hovoříme o \emph{obyčejné diferenciální rovnici} (ODE). 
Pokud funkce závisí na více proměnných a objevují se parciální derivace, jedná se o \emph{parciální diferenciální rovnici} (PDE).
\end{definition}

\begin{remark}
Zatímco algebraické rovnice hledají čísla, která splňují určitý vztah, diferenciální rovnice hledají \emph{funkce}, jejichž derivace splňují danou podmínku.
\end{remark}

\spc

\subsection{Motivační příklady z praxe}

\begin{example}[Volný pád]
Podle Newtonova druhého zákona má těleso hmotnosti $m$ padající volně k Zemi rovnici
\[
m \frac{\dd^2 y}{\dd t^2} = -mg.
\]
Tato rovnice druhého řádu má řešení tvaru
\[
y(t) = -\tfrac{1}{2} g t^2 + C_1 t + C_2,
\]
kde $C_1, C_2$ určíme z počátečních podmínek (počáteční výška, rychlost).
\end{example}

\begin{example}[Rádioaktivní rozpad]
Množství radioaktivní látky $N(t)$ splňuje rovnici
\[
\frac{\dd N}{\dd t} = -\lambda N,
\]
kde $\lambda > 0$ je rozpadová konstanta. 
Řešením je exponenciální zákon rozpadu
\[
N(t) = N_0 \e^{-\lambda t}.
\]
\end{example}

\begin{example}[Růst populace]
Nejjednodušší model růstu populace (Malthusův model) má tvar
\[
\frac{\dd P}{\dd t} = k P,
\]
který vede k exponenciálnímu růstu $P(t) = P_0 \e^{kt}$. 
Pokročilejší model (logistický) zohledňuje omezené zdroje:
\[
\frac{\dd P}{\dd t} = k P \left(1 - \frac{P}{K}\right).
\]
\end{example}

\begin{example}[Ekonomie]
Jednoduchý model cenové adaptace může mít tvar
\[
\frac{\dd p}{\dd t} = \alpha \big(D(p) - S(p)\big),
\]
kde $p(t)$ je cena, $D(p)$ poptávka, $S(p)$ nabídka a $\alpha > 0$ rychlost adaptace.
\end{example}

\spc

\subsection{Základní pojmy}

\begin{definition}[Řešení diferenciální rovnice] Funkce $y=\varphi(x)$ definovaná na intervalu $I$ je řešením diferenciální rovnice na $I$, pokud po jejím dosazení (spolu s derivacemi) získáme identitu. \end{definition}
    

\begin{definition}
    \blocktitle{Obecné a partikulární řešení}
    \begin{romanenum}
    \item \emph{Obecné řešení} obsahuje rodinu funkcí s integračními konstantami.
    \item \emph{Partikulární řešení} vzniká konkrétní volbou konstant, často určenou podmínkami.
    \end{romanenum}
    \end{definition}
    

\begin{remark}
Někdy se vyskytují i tzv. \emph{singulární řešení}, která nelze získat z obecného řešení volbou konstant (např. obálka rodiny křivek).
\end{remark}

\spc

\subsection{Historický exkurz}

\begin{figure}[h]
    \centering
    \begin{tikzpicture}[>=Latex, x=2.2cm, y=1cm]
      % Osa času
      \draw[->] (0,0) -- (5.2,0) node[right] {Čas};
    
      % Značky + roky (dole)
      \foreach \x/\label in {
        0/{17.\,st.},
        1/{18.\,st.},
        2/{19.\,st.},
        3/{20.\,st.},
        4/{21.\,st.}
      }{
        \draw (\x,0.12) -- (\x,-0.12);
        \node[below=4pt] at (\x,-0.12) {\label};
      }
    
      % Názvy (nahoře) – zarovnáno na střed, šířka pro zalomení do 2 řádků
      \node[above=6pt, align=center, text width=2.2cm] at (0, 0.12) {\small Newton\\Leibniz};
      \node[above=6pt, align=center, text width=2.2cm] at (1, 0.12) {\small Bernoulliové\\Euler};
      \node[above=6pt, align=center, text width=2.2cm] at (2, 0.12) {\small Cauchy\\Lipschitz};
      \node[above=6pt, align=center, text width=2.2cm] at (3, 0.12) {\small Poincaré\\Ljapunov};
      \node[above=6pt, align=center, text width=2.2cm] at (4, 0.12) {\small Moderní\\teorie};
    \end{tikzpicture}
    \end{figure}


    \spc

    \paragraph*{17. století — Newton a Leibniz}
    Vznik diferenciálního počtu i diferenciálních rovnic. 
    Newton řešil pohybové zákony a gravitační problémy pomocí rovnic druhého řádu. 
    Leibniz zavedl symboliku a jeho škola (Bernoulliové) rozvinula metody separace proměnných.
    
    \spc
    
    \paragraph*{18. století — Bernoulliové a Euler}
    Bernoulliové přinesli první obecné metody řešení jednoduchých ODE. 
    Euler systematizoval teorii, zavedl metodu variace konstant a ukázal význam lineárních rovnic s konstantními koeficienty.
    
    \spc
    
    \paragraph*{19. století — Cauchy a Lipschitz}
    Nastává éra rigorózní matematiky. 
    Cauchy přesně formuloval pojem řešení diferenciální rovnice a položil základy existenčních vět. 
    Lipschitz rozpracoval podmínky jednoznačnosti řešení, které dnes známe jako Lipschitzovu podmínku.
    
    \spc
    
    \paragraph*{20. století — Poincaré a Ljapunov}
    Pozornost se přesouvá od explicitních řešení k analýze chování systémů. 
    Poincaré zakládá kvalitativní teorii diferenciálních rovnic a dynamických systémů, 
    Ljapunov formuluje teorii stability rovnovážných stavů.
    
    \spc
    
    \paragraph*{21. století — Moderní teorie}
    Současný vývoj se zaměřuje na numerické metody, modelování složitých systémů, chaotické chování a aplikace v biologii, fyzice, ekonomii a strojovém učení.
    

\begin{remark}
Vývoj teorie diferenciálních rovnic prošel od konkrétních příkladů (Newton, Bernoulli, Euler) přes rigorózní existenční a jednoznačnostní věty (Cauchy, Lipschitz) až po kvalitativní teorii a stabilitu (Poincaré, Ljapunov).
\end{remark}

\spc

\subsection*{Shrnutí kapitoly}
\begin{itemize}
\item Diferenciální rovnice popisují vztah mezi funkcí a jejími derivacemi a modelují dynamické procesy.
\item Rozlišujeme ODE (jedna proměnná) a PDE (více proměnných).
\item Řešení může být obecné, partikulární či singulární.
\item Historie ukazuje posun od konkrétních řešení k abstraktní teorii.
\end{itemize}

\spc



