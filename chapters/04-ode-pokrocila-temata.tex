% !TEX root = ../main.tex
\section{Teorie obyčejných diferenciálních rovnic II — pokročilá témata}
\label{sec:ODE-pokrocila-temata}

\blocktitle{Cíl kapitoly}
Tato kapitola rozvíjí teorii ODE za rámec lokální existence a jednoznačnosti. 
Studujeme globální vlastnosti řešení, prodlužování na maximální interval existence, hladkou závislost na počátečních podmínkách a parametrech, 
a pokročilé existenční věty pro případy, kdy není splněna Lipschitzova podmínka.

\spc

\subsection{Maximální interval existence a prodloužení řešení}
\label{sec:maximalni-interval}

\begin{definition}[Maximální řešení]
\label{def:maximalni-reseni}
Řešení $y:I\to\R$ počáteční úlohy $y'=f(x,y)$, $y(x_0)=y_0$ se nazývá \emph{maximální}, jestliže neexistuje řešení $\tilde y:J\to\R$ téže úlohy takové, že $I\subsetneq J$ a $\tilde y|_I=y$.
\end{definition}

\begin{theorem}[O prodlužování řešení]
\label{vet:prodlouzeni}
Nechť $f:D\subset\R^2\to\R$ je na otevřené množině $D$ spojitá a lokálně lipschitzovská vzhledem k $y$. 
Pak pro libovolnou počáteční podmínku $(x_0,y_0)\in D$ existuje právě jedno maximální řešení $y:(\alpha,\omega)\to\R$ a platí:
\begin{romanenum}
\item buď $\omega=+\infty$, nebo $\limsup_{x\to\omega^-}|y(x)|=+\infty$, nebo $(x,y(x))$ míří k hranici $\partial D$ pro $x\to\omega^-$,
\item buď $\alpha=-\infty$, nebo $\limsup_{x\to\alpha^+}|y(x)|=+\infty$, nebo $(x,y(x))$ míří k $\partial D$ pro $x\to\alpha^+$.
\end{romanenum}
\end{theorem}

\begin{proof}
Definujme $\omega=\sup\{\,b>x_0:\exists$ řešení na $[x_0,b]\,\}$. 
Pokud $\omega<\infty$ a trajektorie zůstává v kompaktní podmnožině $D$, Picard–Lindel\"of dovolí prodloužení za $\omega$ — spor. Analogicky vlevo.
\end{proof}

\begin{example}[Konečný blow-up]
\label{ex:blow-up}
Pro $y'=y^2$, $y(0)=1$ je $y(x)=\frac{1}{1-x}$, které exploduje v čase $x=1$. Maximální interval existence je $(-\infty,1)$.
\end{example}

\begin{theorem}[A priori odhad a globální existence]
\label{vet:a-priori}
Nechť $f:\R\times\R\to\R$ je spojitá a existují funkce $\alpha,\beta\in L^1_{\mathrm{loc}}(\R)$ takové, že
\[
|f(x,y)|\le \alpha(x)|y|+\beta(x)\quad \forall x,y.
\]
Pak každé maximální řešení je globální (definované na celé $\R$).
\end{theorem}

\begin{proof}
Aplikací Gr\"onwallova lemmatu na $|y|$ dostaneme lokální ohraničení nezávislé na konci intervalu, což vylučuje blow-up v konečném čase.
\end{proof}

\spc

\subsection{Spojitá a hladká závislost na počátečních podmínkách}
\label{sec:spojita-zavislost}

\begin{theorem}[Spojitá závislost]
\label{vet:spojita-zavislost}
Nechť $f:D\subset\R^2\to\R$ je spojitá a lokálně lipschitzovská vzhledem k $y$. Označme $y(x;x_0,y_0)$ maximální řešení s $y(x_0)=y_0$. 
Pak zobrazení $(x,x_0,y_0)\mapsto y(x;x_0,y_0)$ je spojité na své definiční oblasti.
\end{theorem}

\begin{proof}
Na společném kompaktním pásu je $f$ lipschitzovská. Odhad rozdílu dvou trajektorií a Gr\"onwall dá požadovanou spojitost.
\end{proof}

\begin{theorem}[Diferenciovatelnost podle počáteční podmínky]
\label{vet:diff-zavislost}
Je-li navíc $f\in C^1(D)$, pak $(x_0,y_0)\mapsto y(x;x_0,y_0)$ je třídy $C^1$ a 
$z(x)=\partial_{y_0}y(x;x_0,y_0)$ splňuje variační rovnici
\[
z'=f_y(x,y(x))\,z,\qquad z(x_0)=1.
\]
\end{theorem}

\begin{proof}
Lze odvodit z věty o implicitní funkci aplikované na Picardův operátor.
\end{proof}

\spc

\subsection{Diferenciovatelnost řešení vzhledem k parametrům}
\label{sec:diff-parametry}

\begin{theorem}[Závislost na parametrech]
\label{vet:zavislost-parametry}
Uvažujme $y'=f(x,y,\lambda)$, $y(x_0)=y_0$, kde $\lambda\in\R^k$. 
Nechť $f,f_y,f_\lambda$ jsou spojité na $D\times\Lambda$ (s $\Lambda\subset\R^k$ otevřená). 
Pak $y(x;\lambda)$ je třídy $C^1$ v $\lambda$ a $w=\partial_\lambda y$ splňuje
\[
w' = f_y(x,y(x;\lambda),\lambda)\,w + f_\lambda(x,y(x;\lambda),\lambda),\qquad w(x_0)=0.
\]
\end{theorem}

\begin{example}[Citlivost na parametr]
\label{ex:citlivost-parametry}
Pro $y'=-\lambda y$, $y(0)=1$ je $y(x;\lambda)=e^{-\lambda x}$ a $\partial_\lambda y=-x e^{-\lambda x}$.
\end{example}

\spc

\subsection{Pokročilé existenční věty}
\label{sec:pokrocile-existenci}

\begin{theorem}[Carathéodoryho existenční věta]
\label{vet:caratheodory}
Nechť $f:[x_0,x_0+a]\times\R\to\R$ splňuje:
\begin{romanenum}
\item $f(\cdot,y)$ je měřitelná pro každé $y$,
\item $f(x,\cdot)$ je spojitá pro skoro všechna $x$,
\item existuje $m\in L^1([x_0,x_0+a])$ s $|f(x,y)|\le m(x)$ pro a.\,e.\ $x$ a všechna $y$.
\end{romanenum}
Pak $y'=f(x,y)$, $y(x_0)=y_0$ má alespoň jedno absolutně spojité řešení na $[x_0,x_0+a]$.
\end{theorem}

\begin{proof}
Picardův operátor je dobře definovaný, spojitý a kompaktní na vhodné uzavřené konvexní množině v $C([x_0,x_0+a])$. Použijeme Schauderovu větu.
\end{proof}

\begin{theorem}[Existence pomocí Schauderovy/Leray–Schauderovy metody]
\label{vet:existence-schauder}
Nechť $f:[a,b]\times\R\to\R$ je spojitá a existuje $R>0$ takové, že každé řešení $y$ úlohy $y'=\lambda f(x,y)$, $y(x_0)=y_0$, $\lambda\in[0,1]$ splňuje $\norm{y}_\infty<R$. 
Pak $y'=f(x,y)$, $y(x_0)=y_0$ má alespoň jedno řešení.
\end{theorem}

\begin{example}[Bez jednoznačnosti]
\label{ex:bez-uniqueness}
Pro $y'=2\sqrt{|y|}$, $y(0)=0$ existuje více řešení (např.\ $y\equiv0$ a $y=x^2$). Carathéodoryho věta zajišťuje existenci, Lipschitz však selhává.
\end{example}

\spc

\subsection{Variační rovnice a citlivostní analýza}
\label{sec:variacni-rovnice}

\begin{definition}[Variační rovnice]
\label{def:variacni-rovnice}
Je-li $y(x;p)$ řešení $y'=f(x,y,p)$, $y(x_0)=y_0(p)$, pak $w=\partial_p y$ splňuje
\[
w'=f_y(x,y,p)\,w + f_p(x,y,p),\qquad w(x_0)=y_0'(p).
\]
\end{definition}

\begin{theorem}[Linearizace a stabilita]
\label{vet:linearizace-stabilita}
Je-li $y^\ast$ stacionární řešení $y'=f(y)$, tj.\ $f(y^\ast)=0$, pak linearizace $w'=f'(y^\ast)w$. 
Platí-li $\Re f'(y^\ast)<0$, je $y^\ast$ lokálně asymptoticky stabilní.
\end{theorem}

\begin{example}[Logistická rovnice]
\label{ex:logisticka-citlivost}
Pro $y'=r y(1-y/K)$: body $0$ (nestabilní) a $K$ (stabilní). Variační rovnice v $K$ je $w'=-r w$.
\end{example}

\spc

\subsection{Aplikace v matematické biologii}
\label{sec:aplikace-biologie}

\begin{example}[Modely populační dynamiky]
\label{ex:populacni-modely}
\begin{itemize}
\item Malthusiánský růst: $y'=ry$ (exponenciální růst),
\item Logistický růst: $y'=ry(1-y/K)$ (omezený růst),
\item Alleeho efekt: $y'=ry(y-a)(1-y/K)$ (prahové chování).
\end{itemize}
Analýza existence, stability a citlivosti na parametry.
\end{example}

\begin{example}[Chemické reakce]
\label{ex:chemicke-reakce}
Jednoduchá reakce $A+B\to C$: 
\[
\dot a=-kab,\qquad \dot b=-kab,\qquad \dot c=kab.
\]
Studium existence globálního řešení a asymptotiky.
\end{example}

\spc

\subsection*{Shrnutí kapitoly}
\begin{itemize}
\item Maximální řešení a kritéria globální existence jsou zásadní pro dlouhodobé chování.
\item Spojitá/hladká závislost na počátečních datech a parametrech umožňuje stabilitní a citlivostní analýzu.
\item Carathéodoryho a Schauderovy přístupy pokrývají případy bez Lipschitzovy podmínky.
\item Variační rovnice dávají nástroj pro linearizaci a lokální stabilitu.
\end{itemize}

\spc

\subsection*{Cvičení}
\begin{enumerate}
\item Najděte maximální interval existence a explicitní řešení pro $y'=1+y^2$, $y(0)=0$.
\item Ukažte, že řešení $y'=-y^3$, $y(0)=1$ je globální a určete jeho asymptotiku.
\item Pro $y'=f(y)$ odvoďte rovnici pro $z(x)=\partial_{y_0}y(x;y_0)$.
\item Pro $y'=\lambda y - y^3$ analyzujte stabilitu stacionárních bodů v závislosti na $\lambda$ (bifurkace).
\item Použijte Carathéodoryho větu pro $y'=\mathrm{sign}(x)\,y$, $y(0)=1$ na $[-1,1]$.
\item Je-li $|f(x,y)|\le C(1+|y|)$, dokažte, že každé maximální řešení $y'=f(x,y)$ je globální.
\item Model predátor–kořist: $\dot x=ax-bxy$, $\dot y=-cy+dxy$. Najděte stacionární body a analyzujte jejich stabilitu linearizací.
\end{enumerate}
