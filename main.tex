% !TEX program = pdflatex
% Hlavní kořenový soubor skript
\documentclass[a4paper,12pt]{article}

% Načtení globální preambule a maker
% !TEX root = main.tex
\documentclass[a4paper,12pt]{article}

% --- jazyk a kódování ---
\usepackage[main=czech]{babel}
\usepackage[utf8]{inputenc}   % pro pdfLaTeX
\usepackage[T1]{fontenc}
\usepackage{lmodern}

% --- matematika ---
\usepackage{amsmath,amssymb,amsthm}
\allowdisplaybreaks[3] % dovol rozdělování vícerovnic přes stránky

% --- okraje ---
\usepackage{geometry}
\geometry{margin=2.5cm}

% --- čitelnost ---
\linespread{1.1}

% --- jednoduché prostředí s římským číslováním (bez potřeby enumitem) ---
\newenvironment{romanenum}%
  {\begin{enumerate}\renewcommand\labelenumi{\roman{enumi})}}%
  {\end{enumerate}}

% --- hyperref vždy; cleveref fallback (když není nainstalován) ---
\usepackage{hyperref}
\hypersetup{
  colorlinks=true,
  linkcolor=black,
  citecolor=black,
  urlcolor=black
}

\makeatletter
\@ifundefined{ver@cleveref.sty}{%
  % fallback: základní aliasy, pokud cleveref není k dispozici
  \newcommand{\cref}[1]{\ref{#1}}
  \newcommand{\Cref}[1]{\ref{#1}}
}{%
  \usepackage[capitalize,nameinlink]{cleveref}
}
\makeatother

% --- mikrotypografie ---
\usepackage{microtype}
\emergencystretch=1.5em

% --- theoremstyle: vlastní nastavení s mezerami a nadpisem na novém řádku ---
\newtheoremstyle{mytheoremstyle}%
  {12pt}{12pt}% space above/below
  {\normalfont}{}%
  {\bfseries}{}%
  {\newline}{} % head–body separator = new line

\theoremstyle{mytheoremstyle}

% --- prostředí, každé má vlastní číslování ---
\newtheorem{definition}{Definice}[section]
\newtheorem{example}{Příklad}[section]
\newtheorem{remark}{Poznámka}[section]
\newtheorem{theorem}{Věta}[section]
\newtheorem{lemma}{Lemma}[section]
\newtheorem{proposition}{Tvrzení}[section]
\newtheorem{corollary}{Důsledek}[section]

% --- důkazy ---
\renewcommand{\proofname}{Důkaz}

% --- číslování rovnic, obrázků a tabulek podle sekcí ---
\numberwithin{equation}{section}
\numberwithin{figure}{section}
\numberwithin{table}{section}

% --- vlastní příkazy ---
\providecommand{\diff}{\mathrm{d}}

% --- pomocné blokové nadpisy a jemné svislé mezery ---
\newcommand{\blocktitle}[1]{\vspace{0.6\baselineskip}\noindent\textbf{#1}\par\vspace{0.25\baselineskip}}
\newcommand{\spc}{\vspace{0.6\baselineskip}}

\usepackage{tikz}
\usetikzlibrary{arrows.meta} % volitelné, jen pro hezčí šipky
% !TEX root = main.tex

% --- NASTAVENÍ ZÁ HLAVÍ A ZÁPATÍ ---
\pagestyle{fancy}
\fancyhf{}
\fancyhead[R]{\thepage}
\fancyhead[L]{\leftmark}
\renewcommand{\headrulewidth}{0.4pt}

% --- DEFINICE BAREV PRO KVANTITATIVNÍ APLIKACE ---
\definecolor{quantblue}{RGB}{0, 82, 147}
\definecolor{quantgreen}{RGB}{0, 128, 0}
\definecolor{quantred}{RGB}{192, 0, 0}
\definecolor{quantgray}{RGB}{240, 240, 240}

% --- TCOROLORBOX PROSTŘEDÍ PRO SPECIÁLNÍ BLOKY ---
\newtcolorbox{keyinsight}[1][]{
  breakable,
  enhanced,
  colback=quantgray,
  colframe=quantblue,
  title=#1,
  fonttitle=\bfseries,
  attach boxed title to top left={yshift=-2mm, xshift=5mm},
  boxed title style={colback=quantblue, colframe=quantblue}
}

\newtcolorbox{application}[1][]{
  breakable,
  enhanced,
  colback=white,
  colframe=quantgreen,
  title=#1,
  fonttitle=\bfseries,
  attach boxed title to top left={yshift=-2mm, xshift=5mm},
  boxed title style={colback=quantgreen, colframe=quantgreen}
}

\newtcolorbox{expertnote}[1][]{
  breakable,
  enhanced,
  colback=white,
  colframe=quantred,
  title=#1,
  fonttitle=\bfseries,
  attach boxed title to top left={yshift=-2mm, xshift=5mm},
  boxed title style={colback=quantred, colframe=quantred}
}

\newtcolorbox{transition}{
  breakable,
  enhanced,
  colback=white,
  colframe=black,
  arc=0pt,
  boxrule=1pt,
  left=5mm,
  right=5mm
}

\newtcolorbox{researcharea}[1][]{
  breakable,
  enhanced,
  colback=quantgray!20,
  colframe=quantblue!80,
  title=#1,
  fonttitle=\bfseries,
  attach boxed title to top left={yshift=-2mm, xshift=5mm}
}

% --- ROADMAP BOX ---
\newcommand{\roadmap}[1]{
  \begin{tcolorbox}[
    title=Roadmap, 
    breakable, 
    enhanced, 
    colback=quantgray!30, 
    colframe=quantblue
  ] 
  #1 
  \end{tcolorbox}
}

% --- THEOREM STYLES A PROSTŘEDÍ ---
\newtheoremstyle{mytheoremstyle}%
  {12pt}{12pt}%
  {\normalfont}{}%
  {\bfseries}{}%
  {\newline}{}

\theoremstyle{mytheoremstyle}

\newtheorem{definition}{Definice}[section]
\newtheorem{example}{Příklad}[section]
\newtheorem{remark}{Poznámka}[section]
\newtheorem{theorem}{Věta}[section]
\newtheorem{lemma}{Lemma}[section]
\newtheorem{proposition}{Tvrzení}[section]
\newtheorem{corollary}{Důsledek}[section]
\newtheorem{principle}{Princip}[section]
\newtheorem{innovation}{Inovace}[section]
\newtheorem{researchareaenv}{Výzkumná Oblast}[section]
\newtheorem{assessment}{Hodnocení}[section]

% --- DŮKAZY ---
\renewcommand{\proofname}{Důkaz}

% --- ČÍSLOVÁNÍ PODLE SEKCI ---
\numberwithin{equation}{section}
\numberwithin{figure}{section}
\numberwithin{table}{section}
\numberwithin{algorithm}{section}


% --- NOVÉ SLOUPCE PRO TABULKY ---
\newcolumntype{L}{>{\raggedright\arraybackslash}p{0.2\textwidth}}
\newcolumntype{M}{>{\centering\arraybackslash}p{0.3\textwidth}}
\newcolumntype{R}{>{\raggedleft\arraybackslash}p{0.4\textwidth}}

% --- LISTINGS STYLES ---
\lstset{
  basicstyle=\ttfamily\small,
  breaklines=true,
  frame=single,
  numbers=left,
  numberstyle=\tiny\color{gray},
  showstringspaces=false,
  commentstyle=\color{quantgreen},
  keywordstyle=\color{quantblue},
  stringstyle=\color{quantred},
  backgroundcolor=\color{quantgray!20}
}

\lstdefinestyle{python}{
  language=Python,
  morekeywords={import, def, class, return, yield, lambda, with, async, await}
}

\lstdefinestyle{matlab}{
  language=Matlab,
  morekeywords={function, end, if, else, for, while}
}

% --- MATEMATICKÉ PŘÍKAZY ---
\providecommand{\diff}{\mathrm{d}}
\newcommand{\bmm}[1]{\bm{#1}}

% --- ČÍSELNÉ MNOŽINY ---
\newcommand{\RR}{\mathbb{R}}
\newcommand{\CC}{\mathbb{C}}
\newcommand{\NN}{\mathbb{N}}
\newcommand{\ZZ}{\mathbb{Z}}
\newcommand{\QQ}{\mathbb{Q}}

% --- PRAVDĚPODOBNOSTNÍ SYMBOLY ---
\newcommand{\EE}{\mathbb{E}}
\newcommand{\PP}{\mathbb{P}}
\newcommand{\Var}{\mathrm{Var}}
\newcommand{\Cov}{\mathrm{Cov}}
\newcommand{\Corr}{\mathrm{Corr}}

% --- DALŠÍ MATEMATICKÉ SYMBOLY ---
\newcommand{\supp}{\mathrm{supp}}
\newcommand{\sgn}{\mathrm{sgn}}
\newcommand{\id}{\mathrm{id}}

% --- OPERÁTORY PRO FUNKCIONÁLNÍ ANALÝZU ---
\DeclareMathOperator{\diag}{diag}
\DeclareMathOperator{\tr}{tr}
\DeclareMathOperator{\Span}{span}
\DeclareMathOperator{\grad}{grad}
\DeclareMathOperator{\curl}{curl}
\DeclareMathOperator{\divg}{div}
\DeclareMathOperator{\Hess}{Hess}
\DeclareMathOperator{\Jac}{Jac}

% --- POMOCNÉ BLOKOVÉ NADPISY ---
\newcommand{\blocktitle}[1]{\vspace{0.6\baselineskip}\noindent\textbf{#1}\par\vspace{0.25\baselineskip}}
\newcommand{\spc}{\vspace{0.6\baselineskip}}

% --- HYPERREF NASTAVENÍ ---
\hypersetup{
  colorlinks=true,
  linkcolor=quantblue,
  citecolor=quantgreen,
  urlcolor=quantred,
  pdftitle={Pokročilé Diferenciální Rovnice pro Kvantitativní Odborníky},
  pdfauthor={Quant Expert Team},
  pdfsubject={Matematika, Finanční Modelování},
  pdfkeywords={diferenciální rovnice, kvantitativní finance, matematické modelování}
}

% --- MIKROTYPOGRAFIE ---
\microtypesetup{expansion=true}
\emergencystretch=1.5em

% --- ČESKÉ DĚLENÍ SLOV ---


% ---- Titulní údaje (uprav dle potřeby) ----
\title{\textbf{Calculus 2}\\\large Skripta s důrazem na rigoróznost a příklady}

\date{\today}

\begin{document}

% ---- Titulní strana ----
\maketitle
\thispagestyle{empty}
\clearpage

% ---- Předmluva (nečíslovaná, ale v obsahu) ----
\section*{Předmluva}
\addcontentsline{toc}{section}{Předmluva}
Tato skripta vznikají s cílem poskytnout systematický a rigorózní úvod do Calculusu a diferenciálních rovnic.
Důraz je kladen na jasné definice, věty a důkazy, doplněné o řešené příklady.
Struktura je rozdělena do \emph{Levelů}, aby bylo možné progresivně zvyšovat náročnost.

\bigskip
\noindent\textbf{Jak číst:}
Každý Level obsahuje úvodní motivaci, přehled klíčových výsledků a výběr příkladů.
Doporučujeme procházet kapitoly postupně a průběžně ověřovat porozumění na cvičeních.

\clearpage

% ---- Globální obsah ----
\setcounter{secnumdepth}{3}   % číslování až do \subsubsection
\setcounter{tocdepth}{2}      % obsah zobrazuje do \subsection (změň na 3 dle chuti)
\tableofcontents
\clearpage

% ---- (Volitelně) seznam obrázků/tabulek/vět – odkomentuj, pokud používáš ----
% \listoffigures
% \clearpage
% \listoftables
% \clearpage
% \usepackage{thmtools} % v preamble, pokud chceš seznam vět
% \listoftheorems[ignoreall,show={theorem,lemma,proposition}]
% \clearpage

% ---- Hlavní text (kapitoly) ----
% Každá kapitola je samostatný soubor v 'chapters/'. Přidávej/ubírej dle potřeby.
% !TEX root = ../main.tex
\section{Úvod do diferenciálních rovnic}
\label{sec:uvod-diffeq}

\blocktitle{Cíl kapitoly}
Tato kapitola uvede čtenáře do problematiky diferenciálních rovnic, vysvětlí jejich základní význam a naznačí, proč jsou tak zásadním nástrojem pro modelování reálného světa. 
Důraz je kladen na pochopení toho, že diferenciální rovnice popisují \emph{změny} — dynamiku procesů.

\spc

\subsection{Co je to diferenciální rovnice?}
\begin{definition}[Diferenciální rovnice]
Diferenciální rovnice je vztah mezi neznámou funkcí a jejími derivacemi. 
Pokud funkce závisí pouze na jedné proměnné a v rovnici vystupují pouze obyčejné derivace, hovoříme o \emph{obyčejné diferenciální rovnici} (ODE). 
Pokud funkce závisí na více proměnných a objevují se parciální derivace, jedná se o \emph{parciální diferenciální rovnici} (PDE).
\end{definition}

\begin{remark}
Zatímco algebraické rovnice hledají čísla, která splňují určitý vztah, diferenciální rovnice hledají \emph{funkce}, jejichž derivace splňují danou podmínku.
\end{remark}

\spc

\subsection{Motivační příklady z praxe}

\begin{example}[Volný pád]
Podle Newtonova druhého zákona má těleso hmotnosti $m$ padající volně k Zemi rovnici
\[
m \frac{\dd^2 y}{\dd t^2} = -mg.
\]
Tato rovnice druhého řádu má řešení tvaru
\[
y(t) = -\tfrac{1}{2} g t^2 + C_1 t + C_2,
\]
kde $C_1, C_2$ určíme z počátečních podmínek (počáteční výška, rychlost).
\end{example}

\begin{example}[Rádioaktivní rozpad]
Množství radioaktivní látky $N(t)$ splňuje rovnici
\[
\frac{\dd N}{\dd t} = -\lambda N,
\]
kde $\lambda > 0$ je rozpadová konstanta. 
Řešením je exponenciální zákon rozpadu
\[
N(t) = N_0 \e^{-\lambda t}.
\]
\end{example}

\begin{example}[Růst populace]
Nejjednodušší model růstu populace (Malthusův model) má tvar
\[
\frac{\dd P}{\dd t} = k P,
\]
který vede k exponenciálnímu růstu $P(t) = P_0 \e^{kt}$. 
Pokročilejší model (logistický) zohledňuje omezené zdroje:
\[
\frac{\dd P}{\dd t} = k P \left(1 - \frac{P}{K}\right).
\]
\end{example}

\begin{example}[Ekonomie]
Jednoduchý model cenové adaptace může mít tvar
\[
\frac{\dd p}{\dd t} = \alpha \big(D(p) - S(p)\big),
\]
kde $p(t)$ je cena, $D(p)$ poptávka, $S(p)$ nabídka a $\alpha > 0$ rychlost adaptace.
\end{example}

\spc

\subsection{Základní pojmy}

\begin{definition}[Řešení diferenciální rovnice] Funkce $y=\varphi(x)$ definovaná na intervalu $I$ je řešením diferenciální rovnice na $I$, pokud po jejím dosazení (spolu s derivacemi) získáme identitu. \end{definition}
    

\begin{definition}
    \blocktitle{Obecné a partikulární řešení}
    \begin{romanenum}
    \item \emph{Obecné řešení} obsahuje rodinu funkcí s integračními konstantami.
    \item \emph{Partikulární řešení} vzniká konkrétní volbou konstant, často určenou podmínkami.
    \end{romanenum}
    \end{definition}
    

\begin{remark}
Někdy se vyskytují i tzv. \emph{singulární řešení}, která nelze získat z obecného řešení volbou konstant (např. obálka rodiny křivek).
\end{remark}

\spc

\subsection{Historický exkurz}

\begin{figure}[h]
    \centering
    \begin{tikzpicture}[>=Latex, x=2.2cm, y=1cm]
      % Osa času
      \draw[->] (0,0) -- (5.2,0) node[right] {Čas};
    
      % Značky + roky (dole)
      \foreach \x/\label in {
        0/{17.\,st.},
        1/{18.\,st.},
        2/{19.\,st.},
        3/{20.\,st.},
        4/{21.\,st.}
      }{
        \draw (\x,0.12) -- (\x,-0.12);
        \node[below=4pt] at (\x,-0.12) {\label};
      }
    
      % Názvy (nahoře) – zarovnáno na střed, šířka pro zalomení do 2 řádků
      \node[above=6pt, align=center, text width=2.2cm] at (0, 0.12) {\small Newton\\Leibniz};
      \node[above=6pt, align=center, text width=2.2cm] at (1, 0.12) {\small Bernoulliové\\Euler};
      \node[above=6pt, align=center, text width=2.2cm] at (2, 0.12) {\small Cauchy\\Lipschitz};
      \node[above=6pt, align=center, text width=2.2cm] at (3, 0.12) {\small Poincaré\\Ljapunov};
      \node[above=6pt, align=center, text width=2.2cm] at (4, 0.12) {\small Moderní\\teorie};
    \end{tikzpicture}
    \end{figure}


    \spc

    \paragraph*{17. století — Newton a Leibniz}
    Vznik diferenciálního počtu i diferenciálních rovnic. 
    Newton řešil pohybové zákony a gravitační problémy pomocí rovnic druhého řádu. 
    Leibniz zavedl symboliku a jeho škola (Bernoulliové) rozvinula metody separace proměnných.
    
    \spc
    
    \paragraph*{18. století — Bernoulliové a Euler}
    Bernoulliové přinesli první obecné metody řešení jednoduchých ODE. 
    Euler systematizoval teorii, zavedl metodu variace konstant a ukázal význam lineárních rovnic s konstantními koeficienty.
    
    \spc
    
    \paragraph*{19. století — Cauchy a Lipschitz}
    Nastává éra rigorózní matematiky. 
    Cauchy přesně formuloval pojem řešení diferenciální rovnice a položil základy existenčních vět. 
    Lipschitz rozpracoval podmínky jednoznačnosti řešení, které dnes známe jako Lipschitzovu podmínku.
    
    \spc
    
    \paragraph*{20. století — Poincaré a Ljapunov}
    Pozornost se přesouvá od explicitních řešení k analýze chování systémů. 
    Poincaré zakládá kvalitativní teorii diferenciálních rovnic a dynamických systémů, 
    Ljapunov formuluje teorii stability rovnovážných stavů.
    
    \spc
    
    \paragraph*{21. století — Moderní teorie}
    Současný vývoj se zaměřuje na numerické metody, modelování složitých systémů, chaotické chování a aplikace v biologii, fyzice, ekonomii a strojovém učení.
    

\begin{remark}
Vývoj teorie diferenciálních rovnic prošel od konkrétních příkladů (Newton, Bernoulli, Euler) přes rigorózní existenční a jednoznačnostní věty (Cauchy, Lipschitz) až po kvalitativní teorii a stabilitu (Poincaré, Ljapunov).
\end{remark}

\spc

\subsection*{Shrnutí kapitoly}
\begin{itemize}
\item Diferenciální rovnice popisují vztah mezi funkcí a jejími derivacemi a modelují dynamické procesy.
\item Rozlišujeme ODE (jedna proměnná) a PDE (více proměnných).
\item Řešení může být obecné, partikulární či singulární.
\item Historie ukazuje posun od konkrétních řešení k abstraktní teorii.
\end{itemize}

\spc



 
% !TEX root = ../main.tex
\section{Pokročilá funkcionální analýza ve prostorech funkcí}
\label{sec:funkcionalni-analyza}

\blocktitle{Cíl kapitoly}
V této kapitole vybudujeme abstraktní rámec normovaných a Hilbertových prostorů, který je základním jazykem moderní teorie diferenciálních rovnic. Zavedeme klíčové koncepty jako duální prostory, kompaktnost a spektrální teorii, které jsou nezbytné pro rigorózní formulaci a řešení diferenciálních rovnic v nekonečně-dimenzionálních prostorech.

\spc

\subsection{Normované a Banachovy prostory — rekapitulace a prohloubení}
\label{subsec:banach-prostory}

\begin{definition}[Normovaný prostor]
\label{def:normovany-prostor}
Normovaným prostorem rozumíme dvojici $(X,\norm{\cdot})$, kde $X$ je vektorový prostor nad $\R$ nebo $\C$ a $\norm{\cdot}:X\to\R$ splňuje:
\begin{romanenum}
\item $\norm{x}\ge 0$ pro všechna $x\in X$ a $\norm{x}=0$ právě když $x=0$,
\item $\norm{\alpha x}=|\alpha|\,\norm{x}$ pro všechna $\alpha\in\R$ (resp.\ $\C$) a $x\in X$,
\item $\norm{x+y}\le \norm{x}+\norm{y}$ pro všechna $x,y\in X$ (trojúhelníková nerovnost).
\end{romanenum}
\end{definition}

\begin{example}[Klasické příklady normovaných prostorů]
\label{ex:normovane-prostory}
\begin{enumerate}
\item $\R^n$ s eukleidovskou normou $\norm{\vb{x}}_2=\big(\sum_{i=1}^n x_i^2\big)^{1/2}$.
\item Prostor spojitých funkcí $C([a,b])$ s normou $\norm{f}_\infty=\sup_{x\in[a,b]}|f(x)|$.
\item Prostor $p$-sumovatelných posloupností $\ell^p=\{(x_n)_{n\ge1}:\sum_{n=1}^\infty |x_n|^p<\infty\}$ s normou $\norm{x}_p=\big(\sum_{n=1}^\infty |x_n|^p\big)^{1/p}$.
\end{enumerate}
\end{example}

\begin{definition}[Banachův prostor]
\label{def:banach-prostor}
Normovaný prostor $(X,\norm{\cdot})$ se nazývá Banachův prostor, je-li úplný v metrice indukované normou, tj.\ každá Cauchyovská posloupnost v $X$ konverguje k nějakému prvku $X$.
\end{definition}

\begin{theorem}[Úplnost $C([a,b])$]
\label{vet:uplnost-C}
Prostor $(C([a,b]),\norm{\cdot}_\infty)$ je Banachův prostor.
\end{theorem}

\begin{proof}
Nechť $\{f_n\}$ je Cauchyovská posloupnost v $C([a,b])$. Pak pro každé $\varepsilon>0$ existuje $N$ takové, že pro všechna $m,n\ge N$ platí
\[
\norm{f_m-f_n}_\infty=\sup_{x\in[a,b]}|f_m(x)-f_n(x)|<\varepsilon.
\]
Pro pevné $x$ je tedy $\{f_n(x)\}$ Cauchyovská v $\R$, a tudíž konverguje k $f(x)$. Z nerovnosti přechodem $m\to\infty$ dostaneme $|f(x)-f_n(x)|\le\varepsilon$ pro $n\ge N$ a všechna $x$, tedy $f_n\to f$ stejnoměrně. Stejnoměrná limita spojitých funkcí je spojitá, takže $f\in C([a,b])$.
\end{proof}

\spc

\subsection{Lineární operátory a funkcionály}
\label{subsec:linearni-operatory}

\begin{definition}[Lineární operátor]
\label{def:linearni-operator}
Nechť $X,Y$ jsou normované prostory. Zobrazení $T:X\to Y$ se nazývá lineární operátor, jestliže pro všechna $x,y\in X$ a $\alpha\in\R$ (resp.\ $\C$) platí:
\begin{romanenum}
\item $T(x+y)=T(x)+T(y)$,
\item $T(\alpha x)=\alpha T(x)$.
\end{romanenum}
\end{definition}

\begin{definition}[Omezený operátor]
\label{def:omezeny-operator}
Lineární operátor $T:X\to Y$ je \emph{omezený}, existuje-li konstanta $C>0$ taková, že
\[
\norm{Tx}_Y\le C\,\norm{x}_X \quad \text{pro všechna } x\in X.
\]
\end{definition}

\begin{theorem}[Ekivalence omezenosti a spojitosti]
\label{vet:omezenost-spojitost}
Pro lineární operátor $T:X\to Y$ jsou ekvivalentní:
\begin{romanenum}
\item $T$ je spojitý,
\item $T$ je spojitý v nule,
\item $T$ je omezený.
\end{romanenum}
\end{theorem}

\begin{definition}[Norma operátoru]
\label{def:norma-operatoru}
Je-li $T:X\to Y$ omezený lineární operátor, definujeme jeho normu
\[
\norm{T}=\sup\{\norm{Tx}_Y:\ \norm{x}_X\le1\}
       =\sup\Big\{\frac{\norm{Tx}_Y}{\norm{x}_X}: x\ne0\Big\}.
\]
\end{definition}

\begin{definition}[Prostor omezených operátorů]
\label{def:prostor-operatoru}
Množina všech omezených lineárních operátorů z $X$ do $Y$ tvoří vektorový prostor $\mathcal{L}(X,Y)$. Je-li $Y$ Banachův, pak i $\mathcal{L}(X,Y)$ je Banachův s operátorovou normou.
\end{definition}

\begin{theorem}[Princip stejnoměrné omezenosti]
\label{vet:princip-stejnomerne-omezenosti}
Nechť $X$ je Banachův prostor a $Y$ normovaný prostor. Je-li $\{T_\alpha\}_{\alpha\in A}$ rodina omezených lineárních operátorů $X\to Y$ taková, že pro každé $x\in X$ je množina $\{\norm{T_\alpha x}_Y:\alpha\in A\}$ omezená, pak je omezená i množina $\{\norm{T_\alpha}:\alpha\in A\}$.
\end{theorem}

\spc

\subsection{Hilbertovy prostory}
\label{subsec:hilbert-prostory}

\begin{definition}[Skalární součin]
\label{def:skalarni-soucin}
Nechť $H$ je vektorový prostor nad $\C$. Zobrazení $\ip{\cdot}{\cdot}:H\times H\to\C$ je skalární součin, jestliže:
\begin{romanenum}
\item $\ip{x}{x}\ge 0$ a $\ip{x}{x}=0$ právě když $x=0$,
\item $\ip{x}{y}=\overline{\ip{y}{x}}$,
\item $\ip{\alpha x+\beta y}{z}=\alpha\,\ip{x}{z}+\beta\,\ip{y}{z}$.
\end{romanenum}
\end{definition}

\begin{definition}[Hilbertův prostor]
\label{def:hilbert-prostor}
Hilbertův prostor je prostor se skalárním součinem, který je úplný v metrice indukované normou $\norm{x}=\sqrt{\ip{x}{x}}$.
\end{definition}

\begin{example}[Klasické Hilbertovy prostory]
\label{ex:hilbert-prostory}
\begin{enumerate}
\item $\R^n$ se standardním skalárním součinem $\ip{\vb{x}}{\vb{y}}=\sum_{i=1}^n x_i y_i$.
\item Prostor $\ell^2$ se součinem $\ip{x}{y}=\sum_{n=1}^\infty x_n\overline{y_n}$.
\item Prostor $L^2([a,b])$ se součinem $\ip{f}{g}=\int_a^b f(x)\overline{g(x)}\,\dd x$.
\end{enumerate}
\end{example}

\begin{theorem}[Cauchy–Schwarzova nerovnost]
\label{vet:cauchy-schwarz}
Pro každé $x,y$ v prostoru se skalárním součinem platí
\[
|\ip{x}{y}|\le \norm{x}\,\norm{y},
\]
přičemž rovnost nastává právě tehdy, když $x$ a $y$ jsou lineárně závislé.
\end{theorem}

\begin{theorem}[Věta o projekci]
\label{vet:o-projekci}
Nechť $H$ je Hilbertův prostor a $M\subset H$ jeho neprázdná uzavřená konvexní množina. Pak pro každé $x\in H$ existuje právě jeden bod $y\in M$ (projekce) takový, že
\[
\norm{x-y}=\inf_{z\in M}\norm{x-z}.
\]
Navíc platí charakterizace: $y\in M$ a $\ip{x-y}{z-y}\le 0$ pro všechna $z\in M$.
\end{theorem}

\begin{proof}
Existence: Nechť $d=\inf_{z\in M}\norm{x-z}$ a zvolme posloupnost $\{y_n\}\subset M$ s $\norm{x-y_n}\to d$. Rovnoběžníkový zákon dává Cauchyovskost $\{y_n\}$; uzavřenost a úplnost implikuje existenci limity $y\in M$ s $\norm{x-y}=d$. Jednoznačnost plyne opět z rovnoběžníkového zákona: kdyby $y,y'$ byly dvě projekce, vyšla by $\norm{y-y'}=0$.
\end{proof}

\spc

\subsection{Duální prostory a věta Hahn–Banach}
\label{subsec:dualni-prostory}

\begin{definition}[Duální prostor]
\label{def:dualni-prostor}
Nechť $X$ je normovaný prostor. Duální prostor $X^*$ je prostor všech spojitých lineárních funkcionálů na $X$, tj.\ $X^*=\mathcal{L}(X,\R)$ (resp.\ $\mathcal{L}(X,\C)$).
\end{definition}

\begin{theorem}[Hahn–Banachova věta (reálná verze)]
\label{vet:hahn-banach}
Nechť $X$ je reálný vektorový prostor, $p:X\to\R$ sublineární funkcionál, $Y\subset X$ podprostor a $f:Y\to\R$ lineární funkcionál s $f(y)\le p(y)$ pro všechna $y\in Y$. Pak existuje lineární funkcionál $F:X\to\R$ takový, že $F|_Y=f$ a $F(x)\le p(x)$ pro všechna $x\in X$.
\end{theorem}

\begin{corollary}[Rozšíření funkcionálů]
\label{dusl:rozsireni-funkcionalu}
Nechť $X$ je normovaný prostor, $Y\subset X$ podprostor a $f\in Y^*$. Pak existuje $F\in X^*$ takové, že $F|_Y=f$ a $\norm{F}_{X^*}=\norm{f}_{Y^*}$.
\end{corollary}

\begin{theorem}[Reprezentace duálu k $L^p$]
\label{vet:reprezentace-Lp}
Pro $1<p<\infty$ je $(L^p(\mu))^*$ izometricky izomorfní s $L^q(\mu)$, kde $\frac1p+\frac1q=1$. Konkrétně, každý $\phi\in (L^p(\mu))^*$ má tvar
\[
\phi(f)=\int f\,g\,\dd\mu
\]
pro nějaké $g\in L^q(\mu)$ a $\norm{\phi}=\norm{g}_q$.
\end{theorem}

\spc

\subsection{Kompaktnost v prostorech funkcí}
\label{subsec:kompaktnost}

\begin{definition}[Kompaktní množina]
\label{def:kompaktni-mnozina}
Množina $K$ v metrickém prostoru je kompaktní, jestliže z každé posloupnosti bodů v $K$ lze vybrat podposloupnost konvergentní k bodu z $K$.
\end{definition}

\begin{definition}[Stejná spojitost]
\label{def:stejna-spojitost}
Množina funkcí $\mathcal{F}\subset C([a,b])$ je stejně spojitá, jestliže pro každé $\varepsilon>0$ existuje $\delta>0$ takové, že pro všechny $f\in\mathcal{F}$ a $x,y\in[a,b]$ s $|x-y|<\delta$ platí $|f(x)-f(y)|<\varepsilon$.
\end{definition}

\begin{theorem}[Arzelà–Ascoli]
\label{vet:arzela-ascoli}
Množina $\mathcal{F}\subset C([a,b])$ je relativně kompaktní právě tehdy, když:
\begin{romanenum}
\item je stejně omezená: existuje $M>0$ takové, že $\norm{f}_\infty\le M$ pro všechna $f\in\mathcal{F}$,
\item je stejně spojitá.
\end{romanenum}
\end{theorem}

\begin{proof}
($\Leftarrow$) Z libovolné posloupnosti v $\mathcal{F}$ vyberte diagonální metodou podposloupnost konvergující na husté množině; stejná spojitost dá stejnoměrnou konvergenci.\\
($\Rightarrow$) Sporem: není-li některá z podmínek splněna, sestrojí se posloupnost bez stejnoměrně konvergentní podposloupnosti.
\end{proof}

\begin{example}[Aplikace Arzelà–Ascoliho věty]
\label{ex:aplikace-arzela}
Uvažujme Picardův operátor $(\Phi y)(x)=y_0+\int_{x_0}^x f(t,y(t))\,\dd t$ na množině
\[
M=\{\,y\in C([x_0-h,x_0+h]) : \norm{y-y_0}_\infty\le b\,\}.
\]
Je-li $f$ omezená na příslušném obdélníku, je $\Phi(M)$ stejně omezená; je-li $f$ spojitá, je $\Phi(M)$ stejně spojitá. Tedy $\Phi(M)$ je relativně kompaktní, což je klíčové pro použití Schauderovy věty o pevném bodě.
\end{example}

\begin{theorem}[Kompaktní operátory]
\label{vet:kompaktni-operatory}
Lineární operátor $T:X\to Y$ je \emph{kompaktní}, jestliže obraz jednotkové koule v $X$ má kompaktní uzávěr v $Y$. Každý kompaktní operátor je omezený.
\end{theorem}

\spc

\subsection*{Shrnutí kapitoly}
\begin{itemize}
\item Banachovy a Hilbertovy prostory poskytují přirozený rámec pro studium diferenciálních rovnic.
\item Lineární operátory a funkcionály převádějí diferenciální rovnice na operátorové rovnice.
\item Hahn–Banachova věta garantuje bohatství funkcionálů (variace, dualita).
\item Arzelà–Ascoli a kompaktní operátory jsou klíčové pro existenční výsledky v nelineárních problémech.
\end{itemize}

\spc

\subsection*{Cvičení}
\begin{enumerate}
\item Dokažte, že prostor $C^1([a,b])$ s normou $\norm{f}=\norm{f}_\infty+\norm{f'}_\infty$ je Banachův.
\item Ukažte, že derivace $D:C^1([0,1])\to C([0,1])$, $Df=f'$, je lineární, ale není omezená, pokud $C^1([0,1])$ vezmeme s normou $\norm{\cdot}_\infty$.
\item V Hilbertově prostoru dokažte rovnoběžníkový zákon:
\[
\norm{x+y}^2+\norm{x-y}^2=2\big(\norm{x}^2+\norm{y}^2\big).
\]
\item Pomocí Hahn–Banacha dokažte: pro každý nenulový $x\in X$ existuje $f\in X^*$ s $\norm{f}=1$ a $f(x)=\norm{x}$.
\item Ukažte, že množina $\{f\in C([0,1]) : |f(x)|\le1,\ |f(x)-f(y)|\le |x-y|\ \forall x,y\}$ je kompaktní v $C([0,1])$.
\item Je-li $k\in C([0,1]\times[0,1])$, definujte $(Tf)(x)=\int_0^1 k(x,y)f(y)\,\dd y$. Dokažte, že $T:C([0,1])\to C([0,1])$ je kompaktní lineární operátor.
\end{enumerate}

% !TEX root = ../main.tex
\section{Teorie metrických prostorů a pokročilé věty o pevném bodě}
\label{chap:metrika-pevny-bod}

\blocktitle{Cíl kapitoly}
Tato kapitola rozvíjí abstraktní teorii metrických prostorů a představuje pokročilé nástroje pro důkazy existence řešení diferenciálních rovnic. 
Kromě klasické Banachovy věty o kontrakci zavedeme topologické věty o pevném bodě a variační principy, které umožňují pracovat s širší třídou operátorů 
a jsou klíčové pro moderní teorii nelineárních diferenciálních rovnic.

\spc

\subsection{Základy teorie metrických prostorů}
\label{sec:zaklady-metrika}

\begin{definition}[Metrický prostor]
\label{def:metricky-prostor}
Metrickým prostorem rozumíme dvojici $(X,d)$, kde $X$ je množina a $d: X \times X \to \mathbb{R}$ je zobrazení splňující:
\begin{romanenum}
\item $d(x,y) \geq 0$ a $d(x,y) = 0$ právě když $x = y$,
\item $d(x,y) = d(y,x)$ (symetrie),
\item $d(x,z) \leq d(x,y) + d(y,z)$ (trojúhelníková nerovnost).
\end{romanenum}
\end{definition}

\begin{example}[Metriky na prostorech funkcí]
\label{ex:metriky-funkce}
\begin{enumerate}
\item $C([a,b])$ s metrikou $d(f,g) = \sup_{x \in [a,b]} |f(x) - g(x)|$,
\item $L^p([a,b])$ s metrikou $d(f,g) = \left(\int_a^b |f(x)-g(x)|^p dx\right)^{1/p}$,
\item $C^1([a,b])$ s metrikou $d(f,g) = \sup_{x \in [a,b]} |f(x)-g(x)| + \sup_{x \in [a,b]} |f'(x)-g'(x)|$.
\end{enumerate}
\end{example}

\begin{definition}[Úplný metrický prostor]
\label{def:uplny-prostor}
Metrický prostor $(X,d)$ se nazývá úplný, jestliže každá Cauchyovská posloupnost v $X$ konverguje k nějakému prvku $X$.
\end{definition}

\begin{theorem}[Vztah mezi úplností a kompaktností]
\label{vet:uplnost-kompaktnost}
Každá kompaktní množina v metrickém prostoru je úplná. Opačné tvrzení neplatí.
\end{theorem}

\subsection{Princip kontrahujícího zobrazení (Banachova věta)}
\label{sec:banach-veta}

\begin{definition}[Kontrahující zobrazení]
\label{def:kontrakce}
Nechť $(X,d)$ je metrický prostor. Zobrazení $T: X \to X$ se nazývá kontrahující (kontrakce), jestliže existuje konstanta $q \in [0,1)$ taková, že
\[
d(Tx, Ty) \leq q \cdot d(x,y) \quad \text{pro všechna } x,y \in X.
\]
\end{definition}

\begin{theorem}[Banachova věta o pevném bodě]
\label{vet:banach}
Nechť $(X,d)$ je úplný metrický prostor a $T: X \to X$ je kontrakce. Pak existuje právě jeden pevný bod $x^* \in X$ a pro libovolné $x_0 \in X$ posloupnost $x_{n+1} = Tx_n$ konverguje k $x^*$.
\end{theorem}

\begin{proof}
Zvolme $x_0 \in X$, $x_{n+1} = Tx_n$. Potom
\[
d(x_{n+1}, x_n) \leq q^n d(x_1,x_0).
\]
Tedy
\[
d(x_m, x_n) \leq \frac{q^n}{1-q} d(x_1,x_0), \quad m>n,
\]
a $\{x_n\}$ je Cauchyovská. Z úplnosti plyne $x_n \to x^*$. Navíc $Tx^* = x^*$ a jednoznačnost plyne z $d(x^*,y^*) \leq q d(x^*,y^*)$.
\end{proof}

\begin{corollary}[Odhad rychlosti konvergence]
\label{dusl:rychlost-konvergence}
Za předpokladů Banachovy věty platí:
\[
d(x_n, x^*) \leq \frac{q^n}{1-q} d(x_1, x_0).
\]
\end{corollary}

\subsection{Věta o implicitní funkci}
\label{sec:implicitni-funkce}

\begin{theorem}[Věta o implicitní funkci]
\label{vet:implicitni}
Nechť $F: U \subset \mathbb{R}^2 \to \mathbb{R}$ je třídy $C^1$ na otevřené množině $U$ a $(x_0, y_0) \in U$ splňuje $F(x_0,y_0)=0$, $\partial_y F(x_0,y_0)\neq 0$. Pak existuje okolí $V$ bodu $x_0$ a funkce $f: V \to \mathbb{R}$ třídy $C^1$ taková, že $f(x_0)=y_0$, $F(x,f(x))=0$ a
\[
f'(x) = -\frac{\partial_x F(x,f(x))}{\partial_y F(x,f(x))}.
\]
\end{theorem}

\begin{example}[Aplikace na diferenciální rovnice]
\label{ex:implicitni-dre}
Věta o implicitní funkci umožňuje dokázat spojitou a hladkou závislost řešení ODR na parametrech. Např. pro
\[
y' = f(x,y,\lambda), \quad y(x_0)=y_0,
\]
je řešení $y(x;\lambda)$ hladce závislé na $\lambda$.
\end{example}

\subsection{Variační principy a Ekelandův princip}
\label{sec:variacni-principy}

\begin{definition}[Spodní polospojitost]
\label{def:spodni-polospojitost}
Funkce $f: X \to \mathbb{R}\cup\{+\infty\}$ je spodně polospojitá, jestliže $f(x) \leq \liminf f(x_n)$ vždy, když $x_n \to x$.
\end{definition}

\begin{theorem}[Ekelandův princip]
\label{vet:ekeland}
Je-li $(X,d)$ úplný metrický prostor a $f: X \to \mathbb{R}\cup\{+\infty\}$ vlastní, spodně polospojitá a omezená zdola, pak pro každé $\epsilon>0$ existuje $x_\epsilon\in X$ takové, že $f(x_\epsilon)\leq f(x_0)-\epsilon d(x_0,x_\epsilon)$ a $f(x)>f(x_\epsilon)-\epsilon d(x,x_\epsilon)$ pro všechna $x\neq x_\epsilon$.
\end{theorem}

\begin{remark}
Ekelandův princip zaručuje existenci „téměř minima“ izolovaného vůči malým perturbacím.
\end{remark}

\begin{example}[Aplikace na existenční teorém]
\label{ex:ekeland-dre}
Ekelandův princip umožňuje dokazovat existenci řešení Euler–Lagrangeových rovnic v Hilbertových prostorech pomocí variační metody.
\end{example}

\subsection{Topologické věty o pevném bodě}
\label{sec:topologicke-vety}

\begin{theorem}[Brouwerova věta]
\label{vet:brouwer}
Je-li $K\subset\mathbb{R}^n$ kompaktní a konvexní, pak každé spojité $f:K\to K$ má pevný bod.
\end{theorem}

\begin{theorem}[Schauderova věta]
\label{vet:schauder}
Je-li $K\subset X$ neprázdná uzavřená konvexní množina v Banachově prostoru a $T:K\to K$ kompaktní operátor, pak $T$ má pevný bod.
\end{theorem}

\begin{theorem}[Leray–Schauderův princip]
\label{vet:leray-schauder}
Nechť $T:X\times[0,1]\to X$ je kompaktní a všechna řešení $x=T(x,\lambda)$ jsou ohraničená, pak $x=T(x,1)$ má řešení.
\end{theorem}

\begin{example}[Aplikace na nelineární ODR]
\label{ex:schauder-dre}
Pro $y''=f(x,y,y')$, $y(0)=y(1)=0$, lze integrální formulací a Schauderovou větou dokázat existenci řešení.
\end{example}

\spc

\subsection*{Shrnutí kapitoly}
\begin{itemize}
\item Banachova věta dává konstruktivní metodu existence řešení.
\item Implicitní funkce ukazuje závislost na parametrech.
\item Ekelandův princip slouží k hledání téměř minim funkcionálů.
\item Brouwerova a Schauderova věta umožňují dokazovat existenci i bez kontraktivity.
\end{itemize}

\subsection*{Cvičení}
\begin{enumerate}
\item Dokažte, že každá kontrakce je stejnoměrně spojitá.
\item Najděte pevný bod zobrazení $T(x)=\tfrac{1}{2}x$ na $\mathbb{R}$.
\item Aplikujte větu o implicitní funkci na $x^3+y^3-3xy=0$.
\item Použijte Ekelandův princip na funkcionál $J(y)=\int_0^1(y'^2+y^2)\,dx$.
\item Dokažte existenci pevného bodu operátoru $T(f)(x)=\tfrac12\int_0^1 \tfrac{f(t)}{1+x+t}\,dt$.
\end{enumerate}




% --- (Volitelně) další sekce Calculusu ---
% \input{chapters/20-limity}
% \input{chapters/21-derivace}
% \input{chapters/22-integraly}
% \input{chapters/23-mv-calculus}

% ---- Přílohy (pokud chceš) ----
% \appendix
% \section{Dodatek A: Notace a zkratky}
% \label{app:notace}
% Stručný souhrn používané notace. (Můžeš také držet v macros.tex)

\end{document}
