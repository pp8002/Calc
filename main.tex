% !TEX program = pdflatex
% Hlavní kořenový soubor skript
\documentclass[a4paper,12pt]{article}

% Načtení globální preambule a maker
% !TEX root = main.tex
\documentclass[a4paper,12pt]{article}

% --- jazyk a kódování ---
\usepackage[main=czech]{babel}
\usepackage[utf8]{inputenc}   % pro pdfLaTeX
\usepackage[T1]{fontenc}
\usepackage{lmodern}

% --- matematika ---
\usepackage{amsmath,amssymb,amsthm}
\allowdisplaybreaks[3] % dovol rozdělování vícerovnic přes stránky

% --- okraje ---
\usepackage{geometry}
\geometry{margin=2.5cm}

% --- čitelnost ---
\linespread{1.1}

% --- jednoduché prostředí s římským číslováním (bez potřeby enumitem) ---
\newenvironment{romanenum}%
  {\begin{enumerate}\renewcommand\labelenumi{\roman{enumi})}}%
  {\end{enumerate}}

% --- hyperref vždy; cleveref fallback (když není nainstalován) ---
\usepackage{hyperref}
\hypersetup{
  colorlinks=true,
  linkcolor=black,
  citecolor=black,
  urlcolor=black
}

\makeatletter
\@ifundefined{ver@cleveref.sty}{%
  % fallback: základní aliasy, pokud cleveref není k dispozici
  \newcommand{\cref}[1]{\ref{#1}}
  \newcommand{\Cref}[1]{\ref{#1}}
}{%
  \usepackage[capitalize,nameinlink]{cleveref}
}
\makeatother

% --- mikrotypografie ---
\usepackage{microtype}
\emergencystretch=1.5em

% --- theoremstyle: vlastní nastavení s mezerami a nadpisem na novém řádku ---
\newtheoremstyle{mytheoremstyle}%
  {12pt}{12pt}% space above/below
  {\normalfont}{}%
  {\bfseries}{}%
  {\newline}{} % head–body separator = new line

\theoremstyle{mytheoremstyle}

% --- prostředí, každé má vlastní číslování ---
\newtheorem{definition}{Definice}[section]
\newtheorem{example}{Příklad}[section]
\newtheorem{remark}{Poznámka}[section]
\newtheorem{theorem}{Věta}[section]
\newtheorem{lemma}{Lemma}[section]
\newtheorem{proposition}{Tvrzení}[section]
\newtheorem{corollary}{Důsledek}[section]

% --- důkazy ---
\renewcommand{\proofname}{Důkaz}

% --- číslování rovnic, obrázků a tabulek podle sekcí ---
\numberwithin{equation}{section}
\numberwithin{figure}{section}
\numberwithin{table}{section}

% --- vlastní příkazy ---
\providecommand{\diff}{\mathrm{d}}

% --- pomocné blokové nadpisy a jemné svislé mezery ---
\newcommand{\blocktitle}[1]{\vspace{0.6\baselineskip}\noindent\textbf{#1}\par\vspace{0.25\baselineskip}}
\newcommand{\spc}{\vspace{0.6\baselineskip}}

\usepackage{tikz}
\usetikzlibrary{arrows.meta} % volitelné, jen pro hezčí šipky
% !TEX root = main.tex

% --- NASTAVENÍ ZÁ HLAVÍ A ZÁPATÍ ---
\pagestyle{fancy}
\fancyhf{}
\fancyhead[R]{\thepage}
\fancyhead[L]{\leftmark}
\renewcommand{\headrulewidth}{0.4pt}

% --- DEFINICE BAREV PRO KVANTITATIVNÍ APLIKACE ---
\definecolor{quantblue}{RGB}{0, 82, 147}
\definecolor{quantgreen}{RGB}{0, 128, 0}
\definecolor{quantred}{RGB}{192, 0, 0}
\definecolor{quantgray}{RGB}{240, 240, 240}

% --- TCOROLORBOX PROSTŘEDÍ PRO SPECIÁLNÍ BLOKY ---
\newtcolorbox{keyinsight}[1][]{
  breakable,
  enhanced,
  colback=quantgray,
  colframe=quantblue,
  title=#1,
  fonttitle=\bfseries,
  attach boxed title to top left={yshift=-2mm, xshift=5mm},
  boxed title style={colback=quantblue, colframe=quantblue}
}

\newtcolorbox{application}[1][]{
  breakable,
  enhanced,
  colback=white,
  colframe=quantgreen,
  title=#1,
  fonttitle=\bfseries,
  attach boxed title to top left={yshift=-2mm, xshift=5mm},
  boxed title style={colback=quantgreen, colframe=quantgreen}
}

\newtcolorbox{expertnote}[1][]{
  breakable,
  enhanced,
  colback=white,
  colframe=quantred,
  title=#1,
  fonttitle=\bfseries,
  attach boxed title to top left={yshift=-2mm, xshift=5mm},
  boxed title style={colback=quantred, colframe=quantred}
}

\newtcolorbox{transition}{
  breakable,
  enhanced,
  colback=white,
  colframe=black,
  arc=0pt,
  boxrule=1pt,
  left=5mm,
  right=5mm
}

\newtcolorbox{researcharea}[1][]{
  breakable,
  enhanced,
  colback=quantgray!20,
  colframe=quantblue!80,
  title=#1,
  fonttitle=\bfseries,
  attach boxed title to top left={yshift=-2mm, xshift=5mm}
}

% --- ROADMAP BOX ---
\newcommand{\roadmap}[1]{
  \begin{tcolorbox}[
    title=Roadmap, 
    breakable, 
    enhanced, 
    colback=quantgray!30, 
    colframe=quantblue
  ] 
  #1 
  \end{tcolorbox}
}

% --- THEOREM STYLES A PROSTŘEDÍ ---
\newtheoremstyle{mytheoremstyle}%
  {12pt}{12pt}%
  {\normalfont}{}%
  {\bfseries}{}%
  {\newline}{}

\theoremstyle{mytheoremstyle}

\newtheorem{definition}{Definice}[section]
\newtheorem{example}{Příklad}[section]
\newtheorem{remark}{Poznámka}[section]
\newtheorem{theorem}{Věta}[section]
\newtheorem{lemma}{Lemma}[section]
\newtheorem{proposition}{Tvrzení}[section]
\newtheorem{corollary}{Důsledek}[section]
\newtheorem{principle}{Princip}[section]
\newtheorem{innovation}{Inovace}[section]
\newtheorem{researchareaenv}{Výzkumná Oblast}[section]
\newtheorem{assessment}{Hodnocení}[section]

% --- DŮKAZY ---
\renewcommand{\proofname}{Důkaz}

% --- ČÍSLOVÁNÍ PODLE SEKCI ---
\numberwithin{equation}{section}
\numberwithin{figure}{section}
\numberwithin{table}{section}
\numberwithin{algorithm}{section}


% --- NOVÉ SLOUPCE PRO TABULKY ---
\newcolumntype{L}{>{\raggedright\arraybackslash}p{0.2\textwidth}}
\newcolumntype{M}{>{\centering\arraybackslash}p{0.3\textwidth}}
\newcolumntype{R}{>{\raggedleft\arraybackslash}p{0.4\textwidth}}

% --- LISTINGS STYLES ---
\lstset{
  basicstyle=\ttfamily\small,
  breaklines=true,
  frame=single,
  numbers=left,
  numberstyle=\tiny\color{gray},
  showstringspaces=false,
  commentstyle=\color{quantgreen},
  keywordstyle=\color{quantblue},
  stringstyle=\color{quantred},
  backgroundcolor=\color{quantgray!20}
}

\lstdefinestyle{python}{
  language=Python,
  morekeywords={import, def, class, return, yield, lambda, with, async, await}
}

\lstdefinestyle{matlab}{
  language=Matlab,
  morekeywords={function, end, if, else, for, while}
}

% --- MATEMATICKÉ PŘÍKAZY ---
\providecommand{\diff}{\mathrm{d}}
\newcommand{\bmm}[1]{\bm{#1}}

% --- ČÍSELNÉ MNOŽINY ---
\newcommand{\RR}{\mathbb{R}}
\newcommand{\CC}{\mathbb{C}}
\newcommand{\NN}{\mathbb{N}}
\newcommand{\ZZ}{\mathbb{Z}}
\newcommand{\QQ}{\mathbb{Q}}

% --- PRAVDĚPODOBNOSTNÍ SYMBOLY ---
\newcommand{\EE}{\mathbb{E}}
\newcommand{\PP}{\mathbb{P}}
\newcommand{\Var}{\mathrm{Var}}
\newcommand{\Cov}{\mathrm{Cov}}
\newcommand{\Corr}{\mathrm{Corr}}

% --- DALŠÍ MATEMATICKÉ SYMBOLY ---
\newcommand{\supp}{\mathrm{supp}}
\newcommand{\sgn}{\mathrm{sgn}}
\newcommand{\id}{\mathrm{id}}

% --- OPERÁTORY PRO FUNKCIONÁLNÍ ANALÝZU ---
\DeclareMathOperator{\diag}{diag}
\DeclareMathOperator{\tr}{tr}
\DeclareMathOperator{\Span}{span}
\DeclareMathOperator{\grad}{grad}
\DeclareMathOperator{\curl}{curl}
\DeclareMathOperator{\divg}{div}
\DeclareMathOperator{\Hess}{Hess}
\DeclareMathOperator{\Jac}{Jac}

% --- POMOCNÉ BLOKOVÉ NADPISY ---
\newcommand{\blocktitle}[1]{\vspace{0.6\baselineskip}\noindent\textbf{#1}\par\vspace{0.25\baselineskip}}
\newcommand{\spc}{\vspace{0.6\baselineskip}}

% --- HYPERREF NASTAVENÍ ---
\hypersetup{
  colorlinks=true,
  linkcolor=quantblue,
  citecolor=quantgreen,
  urlcolor=quantred,
  pdftitle={Pokročilé Diferenciální Rovnice pro Kvantitativní Odborníky},
  pdfauthor={Quant Expert Team},
  pdfsubject={Matematika, Finanční Modelování},
  pdfkeywords={diferenciální rovnice, kvantitativní finance, matematické modelování}
}

% --- MIKROTYPOGRAFIE ---
\microtypesetup{expansion=true}
\emergencystretch=1.5em

% --- ČESKÉ DĚLENÍ SLOV ---


% ---- Titulní údaje (uprav dle potřeby) ----
\title{\textbf{Calculus 2}\\\large Skripta s důrazem na rigoróznost a příklady}

\date{\today}

\begin{document}

% ---- Titulní strana ----
\maketitle
\thispagestyle{empty}
\clearpage

% ---- Předmluva (nečíslovaná, ale v obsahu) ----
\section*{Předmluva}
\addcontentsline{toc}{section}{Předmluva}
Tato skripta vznikají s cílem poskytnout systematický a rigorózní úvod do Calculusu a diferenciálních rovnic.
Důraz je kladen na jasné definice, věty a důkazy, doplněné o řešené příklady.
Struktura je rozdělena do \emph{Levelů}, aby bylo možné progresivně zvyšovat náročnost.

\bigskip
\noindent\textbf{Jak číst:}
Každý Level obsahuje úvodní motivaci, přehled klíčových výsledků a výběr příkladů.
Doporučujeme procházet kapitoly postupně a průběžně ověřovat porozumění na cvičeních.

\clearpage

% ---- Globální obsah ----
\setcounter{secnumdepth}{3}   % číslování až do \subsubsection
\setcounter{tocdepth}{2}      % obsah zobrazuje do \subsection (změň na 3 dle chuti)
\tableofcontents
\clearpage

% ---- (Volitelně) seznam obrázků/tabulek/vět – odkomentuj, pokud používáš ----
% \listoffigures
% \clearpage
% \listoftables
% \clearpage
% \usepackage{thmtools} % v preamble, pokud chceš seznam vět
% \listoftheorems[ignoreall,show={theorem,lemma,proposition}]
% \clearpage

% ---- Hlavní text (kapitoly) ----
% Každá kapitola je samostatný soubor v 'chapters/'. Přidávej/ubírej dle potřeby.
% !TEX root = ../main.tex
\section{Úvod do světa diferenciálních rovnic}
\label{sec:uvod-diffeq}

\blocktitle{Cíl kapitoly}
Vytvořit konceptuální a filozofický základ pro chápání diferenciálních rovnic jako univerzálního jazyka dynamických systémů. Zavedeme základní terminologii, klasifikační schémata a filozofii přístupu s důrazem na postupnou gradaci složitosti a propojení s kvantitativními aplikacemi.

\spc

\subsection{Filozofický a historický kontext diferenciálních rovnic}

\subsubsection{Od deskripce stavu k modelování evoluce}

\begin{motivation}
Zatímco algebraické rovnice popisují \emph{stavy} (např. $x^2 = 4$ má řešení $x = \pm 2$ - statický výsledek), diferenciální rovnice modelují \emph{změnu} (např. $\frac{dx}{dt} = x$ popisuje exponenciální růst v čase - dynamický proces). Tento posun od statiky k dynamice představuje fundamentální milník v matematickém modelování reálných systémů.
\end{motivation}

Historický vývoj začíná v 17. století s Newtonem, který nejenže formuloval zákony mechaniky jako diferenciální rovnice, ale také vyvinul první numerické metody při absenci analytických řešení.

\subsubsection{Deterministické paradigma a jeho limity}

\begin{definition}[Laplaceův determinismus]
Přesná znalost současného stavu systému a zákonů jeho evoluce umožňuje principiálně přesnou predikci budoucího vývoje.
\end{definition}

\begin{intuition}
Deterministické ODR jsou ideální pro systémy s nízkou mírou nejistoty, ale v ekonomii a financích narážíme na inherentní limity tohoto přístupu.
\end{intuition}

\begin{example}[Limitace determinismu]
Předpověď dlouhodobého vývoje finančních trhů pomocí čistě deterministických modelů je principiálně omezená kvůli vlivu neočekávaných událostí a behaviorálních faktorů.
\end{example}

\subsubsection{Univerzální vlastnosti diferenciálních rovnic}

\begin{theorem}[Univerzální vlastnosti]
Diferenciální rovnice vykazují:
\begin{itemize}
\item \emph{Univerzalitu}: Stejné rovnice popisují různé fyzikální a ekonomické systémy
\item \emph{Strukturní stabilitu}: Malé změny parametrů vedou k malým změnám řešení
\item \emph{Hierarchii složitosti}: Od integrovatelných systémů po komplexní chování
\end{itemize}
\end{theorem}

\spc

\subsection{Formální základ a systematická klasifikace ODR}

\begin{motivation}
Systematická klasifikace určuje které matematické nástroje jsou aplikovatelné a jaké chování řešení můžeme očekávat.
\end{motivation}

\subsubsection{Základní definice a notace}

\begin{definition}[Obyčejná diferenciální rovnice]
Nechť $n \in \mathbb{N}$ a $F: D \subset \mathbb{R}^{n+2} \to \mathbb{R}$. \emph{Obyčajnou diferenciální rovnicou $n$-tého řádu} rozumíme:
\[
F\left(x, y, y', y'', \dots, y^{(n)}\right) = 0,
\]
kde $y = y(x)$ je neznámá funkce.
\end{definition}

\begin{intuition}
Řád rovnice odpovídá počtu počátečních podmínek potřebných pro jednoznačnou specifikaci řešení.
\end{intuition}

\subsubsection{Hierarchické klasifikační schéma}

\begin{itemize}
\item \textbf{Podle řádu}: Určuje dimenzi fázového prostoru
\item \textbf{Podle linearity}: Lineární vs. nelineární systémy
\item \textbf{Podle autonomnosti}: Závislost na nezávislé proměnné
\end{itemize}

\begin{example}[Autonomní vs. neautonomní systém]
Autonomní: $\dot{x} = -x^2$ (řešení nezávisí na "času začátku") \\
Neautonomní: $\dot{x} = -tx$ (explicitní závislost na čase)
\end{example}



\subsubsection{Singularity a regulární body}

\begin{definition}[Singulární bod]
Bod $x_0$ je \emph{singulární}, pokud funkce $f(x,y)$ není definována nebo není spojitá v jeho okolí.
\end{definition}

\begin{intuition}
Singulární body často odpovídají fyzikálním singularitám nebo bodům, kde se mění kvalitativní chování systému.
\end{intuition}

\spc

\subsection{Řešení a jeho interpretace v aplikacích}

\begin{motivation}
Různé typy řešení poskytují různé úrovně informace o chování systému. Pro aplikace je klíčové pochopení jejich interpretace.
\end{motivation}

\subsubsection{Druhy řešení a jejich význam}

\begin{definition}[Řešení ODR]
Funkce $\varphi: I \to \mathbb{R}$ je \emph{řešením} na intervalu $I$, pokud splňuje rovnici identicky na $I$.
\end{definition}

\begin{definition}[Obecné a partikulární řešení]
\emph{Obecné řešení} obsahuje $n$ integračních konstant, \emph{partikulární řešení} vzniká jejich konkrétní volbou.
\end{definition}

\begin{example}[Fyzikální interpretace]
V mechanice: obecné řešení popisuje všechny možné trajektorie, partikulární řešení konkrétní pohyb při daných počátečních podmínkách.
\end{example}

\subsubsection{Kvalitativní versus kvantitativní analýza}

\begin{intuition}
Někdy je kvalitativní informace (stabilita, periodicita) důležitější než explicitní vyjádření řešení.
\end{intuition}

\begin{definition}[Kvalitativní analýza]
Studium vlastností řešení bez explicitního nalezení jejich tvaru, pomocí geometrických a analytických metod.
\end{definition}

\begin{keyinsight}
Pro kvantové aplikace: kvalitativní vlastnosti (stabilita, periodičnost) jsou často důležitější než explicitní řešení. Pokročilé metody stability budou detailně rozebrány v \hyperref[sec:teorie-stability]{Kapitole 5}.
\end{keyinsight}

\subsubsection{Asymptotické chování}

\begin{definition}[Asymptotická notace]
$f(x) \sim g(x)$ pro $x \to x_0$ znamená $\lim_{x \to x_0} \frac{f(x)}{g(x)} = 1$.
\end{definition}

\begin{intuition}
Asymptotika řešení pro velké časy často poskytuje důležitější informaci než přesný tvar řešení.
\end{intuition}

\spc

\subsection{Formulace matematických problémů}

\begin{motivation}
Správná formulace problému je prvním krokem k jeho řešení. Různé typy podmínek vedou na různé matematické problémy.
\end{motivation}

\subsubsection{Počáteční úloha (Cauchyova úloha)}

\[
\begin{cases}
y^{(n)} = f\left(x, y, y', \dots, y^{(n-1)}\right) \\
y(x_0) = y_0, \ y'(x_0) = y_1, \ \dots, \ y^{(n-1)}(x_0) = y_{n-1}
\end{cases}
\]

\begin{intuition}
Počáteční úloha odpovídá standardní situaci: známe kompletní stav systému v počátečním čase a predikujeme jeho vývoj.
\end{intuition}

\subsubsection{Okrajová úloha}

\[
\begin{cases}
y'' = f(x, y, y') \\
y(a) = \alpha, \ y(b) = \beta
\end{cases}
\]

\begin{intuition}
Okrajové úlohy jsou typické pro problémy statické optimalizace a stacionární stavy.
\end{intuition}

\subsubsection{Problém vlastních čísel}

\begin{definition}[Problém vlastních čísel]
Hledáme netriviální řešení $\mathcal{L}y = \lambda y$ s okrajovými podmínkami.
\end{definition}

\begin{example}[Kvantová mechanika]
Časově nezávislá Schrödingerova rovnice $\hat{H}\psi = E\psi$ je problémem vlastních hodnot.
\end{example}

\spc

\subsection{Základní aplikační domény}

\begin{motivation}
Aplikace motivují teorii a poskytují kontext pro interpretaci výsledků. Zaměříme se na fundamentální modely relevantní pro kvantové experty.
\end{motivation}

\subsubsection{Deterministické modely v ekonomii}

\begin{example}[Model spojitého úročení]
\[
\frac{\dd V}{\dd t} = rV \quad \Rightarrow \quad V(t) = V_0 e^{rt}
\]
Exponenciální růst při spojitém úročení.
\end{example}

\begin{example}[Cenová adaptace]
\[
\frac{\dd p}{\dd t} = \alpha [D(p) - S(p)]
\]
Adaptace ceny na nerovnováhu mezi poptávkou a nabídkou.
\end{example}

\subsubsection{Spojité modely úrokových měr}

\begin{example}[Deterministická část Vasicekova modelu]
\[
\frac{\dd r}{\dd t} = \kappa(\theta - r(t)) \quad \Rightarrow \quad r(t) = \theta + (r_0 - \theta)e^{-\kappa t}
\]
Mean-reverting chování úrokových měr.
\end{example}

\begin{intuition}
Parameter $\kappa$ určuje rychlost návratu k dlouhodobému průměru $\theta$, což je klíčové pro risk management v úrokových produktech. Deterministické modely tvoří základ pro složitější stochastické rozšíření.
\end{intuition}

\subsubsection{Úvod do kvantových systémů}

\begin{example}[Časově nezávislá Schrödingerova rovnice]
\[
\hat{H}\psi = E\psi
\]
Problém vlastních hodnot pro lineární diferenciální operátor.
\end{example}

\begin{intuition}
Kvantové systémy přirozeně vedou na lineární diferenciální rovnice a problémy vlastních hodnot.
\end{intuition}

\spc

\subsection{Geometrická interpretace a základy dynamických systémů}

\begin{motivation}
Geometrický pohled poskytuje intuitivní porozumění chování řešení bez explicitního výpočtu.
\end{motivation}

\subsubsection{Fázové portréty a trajektorie}

\begin{definition}[Fázový portrét]
Geometrická reprezentace všech trajektorií autonomního systému $\dot{\mathbf{x}} = \mathbf{f}(\mathbf{x})$.
\end{definition}

\begin{example}[Harmonický oscilátor]
Pro systém $\dot{x} = y$, $\dot{y} = -x$ jsou trajektorie kružnice v fázové rovině.
\end{example}

\begin{example}[Ekonomická interpretace fázového portrétu]
Pro model $\dot{p} = \alpha(D(p) - S(p))$, $\dot{q} = \beta p$ můžeme v rovině $(p,q)$ vizualizovat společnou dynamiku ceny a množství.
\end{example}

\subsubsection{Rovnovážné body a jejich klasifikace}

\begin{definition}[Rovnovážný bod]
Bod $\mathbf{x}^*$ takový, že $\mathbf{f}(\mathbf{x}^*) = \mathbf{0}$.
\end{definition}

\begin{intuition}
Rovnovážné body odpovídají stacionárním stavům systému.
\end{intuition}

\begin{itemize}
\item \textbf{Stabilní uzel}: Všechny trajektorie konvergují
\item \textbf{Sedlo}: Některé konvergují, některé divergují  
\item \textbf{Střed}: Periodické trajektorie
\end{itemize}

\subsubsection{Elementární linearizace}

\begin{intuition}
V okolí rovnovážného bodu se systém chová přibližně jako jeho linearizace.
\end{intuition}

\begin{definition}[Linearizovaný systém]
Pro systém $\dot{\mathbf{x}} = \mathbf{f}(\mathbf{x})$ v okolí $\mathbf{x}^*$:
\[
\dot{\mathbf{\xi}} = D\mathbf{f}(\mathbf{x}^*)\mathbf{\xi}
\]
kde $\mathbf{\xi} = \mathbf{x} - \mathbf{x}^*$.
\end{definition}

\begin{intuition}
Geometrická interpretace je také klíčová pro design stabilních numerických schémat - numerická metoda by měla zachovávat kvalitativní vlastnosti analytického řešení.
\end{intuition}

\spc


\begin{motivation}
Porozumění celkové struktuře teorie pomáhá vidět souvislosti a směřování dalšího studia.
\end{motivation}

\subsubsection{Logická struktura pokročilé teorie}

\begin{itemize}
\item \textbf{Kapitola 2}: Matematický fundament - teorie míry, funkcionální analýza
\item \textbf{Kapitola 3}: Základní teorémy - existence, jednoznačnost, spojitá závislost
\item \textbf{Kapitola 4}: Pokročilý aparát - distribuce, slabá formulace
\item \textbf{Kapitola 5}: Teorie stability - Ljapunovovy metody, bifurkace
\item \textbf{Kapitola 6}: Pokročilé koncepty - Hamiltonovské systémy, variační principy
\end{itemize}

\subsubsection{Klíčové koncepty pro kvantové experty}

\begin{keyinsight}
Pro kvantové aplikace jsou klíčové: spektrální teorie, stochastické rozšíření, numerická stabilita a variační principy. Tyto pokročilé metody budou rozvíjeny v následujících kapitolách.
\end{keyinsight}

\spc

\subsection*{Shrnutí kapitoly}

\begin{itemize}
\item Diferenciální rovnice poskytují formalismus pro modelování dynamických systémů s konečným počtem stupňů volnosti

\item Systematická klasifikace určuje dostupné matematické nástroje a očekávané chování řešení

\item Různé typy řešení (obecné, partikulární, singulární) poskytují různé úrovně informace o systému

\item Geometrická interpretace pomocí fázových portrétů poskytuje intuitivní porozumění chování řešení

\item Aplikace v ekonomii a kvantových systémech demonstrují relevanci teorie pro modelování reálných systémů

\item Deterministické ODR tvoří teoretický základ pro složitější stochastické modely
\end{itemize}

 
\input{chapters/02-Matematický-fundament}
% !TEX root = ../main.tex
\section{Základní teorémy teorie ODR - Existence, jednoznačnost a kvalitativní vlastnosti}
\label{sec:zakladni-teoremy}

\blocktitle{Cíl kapitoly}
Aplikovat matematický aparát z Kapitoly 2 na fundamentální teorémy existence, jednoznačnosti a kvalitativního chování řešení obyčejných diferenciálních rovnic. Propojit abstraktní funkcionální analýzu s konkrétními dynamickými systémy a kvantovými aplikacemi.

\begin{intermezzo}
\textbf{Navázání na Kapitolu 2:} Zatímco Kapitola 2 poskytla matematický aparát funkcionální analýzy, nyní jej aplikujeme na teorii ODR prostřednictvím \hyperref[sec:matematicky-fundament]{Banachových prostorů pro Picardovu iteraci}, \hyperref[sec:matematicky-fundament]{teorie míry pro Carathéodoryho větu} a \hyperref[sec:matematicky-fundament]{Sobolevových prostorů pro analýzu regularity}.
\end{intermezzo}

\spc

\subsection{Formulace základních problémů a motivace}

\begin{scaffold}
\textbf{Co umíme:} Klasifikace ODR a typy úloh z Kapitoly 1. \\
\textbf{Co se naučíme:} Proč a jak dokazovat existenci a jednoznačnost řešení. \\
\textbf{K čemu to využijeme:} Garance řešení u kvantových a stochastických systémů.
\end{scaffold}

\begin{motivation}
Potřebujeme matematickou jistotu, že naše modely ODR mají smysluplná řešení a že jsou jednoznačná pro dané počáteční podmínky. Bez těchto záruk by jakákoli numerická simulace nebo fyzikální interpretace postrádala pevný základ.
\end{motivation}

\begin{itemize}
\item \textbf{Problém existence:} Existuje alespoň jedno řešení na nějakém intervalu?
\item \textbf{Problém jednoznačnosti:} Je řešení určeno jednoznačně počáteční podmínkou?
\item \textbf{Problém prodloužení:} Lze lokální řešení rozšířit na maximální interval?
\item \textbf{Problém závislosti:} Jak řešení závisí na počátečních podmínkách a parametrech?
\end{itemize}

\begin{intuition}
Geometrický pohled: V fázovém prostoru hledáme křivky (trajektorie), které procházejí daným bodem (počáteční podmínka) a jejichž tečný vektor je dán pravou stranou rovnice. Existence znamená, že z každého bodu vychází alespoň jedna taková křivka, jednoznačnost že právě jedna.
\end{intuition}

\begin{example}[Kvantová motivace]
Pro časově závislou Schrödingerovu rovnici $i\hbar\frac{\partial\psi}{\partial t} = \hat{H}\psi$ potřebujeme garantovat, že pro danou počáteční vlnovou funkci $\psi(0)$ existuje jednoznačné řešení $\psi(t)$ zachovávající normu.
\end{example}

\begin{keyinsight}
Bez teorémů existence a jednoznačnosti není možné spolehlivě modelovat ani numericky simulovat dynamické systémy. Tyto teorémy tvoří matematický základ pro všechny aplikace ODR ve fyzice, inženýrství a kvantových vědách.
\end{keyinsight}

\begin{summary}
\textbf{Klíčové koncepty:} Existence, jednoznačnost, prodloužení řešení, spojitá závislost. \\
\textbf{Hlavní výsledky:} Picardova-Lindelöfova věta, Peanova věta, Carathéodoryho věta. \\
\textbf{Aplikace:} Záruky pro numerické simulace, kvantové systémy, teorii řízení.
\end{summary}

\spc

\subsection{Lipschitzovská spojitost a její varianty}

\begin{scaffold}
\textbf{Co umíme:} Banachovy prostory a normy z Kapitoly 2. \\
\textbf{Co se naučíme:} Lipschitzova vlastnost a její role v teorii ODR. \\
\textbf{K čemu to využijeme:} Základ Picardova-Lindelöfova principu a analýzy stability.
\end{scaffold}

\begin{motivation}
Lipschitzovská podmínka garantuje kontrolu růstu odchylek řešení - malá změna v počáteční podmínce vede k malé změně řešení. Tato vlastnost je klíčová pro stabilitu a numerickou schůdnost problémů.
\end{motivation}

\begin{definition}[Lipschitzovská spojitost]
Funkce $f: D \subset \mathbb{R} \times \mathbb{R}^n \to \mathbb{R}^n$ je \emph{Lipschitzovská vzhledem k $y$} na $D$, jestliže existuje konstanta $L > 0$ taková, že
\[
\|f(t,y_1) - f(t,y_2)\| \leq L\|y_1 - y_2\| \quad \text{pro všechna } (t,y_1), (t,y_2) \in D
\]
\end{definition}

\begin{definition}[Lokální versus globální Lipschitzovskost]
\begin{itemize}
\item \emph{Lokální Lipschitzovskost}: Pro každý kompakt $K \subset D$ existuje $L_K$
\item \emph{Globální Lipschitzovskost}: Jedna konstanta $L$ platí na celém $D$
\end{itemize}
\end{definition}

\begin{intuition}
Globální Lipschitzovskost znamená, že "strmost" funkce $f$ je omezená v celém definičním oboru. Selhání této podmínky může vést k rychlému rozbíhání trajektorií a potenciálnímu blow-up řešení.
\end{intuition}

\begin{theorem}[Postačující podmínky]
Je-li $f$ spojitě diferencovatelná v $y$ na konvexní oblasti $D$, pak je lokálně Lipschitzovská s konstantou $L = \sup_{(t,y)\in D} \|D_y f(t,y)\|$.
\end{theorem}

\begin{keyinsight}
Lipschitzovská podmínka je klíčová pro konvergenci Picardovy iterace - čím menší Lipschitzova konstanta, tím rychlejší konvergence. Pro $Lh < 1$ je Picardova iterace kontrakcí.
\end{keyinsight}

\begin{example}[Lipschitzovskost v kvantových systémech]
Pro Schrödingerovu rovnici $i\hbar\psi_t = -\frac{\hbar^2}{2m}\psi_{xx} + V(x)\psi$ s hladkým potenciálem $V(x)$ je pravá strana Lipschitzovská v $L^2$, což zaručuje existenci a jednoznačnost řešení.
\end{example}

\begin{application}
\begin{verbatim}
# Python: Výpočet Lipschitzovy konstanty pro numerické metody
def estimate_lipschitz(f, domain, samples=1000):
    max_slope = 0
    for _ in range(samples):
        y1 = np.random.uniform(domain[0], domain[1])
        y2 = np.random.uniform(domain[0], domain[1])
        # Výpočet rozdílu v pravé straně
        diff_f = np.linalg.norm(f(0, y1) - f(0, y2))
        diff_y = np.linalg.norm(y1 - y2)
        if diff_y > 1e-10:
            slope = diff_f / diff_y
            max_slope = max(max_slope, slope)
    return max_slope
\end{verbatim}
\end{application}

\begin{summary}
\textbf{Klíčové koncepty:} Lipschitzovská spojitost, lokální/globální vlastnosti. \\
\textbf{Hlavní výsledky:} Vztah k diferencovatelnosti, odhady konstant. \\
\textbf{Aplikace:} Podmínky pro existenci a jednoznačnost, analýza stability.
\end{summary}

\spc

\subsection{Picardova-Lindelöfova věta: Existence a jednoznačnost}

\begin{scaffold}
\textbf{Co umíme:} Kontrakce a Banachova věta o pevném bodě z Kapitoly 2. \\
\textbf{Co se naučíme:} Důkaz existence a jednoznačnosti pomocí Picardovy iterace. \\
\textbf{K čemu to využijeme:} Konstrukce řešení pro kvantové systémy a numerické metody.
\end{scaffold}

\begin{motivation}
Picardova-Lindelöfova věta nejen garantuje existenci a jednoznačnost řešení, ale poskytuje i konstruktivní metodu pro jeho aproximaci. Tento algoritmický aspekt je klíčový pro numerické simulace.
\end{motivation}

\begin{theorem}[Picardova-Lindelöfova věta]
Nechť $f: [t_0-a, t_0+a] \times \overline{B(y_0,b)} \to \mathbb{R}^n$ je spojitá a Lipschitzovská v $y$ s konstantou $L$. Pak počáteční úloha
\[
y' = f(t,y), \quad y(t_0) = y_0
\]
má právě jedno řešení na intervalu $[t_0-h, t_0+h]$, kde $h = \min\left(a, \frac{b}{M}\right)$, $M = \max\|f(t,y)\|$.
\end{theorem}

\begin{proof}[Náčrt důkazu]
Definujeme Picardovu iteraci:
\[
\phi_0(t) = y_0, \quad \phi_{n+1}(t) = y_0 + \int_{t_0}^t f(s, \phi_n(s))\, ds
\]
\begin{enumerate}
\item Kontrakce v $C([t_0-h, t_0+h])$: $\|\phi_{n+1} - \phi_n\|_\infty \leq \frac{Lh}{n!}\|\phi_1 - \phi_0\|_\infty$
\item Konvergence: $\{\phi_n\}$ je Cauchyho posloupnost v úplném prostoru
\item Limita splňuje integrální rovnici, tedy i diferenciální rovnici
\item Jednoznačnost: Dvě řešení splňující stejnou počáteční podmínku musí být totožná
\end{enumerate}
\qed
\end{proof}

\begin{theorem}[Odhad počtu iterací]
Pro dosažení přesnosti $\epsilon$ v Picardově iteraci postačuje počet iterací:
\[
N \geq \frac{\log\left(\frac{\epsilon(1-Lh)}{\|\phi_1-\phi_0\|_\infty}\right)}{\log(Lh)}
\]
za předpokladu $Lh < 1$.
\end{theorem}

\begin{intuition}
Picardova iterace postupně "opravuje" odhad řešení. Každá iterace bere předchozí aproximaci, vypočítá její derivaci pomocí pravé strany rovnice a integruje ji od počátečního bodu. Tento proces konverguje k přesnému řešení.
\end{intuition}

\begin{application}
\begin{verbatim}
# Python: Adaptivní Picardova iterace pro ODR
def adaptive_picard(f, t0, y0, t_span, tol=1e-6, max_iter=100):
    t = np.linspace(t_span[0], t_span[1], 100)
    y = np.full_like(t, y0)
    
    for iteration in range(max_iter):
        y_old = y.copy()
        # Picardův krok: y_{n+1}(t) = y0 + ∫ f(s, y_n(s)) ds
        integrand = f(t[:-1], y_old[:-1])  # Vyhodnocení pravé strany
        y_new = y0 + cumtrapz(integrand, t, initial=0)  # Numerická integrace
        
        error = np.max(np.abs(y_new - y_old))
        y = y_new
        
        if error < tol:
            print(f"Konvergence po {iteration} iteracích, chyba: {error}")
            break
    
    return t, y

# Aplikace na kvantový systém
def schrodinger_rhs(t, psi):
    # Pravá strana Schrödingerovy rovnice
    return -1j * (H @ psi)  # H je Hamiltonián
\end{verbatim}
\end{application}

\begin{keyinsight}
Rychlost konvergence Picardovy iterace je určena Lipschitzovou konstantou a délkou intervalu. Pro malé $Lh$ konverguje velmi rychle, pro větší hodnoty může být pomalá nebo divergovat.
\end{keyinsight}

\begin{summary}
\textbf{Klíčové koncepty:} Picardova iterace, kontrakční zobrazení, integrální rovnice. \\
\textbf{Hlavní výsledky:} Existence a jednoznačnost řešení, odhady konvergence. \\
\textbf{Aplikace:} Konstruktivní důkazy, numerické metody, kvantová dynamika.
\end{summary}

\spc

\subsection{Grönwallovo lemma a aplikace v teorii odhadů}

\begin{scaffold}
\textbf{Co umíme:} Existence řešení a jednoznačnost z předchozí sekce. \\
\textbf{Co se naučíme:} Techniky pro odhady růstu řešení a analýzu stability. \\
\textbf{K čemu to využijeme:} Odhady stability a dlouhodobého chování dynamických systémů.
\end{scaffold}

\begin{motivation}
Grönwallovo lemma je mocný nástroj pro odvozování odhadů řešení diferenciálních rovnic a nerovnic. Umožňuje nám kontrolovat exponenciální růst řešení a analyzovat stabilitu systémů při perturbacích.
\end{motivation}

\begin{theorem}[Grönwallovo lemma - diferenciální forma]
Nechť $u: [t_0, T] \to \mathbb{R}$ je spojitě diferencovatelná a splňuje
\[
u'(t) \leq \beta(t)u(t) + \alpha(t), \quad t \in [t_0, T]
\]
kde $\alpha, \beta$ jsou spojité. Pak pro $t \in [t_0, T]$ platí:
\[
u(t) \leq u(t_0)e^{\int_{t_0}^t \beta(s)ds} + \int_{t_0}^t \alpha(s)e^{\int_s^t \beta(\tau)d\tau}ds
\]
\end{theorem}

\begin{theorem}[Grönwallovo lemma - integrální forma]
Nechť $u: [t_0, T] \to \mathbb{R}$ je spojitá a splňuje
\[
u(t) \leq \alpha(t) + \int_{t_0}^t \beta(s)u(s)ds, \quad t \in [t_0, T]
\]
kde $\alpha$ je spojitá, $\beta \geq 0$ spojitá. Pak:
\[
u(t) \leq \alpha(t) + \int_{t_0}^t \alpha(s)\beta(s)e^{\int_s^t \beta(\tau)d\tau}ds
\]
Speciálně, je-li $\alpha$ konstantní, pak $u(t) \leq \alpha e^{\int_{t_0}^t \beta(s)ds}$.
\qed
\end{theorem}

\begin{intuition}
Grönwallovo lemma říká, že pokud řešení nerovnice neroste rychleji než exponenciála, pak skutečně neroste rychleji než exponenciála. Je to nástroj pro "uzavření" odhadů - z lokálních vlastností odvozujeme globální chování.
\end{intuition}

\begin{figure}[h]
\centering
\begin{tikzpicture}[scale=0.8]
\begin{axis}[
    xlabel={Čas $t$},
    ylabel={Řešení $u(t)$},
    domain=0:3,
    samples=100,
    legend pos=north west
]
% Exponenciální horní odhad
\addplot[red, thick] {exp(x)};
\addlegendentry{Grönwallův odhad $e^t$}

% Skutečné řešení nerovnice
\addplot[blue, thick] {0.5*exp(0.7*x) + 0.3};
\addlegendentry{Skutečné řešení}

% Počáteční podmínka
\addplot[only marks, mark=*, mark size=2pt] coordinates {(0,1)};
\node at (axis cs:0.2,1.2) {$u(0)$};
\end{axis}
\end{tikzpicture}
\caption{Grönwallův odhad exponenciálního růstu řešení}
\label{fig:gronwall-estimate}
\end{figure}

\begin{example}[Aplikace na stabilitu řešení]
Uvažujme perturbovanou rovnici $y' = f(t,y) + g(t,y)$ s $\|g(t,y)\| \leq \epsilon$. Pokud $f$ je Lipschitzovská s konstantou $L$, pak rozdíl řešení $\Delta y$ splňuje:
\[
\|\Delta y(t)\| \leq \|\Delta y(0)\|e^{Lt} + \frac{\epsilon}{L}(e^{Lt} - 1)
\]
\end{example}

\begin{keyinsight}
Grönwallovo lemma je univerzální nástroj pro stabilitní analýzu - umožňuje převést lokální odhady na globální a kontrolovat kumulaci chyb v čase. Je fundamentální pro teorii robustness a citlivosti dynamických systémů.
\end{keyinsight}

\begin{summary}
\textbf{Klíčové koncepty:} Diferenciální a integrální forma, exponenciální odhady. \\
\textbf{Hlavní výsledky:} Různé varianty Grönwallova lemmatu. \\
\textbf{Aplikace:} Analýza stability, odhady perturbací, kontrola chyb.
\end{summary}

\spc

\subsection{Peanova existenční věta: Existence bez jednoznačnosti}

\begin{scaffold}
\textbf{Co umíme:} Picardova věta vyžaduje Lipschitzovskost pro jednoznačnost. \\
\textbf{Co se naučíme:} Existenci řešení za obecnějších podmínek. \\
\textbf{K čemu to využijeme:} Analýza systémů s více režimy chování.
\end{scaffold}

\begin{motivation}
Peanova věta umožňuje existenci řešení i tam, kde Lipschitzovskost neplatí - například pro systémy s nespojitostmi nebo rychle oscilujícími pravými stranami. Tato obecnost je důležitá pro aplikace v teorii řízení a hybridních systémech.
\end{motivation}

\begin{theorem}[Peanova existenční věta]
Nechť $f: [t_0, t_0+a] \times \overline{B(y_0,b)} \to \mathbb{R}^n$ je spojitá a omezená. Pak počáteční úloha
\[
y' = f(t,y), \quad y(t_0) = y_0
\]
má alespoň jedno řešení na intervalu $[t_0, t_0+h]$, kde $h = \min\left(a, \frac{b}{M}\right)$, $M = \sup\|f(t,y)\|$.
\end{theorem}

\begin{proof}[Náčrt důkazu]
\begin{enumerate}
\item Konstrukce Eulerovy lomené čáry:
\begin{align*}
t_k &= t_0 + \frac{kh}{n}, \quad k = 0,1,\dots,n \\
\phi_n(t) &= y_0 + \int_{t_0}^t f(s, \phi_n(s))\, ds \quad \text{na sítích}
\end{align*}
\item Rovnoměrná omezenost a stejnoměrná spojitost: $\|\phi_n(t)\| \leq b$, $\|\phi_n(t) - \phi_n(s)\| \leq M|t-s|$
\item Arzelà-Ascoliho věta: Existuje stejnoměrně konvergentní podposloupnost $\phi_{n_k} \to \phi$
\item Limita splňuje integrální rovnici, tedy i diferenciální rovnici
\end{enumerate}
\qed
\end{proof}

\begin{intuition}
Peanova konstrukce používá Eulerovu lomenou čáru - aproximujeme řešení po malých krocích. I když tyto aproximace nemusejí konvergovat k jediné limitě (kvůli chybějící jednoznačnosti), některá podposloupnost konverguje k řešení.
\end{intuition}

\begin{example}[Případ bez jednoznačnosti]
Uvažujme rovnici $y' = 2\sqrt{|y|}$, $y(0) = 0$. Tato rovnice má nekonečně mnoho řešení:
\[
y(t) = 0 \quad \text{a} \quad y(t) = \begin{cases}
0 & t \leq c \\
(t-c)^2 & t > c
\end{cases}
\]
pro libovolné $c \geq 0$. Funkce $\sqrt{|y|}$ není Lipschitzovská v nule.
\end{example}

\begin{keyinsight}
V kvantových systémech může volba parametru $c$ odpovídat různým časům přechodu mezi kvantovými stavy nebo různým měřicím protokolům. Existence více řešení odráží fundamentální nedeterminismus v některých kvantových procesech.
\end{keyinsight}

\begin{application}
V kvantové kontrole: Systémy s přepínáním mezi různými Hamiltoniány mohou mít více řešení odpovídajících různým řídicím strategiím. Peanova věta garantuje existenci alespoň jednoho řešení.
\end{application}

\begin{summary}
\textbf{Klíčové koncepty:} Eulerova lomená čára, Arzelà-Ascoliho věta. \\
\textbf{Hlavní výsledky:} Existence za podmínek spojitosti bez Lipschitzovskosti. \\
\textbf{Aplikace:} Systémy s nespojitostmi, hybridní dynamika, teorie řízení.
\end{summary}

\spc

\subsection{Carathéodoryho existenční věta: Zobecnění pro měřitelné pravé strany}

\begin{scaffold}
\textbf{Co umíme:} Peanova věta pro spojité pravé strany. \\
\textbf{Co se naučíme:} Teorii pro měřitelné, ne nutně spojité funkce. \\
\textbf{K čemu to využijeme:} Řízení, optimalizace, systémy s impulzy.
\end{scaffold}

\begin{motivation}
Mnohé aplikace v kvantové fyzice a ekonomii používají měřitelné integrandy - například systémy s okamžitými přepnutími, impulzní řízení, nebo modely s náhodnými vlivy. Carathéodoryho teorie rozšiřuje dosah ODR na tyto praktické scénáře.
\end{motivation}

\begin{theorem}[Carathéodoryho existenční věta]
Nechť $f: [t_0, t_0+a] \times \mathbb{R}^n \to \mathbb{R}^n$ splňuje:
\begin{enumerate}
\item Pro skoro všechna $t$ je $f(t,\cdot)$ spojitá
\item Pro každé $y$ je $f(\cdot,y)$ měřitelná
\item Existuje integrovatelná funkce $m(t)$ taková, že $\|f(t,y)\| \leq m(t)$ pro skoro všechna $t$ a všechna $y$
\end{enumerate}
Pak počáteční úloha $y' = f(t,y)$, $y(t_0) = y_0$ má řešení na $[t_0, t_0+a]$.
\qed
\end{theorem}

\begin{intuition}
Carathéodoryho podmínky umožňují pravé straně být "skoro všude" spojitá v čase a spojitá ve stavové proměnné. Absolutní spojitost řešení zaručuje, že derivace existuje skoro všude a řešení lze zrekonstruovat integrací.
\end{intuition}

\begin{example}[Aplikace v kvantové kontrole]
Uvažujme systém s přepínáním mezi Hamiltoniány:
\[
i\hbar\frac{d\psi}{dt} = \hat{H}_k(t)\psi, \quad t \in [t_k, t_{k+1}]
\]
kde přepínací časy $t_k$ tvoří měřitelnou množinu. Carathéodoryho věta garantuje existenci řešení.
\end{example}

\begin{keyinsight}
Carathéodoryho teorie rozšiřuje použití ODR na problémy s měřitelnými vstupy a řídicími strategiemi. To je klíčové pro moderní aplikace v teorii řízení, ekonomii a kvantové informaci.
\end{keyinsight}

\begin{summary}
\textbf{Klíčové koncepty:} Měřitelnost, absolutní spojitost, Carathéodoryho podmínky. \\
\textbf{Hlavní výsledky:} Existence pro měřitelné pravé strany. \\
\textbf{Aplikace:} Teorie řízení, ekonomické modely, hybridní systémy.
\end{summary}

\spc

\subsection{Prodloužování řešení a maximální interval existence}

\begin{scaffold}
\textbf{Co umíme:} Lokální existence řešení z předchozích vět. \\
\textbf{Co se naučíme:} Podmínky pro globální existenci a kritéria "blow-up". \\
\textbf{K čemu to využijeme:} Analýza dlouhodobého chování a detekce singularit.
\end{scaffold}

\begin{motivation}
V aplikacích potřebujeme vědět, zda lze řešení rozšířit na celý časový horizont, nebo zda v konečném čase "exploduje". Tato otázka je klíčová pro stabilitu modelů a bezpečnost řídicích systémů.
\end{motivation}

\begin{definition}[Maximální řešení a interval existence]
Řešení $\phi: I \to \mathbb{R}^n$ se nazývá \emph{maximální}, pokud neexistuje řešení $\psi: J \to \mathbb{R}^n$ s $I \subsetneq J$ a $\psi|_I = \phi$. Interval $I$ je \emph{maximální interval existence}.
\end{definition}

\begin{theorem}[Prodloužení řešení]
Nechť $f: D \subset \mathbb{R} \times \mathbb{R}^n \to \mathbb{R}^n$ je spojitá a $\phi: (\alpha, \beta) \to \mathbb{R}^n$ je maximální řešení. Pak buď $\beta = +\infty$, nebo pro každý kompakt $K \subset D$ existuje $t_K < \beta$ takové, že $(t, \phi(t)) \notin K$ pro $t > t_K$.
\qed
\end{theorem}

\begin{theorem}[Podmínky globální existence]
Je-li $f$ spojitá a existují funkce $\alpha, \beta: \mathbb{R} \to \mathbb{R}$ takové, že
\[
\|f(t,y)\| \leq \alpha(t)\|y\| + \beta(t)
\]
s $\alpha, \beta$ spojitými na $\mathbb{R}$, pak každé maximální řešení existuje na celém $\mathbb{R}$.
\qed
\end{theorem}

\begin{intuition}
Řešení lze prodlužovat, dokud zůstává v kompaktní oblasti. Pokud řešení "uteče do nekonečna" v konečném čase, nastává blow-up. Lineární růst pravé strany zaručuje, že řešení nemůže explodovat v konečném čase.
\end{intuition}

\begin{figure}[h]
\centering
\begin{tikzpicture}[scale=0.8]
\begin{axis}[
    xlabel={Čas $t$},
    ylabel={Řešení $y(t)$},
    domain=0:2.5,
    samples=100,
    legend pos=north west
]
% Blow-up řešení
\addplot[red, thick, domain=0:2.3] {1/(2.3-x)};
\addlegendentry{Blow-up řešení}

% Globální řešení
\addplot[blue, thick] {1 + 0.3*x};
\addlegendentry{Globální řešení}

% Singularita
\addplot[only marks, mark=*, mark size=2pt, red] coordinates {(2.3,8)};
\node at (axis cs:2.1,9) {Singularita};
\end{axis}
\end{tikzpicture}
\caption{Porovnání blow-up a globálního řešení}
\label{fig:blowup-global}
\end{figure}

\begin{example}[Blow-up v nelineární Schrödingerově rovnici]
Pro NLS $i\psi_t + \psi_{xx} + |\psi|^2\psi = 0$ s dostatečně velkou počáteční podmínkou může dojít k blow-up v konečném čase: $\|\psi(t)\|_{L^\infty} \to \infty$ pro $t \to T^-$.
\end{example}

\begin{keyinsight}
Kontrola parametrů a počátečních podmínek může zaručit globální existenci řešení. Pro systémy s potenciálním blow-up je důležité stanovit kritéria, kdy k explozi dochází a kdy ne.
\end{keyinsight}

\begin{summary}
\textbf{Klíčové koncepty:} Maximální řešení, interval existence, blow-up. \\
\textbf{Hlavní výsledky:} Věty o prodlužování, podmínky globální existence. \\
\textbf{Aplikace:} Analýza stability, detekce singularit, kvantová teorie pole.
\end{summary}

\spc

\subsection{Spojitá a hladká závislost na datech a parametrech}

\begin{scaffold}
\textbf{Co umíme:} Existence a jednoznačnost řešení. \\
\textbf{Co se naučíme:} Citlivost řešení na změny vstupů a parametrů. \\
\textbf{K čemu to využijeme:} Bifurkační analýza, optimalizace, robustní řízení.
\end{scaffold}

\begin{motivation}
V kvantových experimentech i technických aplikacích jsou vstupní parametry a počáteční podmínky známy pouze přibližně. Potřebujeme vědět, jak malé chyby v těchto údajích ovlivní řešení.
\end{motivation}

\begin{theorem}[Spojitá závislost na počátečních podmínkách]
Nechť $f$ je spojitá a lokálně Lipschitzovská. Pak řešení $y(t; t_0, y_0)$ závisí spojitě na počáteční podmínce $y_0$. Konkrétně, pro kompaktní interval $I$ existuje $L$ takové, že
\[
\|y(t; t_0, y_0) - y(t; t_0, z_0)\| \leq \|y_0 - z_0\|e^{L|t-t_0|}
\]
\end{theorem}

\begin{theorem}[Hladká závislost na parametrech]
Nechť $f(t,y,\lambda)$ je $C^k$ v $(y,\lambda)$. Pak řešení $y(t;\lambda)$ je $C^k$ v $\lambda$.
\qed
\end{theorem}

\begin{intuition}
Spojitá závislost znamená, že malá změna vstupů vede k malé změně řešení. Hladká závislost umožňuje použít derivační metody pro analýzu citlivosti a optimalizaci.
\end{intuition}

\begin{example}[Derivace řešení podle parametru]
Pro rovnici $y' = f(t,y,\lambda)$, $y(0) = y_0$ platí variační rovnice:
\[
\frac{\partial}{\partial t}\left(\frac{\partial y}{\partial\lambda}\right) = D_y f(t,y,\lambda)\frac{\partial y}{\partial\lambda} + D_\lambda f(t,y,\lambda)
\]
s počáteční podmínkou $\frac{\partial y}{\partial\lambda}(0) = 0$.
\end{example}

\begin{keyinsight}
Diferenciovatelná závislost řešení na parametrech je klíčová pro optimalizační problémy a analýzu citlivosti. Umožňuje použít gradientní metody a studovat stabilitu řešení vůči perturbacím.
\end{keyinsight}

\begin{application}
V bifurkační analýze: Hladká závislost na parametrech umožňuje studovat, jak se kvalitativní chování systému mění s parametry. Body, kde derivace neexistuje, často odpovídají bifurkačním bodům.
\end{application}

\begin{summary}
\textbf{Klíčové koncepty:} Spojitá závislost, hladká závislost, derivace řešení. \\
\textbf{Hlavní výsledky:} Věty o spojité a hladké závislosti. \\
\textbf{Aplikace:} Bifurkační analýza, optimalizace, studie citlivosti.
\end{summary}

\spc

\subsection{Wazewskiho topologická metoda}

\begin{scaffold}
\textbf{Co umíme:} Analytické metody existence z předchozích vět. \\
\textbf{Co se naučíme:} Topologický přístup k existenci řešení. \\
\textbf{K čemu to využijeme:} Nehomogenní a okrajové úlohy, systémy s vícero řešeními.
\end{scaffold}

\begin{motivation}
Topologické metody doplňují čistě analytické důkazy existence a jsou zvláště účinné pro problémy s okrajovými podmínkami nebo systémy, kde analytické podmínky jsou obtížně ověřitelné.
\end{motivation}

\begin{definition}[Wazewskiho množina]
Množina $W \subset \mathbb{R} \times \mathbb{R}^n$ se nazývá \emph{Wazewskiho množina} pro rovnici $y' = f(t,y)$, jestliže:
\begin{itemize}
\item Je-li $(t_0, y_0) \in W$ a řešení opustí $W$ v čase $t_1 > t_0$, pak opustí přes hranici $\partial W$
\item Množina okamžitého úniku je uzavřená v $\partial W$
\end{itemize}
\end{definition}

\begin{theorem}[Wazewskiho princip]
Nechť $W$ je Wazewskiho množina a existuje podmnožina $Z \subset \partial W$ taková, že:
\begin{itemize}
\item Žádné řešení neopouští $W$ přes $Z$
\item $Z$ je deformačním retraktem $W$ ale není deformačním retraktem $W \cup Z$
\end{itemize}
Pak existuje řešení, které zůstává v $W$ pro všechna $t \geq t_0$.
\qed
\end{theorem}

\begin{intuition}
Wazewskiho princip je topologická verze "pastí na řešení" - pokud řešení nemůže uniknout určitými částmi hranice, musí existovat řešení, které zůstane uvnitř množiny.
\end{intuition}

\begin{application}
V kvantové mechanice: Wazewskiho metoda může garantovat existenci vázaných stavů v nehomogenních potenciálech nebo existenci periodických orbit v kvantových dotacích.
\end{application}

\begin{keyinsight}
Topologické metody rozšiřují dosah existenčních teorémů na problémy, kde analytické podmínky jsou obtížně ověřitelné. Poskytují kvalitativní kritéria existence bez nutnosti explicitní konstrukce řešení.
\end{keyinsight}

\begin{summary}
\textbf{Klíčové koncepty:} Wazewskiho množina, topologický princip, deformační retrakt. \\
\textbf{Hlavní výsledky:} Wazewskiho princip existence řešení. \\
\textbf{Aplikace:} Okrajové úlohy, periodická řešení, kvalitativní analýza.
\end{summary}

\spc

\subsection{Diferenciální inkluze a jejich aplikace}

\begin{scaffold}
\textbf{Co umíme:} Jednohodnotový přístup k ODR. \\
\textbf{Co se naučíme:} Multi-valued dynamiku a její vlastnosti. \\
\textbf{K čemu to využijeme:} Teorie řízení, stochastické modely, hybridní systémy.
\end{scaffold}

\begin{motivation}
Mnohé reálné systémy - jako systémy s přepínáním, s hysterézí, nebo s neurčitými parametry - nelze adekvátně modelovat jednohodnotovými ODR. Diferenciální inkluce poskytují framework pro takové multi-valued dynamiky.
\end{motivation}

\begin{definition}[Diferenciální inkluze]
\emph{Diferenciální inklucí} rozumíme problém
\[
y'(t) \in F(t, y(t)), \quad y(t_0) = y_0
\]
kde $F: \mathbb{R} \times \mathbb{R}^n \to 2^{\mathbb{R}^n}$ je multi-valued zobrazení.
\end{definition}

\begin{theorem}[Filippovova existenční věta]
Nechť $F$ je multi-valued zobrazení splňující:
\begin{itemize}
\item $F(t,y)$ je neprázdná, konvexní a kompaktní pro všechna $(t,y)$
\item $F$ je měřitelné v $t$ a Lipschitzovské v $y$
\item $\|F(t,y)\| \leq m(t)(1 + \|y\|)$ pro integrovatelnou $m(t)$
\end{itemize}
Pak diferenciální inkluce má řešení na $[t_0, t_0+a]$.
\qed
\end{theorem}

\begin{intuition}
Diferenciální inkluce modelují systémy, kde v každém okamžiku může být rychlost změny stavu vybrána z určité množiny možností. To odpovídá situacím s nedeterministickým chováním nebo více možnými režimy.
\end{intuition}

\begin{example}[Filippovova regularizace]
Pro systém s přepínáním $y' = f_1(y)$ pro $h(y) > 0$, $y' = f_2(y)$ pro $h(y) < 0$ definujeme Filippovovu inkluci na přepínací hranici $h(y) = 0$:
\[
y' \in \overline{\text{co}}\{f_1(y), f_2(y)\}
\]
\end{example}

\begin{keyinsight}
Konvexní uzávěr v Filippovově regularizaci zajišťuje existenci řešení na přepínacích hranicích a zachovává fyzikální interpretaci hybridních kvantových systémů, kde přechody mezi stavy mohou být spojité.
\end{keyinsight}

\begin{application}
V hybridní kvantové kontrole: Systémy přepínající mezi různými Hamiltoniány podle měření lze modelovat diferenciálními inklucemi, kde množina $F(t,\psi)$ obsahuje všechny možné Hamiltoniány kompatibilní s aktuálním měřením.
\end{application}

\begin{summary}
\textbf{Klíčové koncepty:} Multi-valued zobrazení, diferenciální inkluze, Filippovova regularizace. \\
\textbf{Hlavní výsledky:} Existenční věty pro diferenciální inkluze. \\
\textbf{Aplikace:} Hybridní systémy, teorie řízení, systémy s přepínáním.
\end{summary}

\spc

\subsection*{Shrnutí kapitoly}

\begin{itemize}
\item Představili jsme fundamentální teorémy existence a jednoznačnosti řešení ODR
\item Picardova-Lindelöfova věta poskytuje konstruktivní důkaz za Lipschitzovských podmínek
\item Peanova a Carathéodoryho věta rozšiřují existenci na obecnější třídy pravých stran
\item Grönwallovo lemma je univerzální nástroj pro odhady a analýzu stability
\item Teorie prodlužování řešení popisuje podmínky globální existence a kritéria blow-up
\item Spojitá a hladká závislost na parametrech je klíčová pro analýzu citlivosti
\item Topologické metody a diferenciální inkluce rozšiřují dosah klasické teorie
\end{itemize}

\vspace{0.5cm}

\begin{tcolorbox}[title=\textbf{Doporučená literatura}, colback=blue!5!white, colframe=blue!75!black]
\begin{itemize}
\item \textbf{Hartman, P.} \emph{Ordinary Differential Equations}
\item \textbf{Hale, J. K.} \emph{Ordinary Differential Equations}  
\item \textbf{Coddington, E. A.; Levinson, N.} \emph{Theory of Ordinary Differential Equations}
\item \textbf{Filippov, A. F.} \emph{Differential Equations with Discontinuous Righthand Sides}
\end{itemize}
\end{tcolorbox}

\begin{transition}
V \hyperref[sec:pokrocila-teorie]{Kapitole 4} prohloubíme kvalitativní analýzu dynamických systémů pomocí teorie stability, bifurkací a Ljapunovových metod. Aplikujeme získané výsledky na studium dlouhodobého chování řešení a přechodů mezi různými režimy dynamiky.
\end{transition}

\spc
% !TEX root = ../main.tex
\section{Pokročilá teorie ODR - Stabilita, bifurkace a globální vlastnosti}
\label{sec:pokrocila-teorie}

\blocktitle{Cíl kapitoly}
Prohloubit porozumění dlouhodobému a kvalitativnímu chování řešení obyčejných diferenciálních rovnic. Představit metody stability, bifurkací a globálních atraktorů, které propojí lokální teorémy kapitol 1–3 s pokročilými koncepty dynamických systémů a kvantových aplikací.

\begin{intermezzo}
\textbf{Navázání na předchozí kapitoly:} Zatímco Kapitola 3 se zabývala existencí a jednoznačností řešení, nyní studujeme jejich kvalitativní chování pomocí \hyperref[sec:zakladni-teoremy]{teorie stability}, \hyperref[sec:zakladni-teoremy]{bifurkací} a \hyperref[sec:zakladni-teoremy]{globálních vlastností} dynamických systémů.
\end{intermezzo}

\spc

\subsection{Základy teorie stability}

\begin{scaffold}
\item[] \textbf{Co umíme:} Existence, jednoznačnost, Grönwallovo lemma z Kapitoly 3
\item[] \textbf{Co se naučíme:} Definice stability rovnovážných bodů a lokální analýzu  
\item[] \textbf{K čemu to využijeme:} Odhad tolerance systémů vůči rušení, kvantová stabilita
\end{scaffold}

\begin{motivation}
Pochopit, co znamená, že řešení zůstává v okolí rovnovážného stavu. Stabilita je klíčová pro robustnost fyzikálních modelů a spolehlivost technických systémů.
\end{motivation}

\begin{itemize}
\item \textbf{Problém lokální stability:} Chování řešení v okolí rovnovážných bodů
\item \textbf{Problém globální stability:} Dlouhodobé chování všech řešení
\item \textbf{Problém robustness:} Citlivost na perturbace parametrů
\end{itemize}

\begin{definition}[Ljapunovova stabilita]
Rovnovážný bod $y^*$ systému $y' = f(y)$ je \emph{stabilní}, jestliže pro každé $\epsilon > 0$ existuje $\delta > 0$ takové, že
\[
\|y(0) - y^*\| < \delta \Rightarrow \|y(t) - y^*\| < \epsilon \quad \text{pro všechna } t \geq 0
\]
\end{definition}

\begin{figure}[h]
\centering
\begin{tikzpicture}[scale=0.8]
% Stabilní uzel
\draw[->] (0,0) -- (3,0) node[right] {$x$};
\draw[->] (0,0) -- (0,3) node[above] {$y$};
\draw[red, thick, ->] (1.5,1.5) -- (1.3,1.3);
\draw[red, thick, ->] (1.5,1.5) -- (1.7,1.3);
\draw[red, thick, ->] (1.5,1.5) -- (1.3,1.7);
\draw[red, thick, ->] (1.5,1.5) -- (1.7,1.7);
\fill[red] (1.5,1.5) circle (2pt);
\node at (1.5,2) {Stabilní uzel};

% Sedlo
\draw[->] (4,0) -- (7,0) node[right] {$x$};
\draw[->] (4,0) -- (4,3) node[above] {$y$};
\draw[blue, thick, ->] (5.5,1.5) -- (5.3,1.3);
\draw[blue, thick, ->] (5.5,1.5) -- (5.7,1.7);
\draw[blue, thick, ->] (5.5,1.5) -- (5.7,1.3);
\draw[blue, thick, ->] (5.5,1.5) -- (5.3,1.7);
\fill[blue] (5.5,1.5) circle (2pt);
\node at (5.5,2) {Sedlo};
\end{tikzpicture}
\caption{Stabilní uzel versus sedlo ve fázovém prostoru}
\label{fig:stability-types}
\end{figure}

\begin{theorem}[Hartman-Grobmanova věta]
Nechť $y^*$ je hyperbolický rovnovážný bod $y' = f(y)$ (žádná vlastní čísla $Df(y^*)$ nemá nulovou reálnou část). Pak existuje homeomorfismus mezi okolím $y^*$ a okolím počátku, který zobrazuje trajektorie nelineárního systému na trajektorie linearizovaného systému $z' = Df(y^*)z$.
\qed
\end{theorem}

\begin{intuition}
V okolí hyperbolického rovnovážného bodu se nelineární systém chová kvalitativně stejně jako jeho linearizace. Stabilní uzel přitahuje trajektorie, sedlo je nestabilní.
\end{intuition}

\begin{keyinsight}
Hartman-Grobmanova věta umožňuje analyzovat lokální stabilitu pomocí linearizace. Pro kvantové systémy to znamená studium spektra Hamiltoniánu kolem stacionárních bodů.
\end{keyinsight}

\begin{summary}
\textbf{Klíčové koncepty:} Ljapunovova stabilita, hyperbolické body, linearizace \\
\textbf{Hlavní výsledky:} Hartman-Grobmanova věta, klasifikace stability \\
\textbf{Aplikace:} Analýza stability kvantových stavů, robustní systémy
\end{summary}

\spc

\subsection{Přímá metoda Ljapunova}

\begin{scaffold}
\item[] \textbf{Co umíme:} Linearizace a stabilita z předchozí sekce
\item[] \textbf{Co se naučíme:} Konstrukci Ljapunovových funkcí a princip invariance  
\item[] \textbf{K čemu to využijeme:} Analýza nelineárních kvantových systémů
\end{scaffold}

\begin{motivation}
Přímá konstrukce funkce, která měří "energii" nebo "vzdálenost" od rovnováhy. Ljapunovova metoda umožňuje analyzovat stabilitu bez explicitního řešení rovnic.
\end{motivation}

\begin{definition}[Ljapunovova funkce]
Funkce $V: D \to \mathbb{R}$ je \emph{Ljapunovova funkce} pro systém $y' = f(y)$ s rovnováhou $y^*$, jestliže:
\begin{itemize}
\item $V(y^*) = 0$ a $V(y) > 0$ pro $y \neq y^*$
\item $\dot{V}(y) = \nabla V(y) \cdot f(y) \leq 0$ na $D$
\end{itemize}
\end{definition}

\begin{theorem}[Ljapunovova věta o stabilitě]
Existuje-li Ljapunovova funkce s $\dot{V}(y) \leq 0$, pak $y^*$ je stabilní. Je-li $\dot{V}(y) < 0$ pro $y \neq y^*$, pak $y^*$ je asymptoticky stabilní.
\qed
\end{theorem}

\begin{example}[Ljapunovova funkce pro nelineární oscilátor]
Pro Duffingův oscilátor $y'' + \delta y' + \alpha y + \beta y^3 = 0$ lze konstruovat Ljapunovovu funkci:
\begin{align*}
V(y,y') &= \frac{1}{2}y'^2 + \frac{\alpha}{2}y^2 + \frac{\beta}{4}y^4 \\
\dot{V}(y,y') &= -\delta y'^2 \leq 0
\end{align*}
což dokazuje stabilitu pro $\delta > 0$.
\end{example}

\begin{application}
\begin{verbatim}
# Python: Numerická verifikace Ljapunovovy stability
def verify_lyapunov(f, V, gradV, equilibrium, domain):
    # Ověření podmínky V(y*) = 0
    assert abs(V(equilibrium)) < 1e-10
    
    # Testování dV/dt ≤ 0 v náhodných bodech
    for _ in range(1000):
        y_test = np.random.uniform(domain[0], domain[1])
        dV_dt = np.dot(gradV(y_test), f(y_test))
        if dV_dt > 1e-10:  # Porušení podmínky
            return False
    return True
\end{verbatim}
\end{application}

\begin{keyinsight}
Ljapunovova metoda umožňuje analyzovat stabilitu komplexních nelineárních systémů, kde linearizace selhává. V kvantových systémech často hraje roli Ljapunovovy funkce Hamiltonián nebo jiné zachovávající se veličiny.
\end{keyinsight}

\begin{summary}
\textbf{Klíčové koncepty:} Ljapunovova funkce, energetické metody \\
\textbf{Hlavní výsledky:} Věty o stabilitě, konstrukce Ljapunovových funkcí \\
\textbf{Aplikace:} Nelineární oscilátory, kvantová stabilita
\end{summary}

\spc

\subsection{Teorie bifurkací}

\begin{scaffold}
\item[] \textbf{Co umíme:} Stabilita a Ljapunovovy metody
\item[] \textbf{Co se naučíme:} Jak se kvalita rovnováh mění s parametry  
\item[] \textbf{K čemu to využijeme:} Kvantové fáze, přechody dynamických režimů
\end{scaffold}

\begin{motivation}
Bifurkace popisují vznik a zánik rovnováh při změně parametrů. Tyto kvalitativní změny odpovídají fázovým přechodům ve fyzikálních systémech.
\end{motivation}

\begin{theorem}[Vidličková bifurkace]
Pro systém $y' = \lambda y - y^3$:
\begin{itemize}
\item $\lambda < 0$: jeden stabilní uzel v $y = 0$
\item $\lambda > 0$: tři rovnováhy - $y = 0$ (sedlo), $y = \pm\sqrt{\lambda}$ (stabilní uzly)
\end{itemize}
\end{theorem}

\begin{theorem}[Hopfova bifurkace]
Nechť $y' = f(y,\lambda)$ má pro $\lambda = \lambda_0$ rovnováhu $y^*$ s vlastními čísly $\alpha(\lambda) \pm i\beta(\lambda)$, kde $\alpha(\lambda_0) = 0$, $\beta(\lambda_0) \neq 0$ a $\frac{d\alpha}{d\lambda}(\lambda_0) \neq 0$. Pak v okolí $\lambda_0$ vzniká limitní cyklus.
\qed
\end{theorem}

\begin{figure}[h]
\centering
\begin{tikzpicture}[scale=0.7]
% Vidličková bifurkace
\begin{scope}
\draw[->] (-1.5,0) -- (1.5,0) node[right] {$\lambda$};
\draw[->] (0,-1.5) -- (0,1.5) node[above] {$y^*$};
\draw[red, thick] (-1.5,0) -- (0,0);
\draw[blue, thick] (0,0) -- (1.5,0);
\draw[blue, thick, domain=0:1.5] plot (\x, {sqrt(\x)});
\draw[blue, thick, domain=0:1.5] plot (\x, {-sqrt(\x)});
\node at (-0.7,0.3) {Stabilní};
\node at (0.7,0.8) {Stabilní};
\node at (0.7,-0.8) {Stabilní};
\node at (0.2,-1.2) {Vidličková bifurkace};
\end{scope}

% Hopfova bifurkace
\begin{scope}[shift={(4,0)}]
\draw[->] (-1.5,0) -- (1.5,0) node[right] {$\lambda$};
\draw[->] (0,-1.5) -- (0,1.5) node[above] {Amplituda};
\draw[red, thick] (-1.5,0) -- (0,0);
\draw[blue, thick, domain=0:1.2] plot (\x, {0.7*sqrt(\x)});
\node at (-0.7,0.3) {Stabilní};
\node at (0.7,0.5) {Limitní cyklus};
\node at (0.2,-1.2) {Hopfova bifurkace};
\end{scope}
\end{tikzpicture}
\caption{Bifurkační diagramy: vidličková a Hopfova bifurkace}
\label{fig:bifurcation-diagrams}
\end{figure}

\begin{intuition}
Bifurkace představují "větvení" řešení - při překročení kritické hodnoty parametru se systém kvalitativně mění. Vidličková bifurkace vytváří nové rovnováhy, Hopfova bifurkace generuje oscilace.
\end{intuition}

\begin{keyinsight}
Bifurkační teorie poskytuje framework pro pochopení fázových přechodů v kvantových systémech. Kritické body odpovídají hodnotám parametrů, kde se mění kvalitativní chování systému.
\end{keyinsight}

\begin{summary}
\textbf{Klíčové koncepty:} Bifurkační body, kvalitativní změny \\
\textbf{Hlavní výsledky:} Vidličková a Hopfova bifurkace \\
\textbf{Aplikace:} Fázové přechody, vznik oscilací
\end{summary}

\spc

\subsection{Strukturální stabilita}

\begin{scaffold}
\item[] \textbf{Co umíme:} Bifurkace a stabilita
\item[] \textbf{Co se naučíme:} Odolnost dynamiky vůči perturbacím  
\item[] \textbf{K čemu to využijeme:} Robustní návrh kvantových systémů
\end{scaffold}

\begin{motivation}
Určit, kdy je dynamický systém "generický" a odolný vůči malým změnám. Strukturálně stabilní systémy si zachovávají kvalitativní chování při malých perturbacích.
\end{motivation}

\begin{definition}[Strukturální stabilita]
Systém $y' = f(y)$ je \emph{strukturálně stabilní} na kompaktní varietě $M$, jestliže existuje okolí $f$ v $C^1$-topologii takové, že každý systém v tomto okolí je topologicky ekvivalentní $f$.
\end{definition}

\begin{theorem}[Peixotova věta]
Na kompaktní dvourozměrné varietě je generický systém strukturálně stabilní právě tehdy, když:
\begin{itemize}
\item Všechny rovnováhy jsou hyperbolické
\item Všechny periodické orbity jsou hyperbolické  
\item Neexistují homoklinické nebo heteroklinické orbity
\end{itemize}
\qed
\end{theorem}

\begin{intuition}
Strukturálně stabilní systém je "typický" - malé změny nemění topologii fázového portrétu. Nestabilní systémy jsou "výjimečné" a citlivé na perturbace, nacházejí se na bifurkacích.
\end{intuition}

\begin{keyinsight}
Strukturální stabilita zaručuje, že matematické modely zůstávají platné i při nevyhnutelných aproximacích a šumu. V kvantových systémech to znamená robustnost vůči decoherenci a perturbacím.
\end{keyinsight}

\begin{summary}
\textbf{Klíčové koncepty:} Strukturální stabilita, generické vlastnosti \\
\textbf{Hlavní výsledky:} Peixotova věta, klasifikace stability \\
\textbf{Aplikace:} Robustní modely, kvantová odolnost
\end{summary}

\spc

\subsection{Hamiltonovské systémy}

\begin{scaffold}
\item[] \textbf{Co umíme:} Matematický fundament z Kapitol 2–3
\item[] \textbf{Co se naučíme:} Geometrický rámec pro konzervativní systémy  
\item[] \textbf{K čemu to využijeme:} Kvantové Hamiltoniány, symplektická integrace
\end{scaffold}

\begin{motivation}
Hamiltonova formulace zdůrazňuje zachování energie a symplektickou geometrii. Tento přístup je fundamentální pro kvantovou mechaniku a teorii integrabilních systémů.
\end{motivation}

\begin{definition}[Symplektická struktura]
\emph{Symplektická forma} na varietě $M$ je uzavřená nedegenerovaná 2-forma $\omega$. Dvojice $(M,\omega)$ se nazývá \emph{symplektická varieta}.
\end{definition}

\begin{theorem}[Liouville-Arnold]
Pro úplně integrabilní systém jsou invariantní tory dány $F_i = \text{konst.}$ a pohyb na nich je kvaziperiodický.
\qed
\end{theorem}

\begin{intuition}
Symplektická struktura zachovává "fázový objem" - Hamiltonovský tok je kanonickou transformací. Akční proměnné popisují invarianty pohybu, úhlové proměnné fázové posuvy.
\end{intuition}

\begin{application}
\begin{verbatim}
# Python: Symplektická integrace pro Hamiltonovské systémy
def symplectic_euler(q, p, H, dt, steps):
    # Symplektická Eulerova metoda zachovává strukturu
    trajectory = []
    for _ in range(steps):
        # Aktualizace hybnosti pak polohy
        p_new = p - dt * H.dH_dq(q, p)    # ∂H/∂q
        q_new = q + dt * H.dH_dp(q, p_new) # ∂H/∂p
        
        q, p = q_new, p_new
        trajectory.append((q.copy(), p.copy()))
    return trajectory
\end{verbatim}
\end{application}

\begin{keyinsight}
Hamiltonovská struktura je fundamentální pro kvantovou mechaniku - kvantování spočívá v nahrazení Poissonových závorek komutátory. Symplektické integrátory zachovávají tuto strukturu v numerických simulacích.
\end{keyinsight}

\begin{summary}
\textbf{Klíčové koncepty:} Symplektická geometrie, integrabilita \\
\textbf{Hlavní výsledky:} Liouville-Arnoldova věta, kanonické rovnice \\
\textbf{Aplikace:} Kvantová mechanika, nebeská mechanika
\end{summary}

\spc

\subsection{Globální atraktory a dissipativní systémy}

\begin{scaffold}
\item[] \textbf{Co umíme:} Asymptotická stabilita
\item[] \textbf{Co se naučíme:} Existence a struktura globálních atraktorů  
\item[] \textbf{K čemu to využijeme:} Dlouhodobé chování otevřených systémů
\end{scaffold}

\begin{motivation}
V dissipativních systémech řešení konverguje k malému atraktoru. Globální atraktor zachycuje veškeré dlouhodobé chování systému.
\end{motivation}

\begin{definition}[Globální atraktor]
Kompaktní množina $\mathcal{A} \subset X$ je \emph{globálním atraktorem}, jestliže:
\begin{itemize}
\item Je invariantní: $S(t)\mathcal{A} = \mathcal{A}$ pro všechna $t \geq 0$
\item Je atrahující: $\lim_{t\to\infty} \text{dist}(S(t)B, \mathcal{A}) = 0$ pro každou ohraničenou $B \subset X$
\end{itemize}
\end{definition}

\begin{theorem}[Existence globálního atraktoru]
Nechť semigroupa $S(t)$ je spojitá, uniformně kompaktní a existuje ohraničená absorbující množina. Pak existuje globální atraktor.
\qed
\end{theorem}

\begin{figure}[h]
\centering
\begin{tikzpicture}[scale=0.6]
% Lorenzův atraktor - zjednodušená schematická reprezentace
\draw[->] (-3,0) -- (3,0) node[right] {$x$};
\draw[->] (0,-2) -- (0,4) node[above] {$z$};
\draw[->] (135:2) -- (315:2) node[below] {$y$};

% Atraktor jako fraktální struktura
\draw[red, thick, rotate=20] 
    (0,0) .. controls (1,1) and (1.5,0.5) .. (2,2)
    .. controls (1.5,3) and (0.5,2.5) .. (0,2)
    .. controls (-0.5,1.5) and (-1,2) .. (-1.5,1)
    .. controls (-1,0) and (0,0.5) .. (0,0);
    
\draw[red, thick, rotate=-20] 
    (0.5,0.3) .. controls (1.2,1) and (1.7,0.8) .. (2.2,2.2);

\node[red] at (1.5,3.2) {Lorenzův atraktor};
\node at (0,-2.5) {Dimense ≈ 2.06};
\end{tikzpicture}
\caption{Lorenzův atraktor - příklad podivného atraktoru s fraktální strukturou}
\label{fig:lorenz-attractor}
\end{figure}

\begin{intuition}
Globální atraktor je "srdce" dynamického systému - všechny trajektorie se k němu nakonec přiblíží a na něm se odehrává zajímavá dynamika. Fraktální dimenze měří komplexitu této struktury.
\end{intuition}

\begin{keyinsight}
Teorie atraktorů popisuje dlouhodobé chování dissipativních systémů. V kvantových systémech mohou atraktory modelovat stacionární stavy otevřených kvantových systémů interagujících s prostředím.
\end{keyinsight}

\begin{summary}
\textbf{Klíčové koncepty:} Globální atraktor, dissipativní systémy \\
\textbf{Hlavní výsledky:} Věty o existenci, fraktální dimenze \\
\textbf{Aplikace:} Chaotické systémy, kvantová optika
\end{summary}

\spc

\subsection{Ergodická teorie dynamických systémů}

\begin{scaffold}
\item[] \textbf{Co umíme:} Globální chování systémů
\item[] \textbf{Co se naučíme:} Statistickou vlastnost trajektorií  
\item[] \textbf{K čemu to využijeme:} Kvantová ergodicita, statistická mechanika
\end{scaffold}

\begin{motivation}
Ergodická teorie propojuje dlouhodobé časové průměry s prostorovými průměry. To je fundamentální pro statistickou mechaniku a kvantovou teorii chaosu.
\end{motivation}

\begin{definition}[Ergodický systém]
Dynamický systém $(X, \mathcal{F}, \mu, T)$ je \emph{ergodický}, jestliže každá $T$-invariantní množina má míru 0 nebo 1.
\end{definition}

\begin{theorem}[Birkhoffův ergodický teorém]
Je-li $f \in L^1(X,\mu)$, pak pro skoro všechna $x \in X$ platí:
\[
\lim_{n \to \infty} \frac{1}{n} \sum_{k=0}^{n-1} f(T^k x) = \int_X f\, d\mu
\]
\qed
\end{theorem}

\begin{example}[Bernoulliho posuv jako mixing systém]
Pro Bernoulliho posuv na $\{0,1\}^\mathbb{Z}$ s mírou $\mu(\{0\}) = \mu(\{1\}) = \frac{1}{2}$ platí:
\[
\lim_{n \to \infty} \mu(T^{-n}A \cap B) = \mu(A)\mu(B)
\]
což dokazuje mixing - silnější vlastnost než ergodicita.
\end{example}

\begin{intuition}
Ergodicita znamená, že "časový průměr = prostorový průměr". Jedna typická trajektorie prozkoumá celý fázový prostor rovnoměrně. Mixing znamená, že korelace mezi stavy v čase mizí.
\end{intuition}

\begin{keyinsight}
Ergodická teorie poskytuje matematický základ pro statistickou mechaniku. V kvantových systémech ergodicita vysvětluje, proč izolované systémy dosahují tepelné rovnováhy a proč vlastní funkce chaotických systémů jsou equidistribuované.
\end{keyinsight}

\begin{summary}
\textbf{Klíčové koncepty:} Ergodicita, mixing, časové průměry \\
\textbf{Hlavní výsledky:} Birkhoffův teorém, vlastnosti mixing \\
\textbf{Aplikace:} Statistická mechanika, kvantová ergodicita
\end{summary}

\spc

\subsection*{Shrnutí kapitoly}

\begin{itemize}
\item Představili jsme pokročilé metody analýzy stability dynamických systémů
\item Ljapunovova metoda umožňuje studovat stabilitu bez explicitního řešení
\item Teorie bifurkací klasifikuje kvalitativní změny při variaci parametrů
\item Strukturální stabilita charakterizuje robustnost dynamiky
\item Hamiltonovská mechanika poskytuje geometrický framework pro konzervativní systémy
\item Teorie atraktorů popisuje dlouhodobé chování dissipativních systémů
\item Ergodická teorie spojuje časové a prostorové středování
\end{itemize}

\vspace{0.5cm}

\begin{tcolorbox}[title=\textbf{Doporučená literatura}, colback=blue!5!white, colframe=blue!75!black]
\begin{itemize}
\item \textbf{Arnold, V. I.} \emph{Mathematical Methods of Classical Mechanics}
\item \textbf{Guckenheimer, J.; Holmes, P.} \emph{Nonlinear Oscillations, Dynamical Systems, and Bifurcations of Vector Fields}  
\item \textbf{Katok, A.; Hasselblatt, B.} \emph{Introduction to the Modern Theory of Dynamical Systems}
\item \textbf{Temam, R.} \emph{Infinite-Dimensional Dynamical Systems in Mechanics and Physics}
\end{itemize}
\end{tcolorbox}

\begin{transition}
V následující kapitole rozšíříme naše porozumění na parciální diferenciální rovnice, kde infinite-dimenzionální povaha problémů vyžaduje sofistikovanější matematický aparát a přináší nové kvalitativní jevy jako disperzi, difúzi a vlnové jevy.
\end{transition}

\spc
% !TEX root = ../main.tex
\section{Teorie Stability Dynamických Systémů}
\label{sec:teorie-stability}

\blocktitle{Cíl kapitoly}
Tato kapitola systematicky buduje aparát pro analýzu stability dynamických systémů, který je klíčový pro robustní kvantitativní modelování. Pokrývá jak teoretické základy Ljapunovovy stability, tak praktické metody pro finanční modely, numerické schémata a stochastické systémy.

Provedeme čtenáře od základních pojmů stability přes pokročilé koncepty jako strukturální stabilita a input-to-state stabilita až po aplikace v analýze finančních modelů a numerických metod. Každý koncept je ilustrován na konkrétních příkladech z kvantitativních věd.

\begin{figure}[H]
\begin{tcolorbox}[title=Roadmap Kapitoly 5]
\item[] \textbf{5.1 Základy Stability} \\- Definice stability, atraktivity, klasifikace bodů
\item[] \textbf{5.2 Lineární Systémy} \\- Spektrální kritéria, Jordanova forma, explicitní řešení  
\item[] \textbf{5.3 Linearizace} \\- Hartman-Grobman věta, středová varieta, redukce dimenze
\item[] \textbf{5.4 Ljapunovova Metoda} \\- Konstrukce Ljapunovských funkcí, věty o stabilitě
\item[] \textbf{5.5 Pokročilé Koncepty} \\- LaSalleův princip, strukturální stabilita, ISS
\item[] \textbf{5.6 Fázová Analýza} \\- Klasifikace v rovině, limitní cykly, bifurkace
\item[] \textbf{5.7 Aplikace} \\- Finanční modely, numerické schémata, stochastické systémy
\end{tcolorbox}
\caption{Roadmap kapitoly 5: Teorie Stability Dynamických Systémů}
\label{fig:roadmap-chapter5}
\end{figure}

\spc

\subsection{Základní Pojmy Stability}

\subsubsection{Rovnovážné body a jejich klasifikace}

\begin{definition}[Fixní bod]
Nechť $\dot{x} = f(x)$ je autonomní dynamický systém. Bod $x^* \in \mathbb{R}^n$ se nazývá \emph{rovnovážný bod} (fixní bod), jestliže $f(x^*) = 0$.
\end{definition}

\begin{intuition}
Rovnovážné body reprezentují stacionární stavy systému, kde dynamika "stojí". V ekonomických modelech odpovídají steady-state rovnováhám, v mechanice bodům, kde jsou síly vyrovnány.
\end{intuition}

\begin{definition}[Hyperbolický bod]
Rovnovážný bod $x^*$ se nazývá \emph{hyperbolický}, jestliže všechny vlastní hodnoty Jacobiho matice $Df(x^*)$ mají nenulové reálné části.
\end{definition}

\begin{intuition}
Hyperbolicita zaručuje, že lokální chování je určeno linearizací. Nehyperbolické body jsou "hranicí stability" a vyžadují speciální analýzu pomocí středové variety. Viz Obrázek \ref{fig:hyperbolic_classification}.
\end{intuition}

\begin{figure}[H]
\centering
\includegraphics[width=0.8\textwidth]{hyperbolic_classification.pdf}
\caption{Klasifikace rovnovážných bodů podle spektra Jacobiho matice}
\label{fig:hyperbolic_classification}
\end{figure}

\subsubsection{Stabilita dle Ljapunova - přesné definice}

\begin{definition}[Lokální stabilita]
Rovnovážný bod $x^*$ je \emph{lokálně stabilní} (ve smyslu Ljapunova), jestliže:
\[
\forall \epsilon > 0, \exists \delta > 0: \|x(0) - x^*\| < \delta \implies \|x(t) - x^*\| < \epsilon \quad \forall t \geq 0.
\]
\end{definition}

\begin{intuition}
δ-ε podmínka znamená: "Pro libovolně malou toleranci ε existuje výchozí okolí δ takové, že trajektorie nikdy neopustí ε-okí." Toto je základní koncept kvalitativní robustness.
\end{intuition}

\begin{example}[Stabilita Cobb-Douglasova modelu]
Uvažujme dynamický systém popisující ekonomiku s Cobb-Douglasovou produkční funkcí:
\begin{align*}
\dot{k} &= s k^\alpha - (\delta + n)k, \\
k^* &= \left(\frac{s}{\delta + n}\right)^{1/(1-\alpha)}.
\end{align*}
Rovnovážný bod $k^*$ je lokálně stabilní pro $\alpha < 1$, neboť $Df(k^*) = \alpha s (k^*)^{\alpha-1} - (\delta + n) < 0$.
\end{example}

\begin{definition}[Globální stabilita]
Rovnovážný bod $x^*$ je \emph{globálně stabilní}, jestliže je stabilní a navíc:
\[
\lim_{t \to \infty} x(t) = x^* \quad \text{pro všechny počáteční podmínky } x(0) \in \mathbb{R}^n.
\]
\end{definition}

\begin{example}[Globální stabilita v ekonomických modelech]
Model Solowova růstu s konstantními výnosy z rozsahu vykazuje globální stabilitu. Ljapunovova funkce $V(k) = (k - k^*)^2$ splňuje $\dot{V} < 0$ pro všechna $k > 0$, což zaručuje konvergenci z libovolné pozitivní počáteční podmínky.
\end{example}

\subsubsection{Atraktivita, asymptotická a exponenciální stabilita}

\begin{definition}[Atraktivita]
Rovnovážný bod $x^*$ je \emph{lokálně atrahující}, jestliže existuje $\delta > 0$ takové, že:
\[
\|x(0) - x^*\| < \delta \implies \lim_{t \to \infty} x(t) = x^*.
\]
\end{definition}

\begin{definition}[Asymptotická a exponenciální stabilita]
Rovnovážný bod $x^*$ je \emph{asymptoticky stabilní}, jestliže je stabilní a atrahující. Je \emph{exponenciálně stabilní}, jestliže existují $C, \lambda > 0$ takové, že:
\[
\|x(t) - x^*\| \leq C e^{-\lambda t} \|x(0) - x^*\| \quad \forall t \geq 0.
\]
\end{definition}

\begin{figure}[H]
\centering
\includegraphics[width=0.8\textwidth]{convergence_types.pdf}
\caption{Srovnání asymptotické a exponenciální konvergence}
\label{fig:convergence_types}
\end{figure}

\begin{application}[Rychlost konvergence v Ramseyově modelu]
V Ramsey-Cass-Koopmansově modelu konverguje spotřeba k steady-state exponenciálně:
\[
|c(t) - c^*| \approx e^{-\theta t}|c(0) - c^*|,
\]
kde $\theta$ je určeno mezní mírou substituce a časovou preferencí. Tato rychlost konvergence je klíčová pro kalibraci modelu na empirická data.
\end{application}

\spc

\subsection{Stabilita Lineárních Systémů}

\subsubsection{Autonomní lineární systémy a jejich explicitní řešení}

\begin{theorem}[Explicitní řešení lineárního systému]
Nechť $\dot{x} = Ax$ je autonomní lineární systém. Pak řešení s počáteční podmínkou $x(0) = x_0$ je dáno vztahem:
\begin{align*}
x(t) &= e^{At}x_0, \\
\text{kde } e^{At} &= \sum_{k=0}^\infty \frac{(At)^k}{k!}.
\end{align*}
\end{theorem}

\begin{proof}
Přímým dosazením ověříme:
\begin{align*}
\dot{x}(t) &= \frac{d}{dt}\left(e^{At}x_0\right) = A e^{At}x_0 = Ax(t), \\
x(0) &= e^{A\cdot 0}x_0 = I x_0 = x_0. \qed
\end{align*}
\end{proof}

\subsubsection{Vztah spektra a stability}

\begin{theorem}[Stabilita lineárních systémů]
Nechť $\dot{x} = Ax$ je lineární systém. Pak:
\begin{itemize}
\item Systém je stabilní právě tehdy, když všechny vlastní hodnoty $A$ mají nekladné reálné části a vlastní hodnoty s nulovou reálnou částí mají algebraickou násobnost rovnou geometrické.
\item Systém je asymptoticky stabilní právě tehdy, když všechny vlastní hodnoty $A$ mají záporné reálné části.
\item Systém je exponenciálně stabilní právě tehdy, když je asymptoticky stabilní.
\end{itemize}
\end{theorem}

\begin{keyinsight}
Spektrum matice $A$ určuje asymptotickou rychlost konvergence. Vlastní hodnota s největší reálnou částí (dominantní vlastní hodnota) určuje pomalost konvergence systému k rovnováze. Pro exponenciální stabilitu musí být $\max_i \mathrm{Re}(\lambda_i) < 0$.
\end{keyinsight}

\subsubsection{Jordanova forma a její role v multiplicitě}

\begin{theorem}[Jordanova forma a stabilita]
Nechť $J$ je Jordanova forma matice $A$. Systém $\dot{x} = Ax$ je stabilní právě tehdy, když všechny Jordanovy bloky odpovídající vlastním číslům s nezápornou reálnou částí jsou $1 \times 1$.
\end{theorem}

\begin{proofsketch}
Jordanova forma umožňuje explicitní výpočet maticové exponenciály. Pro Jordanův blok $J_k(\lambda)$ platí:
\begin{align*}
e^{J_k(\lambda)t} &= e^{\lambda t} \begin{bmatrix}
1 & t & \frac{t^2}{2!} & \cdots & \frac{t^{k-1}}{(k-1)!} \\
0 & 1 & t & \cdots & \frac{t^{k-2}}{(k-2)!} \\
\vdots & \vdots & \ddots & \ddots & \vdots \\
0 & 0 & \cdots & 1 & t \\
0 & 0 & \cdots & 0 & 1
\end{bmatrix}.
\end{align*}
Polynomiální členy rostou, pokud $\mathrm{Re}(\lambda) \geq 0$. \qed
\end{proofsketch}

\begin{keyinsight}
Polynomiální členy při defektech (když geometrická násobnost < algebraická) vedou k členům tvaru $t^k e^{\lambda t}$. I pro $\mathrm{Re}(\lambda) = 0$ mohou tyto polynomiální faktory způsobit nestabilitu, což je klíčové pro systémy s násobnými vlastními čísly na imaginární ose.
\end{keyinsight}

\spc

\subsection{Linearizace Nelineárních Systémů}

\subsubsection{Linearizace v okolí rovnovážného bodu}

\begin{definition}[Jacobiho matice]
Nechť $\dot{x} = f(x)$ je nelineární systém s rovnovážným bodem $x^*$. \emph{Jacobiho matice} v $x^*$ je definována jako:
\[
J = Df(x^*) = \left[\frac{\partial f_i}{\partial x_j}(x^*)\right]_{i,j=1}^n.
\]
\end{definition}

\subsubsection{Hartman-Grobmanova věta o linearizaci}

\begin{theorem}[Hartman-Grobman]
Nechť $x^*$ je hyperbolický rovnovážný bod systému $\dot{x} = f(x)$, kde $f \in C^1$. Pak existuje homeomorfismus $h$ definovaný na okolí $x^*$, který zobrazuje trajektorie nelineárního systému na trajektorie linearizovaného systému $\dot{y} = Jy$.
\end{theorem}

\begin{proofsketch}
\begin{itemize}
\item Konstrukce h pomocí integrálních rovnic podél trajektorií
\item Použití Banachovy věty o pevném bodě pro malá okolí
\item Důkaz, že h je homeomorfismus zachovávající směr času
\end{itemize}
\end{proofsketch}

\begin{figure}[H]
\centering
\includegraphics[width=0.7\textwidth]{hartman_grobman.pdf}
\caption{Lokální homeomorfismus mezi nelineárním a linearizovaným systémem}
\label{fig:hartman_grobman}
\end{figure}

\subsubsection{Středová varieta a redukce dimenze}

\begin{definition}[Středová varieta]
Nechť $x^*$ je rovnovážný bod s vlastními čísly majícími nulové reálné části. \emph{Středová varieta} je invariantní manifold tangentní ke středovému vlastnímu podprostoru.
\end{definition}

\begin{theorem}[Existence středové variety]
Za předpokladu dostatečné hladkosti $f$ existuje lokálně hladká středová varieta dimenze rovné počtu vlastních čísel s nulovou reálnou částí.
\end{theorem}

\begin{algorithm}[Numerická konstrukce středové variety]
\label{alg:center_manifold}
\begin{enumerate}
\item Vypočti Jacobiho matici $J = Df(x^*)$ a diagonalizuj ji
\item Identifikuj středový podprostor $E^c$ odpovídající vlastním číslům s $\mathrm{Re}(\lambda) = 0$
\item Polož ansatz pro středovou varietu: $y = h(z)$, kde $z \in E^c$
\item Řeš invariantní podmínku: $Dh(z) \cdot g(z) = f(h(z))$
\item Aplikuj Newtonovu metodu pro refinemet tvaru $h(z)$
\end{enumerate}
\end{algorithm}

\begin{application}[Redukce dimenze v Hopfově bifurkaci]
Pro systém exhibující Hopfovu bifurkaci s dvourozměrnou středovou varietou lze analýzu redukovat z $\mathbb{R}^n$ na $\mathbb{R}^2$ pomocí Algoritmu \ref{alg:center_manifold}, což podstatně zjednodušuje studium vzniku limitních cyklů.
\end{application}

\spc

\subsection{Přímá Metoda Ljapunova}

\subsubsection{Ljapunovské funkce a jejich vlastnosti}

\begin{definition}[Ljapunovská funkce]
Spojitě diferencovatelná funkce $V: U \to \mathbb{R}$ na okolí $U$ rovnovážného bodu $x^*$ se nazývá \emph{Ljapunovská funkce}, jestliže:
\begin{enumerate}
\item $V(x^*) = 0$ a $V(x) > 0$ pro $x \neq x^*$ (kladně definitní)
\item $\dot{V}(x) = \nabla V(x) \cdot f(x) \leq 0$ pro $x \in U \setminus \{x^*\}$
\end{enumerate}
\end{definition}

\subsubsection{Hlavní teorémy stability a instability}

\begin{theorem}[Ljapunovova věta o stabilitě]
Existuje-li Ljapunovská funkce, pak $x^*$ je stabilní.
\end{theorem}

\begin{theorem}[Ljapunovova věta o asymptotické stabilitě]
Je-li navíc $\dot{V}(x) < 0$ pro $x \neq x^*$, pak $x^*$ je asymptoticky stabilní.
\end{theorem}

\subsubsection{Konstrukce Ljapunovských funkcí}

\begin{theorem}[Ljapunovova rovnice]
Pro lineární systém $\dot{x} = Ax$ je funkce $V(x) = x^T P x$ Ljapunovskou funkcí, jestliže $P$ je symetrická pozitivně definitní matice splňující:
\begin{align*}
A^T P + P A = -Q,
\end{align*}
kde $Q$ je libovolná symetrická pozitivně definitní matice.
\end{theorem}

\begin{pseudocode}[Řešení Ljapunovovy rovnice v Pythonu]
\begin{verbatim}
import scipy.linalg as la
import numpy as np

def solve_lyapunov(A, Q):
    """
    Řeší Ljapunovovu rovnici AᵀP + PA = -Q
    pro symetrické pozitivně definitní matice.
    
    Parameters:
    A: systémová matice (n x n)
    Q: pravostranná matice (n x n)
    
    Returns:
    P: řešení Ljapunovovy rovnice
    """
    # Kontrola stability A
    if np.max(np.real(np.linalg.eigvals(A))) >= 0:
        raise ValueError("Matice A není stabilní")
    
    # Řešení spojité Ljapunovovy rovnice
    P = la.solve_continuous_lyapunov(A.T, -Q)
    return P

# Příklad použití pro stabilní systém
A = np.array([[-2, 1], [0, -1]])
Q = np.eye(2)  # Volba jednotkové matice
P = solve_lyapunov(A, Q)
\end{verbatim}
\end{pseudocode}

\begin{intuition}
Volba matice $Q$ ovlivňuje tvar Ljapunovské funkce. Jednotková matice $Q = I$ často vede k dobře podmíněným řešením. V praxi lze volit $Q$ tak, aby reflektovala důležité proměnné systému.
\end{intuition}

\begin{example}[Hamiltonovský oscilátor s tlumením]
Uvažujme systém:
\begin{align*}
\dot{q} &= p, \\
\dot{p} &= -q - \delta p.
\end{align*}
Ljapunovova funkce $V(q,p) = \frac{1}{2}(q^2 + p^2)$ splňuje $\dot{V} = -\delta p^2 \leq 0$, což dokazuje stabilitu pro $\delta > 0$.
\end{example}

\spc

\subsection{Pokročilé Koncepty Stability}

\subsubsection{LaSalleův princip invariance}

\begin{theorem}[LaSalleův princip invariance]
Nechť $V$ je Ljapunovská funkce na kompaktní invariantní množině $\Omega$ a nechť $E = \{x \in \Omega : \dot{V}(x) = 0\}$. Pak každé řešení začínající v $\Omega$ konverguje k největší invariantní množině obsažené v $E$.
\end{theorem}

\begin{figure}[H]
\centering
\includegraphics[width=0.7\textwidth]{lasalle_invariance.pdf}
\caption{Ilustrace invariantní množiny kde $\dot{V} = 0$}
\label{fig:lasalle_invariance}
\end{figure}

\begin{proofsketch}
\begin{itemize}
\item Konstrukce Ljapunovské funkce $V$ s $\dot{V} \leq 0$
\item Identifikace množiny $E$ kde $\dot{V} = 0$
\item Analýza největší invariantní množiny v $E$
\item Aplikace věty o limitní množině
\end{itemize}
\end{proofsketch}

\subsubsection{Strukturální stabilita a robustní systémy}

\begin{theorem}[Peixotova věta]
Generický dynamický systém na kompaktní varietě dimenze 2 je strukturálně stabilní právě tehdy, když:
\begin{itemize}
\item Všechny rovnovážné body jsou hyperbolické
\item Všechny periodické orbity jsou hyperbolické
\item Neexistuje trajektorie spojující sedlové body
\end{itemize}
\end{theorem}

\begin{keyinsight}
Strukturální stabilita zaručuje, že malé perturbace nemění kvalitativní chování systému. Tato vlastnost je klíčová pro robustní kvantitativní modely, které musí být odolné vůči malým změnám parametrů a měření šumu.
\end{keyinsight}

\subsubsection{Input-to-state stabilita a stabilita vůči poruchám}

\begin{definition}[Input-to-State Stabilita (ISS)]
Systém $\dot{x} = f(x,u)$ je \emph{input-to-state stabilní}, jestliže existují funkce $\beta \in \mathcal{KL}$ a $\gamma \in \mathcal{K}$ takové, že:
\[
\|x(t)\| \leq \beta(\|x(0)\|, t) + \gamma\left(\sup_{0\leq \tau \leq t} \|u(\tau)\|\right).
\]
\end{definition}

\begin{proofsketch}[Důkaz ISS pomocí Ljapunovské funkce]
\begin{itemize}
\item Konstrukce Ljapunovské funkce $V(x)$ splňující $\alpha_1(\|x\|) \leq V(x) \leq \alpha_2(\|x\|)$
\item Odhad derivace: $\dot{V} \leq -\alpha_3(\|x\|) + \sigma(\|u\|)$
\item Aplikace comparačního lemmatu
\item Odvození konečného odhadu pomocí $\mathcal{KL}$ a $\mathcal{K}$ funkcí
\end{itemize}
\end{proofsketch}

\begin{application}[ISS analýza pro lineární SDE]
Uvažujme systém $\dot{x} = Ax + Bw + \sigma(x)dW_t$, kde $w$ je deterministická porucha a $dW_t$ je Wienerův proces. Je-li $A$ Hurwitzova, pak systém je ISS s:
\[
\gamma(r) = \frac{\|B\|}{\lambda_{\min}(Q)}r + \frac{\|\sigma\|^2}{2\lambda_{\min}(Q)}r^2,
\]
kde $A^T P + PA = -Q$. Tento výsledek je klíčový pro analýzu robustnosti finančních modelů vůči tržnímu šumu.
\end{application}

\spc

\subsection{Fázová Analýza v Rovině}

\subsubsection{Klasifikace lineárních systémů v $\mathbb{R}^2$}

\begin{theorem}[Klasifikace lineárních systémů v rovině]
Nechť $\dot{x} = Ax$ s $A \in \mathbb{R}^{2\times 2}$. Podle vlastních čísel rozlišujeme:
\begin{itemize}
\item \textbf{Uzel}: reálná vlastní čísla stejného znaménka
\item \textbf{Sedlo}: reálná vlastní čísla opačných znamének  
\item \textbf{Ohnisko}: komplexní vlastní čísla s nenulovou reálnou částí
\item \textbf{Střed}: čistě imaginární vlastní čísla
\end{itemize}
\end{theorem}

\subsubsection{Limitační cykly a Poincarého-Bendixsonova věta}

\begin{theorem}[Poincarého-Bendixson]
Nechť $\dot{x} = f(x)$ je spojitý dynamický systém v $\mathbb{R}^2$ a nechť $\omega$-limita množina trajektorie je neprázdná, kompaktní a neobsahuje rovnovážné body. Pak je tato množina periodickou orbitou.
\end{theorem}

\begin{figure}[H]
\centering
\includegraphics[width=0.8\textwidth]{poincare_map.pdf}
\caption{Poincarého mapa a průřez pro analýzu limitního cyklu}
\label{fig:poincare_map}
\end{figure}

\begin{application}[Numerická detekce limitního cyklu]
\label{app:limit_cycle_detection}
\begin{enumerate}
\item Zvol Poincarého průřez $\Sigma$ transverzální k předpokládanému cyklu
\item Definuj Poincarého mapu $P: \Sigma \to \Sigma$ pomocí integrace trajektorie
\item Hledej pevný bod $p^* = P(p^*)$ pomocí Newtonovy metody
\item Analyzuj stabilitu pomocí derivace $DP(p^*)$:
   \begin{itemize}
   \item $|DP(p^*)| < 1$: stabilní limitní cyklus
   \item $|DP(p^*)| > 1$: nestabilní limitní cyklus
   \end{itemize}
\end{enumerate}
\end{application}

\subsubsection{Úvod do bifurkací v rovině}

\begin{definition}[Základní bifurkace]
\begin{itemize}
\item \textbf{Saddle-node}: Zánik dvojice rovnovážných bodů
\item \textbf{Transkritická}: Výměna stability mezi dvěma rovnovážnými body  
\item \textbf{Pitchfork}: Vznik nebo zánik symetrických větví
\item \textbf{Hopfova}: Vznik limitního cyklu z rovnovážného bodu
\end{itemize}
\end{definition}

\spc

\subsection{Aplikace Stability v Kvantitativních Vědách}

\subsubsection{Stabilita finančních a ekonomických modelů}

\begin{application}[Stabilita Black-Scholesova modelu]
Uvažujme stochastickou diferenciální rovnici pro cenový proces:
\[
dS_t = \mu S_t dt + \sigma S_t dW_t.
\]
Deterministická část $\dot{S} = \mu S$ má rovnovážný bod $S^* = 0$, který je nestabilní pro $\mu > 0$, což odpovídá exponenciálnímu růstu cen.
\end{application}

\begin{definition}[Euler-Maruyama metoda]
Numerická schéma pro SDE $dX_t = f(X_t)dt + g(X_t)dW_t$:
\[
X_{n+1} = X_n + f(X_n)\Delta t + g(X_n)\Delta W_n,
\]
kde $\Delta W_n \sim \mathcal{N}(0, \Delta t)$.
\end{definition}

\begin{theorem}[Stabilita Euler-Maruyama metody]
Pro lineární SDE $dX_t = aX_t dt + bX_t dW_t$ je Euler-Maruyama schéma stabilní, jestliže:
\[
|1 + a\Delta t|^2 + b^2\Delta t < 1.
\]
\end{theorem}

\begin{code}[Implementace Euler-Maruyama v Pythonu]
\begin{verbatim}
import numpy as np
import matplotlib.pyplot as plt

def euler_maruyama(f, g, x0, T, dt, params):
    """
    Euler-Maruyama metoda pro SDE.
    
    Parameters:
    f: drift funkce
    g: difúzní funkce  
    x0: počáteční podmínka
    T: časový horizont
    dt: časový krok
    params: parametry modelu
    
    Returns:
    t: časová mřížka
    x: numerická trajektorie
    """
    n_steps = int(T/dt)
    t = np.linspace(0, T, n_steps+1)
    x = np.zeros(n_steps+1)
    x[0] = x0
    
    for i in range(n_steps):
        dW = np.sqrt(dt) * np.random.normal()
        x[i+1] = x[i] + f(x[i], *params)*dt + g(x[i], *params)*dW
    
    return t, x

# Příklad: Geometrický Brownův pohyb
def gbm_drift(S, mu):
    return mu * S

def gbm_diffusion(S, sigma):
    return sigma * S

# Simulace
t, S = euler_maruyama(gbm_drift, gbm_diffusion, 
                      x0=100, T=1, dt=0.001, 
                      params=(0.05, 0.2))
\end{verbatim}
\end{code}

\begin{keyinsight}
Parametry volatility přímo ovlivňují stabilitu řešení - vysoká volatilita může vést k numerické nestabilitě a vyžaduje menší časové kroky v simulacích Monte Carlo. Stabilní volba $\Delta t$ je klíčová pro konzistentní výsledky.
\end{keyinsight}

\subsubsection{Stabilita numerických schémat}

\begin{definition}[A-stabilita]
Numerické schéma pro ODE je \emph{A-stabilní}, jestliže jeho oblast stability obsahuje celou levou komplexní polorovinu $\{z \in \mathbb{C} : \mathrm{Re}(z) < 0\}$.
\end{definition}

\begin{theorem}[Dahlquistova bariéra]
A-stabilní lineární multistep metoda může mít maximálně řád 2. Nejvýše druhého řádu je i A-stabilní implicitní Runge-Kuttova metoda.
\end{theorem}

\begin{proofsketch}
\begin{itemize}
\item Analýza charakteristického polynomu numerického schématu
\item Aplikace maximum principu pro komplexní funkce
\item Důkaz, že vyšší řády vedou k neomezeným oblastem stability
\item Konstrukce protipříkladů pro řády vyšší než 2
\end{itemize}
\end{proofsketch}

\begin{intuition}
Explicitní metody mají omezené oblasti stability, což v praxi znamená nutnost velmi malých časových kroků pro stiff rovnice. Implicitní metody obětují řád přesnosti za lepší stabilní vlastnosti, což je výhodné pro dlouhodobé simulace finančních modelů.
\end{intuition}

\subsubsection{Stabilita kvantových a stochastických systémů}

\begin{definition}[Středně kvadratická stabilita]
Stochastický systém $dX_t = f(X_t)dt + g(X_t)dW_t$ je \emph{středně kvadraticky stabilní}, jestliže:
\[
\lim_{t \to \infty} \mathbb{E}[\|X_t\|^2] = 0.
\]
\end{definition}

\begin{theorem}[Ljapunova metoda pro SDE]
Nechť existuje funkce $V(x)$ taková, že pro infinitesimální generátor $\mathcal{L}$ platí:
\[
\mathcal{L}V(x) = \nabla V \cdot f + \frac{1}{2}\mathrm{tr}(g^T \nabla^2 V g) \leq -cV(x).
\]
Pak je systém exponenciálně středně kvadraticky stabilní.
\end{theorem}

\begin{example}[Ornstein-Uhlenbeckův proces]
Pro OU proces $dX_t = -\theta X_t dt + \sigma dW_t$:
\begin{align*}
\mathbb{E}[X_t] &= X_0 e^{-\theta t} \to 0, \\
\mathrm{Var}(X_t) &= \frac{\sigma^2}{2\theta}(1 - e^{-2\theta t}) \to \frac{\sigma^2}{2\theta}.
\end{align*}
Systém je středně kvadraticky stabilní pro $\theta > 0$.
\end{example}

\begin{figure}[H]
\centering
\includegraphics[width=0.8\textwidth]{ou_process_empirical.pdf}
\caption{Empirický časový průběh OU procesu s fit exponenciálního rozkladu}
\label{fig:ou_empirical}
\end{figure}

\spc

\subsection*{Shrnutí kapitoly}

\begin{table}[H]
\centering
\begin{tabular}{|p{0.22\textwidth}|p{0.28\textwidth}|p{0.4\textwidth}|}
\hline
\textbf{Metoda} & \textbf{Typ systému} & \textbf{Kvantitativní aplikace} \\
\hline
\hyperref[sec:teorie-stability]{Linearizace a spektrální analýza} & Lineární a lokálně nelineární systémy & Rychlá analýza stability ekonomických modelů \\
\hline
\hyperref[sec:teorie-stability]{Ljapunovova přímá metoda} & Globálně nelineární systémy & Důkazy stability tradingových strategií \\
\hline
\hyperref[sec:teorie-stability]{Input-to-State Stabilita (ISS)} & Systémy s vnějšími poruchami & Robustnost vůči tržnímu šumu a nejistotě \\
\hline
\hyperref[sec:teorie-stability]{Bifurkační analýza} & Nelineární systémy s parametry & Detekce regime changes v finančních datech \\
\hline
\hyperref[sec:teorie-stability]{Numerická stabilita analýza} & Diskrétní schémata & Optimalizace časových kroků v Monte Carlo simulacích \\
\hline
\hyperref[sec:teorie-stability]{Stochastická stabilita} & SDE a jump procesy & Analýza riika v kvantových a finančních modelech \\
\hline
\end{tabular}
\caption{Přehled metod stability a jejich aplikací v kvantitativních vědách}
\label{tab:stability_methods}
\end{table}

\begin{tcolorbox}[title=Doporučená literatura: Teoretické základy, floatplacement=H]
\begin{itemize}
\item Khalil, H.K. - \emph{Nonlinear Systems} (3rd ed.) - Komplexní pokrytí Ljapunovovy teorie
\item Evans, L.C. - \emph{Partial Differential Equations} - Propojení s PDE a regularitou
\end{itemize}
\begin{center}
\begin{tabular}{m{0.2\textwidth}m{0.2\textwidth}}
\includegraphics[width=0.35\textwidth]{qr_khalil.pdf} & \includegraphics[width=0.35\textwidth]{qr_evans.pdf} \\
Khalil (2015) & Evans (2010) \\
\end{tabular}
\end{center}
\end{tcolorbox}

\begin{tcolorbox}[title=Doporučená literatura: Numerické metody, floatplacement=H]
\begin{itemize}
\item Oksendal, B. - \emph{Stochastic Differential Equations} - Stochastická stabilita
\item Hairer, E., Wanner, G. - \emph{Solving Ordinary Differential Equations II} - Numerická stabilita
\end{itemize}
\begin{center}
\begin{tabular}{m{0.2\textwidth}m{0.2\textwidth}}
\includegraphics[width=0.35\textwidth]{qr_oksendal.pdf} & \includegraphics[width=0.35\textwidth]{qr_hairer.pdf} \\
Oksendal (2013) & Hairer \& Wanner (2010) \\
\end{tabular}
\end{center}
\end{tcolorbox}

\begin{transition}
Pro Kapitolu 6 (Parciální Diferenciální Rovnice v Kvantitativních Vědách) bude klíčové navázat na teorii Sobolevových prostorů z Kapitoly 2. Tyto prostory poskytují přirozený rámec pro formulaci a analýzu slabých řešení PDE, která jsou základem moderních numerických metod jako konečné prvky a spektrální metody. Stabilita řešení PDE bude vyžadovat rozšíření Ljapunovovy teorie do nekonečně-dimenzionálních prostorů.
\end{transition}
% !TEX root = ../main.tex
\section{Pokročilé Teoretické Koncepty a Syntéza}
\label{sec:pokrocile-teoreticke-koncepty}

\blocktitle{Cíl kapitoly}
Tato kapitola představuje syntézu a vyvrcholení teoretického fundamentu, propojující koncepty z předchozích kapitol do jednotného rámce pokročilé teorie dynamických systémů. Zaměřujeme se na bifurkace jako mechanismy kvalitativních změn, geometrické struktury Hamiltonovských systémů a statistický popis komplexní dynamiky.

Provedeme čtenáře od lokálních bifurkací rovnováh přes globální bifurkační jevy až k ergodické teorii a teorii chaosu. Každý koncept je ilustrován na konkrétních aplikacích z kvantitativních věd s důrazem na finanční modelování a analýzu komplexních systémů.


\spc

\subsection{Pokročilá Teorie Bifurkací pro Kvantitativní Modely}

\subsubsection{Lokální Bifurkace Rovnováh}

\begin{definition}[Bifurkační bod]
Bod $(\mu_0, x_0)$ v parametrickém prostoru se nazývá \emph{bifurkační bod}, jestliže pro každé okolí $U$ bodu $(\mu_0, x_0)$ existuje $(\mu, x) \in U$ takové, že kvalitativní chování systému $\dot{x} = f(x, \mu)$ se liší od chování v $(\mu_0, x_0)$.
\end{definition}

\begin{theorem}[Saddle-node bifurkace]
Uvažujme jednodimenzionální systém $\dot{x} = f(x, \mu)$ s následujícími podmínkami v $(0,0)$:
\begin{align*}
f(0,0) &= 0, \quad \frac{\partial f}{\partial x}(0,0) = 0, \\
\frac{\partial f}{\partial \mu}(0,0) &\neq 0, \quad \frac{\partial^2 f}{\partial x^2}(0,0) \neq 0.
\end{align*}
Pak existuje lokální difeomorfismus převádějící systém na normální formu $\dot{y} = \mu - y^2$.
\end{theorem}

\begin{proofsketch}
\begin{itemize}
\item Aplikace implicitní funkční věty k eliminaci vyšších členů
\item Použití center manifold reduction
\item Transformace do normální formy pomocí near-identity transformace
\end{itemize}
\end{proofsketch}

\begin{application}[Kritické body v ekonomických modelech]
V modelu ekonomického růstu s produkční funkcí $Y = K^\alpha L^{1-\alpha}$ nastává saddle-node bifurkace při kritické hodnotě úsporové míry $s_c$, kdy mizí rovnovážný bod a systém prochází kvalitativní změnou růstového režimu.
\end{application}

\subsubsection{Hopfova Bifurkace a Vznik Oscilací}

\begin{theorem}[Hopfova bifurkace]
Nechť $\dot{x} = f(x, \mu)$ je systém s rovnovážným bodem $x_0(\mu)$ a nechť Jacobiho matice $Df(x_0(\mu), \mu)$ má komplexně sdružená vlastní čísla $\lambda(\mu), \bar{\lambda}(\mu)$ splňující:
\begin{align*}
\mathrm{Re}(\lambda(0)) &= 0, \quad \mathrm{Im}(\lambda(0)) \neq 0, \\
\frac{d}{d\mu}\mathrm{Re}(\lambda(\mu))\big|_{\mu=0} &> 0.
\end{align*}
Pak za obecných podmínek nastává Hopfova bifurkace vedoucí k vzniku limitního cyklu.
\end{theorem}


\begin{application}[Vznik business cyklů]
V Kaldorově modelu obchodního cyklu vede Hopfova bifurkace k vzniku endogenních oscilací v ekonomické aktivitě. Kritický parametr je určen mezní mírou investic a úspor, přičemž vznikající limitní cyklus popisuje pravidelnost ekonomických cyklů.
\end{application}

\subsubsection{Globální Bifurkace a Konexe}

\begin{definition}[Homoklinická a heteroklinická orbita]
\emph{Homoklinická orbita} je trajektorie konvergující k témuž rovnovážnému bodu pro $t \to \pm\infty$. \emph{Heteroklinická orbita} spojuje dva různé rovnovážné body.
\end{definition}

\begin{theorem}[Šilnikovova věta]
Nechť trojdimenzionální systém má homoklinickou orbitu k sedlovému bodu s vlastními čísly $\gamma, \rho \pm i\omega$ splňujícími $\gamma > -\mathrm{Re}(\rho) > 0$. Pak v okolí homoklinické orbity existuje nekonečně mnoho nestabilních periodických orbit.
\end{theorem}

\begin{application}[Regime changes v klimatických modelech]
Homoklinické bifurkace v komplexních klimatických modelech vysvětlují náhlé přechody mezi různými klimatickými režimy (např. mezi dobou ledovou a meziledovou), charakterizované hysteretickým chováním a citlivostí na počáteční podmínky.
\end{application}

\spc

\subsection{Hamiltonovské Systémy a Symplektická Geometrie}

\subsubsection{Symplektická Geometrie a Hamiltonovská Formulace}

\begin{definition}[Symplektická struktura]
\emph{Symplektická forma} na varietě $M$ je diferenciální 2-forma $\omega$ která je:
\begin{itemize}
\item Uzavřená: $d\omega = 0$
\item Nedegenerovaná: $\omega(v, w) = 0\ \forall w \implies v = 0$
\end{itemize}
\end{definition}

\begin{theorem}[Darbouxova věta]
Lokálně existují souřadnice $(q^1, \dots, q^n, p_1, \dots, p_n)$ takové, že:
\[
\omega = \sum_{i=1}^n dq^i \wedge dp_i.
\]
Tyto souřadnice se nazývají \emph{kanonické}.
\end{theorem}

\begin{definition}[Hamiltonův systém]
Nechť $(M, \omega)$ je symplektická varieta a $H: M \to \mathbb{R}$ je Hamiltonova funkce. \emph{Hamiltonovy rovnice} jsou:
\[
\dot{x} = X_H(x), \quad \text{kde } \omega(X_H, \cdot) = -dH.
\]
V kanonických souřadnicích: $\dot{q}^i = \frac{\partial H}{\partial p_i}, \quad \dot{p}_i = -\frac{\partial H}{\partial q^i}$.
\end{definition}

\begin{application}[Konzervativní mechanické systémy]
V klasické mechanice popisuje Hamiltonova formulace konzervativní systémy, kde $H$ představuje celkovou energii. Symplektická struktura zaručuje zachování fázového objemu (Liouvilleova věta) a poskytuje přirozený rámec pro kvantování.
\end{application}

\subsubsection{Integrabilní Systémy a Úplná Separability}

\begin{definition}[Liouvilleova integrabilita]
Hamiltonův systém s $n$ stupni volnosti je \emph{Liouvilleovsky integrabilní}, jestliže existuje $n$ funkcí $F_1, \dots, F_n$ splňujících:
\begin{itemize}
\item $\{F_i, F_j\} = 0$ (komutují v Poissonově závorce)
\item $dF_1 \wedge \dots \wedge dF_n \neq 0$ (jsou nezávislé)
\end{itemize}
\end{definition}

\begin{theorem}[Arnold-Liouville]
Nechť Hamiltonův systém je integrabilní s integrály $F_1, \dots, F_n$. Pak:
\begin{itemize}
\item Invariantní variety $M_f = \{F_i = f_i\}$ jsou torusy
\item Na $M_f$ existují akčně-úhlové proměnné $(I, \theta)$
\item Pohyb je kvaziperiodický na torusech
\end{itemize}
\end{theorem}

\begin{application}[Kalibrace finančních modelů]
Integrabilní systémy v matematické finance umožňají exaktní řešení pro ceny derivátů a optimalizační problémy. Např. Black-Scholesova rovnice může být transformována na heat rovnici, která je integrabilní.
\end{application}

\subsubsection{Teorie Perturbací a KAM Teorie}

\begin{theorem}[KAM teorie]
Nechť $H(I, \theta) = H_0(I) + \epsilon H_1(I, \theta)$ je malá perturbace integrabilního Hamiltoniánu. Pak pro dostatečně malé $\epsilon$ a Diophantinské frekvence přežije většina invariantních torů.
\end{theorem}

\begin{proofsketch}
\begin{itemize}
\item Konstrukce kanonických transformací eliminující závislost na úhlech
\item Aplikace Newtonovy metody v prostoru analytických funkcí
\item Důkaz konvergence pomocí rychlé kvadratické konvergence
\end{itemize}
\end{proofsketch}

\begin{application}[Stabilita sluneční soustavy]
KAM teorie vysvětluje dlouhodobou stabilitu planetárních drah navzdory gravitačním perturbacím. Aplikace na problém N-těles ukazuje, že chaos je omezen na tenké vrstvy mezi invariantními torii.
\end{application}

\spc

\subsection{Ergodická Teorie a Statistický Popis Dynamiky}

\subsubsection{Základní Pojmy Ergodické Teorie}

\begin{definition}[Dynamický systém s mírou]
Čtveřice $(X, \mathcal{B}, \mu, T)$, kde:
\begin{itemize}
\item $X$: fázový prostor
\item $\mathcal{B}$: σ-algebra měřitelných množin
\item $\mu$: pravděpodobnostní míra
\item $T: X \to X$: měřitelné zobrazení zachovávající míru ($\mu(T^{-1}A) = \mu(A)$)
\end{itemize}
\end{definition}

\begin{definition}[Ergodicita]
Dynamický systém je \emph{ergodický}, jestliže každá $T$-invariantní množina má míru 0 nebo 1.
\end{definition}

\begin{application}[Základ statistické mechaniky]
Ergodicita ospravedlňuje nahrazení časových průměrů prostorovými průměry v statistické fyzice. Pro ergodické systémy platí rovnost mikrokanonického a časového průměru.
\end{application}

\subsubsection{Birkhoffův a Von Neumannův Ergodický Teorém}

\begin{theorem}[Birkhoffův ergodický teorém]
Nechť $(X, \mathcal{B}, \mu, T)$ je dynamický systém s mírou a $f \in L^1(X, \mu)$. Pak pro skoro všechna $x \in X$ existuje časový průměr:
\[
\lim_{n \to \infty} \frac{1}{n} \sum_{k=0}^{n-1} f(T^k x) = \bar{f}(x),
\]
kde $\bar{f}$ je $T$-invariantní a $\int_X f d\mu = \int_X \bar{f} d\mu$.
\end{theorem}

\begin{theorem}[Von Neumannův ergodický teorém]
Pro $f \in L^2(X, \mu)$ platí:
\[
\lim_{n \to \infty} \left\| \frac{1}{n} \sum_{k=0}^{n-1} f \circ T^k - P_T f \right\|_{L^2} = 0,
\]
kde $P_T$ je ortogonální projekce na podprostor $T$-invariantních funkcí.
\end{theorem}

\begin{application}[Odhad dlouhodobých průměrů]
V kvantitativních financích umožňují ergodické teorémy odhadovat dlouhodobé výnosy a riika z historických dat za předpokladu ergodicity tržních procesů.
\end{application}

\subsubsection{Entropie a Komplexita Dynamických Systémů}

\begin{definition}[Metrická entropie]
Nechť $\alpha$ je měřitelný rozklad $X$. \emph{Entropie rozkladu} je:
\[
H_\mu(\alpha) = -\sum_{A \in \alpha} \mu(A) \log \mu(A).
\]
\emph{Metrická entropie} dynamického systému je:
\[
h_\mu(T) = \sup_{\alpha} \lim_{n \to \infty} \frac{1}{n} H_\mu\left( \bigvee_{k=0}^{n-1} T^{-k} \alpha \right).
\]
\end{definition}

\begin{theorem}[Ornsteinova teorie]
Dva Bernoulliho systémy jsou izomorfní právě tehdy, když mají stejnou entropii.
\end{theorem}

\begin{application}[Kvantifikace chaosu v finančních datech]
Metrická entropie slouží jako míra prediktability finančních časových řad. Vysoká entropie indikuje chaotické chování a nízkou předpověditelnost, zatímco nízká entropie naznačuje pravidelnosti využitelné pro tradingové strategie.
\end{application}

\spc

\subsection{Teorie Chaosu a Nelineární Dynamika}

\subsubsection{Deterministický Chaos a Citlivá Závislost}

\begin{definition}[Ljapunovovy exponenty]
Pro diferenciovatelné zobrazení $f: \mathbb{R}^n \to \mathbb{R}^n$ jsou \emph{Ljapunovovy exponenty} v bodě $x$ definovány jako:
\[
\lambda_i = \lim_{n \to \infty} \frac{1}{n} \log \sigma_i(Df^n(x)),
\]
kde $\sigma_i$ jsou singulární hodnoty.
\end{definition}

\begin{theorem}[Pesinova formule]
Pro hyperbolický difeomorfismus zachovávající míru $\mu$ platí:
\[
h_\mu(f) = \int \sum_{\lambda_i > 0} \lambda_i d\mu.
\]
\end{theorem}

\begin{application}[Předpověditelnost v komplexních systémech]
Ljapunovovy exponenty kvantifikují "efekt motýlích křídel" - citlivou závislost na počátečních podmínkách. V ekonomických modelech kladné Ljapunovovy exponenty implikují fundamentální limity předpověditelnosti.
\end{application}

\subsubsection{Podivné Atraktory a Fraktální Dimenze}

\begin{definition}[Podivný atraktor]
\emph{Podivný atraktor} je atraktor vykazující:
\begin{itemize}
\item Citlivou závislost na počátečních podmínkách
\item Fraktální strukturu
\item Nekomplikovanou topologii, ale komplikovanou geometrii
\end{itemize}
\end{definition}

\begin{definition}[Box-counting dimenze]
Pro množinu $A \subset \mathbb{R}^n$ je \emph{box-counting dimenze}:
\[
\dim_B(A) = \lim_{\epsilon \to 0} \frac{\log N(\epsilon)}{\log(1/\epsilon)},
\]
kde $N(\epsilon)$ je minimální počet $\epsilon$-kostek pokrývajících $A$.
\end{definition}

\begin{application}[Identifikace chaosu v experimentálních datech]
V kvantitativních financích slouží odhad fraktální dimenze a Ljapunovových exponentů k rozlišení mezi stochastickým šumem a deterministickým chaosem v cenových řadách.
\end{application}

\subsubsection{Symbolická Dynamika a Shift Spaces}

\begin{definition}[Symbolická dynamika]
Nechť $\mathcal{A}$ je konečná abeceda. \emph{Plný shift} je dynamický systém $(\mathcal{A}^\mathbb{Z}, \sigma)$, kde $\sigma$ je posun:
\[
\sigma(\dots x_{-1}.x_0x_1\dots) = \dots x_{-1}x_0.x_1\dots
\]
\end{definition}

\begin{theorem}[Vztah entropie a shiftu]
Pro topologický Markovův shift s přechodovou maticí $A$ platí:
\[
h_{top}(\sigma_A) = \log \lambda_{max}(A),
\]
kde $\lambda_{max}$ je největší vlastní číslo $A$.
\end{theorem}

\begin{application}[Data compression a teorie informace]
Symbolická dynamika poskytuje teoretický základ pro kompresi dat a analýzu informačního toku v dynamických systémech. Aplikace zahrnují analýzu DNA sekvencí a finančních časových řad.
\end{application}

\spc

\subsection{Syntéza Teoretického Fundamentu}

\subsubsection{Propojení Konceptů: Unifikující Teoretická Mapa}


\begin{keyinsight}[Hierarchie složitosti dynamických systémů]
\begin{itemize}
\item \textbf{Lineární systémy}: Úplná analytická řešitelnost, spektrální teorie
\item \textbf{Integrabilní systémy}: Akčně-úhlové proměnné, kvaziperiodicita  
\item \textbf{Hyperbolické systémy}: Strukturální stabilita, shadowing lemma
\item \textbf{Částečně hyperbolické systémy}: Zentrum manifold, bifurkace
\item \textbf{Obecné nelineární systémy}: Chaos, podivné atraktory, ergodicita
\end{itemize}
\end{keyinsight}

\begin{application}[Rozhodovací strom pro výběr metod]
\begin{enumerate}
\item Je systém lineární? → Spektrální analýza
\item Je Hamiltonovský? → Symplektické metody
\item Má atrahující invariantní množinu? → Ljapunovovy funkce
\item Je hyperbolický? → Strukturální analýza
\item Vykazuje komplexní chování? → Ergodická teorie, entropie
\end{enumerate}
\end{application}

\subsubsection{Typická Úskalí a Expertní Doporučení}

\begin{warning}[Časté chyby v analýze nelineárních systémů]
\begin{itemize}
\item Zaměnění stability linearizovaného systému za stabilitu nelineárního systému
\item Ignorování globálních bifurkací při lokální analýze
\item Podcenění numerických chyb v odhadu Ljapunovových exponentů
\item Přecenění ergodicity bez testování míry invariantnosti
\end{itemize}
\end{warning}

\begin{expertnote}[Numerická spolehlivost]
\begin{itemize}
\item Vždy ověřujte konvergenci numerických schémat pro různé počáteční podmínky
\item Používejte více metod pro odhad fraktální dimenze a entropie
\item Testujte robustnost výsledků vůči perturbacím parametrů
\item Validujte modely na nezávislých datech
\end{itemize}
\end{expertnote}

\subsubsection{Příprava na Aplikované Kapitoly}

\begin{roadmap}[Navigace aplikovanými kapitolami]
\begin{itemize}
\item \textbf{Parciální diferenciální rovnice}: Rozšíření na nekonečně-dimenzionální systémy
\item \textbf{Stochastické diferenciální rovnice}: Náhodné perturbace deterministické dynamiky
\item \textbf{Finanční aplikace}: Modelování tržní dynamiky a oceňování derivátů
\item \textbf{Kvantové systémy}: Aplikace v kvantové mechanice a teorii pole
\end{itemize}
\end{roadmap}

\spc

\subsection{Pokročilá Literatura a Směry Výzkumu}

\subsubsection{Fundamentální Monografie a Přehledové Články}

\begin{itemize}
\item \textbf{Arnold, V.I.} - \emph{Mathematical Methods of Classical Mechanics} - Základ symplektické geometrie
\item \textbf{Katok, A., Hasselblatt, B.} - \emph{Introduction to the Modern Theory of Dynamical Systems} - Komplexní přehled teorie
\item \textbf{Cornfeld, I.P., Fomin, S.V., Sinai, Ya.G.} - \emph{Ergodic Theory} - Hluboký vhled do ergodické teorie
\end{itemize}

\subsubsection{Aktuální Směry Výzkumu v Teorii Dynamických Systémů}

\begin{researcharea}[Parciálně hyperbolické systémy]
\begin{itemize}
\item Klasifikace parciálně hyperbolických difeomorfizmů
\item Stabilitní ergodictví a robustní transitivity
\item Aplikace v geometrické grupové teorii
\end{itemize}
\end{researcharea}

\begin{researcharea}[Teorie rigidity]
\begin{itemize}
\item Lokální rigidita grupových akcí
\item Mostowova rigidita a její zobecnění
\item Aplikace v aritmetické geometrii
\end{itemize}
\end{researcharea}

\subsubsection{Interdisciplinární Spojení}

\begin{application}[Dynamické systémy v machine learning]
\begin{itemize}
\item Analýza dynamiky gradient descent algoritmů
\item Teorie rekurentních neuronových sítí jako dynamických systémů
\item Aplikace ergodické teorie v reinforcement learning
\end{itemize}
\end{application}

\begin{application}[Kvantitativní finance a teorie chaosu]
\begin{itemize}
\item Detekce deterministického chaosu v finančních datech
\item Modelování regime changes pomocí bifurkací
\item Aplikace KAM teorie v portfoliové optimalizaci
\end{itemize}
\end{application}

\subsection{Shrnutí kapitoly}

\begin{itemize}
    \item \textbf{Pokročilá teorie bifurkací} — Rozbor lokálních bifurkací, včetně saddle-node, transkritické a pitchfork bifurkace, dále Hopfovo bifurkace s rozborem vzniku oscilací a globálních bifurkací jako homoklinických či heteroklinických jevů. Aplikace v ekonomických modelech a klimatických přechodech.
    
    \item \textbf{Hamiltonovské systémy a symplektická geometrie} — Definice symplektické struktury a Hamiltonovy formulace, Liouvilleova integrabilita, Arnold–Liouvilleův teorém integrability s akčně-úhlovými proměnnými. KAM teorie jako hlavní metoda stability téměř integrabilních systémů. Finanční a fyzikální aplikace.
    
    \item \textbf{Ergodická teorie a statistický popis dynamiky} — Zavedení invariantních měr, ergodicity a mixing, Birkhoffův a Von Neumannův ergodický teorém jako základ pro statistickou mechaniku a analýzu časových řad. Metrická a topologická entropie jako měřítka chaosu a komplexity.
    
    \item \textbf{Teorie chaosu a nelineární dynamika} — Definice Ljapunovových exponentů a jejich význam pro citlivost na počáteční podmínky, charakterizace podivných atraktorů a fraktálních dimenzí, symbolická dynamika a vztah k teorii informace. Aplikace v prediktabilitě ekonomických dat a fyzikálních systémech.
    
    \item \textbf{Syntéza teoretického fundamentu} — Unifikující mapa vztahů mezi různými třídami dynamických systémů, rozhodovací schéma výběru správné analytické metody podle charakteru problému a praktické rady pro analýzu nelineárních systémů, včetně varování před běžnými chybami a doporučení numerické robustnosti.
    
    \item \textbf{Pokročilá literatura a výzkumné směry} — Přehled klíčových monografií a článků, moderních trendů jako parciální hyperboličnost a rigidita, interdisciplinárních aplikací v machine learningu, kvantové teorii pole a ekonomii.
  \end{itemize}
    
% !TEX root = ../main.tex
\section{Systém Levels: cesta kvantitativního matematika}
\label{sec:system-levels}

\blocktitle{Cíl kapitoly}
Tato část představuje revoluční systém \emph{14 Levels} -- strukturovaný přístup k osvojení si
matematických metod pro kvantitativní finance. Ukážeme, jak budeme systematicky postupovat od
elementárních rovnic k pokročilým výzkumným tématům a jak každá úroveň navazuje na teoretický
základ z předchozích kapitol.

\spc

\subsection{Filosofie naší cesty}
\label{sec:filosofie-cesty}

\paragraph{Proč právě systém Levels?}
V kvantitativních financích se setkáváme s obrovským množstvím metod. Tradiční přístupy často selhávají v tom, že:
\begin{itemize}
  \item studenti se ztrácejí v množství technik bez jasné struktury,
  \item chybí přímé propojení mezi matematikou a finančními aplikacemi,
  \item neexistuje systematický postup od základů k expertní úrovni.
\end{itemize}
Náš systém \textbf{14 Levels} řeší tyto problémy tím, že:
\begin{itemize}
  \item \textbf{poskytuje jasnou mapu} -- vždy víš, kde jsi a kam směřuješ,
  \item \textbf{propojuje teorii s praxí} -- každá metoda má okamžité finanční aplikace,
  \item \textbf{buduje znalosti postupně} -- žádné skoky, žádné mezery,
  \item \textbf{připravuje na reálné výzvy} -- od akademických cvičení po tradingové aplikace.
\end{itemize}

\paragraph{Jak systém funguje v praxi}
Každý Level obsahuje tři klíčové komponenty:
\begin{romanenum}
  \item \textbf{Matematický aparát} -- přesné definice, věty a řešicí metody,
  \item \textbf{Finanční motivace} -- proč se danou metodou zabývat a kde se používá,
  \item \textbf{Praktická implementace} -- jak metodu naprogramovat a použít na reálných datech.
\end{romanenum}

\spc

\subsection{Přehled 14 Levels: od základů k výzkumu}
\label{sec:prehled-levels}

\paragraph{Level 1--4: matematické základy finance}
\textbf{Level 1: Základní ODE 1.\ řádu}\\
\emph{Matematika:} separovatelné, lineární a exaktní rovnice.\\
\emph{Finance:} růstové modely, diskontování, jednoduché pricingy.

\medskip
\textbf{Level 2: Speciální nelineární ODE 1.\ řádu}\\
\emph{Matematika:} Bernoulli, Riccati, Clairaut.\\
\emph{Finance:} utility funkce, optimální spotřeba, Mertonův model.

\medskip
\textbf{Level 3: Lineární ODE 2.\ řádu}\\
\emph{Matematika:} konstantní koeficienty, charakteristická rovnice.\\
\emph{Finance:} Vasicek, bond pricing, mean--reverting procesy.

\medskip
\textbf{Level 4: ODE s proměnnými koeficienty}\\
\emph{Matematika:} Eulerovy rovnice, Frobeniova metoda.\\
\emph{Finance:} time--dependent modely, lokální volatility, Hull--White.

\spc

\paragraph{Level 5--8: pokročilé analytické metody}
\textbf{Level 5: ODE vyšších řádů}\\
\emph{Matematika:} fundamentální systémy, variace konstant.\\
\emph{Finance:} konstrukce yield curve, spline modely.

\medskip
\textbf{Level 6: Systémy ODE}\\
\emph{Matematika:} maticové exponenciály, Jordanova forma.\\
\emph{Finance:} multi--asset modely, systémové riziko, korelace.

\medskip
\textbf{Level 7: Stabilita a kvalitativní analýza}\\
\emph{Matematika:} Ljapunovovy funkce, fázové portréty.\\
\emph{Finance:} finanční stabilita, analýza tržních ekvilibrií.

\medskip
\textbf{Level 8: Laplaceova transformace}\\
\emph{Matematika:} integrální transformace, konvoluce.\\
\emph{Finance:} exotické opce, komplexní deriváty, barierové opce.

\spc

\paragraph{Level 9--12: kvantitativní nástroje}
\textbf{Level 9: Nelineární ODE vyšších řádů}\\
\emph{Matematika:} redukce řádu, symetrie.\\
\emph{Finance:} HJB, optimální portfolio, stochastic control.

\medskip
\textbf{Level 10: Okrajové úlohy (BVP)}\\
\emph{Matematika:} Sturm--Liouville, spektrální metody.\\
\emph{Finance:} americké opce, free--boundary problémy.

\medskip
\textbf{Level 11: Speciální funkce a transformace}\\
\emph{Matematika:} Fourierovy řady, Besselovy funkce.\\
\emph{Finance:} Lévy modely, FFT pricing, charakteristické funkce.

\medskip
\textbf{Level 12: Numerické metody}\\
\emph{Matematika:} konečné diference, Runge--Kutta metody.\\
\emph{Finance:} PDE řešiče, Monte Carlo, lattice metody.

\spc

\paragraph{Level 13--14: výzkumné horizonty}
\textbf{Level 13: Dynamické systémy}\\
\emph{Matematika:} bifurkace, chaos, atraktory.\\
\emph{Finance:} ekonomické cykly, market microstructure.

\medskip
\textbf{Level 14: Pokročilé moderní metody}\\
\emph{Matematika:} forward--backward SDE, mean--field games.\\
\emph{Finance:} high--frequency trading, portfolio optimization.

\spc

\subsection{Propojení s teoretickým základem}
\label{sec:propojeni-s-teorii}

\paragraph{Kapitoly 1--2: matematický fundament}
\begin{itemize}
  \item \textbf{Prostory funkcí} (Kap.~2): formulace metod v odpovídajících funkčních prostorech.
  \item \textbf{Banachovy prostory} (Kap.~2): základ pro existence/unikátnost.
  \item \textbf{Lineární operátory} (Kap.~2): klíčové pro Level~6.
\end{itemize}

\paragraph{Kapitola 3: metriky a pevné body}
\begin{itemize}
  \item \textbf{Banachova věta}: kontrakce $\Rightarrow$ existence a jednoznačnost.
  \item \textbf{Schauder}: kompaktní operátory a nelineární existence.
  \item \textbf{Ekeland}: variační přístupy v řízení.
\end{itemize}

\paragraph{Kapitoly 4--5: teorie ODE}
\begin{itemize}
  \item \textbf{Picard--Lindelöf}: existence/unikátnost pro naše modely.
  \item \textbf{Grönwall}: odhady a stabilita.
  \item \textbf{Maximální řešení} a \textbf{závislost na parametrech}: kalibrace modelů.
\end{itemize}

\paragraph{Kapitoly 6--7: stabilita a dynamika}
\begin{itemize}
  \item \textbf{Ljapunov}: analýza finančních systémů.
  \item \textbf{Hartman--Grobman}: lokální klasifikace nelineárních systémů.
  \item \textbf{Bifurkace}: kvalitativní změny v ekonomických modelech.
\end{itemize}

\spc

\subsection{Studijní strategie a doporučení}
\label{sec:studijni-strategie}

\paragraph{Pro různé typy studentů}
\textbf{Začátečníci:}
\begin{itemize}
  \item postupujte lineárně Level po Levelu,
  \item věnujte čas každému teoretickému konceptu,
  \item řešte všechny základní příklady.
\end{itemize}
\textbf{Pokročilí:}
\begin{itemize}
  \item rychlejší průchod Levels 1--6,
  \item fokus na finanční aplikace,
  \item důraz na numerickou implementaci.
\end{itemize}
\textbf{Praktici:}
\begin{itemize}
  \item používejte jako referenční příručku,
  \item hledejte konkrétní metody pro projekty,
  \item zaměřte se na Levels 12--14.
\end{itemize}

\paragraph{Časová náročnost a milníky}
\begin{itemize}
  \item \textbf{Level 1--4}: 4--6 týdnů -- zvládnutí základních pricing modelů,
  \item \textbf{Level 5--8}: 6--8 týdnů -- pokročilé analytické metody,
  \item \textbf{Level 9--12}: 8--10 týdnů -- kvantitativní nástroje,
  \item \textbf{Level 13--14}: 4--6 týdnů -- výzkumná témata.
\end{itemize}

\paragraph{Praktické projekty a portfolio}
\begin{itemize}
  \item \textbf{Level 4}: kalibrace úrokového modelu na tržní data,
  \item \textbf{Level 8}: pricing engine pro exotické opce,
  \item \textbf{Level 9}: optimální portfolio strategie,
  \item \textbf{Level 12}: numerický PDE řešič pro americké opce.
\end{itemize}

\spc

\subsection*{Co získáte po absolvování}
\label{sec:co-ziskate}
\begin{itemize}
  \item \textbf{Matematická výbava}: kompletní porozumění metodám ODE relevantním pro finance,
  \item \textbf{Praktická připravenost}: schopnost implementovat komplexní finanční modely,
  \item \textbf{Výzkumná orientace}: připravenost na pokročilá témata v quant finance,
  \item \textbf{Průmyslová relevance}: konkurenceschopnost na trhu práce.
\end{itemize}

\subsection*{Závěrem}
Systém \textbf{14 Levels} není jen učební pomůcka, ale \emph{kompletní vzdělávací ekosystém} pro kvantitativní finance.
Kombinuje matematickou rigoróznost s praktickou relevancí a připraví tě na reálné výzvy v tradingu, risk managementu
a finančním inženýrství.

\spc

\noindent\emph{Cesta kvantitativního matematika pokračuje v následující části Level~1.}

% !TEX root = ../main.tex
\section{Úroveň 1: Základní ODE 1. Řádu - Expertní Kvantitativní Fundament}
\label{sec:uroven-1}

\subsection{Úvod do Úrovně 1}
\label{subsec:uvod-uroven-1}

Tato úroveň představuje kompletní a rigorózní úvod do teorie obyčejných diferenciálních rovnic prvního řádu. Naším cílem není pouze mechanické osvojení řešicích technik, ale hluboké porozumění matematické struktuře těchto rovnic a jejich bezprostřední aplikace v kvantitativních vědách.

\vspace{0.8\baselineskip}

\begin{principle}[Filozofie úrovně 1]
Úroveň 1 kombinuje matematickou preciznost s praktickou relevancí. Každý teoretický koncept je okamžitě ilustrován na reálných kvantitativních problémech, čímž vytváříme most mezi abstraktní matematikou a aplikovaným výzkumem.
\end{principle}

\vspace{0.8\baselineskip}

\subsubsection*{Organizace a cíle úrovně}

Úroveň 1 je strukturována do čtyř hlavních typů ODE 1. řádu, které tvoří fundament pro veškeré pokročilejší techniky:

\begin{itemize}
\item \textbf{Separabilní rovnice} - základní stavební kámen
\item \textbf{Lineární rovnice} - systematický přístup s integračními faktory
\item \textbf{Homogenní rovnice} - geometrická interpretace a substituce
\item \textbf{Exaktní rovnice} - teorie potenciálů a konzervativních systémů
\end{itemize}

\vspace{0.8\baselineskip}

\subsubsection*{Kvantitativní význam}

Pro kvantitativního experta představují ODE 1. řádu fundamentální nástroj pro:

\begin{itemize}
\item Modelování časového vývoje finančních proměnných
\item Kalibraci parametrů na historická data
\item Analýzu stability ekonomických systémů
\item Přípravu na pokročilejší modely (SDE, PDE)
\end{itemize}

\vspace{0.8\baselineskip}

\begin{example}[Motivační příklad z kvantitativních financí]
Uvažujme jednoduchý model pro cenu akcie $S(t)$ s konstantní expected return $\mu$:
\[
\frac{dS}{dt} = \mu S, \quad S(0) = S_0
\]
Tato separabilní rovnice má řešení $S(t) = S_0 e^{\mu t}$, které tvoří základ pro geometrický Brownův pohyb v Black-Scholesově modelu.
\end{example}

\vspace{0.8\baselineskip}

\subsection{Separabilní Rovnice - Kompletní Teorie}


\label{subsec:separabilni-rovnice}

\subsubsection{Teoretický Fundament}
\label{subsubsec:teoreticky-fundament}

\begin{definition}[Separabilní diferenciální rovnice]
Rovnice je \emph{separabilní}, jestliže ji lze zapsat ve tvaru:
\[
\frac{dy}{dx} = f(x)g(y)
\]
kde $f(x)$ je funkce závislá pouze na $x$ a $g(y)$ je funkce závislá pouze na $y$.
\end{definition}

\vspace{0.6\baselineskip}

\begin{theorem}[Existence a jednoznačnost řešení]
Nechť $f(x)$ je spojitá na intervalu $I$ a $g(y)$ je spojitá na intervalu $J$ a Lipschitzovská v $y$ na $J$. Pak pro každý počáteční bod $(x_0, y_0) \in I \times J$ existuje právě jedno řešení separabilní rovnice splňující $y(x_0) = y_0$, pokud $g(y_0) \neq 0$.
\end{theorem}

\vspace{0.4\baselineskip}

\begin{proof}
Důkaz využívá Picard-Lindelöfovu větu z Kapitoly 3. Separabilní tvar umožňuje přímou integraci:
\[
\int_{y_0}^y \frac{d\eta}{g(\eta)} = \int_{x_0}^x f(\xi)  d\xi
\]
Lipschitzovskost $g(y)$ zaručuje jednoznačnost řešení na maximálním intervalu existence.
\end{proof}

\vspace{0.6\baselineskip}

\begin{theorem}[Maximální interval existence]
Nechť $f(x)$ je spojitá na $(a,b)$ a $g(y)$ je spojitá na $(c,d)$. Pak řešení separabilní rovnice existuje na maximálním otevřeném intervalu $(\alpha,\beta) \subset (a,b)$ a platí buď $\beta = b$, nebo $\lim_{x\to\beta^-} y(x)$ existuje a patří do $\{c,d,\pm\infty\}$.
\end{theorem}

\vspace{0.8\baselineskip}

\subsubsection{Kompletní Metodologie Řešení}
\label{subsubsec:kompletni-metodologie}

\begin{method}[Metoda 1: Přímá separace proměnných]
\label{met:primaseparace}
\begin{enumerate}
\item \textbf{Identifikace tvaru}: Ověřte, že rovnici lze zapsat jako $\frac{dy}{dx} = f(x)g(y)$

\item \textbf{Analýza singulárních bodů}: Najděte všechny $y_c$ takové, že $g(y_c) = 0$

\item \textbf{Konstantní řešení}: $y(x) = y_c$ jsou řešeními pro každé $y_c$ z předchozího kroku

\item \textbf{Separace proměnných}: Pro $g(y) \neq 0$ přepište na $\frac{dy}{g(y)} = f(x)dx$

\item \textbf{Integrace}: 
\[
\int \frac{dy}{g(y)} = \int f(x)dx + C
\]

\item \textbf{Explicitní vyjádření}: Pokud možno, vyjádřete $y$ explicitně jako funkci $x$

\item \textbf{Analýza definičního oboru}: Určete maximální intervaly existence řešení

\item \textbf{Ověření}: Dosazením ověřte, že získané řešení splňuje původní rovnici
\end{enumerate}
\end{method}

\vspace{0.8\baselineskip}

\begin{method}[Metoda 2: Substituční přístupy]
\label{met:substituce}
Pro rovnice tvaru $\frac{dy}{dx} = f(ax + by + c)$ použijte substituci:
\[
u = ax + by + c \implies \frac{du}{dx} = a + b\frac{dy}{dx} = a + bf(u)
\]
což je separabilní rovnice pro $u(x)$.
\end{method}

\vspace{0.6\baselineskip}

\begin{method}[Metoda 3: Numerická verifikace]
\label{met:numerickaverifikace}
Pro ověření analytického řešení použijte Eulerovu metodu:
\[
y_{n+1} = y_n + h f(x_n)g(y_n)
\]
kde $h$ je krok metody. Chyba metody je $O(h)$.
\end{method}

\vspace{0.8\baselineskip}

\subsubsection{Klasifikace Speciálních Případů}
\label{subsubsec:klasifikace-specialnich-pripadu}

\begin{remark}[Rovnice s absolutními hodnotami]
Pro rovnice obsahující absolutní hodnoty je nutné uvažovat případné dělení definičního oboru a zkoumat spojitost řešení v bodech, kde se mění znaménko výrazu uvnitř absolutní hodnoty.
\end{remark}

\vspace{0.6\baselineskip}

\begin{remark}[Po částech definované pravé strany]
Pokud je $f(x)$ nebo $g(y)$ po částech definovaná, řešíme rovnici separátně na každém intervalu spojitosti a poté zkoumáme spojitost řešení v hraničních bodech.
\end{remark}

\vspace{0.6\baselineskip}

\begin{remark}[Rovnice s parametry - bifurkace]
Rovnice tvaru $\frac{dy}{dx} = y(\lambda - y^2)$ vykazuje pitchfork bifurkaci při $\lambda = 0$. Pro $\lambda < 0$ existuje jedno stabilní řešení $y = 0$, pro $\lambda > 0$ existují tři rovnováhy: $y = 0$ (nestabilní) a $y = \pm\sqrt{\lambda}$ (stabilní).
\end{remark}

\vspace{0.6\baselineskip}

\begin{remark}[Singularity a jejich klasifikace]
Body, kde $g(y) = 0$, mohou být:
\begin{itemize}
\item \textbf{Regulární singularita}: Řešení lze prodloužit přes tento bod
\item \textbf{Esenciální singularita}: Řešení nelze prodloužit
\item \textbf{Bod větvení}: Řešení není jednoznačné
\end{itemize}
\end{remark}

\vspace{0.8\baselineskip}

\subsubsection{Početní Sekce - Hierarchická}
\label{subsubsec:pocetni-sekce}

\paragraph*{Úroveň 1: Lehké příklady}

\begin{example}[Základní separace]
    Řešte rovnici: $\frac{dy}{dx} = 2xy$
    
    \vspace{0.3\baselineskip}
    
    \textbf{Řešení}: 
    \begin{enumerate}
    \item \textbf{Singulární body}: $g(y) = y = 0 \implies y = 0$ je konstantní řešení
    
    \item \textbf{Obecné řešení pro $y \neq 0$}: Separace proměnných:
    \[
    \int \frac{dy}{y} = \int 2x  dx \implies \ln|y| = x^2 + C \implies |y| = e^{x^2 + C}
    \]
    \[
    y = \pm e^C e^{x^2} = Ae^{x^2}, \quad A \in \mathbb{R}\setminus\{0\}
    \]
    
    \item \textbf{Kompletní řešení}: 
    \[
    y(x) = Ae^{x^2}, \quad A \in \mathbb{R}
    \]
    Toto zahrnuje i původní konstantní řešení $y = 0$ (pro $A = 0$)
    
    \item \textbf{Interval existence}: Řešení existuje na $\mathbb{R}$ pro všechny $A \in \mathbb{R}$
    \end{enumerate}
    \end{example}

\vspace{0.6\baselineskip}

\begin{example}[S počáteční podmínkou]
    Řešte: $\frac{dy}{dx} = \frac{x}{y}$, $y(0) = 2$
    
    \vspace{0.3\baselineskip}
    
    \textbf{Řešení}: 
    \begin{enumerate}
    \item \textbf{Singulární body}: $g(y) = \frac{1}{y}$ má singularitu v $y = 0$
    
    \item \textbf{Obecné řešení pro $y \neq 0$}: 
    \[
    \int y  dy = \int x  dx \implies \frac{y^2}{2} = \frac{x^2}{2} + C \implies y^2 = x^2 + 2C
    \]
    
    \item \textbf{Určení konstanty z počáteční podmínky}:
    \[
    y(0) = 2 \implies 4 = 0 + 2C \implies C = 2 \implies y^2 = x^2 + 4
    \]
    
    \item \textbf{Výběr větve řešení}: Protože $y(0) = 2 > 0$, volíme kladnou větev:
    \[
    y = \sqrt{x^2 + 4}
    \]
    
    \item \textbf{Interval existence}: Řešení existuje na $\mathbb{R}$ a platí $y(x) > 0$ pro všechna $x \in \mathbb{R}$
    
    \item \textbf{Ověření}: Dosazením do původní rovnice:
    \[
    \frac{dy}{dx} = \frac{x}{\sqrt{x^2 + 4}} = \frac{x}{y} \quad \checkmark
    \]
    \end{enumerate}
    \end{example}

\vspace{0.8\baselineskip}

\paragraph*{Úroveň 2: Střední příklady}

\begin{example}[Analýza singulárních bodů]
    Řešte: $\frac{dy}{dx} = x(y^2 - 4)$
    \vspace{0.3\baselineskip}
    
    \textbf{Řešení}: 
    \begin{enumerate}
    \item \textbf{Singulární body}: $g(y) = y^2 - 4 = 0 \implies y = \pm 2$ jsou konstantní řešení
    
    \item \textbf{Obecné řešení pro $y \neq \pm 2$}:
    \[
    \int \frac{dy}{y^2 - 4} = \int x  dx
    \]
    Rozklad na parciální zlomky:
    \[
    \frac{1}{y^2 - 4} = \frac{1}{4}\left(\frac{1}{y-2} - \frac{1}{y+2}\right)
    \]
    Integrace:
    \[
    \frac{1}{4} \ln\left|\frac{y-2}{y+2}\right| = \frac{x^2}{2} + C
    \]
    \[
    \left|\frac{y-2}{y+2}\right| = e^{2x^2 + 4C} = Ke^{2x^2}, \quad K > 0
    \]
    
    \item \textbf{Explicitní vyjádření}:
    \[
    \frac{y-2}{y+2} = \pm Ke^{2x^2} = Ae^{2x^2}, \quad A \in \mathbb{R}\setminus\{0\}
    \]
    \[
    y-2 = Ae^{2x^2}(y+2) \implies y(1 - Ae^{2x^2}) = 2 + 2Ae^{2x^2}
    \]
    \[
    y = \frac{2(1 + Ae^{2x^2})}{1 - Ae^{2x^2}}, \quad A \in \mathbb{R}
    \]
    
    \item \textbf{Kompletní řešení včetně singulárních}:
    \[
    y(x) = \frac{2(1 + Ae^{2x^2})}{1 - Ae^{2x^2}}, \quad A \in \mathbb{R} \quad \text{NEBO} \quad y(x) = \pm 2
    \]
    Konstantní řešení $y = \pm 2$ odpovídají limitám $A \to \infty$
    
    \item \textbf{Intervaly existence}: 
    \begin{itemize}
    \item Pro $A < 0$: řešení existuje na $\mathbb{R}$
    \item Pro $A > 0$: řešení má singularitu když $1 - Ae^{2x^2} = 0$
    \end{itemize}
    \end{enumerate}
    \end{example}

\vspace{0.6\baselineskip}

\begin{example}[Substituční metoda]
    Řešte: $\frac{dy}{dx} = (x + y)^2$
    
    \vspace{0.3\baselineskip}
    
    \textbf{Řešení}: 
    \begin{enumerate}
    \item \textbf{Substituce}: $u = x + y \implies \frac{du}{dx} = 1 + \frac{dy}{dx} = 1 + u^2$
    
    \item \textbf{Separace proměnných}:
    \[
    \int \frac{du}{1 + u^2} = \int dx \implies \arctan u = x + C
    \]
    
    \item \textbf{Návrat k původní proměnné}:
    \[
    u = \tan(x + C) \implies y = \tan(x + C) - x
    \]
    
    \item \textbf{Singulární body}: Řešení má singularity když $\cos(x + C) = 0$, tedy pro $x + C = \frac{\pi}{2} + k\pi$, $k \in \mathbb{Z}$
    
    \item \textbf{Intervaly existence}: Řešení existuje na intervalech tvaru $\left(C - \frac{\pi}{2} + k\pi, C + \frac{\pi}{2} + k\pi\right)$
    
    \item \textbf{Ověření}:
    \[
    \frac{dy}{dx} = \frac{1}{\cos^2(x + C)} - 1 = \tan^2(x + C) + 1 - 1 = (x + y)^2 \quad \checkmark
    \]
    \end{enumerate}
    \end{example}

\vspace{0.8\baselineskip}

\paragraph*{Úroveň 3: Složité příklady}

\begin{example}[Rovnice s absolutní hodnotou]
    Řešte: $\frac{dy}{dx} = |x|y$
    \vspace{0.3\baselineskip}
    
    \textbf{Řešení}: 
    \begin{enumerate}
    \item \textbf{Singulární body}: $g(y) = y = 0 \implies y = 0$ je konstantní řešení
    
    \item \textbf{Řešení pro $x \geq 0$}: $|x| = x$
    \[
    \frac{dy}{dx} = xy \implies \int \frac{dy}{y} = \int x  dx \implies \ln|y| = \frac{x^2}{2} + C_1
    \]
    \[
    y = \pm e^{C_1} e^{x^2/2} = A e^{x^2/2}, \quad A \in \mathbb{R}
    \]
    
    \item \textbf{Řešení pro $x < 0$}: $|x| = -x$
    \[
    \frac{dy}{dx} = -xy \implies \int \frac{dy}{y} = -\int x  dx \implies \ln|y| = -\frac{x^2}{2} + C_2
    \]
    \[
    y = \pm e^{C_2} e^{-x^2/2} = B e^{-x^2/2}, \quad B \in \mathbb{R}
    \]
    
    \item \textbf{Spojitost v $x = 0$}: 
    \[
    \lim_{x \to 0^-} y(x) = B = \lim_{x \to 0^+} y(x) = A
    \]
    Tedy $A = B = C$
    
    \item \textbf{Kompletní řešení}:
    \[
    y(x) = \begin{cases}
    C e^{-x^2/2} & \text{pro } x < 0 \\
    C e^{x^2/2} & \text{pro } x \geq 0
    \end{cases}, \quad C \in \mathbb{R}
    \]
    Toto zahrnuje i konstantní řešení $y = 0$ (pro $C = 0$)
    
    \item \textbf{Interval existence}: Řešení existuje na $\mathbb{R}$ pro všechna $C \in \mathbb{R}$
    \end{enumerate}
    \end{example}

\vspace{0.6\baselineskip}

\begin{example}[Rovnice s parametrem]
    Analyzujte rovnici: $\frac{dy}{dx} = \lambda y - y^3$ v závislosti na parametru $\lambda$.
    \vspace{0.3\baselineskip}
    
    \textbf{Řešení}: 
    \begin{enumerate}
    \item \textbf{Singulární body}: $g(y) = \lambda y - y^3 = y(\lambda - y^2) = 0$
    \[
    y = 0, \quad y = \pm\sqrt{\lambda} \quad \text{(pro $\lambda > 0$)}
    \]
    
    \item \textbf{Konstantní řešení}:
    \begin{itemize}
    \item Pro $\lambda < 0$: pouze $y = 0$
    \item Pro $\lambda = 0$: pouze $y = 0$  
    \item Pro $\lambda > 0$: $y = 0$, $y = \sqrt{\lambda}$, $y = -\sqrt{\lambda}$
    \end{itemize}
    
    \item \textbf{Obecné řešení pro $y \neq 0, \pm\sqrt{\lambda}$}:
    \[
    \int \frac{dy}{y(\lambda - y^2)} = \int dx
    \]
    Rozklad na parciální zlomky:
    \[
    \frac{1}{y(\lambda - y^2)} = \frac{1}{\lambda y} + \frac{y}{\lambda(\lambda - y^2)}
    \]
    Integrace:
    \[
    \frac{1}{\lambda} \ln|y| - \frac{1}{2\lambda} \ln|\lambda - y^2| = x + C
    \]
    \[
    \ln\left|\frac{y}{\sqrt{|\lambda - y^2|}}\right| = \lambda x + \lambda C
    \]
    
    \item \textbf{Stabilita rovnovážných bodů}:
    \begin{itemize}
    \item Pro $\lambda < 0$: $y = 0$ stabilní
    \item Pro $\lambda = 0$: $y = 0$ nestabilní
    \item Pro $\lambda > 0$: $y = 0$ nestabilní, $y = \pm\sqrt{\lambda}$ stabilní
    \end{itemize}
    
    \item \textbf{Intervaly existence}: Závisí na počáteční podmínce a parametru $\lambda$
    
    \item \textbf{Bifurkační analýza}: Jedná se o \emph{pitchfork bifurkaci} při $\lambda = 0$
    \end{enumerate}
    \end{example}

\vspace{0.8\baselineskip}

\paragraph*{Úroveň 4: Insane příklady}

\begin{example}[Rovnice s neelementárním integrálem]
    Řešte: $\frac{dy}{dx} = \frac{e^{-x^2}}{y}$
    
    \vspace{0.3\baselineskip}
    
    \textbf{Řešení}: 
    \begin{enumerate}
    \item \textbf{Singulární body}: $g(y) = \frac{1}{y}$ má singularitu v $y = 0$
    
    \item \textbf{Obecné řešení pro $y \neq 0$}: Separace:
    \[
    \int y  dy = \int e^{-x^2} dx \implies \frac{y^2}{2} = \frac{\sqrt{\pi}}{2} \erf(x) + C
    \]
    kde $\erf(x) = \frac{2}{\sqrt{\pi}} \int_0^x e^{-t^2} dt$ je error funkce.
    
    \item \textbf{Explicitní vyjádření}:
    \[
    y = \pm \sqrt{\sqrt{\pi} \erf(x) + 2C}
    \]
    
    \item \textbf{Podmínka existence}: Aby řešení bylo reálné, musí platit:
    \[
    \sqrt{\pi} \erf(x) + 2C \geq 0
    \]
    
    \item \textbf{Vlastnosti error funkce}: 
    \begin{itemize}
    \item $\erf(0) = 0$, $\lim_{x \to \infty} \erf(x) = 1$, $\lim_{x \to -\infty} \erf(x) = -1$
    \item $\erf(x)$ je lichá funkce
    \end{itemize}
    
    \item \textbf{Intervaly existence}: Závisí na konstantě $C$:
    \begin{itemize}
    \item Pro $C > \frac{\sqrt{\pi}}{2}$: řešení existuje na $\mathbb{R}$
    \item Pro $-\frac{\sqrt{\pi}}{2} < C \leq \frac{\sqrt{\pi}}{2}$: řešení existuje na omezeném intervalu
    \item Pro $C \leq -\frac{\sqrt{\pi}}{2}$: žádné reálné řešení
    \end{itemize}
    \end{enumerate}
    \end{example}

\vspace{0.6\baselineskip}

\begin{example}[Singularita a limitní chování]
    Analyzujte chování řešení: $\frac{dy}{dx} = \frac{1}{y^2 - 1}$ s $y(0) = 0$.
    
    \vspace{0.3\baselineskip}
    
    \textbf{Řešení}: 
    \begin{enumerate}
    \item \textbf{Singulární body}: $g(y) = \frac{1}{y^2 - 1}$ má singularity v $y = \pm 1$
    Konstantní řešení: $y = \pm 1$ (ověříme dosazením: $\frac{d(\pm 1)}{dx} = 0 = \frac{1}{1-1}$ - singularita)
    
    \item \textbf{Obecné řešení pro $y \neq \pm 1$}: Separace:
    \[
    \int (y^2 - 1) dy = \int dx \implies \frac{y^3}{3} - y = x + C
    \]
    
    \item \textbf{Určení konstanty}:
    \[
    y(0) = 0 \implies 0 - 0 = 0 + C \implies C = 0
    \]
    Tedy $\frac{y^3}{3} - y = x$
    
    \item \textbf{Analýza chování}:
    \begin{itemize}
    \item Pro $x \to -\frac{2}{3}$: $y \to -1$ (singularita)
    \item Pro $x \to \frac{2}{3}$: $y \to 1$ (singularita)
    \item Pro $x = 0$: $y = 0$
    \end{itemize}
    
    \item \textbf{Interval existence}: Řešení existuje na $(-\frac{2}{3}, \frac{2}{3})$
    
    \item \textbf{Implicitní křivka}: $\frac{y^3}{3} - y - x = 0$ je kubická v $y$
    
    \item \textbf{Stabilita}: 
    \begin{itemize}
    \item Pro $y < -1$: $\frac{dy}{dx} > 0$ (rostoucí)
    \item Pro $-1 < y < 1$: $\frac{dy}{dx} < 0$ (klesající)  
    \item Pro $y > 1$: $\frac{dy}{dx} > 0$ (rostoucí)
    \end{itemize}
    \end{enumerate}
    \end{example}

\vspace{0.8\baselineskip}

\paragraph*{Úroveň 5: Quant Level}

\begin{example}[Logistický růst s kalibrací]
    Mějme data o růstu tržního podílu: počáteční podíl 5\%, maximální kapacita 80\%, po 2 letech podíl 20\%. Kalibrujte logistický model.
    
    \vspace{0.3\baselineskip}
    
    \textbf{Řešení}: 
    \begin{enumerate}
    \item \textbf{Logistická rovnice}: $\frac{dP}{dt} = rP(1 - \frac{P}{K})$
    
    \item \textbf{Singulární body}: $P = 0$ a $P = K$ jsou konstantní řešení
    
    \item \textbf{Obecné řešení pro $0 < P < K$}:
    \[
    P(t) = \frac{K}{1 + Ae^{-rt}}, \quad \text{kde } A = \frac{K - P_0}{P_0}
    \]
    
    \item \textbf{Dosazení parametrů}:
    \[
    P_0 = 0.05, \quad K = 0.8, \quad P(2) = 0.2
    \]
    \[
    A = \frac{0.8 - 0.05}{0.05} = 15
    \]
    \[
    0.2 = \frac{0.8}{1 + 15e^{-2r}} \implies 1 + 15e^{-2r} = 4 \implies e^{-2r} = 0.2
    \]
    \[
    r = -\frac{1}{2}\ln(0.2) \approx 0.8047
    \]
    
    \item \textbf{Interval existence}: Řešení existuje na $\mathbb{R}$ a platí $0 < P(t) < K$ pro všechna $t \in \mathbb{R}$
    
    \item \textbf{Predikce}: Po 5 letech: $P(5) = \frac{0.8}{1 + 15e^{-0.8047 \cdot 5}} \approx 0.634$ (63.4\%)
    \end{enumerate}
    \end{example}

\vspace{0.6\baselineskip}

\begin{example}[Mertonův model defaultu]
    Zjednodušený Mertonův model: $\frac{dV}{dt} = \mu V - C$, kde $V$ je hodnota firmy, $C$ konstantní výplaty. Určete podmínky pro nesplatnost.
    
    \vspace{0.3\baselineskip}
    
    \textbf{Řešení}: 
    \begin{enumerate}
    \item \textbf{Singulární body}: Žádné konstantní řešení kromě případu $\mu = 0$, $C = 0$
    
    \item \textbf{Obecné řešení}: Rovnici přepíšeme:
    \[
    \frac{dV}{dt} = \mu\left(V - \frac{C}{\mu}\right)
    \]
    Substituce $U = V - \frac{C}{\mu}$:
    \[
    \frac{dU}{dt} = \mu U \implies U(t) = U_0 e^{\mu t}
    \]
    Tedy:
    \[
    V(t) = \frac{C}{\mu} + \left(V_0 - \frac{C}{\mu}\right)e^{\mu t}
    \]
    
    \item \textbf{Podmínka nesplatnosti}: $V(t) \leq 0$ pro nějaké $t > 0$
    
    \item \textbf{Analýza případů}:
    \begin{itemize}
    \item \textbf{Případ 1}: $\mu > 0$, $V_0 > \frac{C}{\mu}$ - hodnota roste, žádná nesplatnost
    \item \textbf{Případ 2}: $\mu > 0$, $V_0 = \frac{C}{\mu}$ - konstantní hodnota $V(t) = \frac{C}{\mu}$
    \item \textbf{Případ 3}: $\mu > 0$, $V_0 < \frac{C}{\mu}$ - hodnota klesá k $-\infty$, nesplatnost v čase $t^*$
    \item \textbf{Případ 4}: $\mu = 0$ - lineární poklad $V(t) = V_0 - Ct$, nesplatnost při $t > \frac{V_0}{C}$
    \item \textbf{Případ 5}: $\mu < 0$ - hodnota konverguje k $\frac{C}{\mu} < 0$, vždy nastane nesplatnost
    \end{itemize}
    
    \item \textbf{Interval existence}: Řešení existuje na $\mathbb{R}$ dokud nenastane nesplatnost
    
    \item \textbf{Čas nesplatnosti}: Pro $\mu > 0$, $V_0 < \frac{C}{\mu}$:
    \[
    V(t^*) = 0 \implies t^* = \frac{1}{\mu} \ln\left(\frac{C}{C - \mu V_0}\right)
    \]
    \end{enumerate}
    \end{example}

\vspace{0.8\baselineskip}

\subsubsection{Kvantitativní Aplikace}
\label{subsubsec:kvantitativni-aplikace}

\begin{application}[Exponenciální růst a úročení]
Model spojitého úročení: $\frac{dA}{dt} = rA$ má řešení $A(t) = A_0 e^{rt}$.

\textbf{Kvantitativní interpretace}:
\begin{itemize}
\item $r$: okamžitá úroková míra (force of interest)
\item $A(t)$: hodnota investice v čase $t$
\item Aplikace: Oceňování zero-coupon bondů, diskontování cash flow
\end{itemize}
\end{application}

\vspace{0.6\baselineskip}

\begin{application}[Logistické modely v ekonomii]
Logistická rovnice: $\frac{dP}{dt} = rP(1 - \frac{P}{K})$ modeluje:
\begin{itemize}
\item Difúzi inovací na trhu
\item Růst tržního podílu
\item Saturaci poptávky
\item Adopci technologií
\end{itemize}
Parametr $K$ představuje nosnou kapacitu trhu.
\end{application}

\vspace{0.6\baselineskip}

\begin{application}[Stochastická volatilita - příprava]
Deterministická aproximace: $\frac{d\sigma}{dt} = \kappa(\theta - \sigma)$ má řešení:
\[
\sigma(t) = \theta + (\sigma_0 - \theta)e^{-\kappa t}
\]
Tento model připravuje půdu pro Hestonův stochastický model volatility.
\end{application}

\vspace{0.8\baselineskip}

\subsubsection{Shrnutí a Přechod}
\label{subsubsec:shrnuti-presun}

Separabilní rovnice představují nejzákladnější, ale překvapivě mocnou třídu ODE 1. řádu. Zvládnutí této třídy je nezbytné pro:

\begin{itemize}
\item Porozumění základním růstovým modelům v ekonomii a financích
\item Analýzu stability dynamických systémů
\item Přípravu na komplexnější nelineární modely
\item Rozvoj intuice pro chování diferenciálních rovnic
\end{itemize}

\vspace{0.6\baselineskip}

\begin{remark}[Časté chyby a jak se jim vyhnout]
\begin{itemize}
\item \textbf{Ztráta řešení}: Při dělení $g(y)$ vždy zkontrolujte body kde $g(y) = 0$
\item \textbf{Špatný definiční obor}: Vždy určete maximální interval existence
\item \textbf{Chybná integrační konstanta}: Pečlivě pracujte s absolutními hodnotami a znaménky
\item \textbf{Ignorování singularity}: Analyzujte chování řešení v singulárních bodech
\end{itemize}
\end{remark}

\vspace{0.8\baselineskip}

\begin{transition}
S pevným pochopením separabilních rovnic jsme připraveni přejít k lineárním rovnicím 1. řádu, které představují jejich přirozené zobecnění a otevírají cestu k modelování systémů s vnějšími vstupy a řízením. Lineární rovnice nám umožní systematicky řešit problémy, kde separace proměnných není možná.
\end{transition}

% !TEX root = ../main.tex
\subsection{Lineární Rovnice 1. Řádu - Kompletní Teorie}
\label{subsec:linearni-rovnice}

\subsubsection{Teoretický Fundament}
\label{subsubsec:teoreticky-fundament-linearni}

\begin{definition}[Lineární diferenciální rovnice 1. řádu]
Rovnice je \emph{lineární 1. řádu}, jestliže ji lze zapsat ve standardním tvaru:
\[
\frac{dy}{dx} + P(x)y = Q(x)
\]
kde $P(x)$ a $Q(x)$ jsou funkce definované na nějakém intervalu $I \subseteq \mathbb{R}$.
\end{definition}

\vspace{0.6\baselineskip}

\begin{theorem}[Existence a jednoznačnost řešení]
Nechť $P(x)$ a $Q(x)$ jsou spojité funkce na otevřeném intervalu $I$. Pak pro libovolný bod $x_0 \in I$ a libovolnou počáteční hodnotu $y_0 \in \mathbb{R}$ existuje právě jedno řešení $y(x)$ rovnice definované na celém intervalu $I$ splňující $y(x_0) = y_0$.
\end{theorem}

\vspace{0.4\baselineskip}

\begin{proof}
Důkaz využívá Picard-Lindelöfovu větu z Kapitoly 3. Pro lineární rovnici je pravá strana $f(x,y) = Q(x) - P(x)y$ spojitá v $x$ a Lipschitzovská v $y$ na každém kompaktním podintervalu $I$, což zaručuje existenci a jednoznačnost řešení.
\end{proof}

\vspace{0.6\baselineskip}

\begin{theorem}[Struktura řešení lineární rovnice]
Obecné řešení lineární rovnice má tvar:
\[
y(x) = y_h(x) + y_p(x)
\]
kde $y_h(x)$ je obecné řešení homogenní rovnice $\frac{dy}{dx} + P(x)y = 0$ a $y_p(x)$ je partikulární řešení nehomogenní rovnice.
\end{theorem}

\vspace{0.4\baselineskip}

\begin{proof}
Homogenní rovnice má řešení $y_h(x) = Ce^{-\int P(x)dx}$. Partikulární řešení najdeme metodou integračního faktoru nebo variace konstant.
\end{proof}

\vspace{0.8\baselineskip}

\subsubsection{Metoda Integračního Faktoru - Kompletní Analýza}
\label{subsubsec:metoda-integracniho-faktoru}

\begin{method}[Metoda integračního faktoru]
\label{met:integracni-faktor}
\begin{enumerate}
\item \textbf{Standardní tvar}: Ujistěte se, že rovnice je ve tvaru $\frac{dy}{dx} + P(x)y = Q(x)$

\item \textbf{Integrační faktor}: Vypočítejte
\[
\mu(x) = e^{\int P(x)  dx}
\]
Poznámka: Integrační konstanta se zde neuvádí.

\item \textbf{Násobení rovnice}: Vynásobte celou rovnici integračním faktorem:
\[
\mu(x)\frac{dy}{dx} + \mu(x)P(x)y = \mu(x)Q(x)
\]

\item \textbf{Rozpoznání derivace součinu}: Levou stranu lze zapsat jako:
\[
\frac{d}{dx}[\mu(x)y] = \mu(x)Q(x)
\]
Toto platí protože $\frac{d\mu}{dx} = P(x)\mu(x)$.

\item \textbf{Integrace}:
\[
\mu(x)y = \int \mu(x)Q(x)  dx + C
\]

\item \textbf{Vyjádření řešení}:
\[
y(x) = \frac{1}{\mu(x)} \left[\int \mu(x)Q(x)  dx + C\right]
\]
\end{enumerate}
\end{method}

\vspace{0.8\baselineskip}

\begin{theorem}[Vlastnosti integračního faktoru]
Integrační faktor $\mu(x) = e^{\int P(x)dx}$ má následující vlastnosti:
\begin{itemize}
\item $\mu(x) > 0$ pro všechna $x \in I$
\item $\mu(x)$ je spojitě diferencovatelná na $I$
\item $\frac{d\mu}{dx} = P(x)\mu(x)$
\item $\mu(x)$ je určen jednoznačně až na multiplikativní konstantu
\end{itemize}
\end{theorem}

\vspace{0.6\baselineskip}

\begin{example}[Odvození metody integračního faktoru]
Uvažujme rovnici $\frac{dy}{dx} + P(x)y = Q(x)$. Hledáme funkci $\mu(x)$ takovou, že:
\[
\frac{d}{dx}[\mu(x)y] = \mu(x)\frac{dy}{dx} + \frac{d\mu}{dx}y = \mu(x)Q(x)
\]
Porovnáním s původní rovnicí dostaneme:
\[
\frac{d\mu}{dx} = P(x)\mu(x)
\]
Tato rovnice je separabilní:
\[
\int \frac{d\mu}{\mu} = \int P(x)dx \implies \ln|\mu| = \int P(x)dx \implies \mu(x) = e^{\int P(x)dx}
\]
\end{example}

\vspace{0.8\baselineskip}

\subsubsection{Metoda Variace Konstant}
\label{subsubsec:metoda-variance-konstant}

\begin{method}[Metoda variace konstant]
\label{met:variace-konstant}
\begin{enumerate}
\item \textbf{Řešení homogenní rovnice}: Nejprve vyřešte $\frac{dy}{dx} + P(x)y = 0$:
\[
y_h(x) = Ce^{-\int P(x)dx}
\]

\item \textbf{Variace konstanty}: Předpokládejte řešení ve tvaru:
\[
y(x) = C(x)e^{-\int P(x)dx}
\]
kde $C(x)$ je neznámá funkce.

\item \textbf{Dosazení do původní rovnice}:
\[
\frac{dy}{dx} = C'(x)e^{-\int P(x)dx} - C(x)P(x)e^{-\int P(x)dx}
\]
Dosazením do $\frac{dy}{dx} + P(x)y = Q(x)$ dostaneme:
\[
C'(x)e^{-\int P(x)dx} = Q(x)
\]

\item \textbf{Řešení pro C(x)}:
\[
C'(x) = Q(x)e^{\int P(x)dx} \implies C(x) = \int Q(x)e^{\int P(x)dx}dx + K
\]

\item \textbf{Celkové řešení}:
\[
y(x) = e^{-\int P(x)dx}\left[\int Q(x)e^{\int P(x)dx}dx + K\right]
\]
\end{enumerate}
\end{method}

\vspace{0.8\baselineskip}

\begin{theorem}[Ekvivalence metod]
Metoda integračního faktoru a metoda variace konstant jsou ekvivalentní a vedou ke stejnému výsledku.
\end{theorem}

\vspace{0.4\baselineskip}

\begin{proof}
V metodě integračního faktoru máme:
\[
y(x) = \frac{1}{\mu(x)}\left[\int \mu(x)Q(x)dx + C\right]
\]
V metodě variace konstant:
\[
y(x) = e^{-\int P(x)dx}\left[\int Q(x)e^{\int P(x)dx}dx + K\right]
\]
Protože $\mu(x) = e^{\int P(x)dx}$, jsou oba výrazy identické.
\end{proof}

\vspace{0.8\baselineskip}

\subsubsection{Klasifikace Speciálních Případů P(x)}
\label{subsubsec:klasifikace-px}

\begin{remark}[Konstantní P(x)]
Pro $P(x) = a$ (konstanta) je integrační faktor:
\[
\mu(x) = e^{\int a  dx} = e^{ax}
\]
Řešení homogenní rovnice: $y_h(x) = Ce^{-ax}$
\end{remark}

\vspace{0.6\baselineskip}

\begin{remark}[Polynomiální P(x)]
Pro $P(x) = a_nx^n + a_{n-1}x^{n-1} + \cdots + a_0$:
\[
\mu(x) = e^{\int P(x)dx} = e^{\frac{a_n}{n+1}x^{n+1} + \frac{a_{n-1}}{n}x^n + \cdots + a_0x}
\]
Integrace vyžaduje výpočet integrálu polynomu.
\end{remark}

\vspace{0.6\baselineskip}

\begin{remark}[Racionální P(x)]
Pro $P(x) = \frac{N(x)}{D(x)}$ je třeba integrovat racionální funkci. Použijeme rozklad na parciální zlomky:
\[
\int \frac{N(x)}{D(x)} dx = \int \left(\text{parciální zlomky}\right) dx
\]
\end{remark}

\vspace{0.6\baselineskip}

\begin{remark}[Goniometrické P(x)]
Pro $P(x)$ obsahující goniometrické funkce může integrace vyžadovat trigonometrické substituce nebo použití speciálních technik.
\end{remark}

\vspace{0.8\baselineskip}

\subsubsection{Početní Sekce - Kategorie A: Podle typu P(x)}
\label{subsubsec:pocetni-kategorie-a-px}

\paragraph*{A1: Konstantní P(x)}

\begin{example}[Lehký příklad - konstantní P(x)]
Řešte: $\frac{dy}{dx} + 2y = 1$
\vspace{0.3\baselineskip}

\textbf{Řešení}: 
\begin{enumerate}
\item $P(x) = 2$, $Q(x) = 1$ (konstantní)

\item \textbf{Integrační faktor}:
\[
\mu(x) = e^{\int 2  dx} = e^{2x}
\]

\item \textbf{Násobení rovnice}:
\[
e^{2x}\frac{dy}{dx} + 2e^{2x}y = e^{2x}
\]

\item \textbf{Integrace}:
\[
\frac{d}{dx}[e^{2x}y] = e^{2x} \implies e^{2x}y = \int e^{2x} dx = \frac{1}{2}e^{2x} + C
\]

\item \textbf{Výsledek}:
\[
y(x) = \frac{1}{2} + Ce^{-2x}
\]

\item \textbf{Ověření}:
\[
\frac{dy}{dx} = -2Ce^{-2x}, \quad 2y = 1 + 2Ce^{-2x}
\]
\[
\frac{dy}{dx} + 2y = -2Ce^{-2x} + 1 + 2Ce^{-2x} = 1 \quad \checkmark
\end{enumerate}
\end{example}

\vspace{0.6\baselineskip}

\begin{example}[Střední příklad - konstantní P(x)]
Řešte: $\frac{dy}{dx} - 3y = e^x$ s $y(0) = 1$
\vspace{0.3\baselineskip}

\textbf{Řešení}: 
\begin{enumerate}
\item $P(x) = -3$, $Q(x) = e^x$

\item \textbf{Integrační faktor}:
\[
\mu(x) = e^{\int -3  dx} = e^{-3x}
\]

\item \textbf{Násobení a integrace}:
\[
\frac{d}{dx}[e^{-3x}y] = e^{-3x}e^x = e^{-2x}
\]
\[
e^{-3x}y = \int e^{-2x} dx = -\frac{1}{2}e^{-2x} + C
\]

\item \textbf{Obecné řešení}:
\[
y(x) = -\frac{1}{2}e^{x} + Ce^{3x}
\]

\item \textbf{Určení konstanty}:
\[
y(0) = 1 \implies -\frac{1}{2} + C = 1 \implies C = \frac{3}{2}
\]

\item \textbf{Výsledek}:
\[
y(x) = -\frac{1}{2}e^{x} + \frac{3}{2}e^{3x}
\end{enumerate}
\end{example}

\vspace{0.6\baselineskip}

\begin{example}[Složitý příklad - konstantní P(x)]
Řešte: $\frac{dy}{dx} + 5y = \sin(2x)$
\vspace{0.3\baselineskip}

\textbf{Řešení}: 
\begin{enumerate}
\item $P(x) = 5$, $Q(x) = \sin(2x)$

\item \textbf{Integrační faktor}:
\[
\mu(x) = e^{\int 5  dx} = e^{5x}
\]

\item \textbf{Integrace}:
\[
\frac{d}{dx}[e^{5x}y] = e^{5x}\sin(2x)
\]
\[
e^{5x}y = \int e^{5x}\sin(2x) dx
\]

\item \textbf{Integrace per partes}: Použijeme metodu pro $\int e^{ax}\sin(bx)dx$:
\[
\int e^{5x}\sin(2x)dx = \frac{e^{5x}}{29}(5\sin(2x) - 2\cos(2x)) + K
\]

\item \textbf{Výsledek}:
\[
y(x) = \frac{1}{29}(5\sin(2x) - 2\cos(2x)) + Ke^{-5x}
\end{enumerate}
\end{example}

\vspace{0.8\baselineskip}

\paragraph*{A2: Polynomiální P(x)}

\begin{example}[Lehký příklad - polynomiální P(x)]
Řešte: $\frac{dy}{dx} + xy = 1$
\vspace{0.3\baselineskip}

\textbf{Řešení}: 
\begin{enumerate}
\item $P(x) = x$, $Q(x) = 1$

\item \textbf{Integrační faktor}:
\[
\mu(x) = e^{\int x  dx} = e^{x^2/2}
\]

\item \textbf{Integrace}:
\[
\frac{d}{dx}[e^{x^2/2}y] = e^{x^2/2}
\]
\[
e^{x^2/2}y = \int e^{x^2/2} dx
\]

\item \textbf{Integrál nelze vyjádřit elementárními funkcemi}:
\[
y(x) = e^{-x^2/2}\left[\int e^{x^2/2} dx + C\right]
\]

\item \textbf{Řešení pomocí error funkce}:
\[
\int e^{x^2/2} dx = \sqrt{\frac{\pi}{2}} \erf\left(\frac{x}{\sqrt{2}}\right) + K
\]
\[
y(x) = e^{-x^2/2}\left[\sqrt{\frac{\pi}{2}} \erf\left(\frac{x}{\sqrt{2}}\right) + C\right]
\end{enumerate}
\end{example}

\vspace{0.6\baselineskip}

\begin{example}[Střední příklad - polynomiální P(x)]
Řešte: $\frac{dy}{dx} + x^2y = x$
\vspace{0.3\baselineskip}

\textbf{Řešení}: 
\begin{enumerate}
\item $P(x) = x^2$, $Q(x) = x$

\item \textbf{Integrační faktor}:
\[
\mu(x) = e^{\int x^2 dx} = e^{x^3/3}
\]

\item \textbf{Integrace}:
\[
\frac{d}{dx}[e^{x^3/3}y] = xe^{x^3/3}
\]
\[
e^{x^3/3}y = \int xe^{x^3/3} dx
\]

\item \textbf{Substituce}: $u = x^3/3 \implies du = x^2 dx$ - nefunguje přímo

\item \textbf{Řešení pomocí řady}: Rozvineme $e^{x^3/3}$ do řady:
\[
e^{x^3/3} = \sum_{n=0}^{\infty} \frac{x^{3n}}{3^n n!}
\]
\[
xe^{x^3/3} = \sum_{n=0}^{\infty} \frac{x^{3n+1}}{3^n n!}
\]
\[
\int xe^{x^3/3} dx = \sum_{n=0}^{\infty} \frac{x^{3n+2}}{(3n+2)3^n n!} + C
\]

\item \textbf{Výsledek}:
\[
y(x) = e^{-x^3/3}\left[\sum_{n=0}^{\infty} \frac{x^{3n+2}}{(3n+2)3^n n!} + C\right]
\end{enumerate}
\end{example}

\vspace{0.8\baselineskip}


\subsubsection{Početní Sekce - Kategorie A (pokračování)}
\label{subsubsec:pocetni-kategorie-a-px-pokracovani}

\paragraph*{A3: Racionální P(x)}

\begin{example}[Lehký příklad - racionální P(x)]
Řešte: $\frac{dy}{dx} + \frac{1}{x}y = 1$
\vspace{0.3\baselineskip}

\textbf{Řešení}: 
\begin{enumerate}
\item $P(x) = \frac{1}{x}$, $Q(x) = 1$, $x \neq 0$

\item \textbf{Integrační faktor}:
\[
\mu(x) = e^{\int \frac{1}{x} dx} = e^{\ln|x|} = |x|
\]
Pro $x > 0$: $\mu(x) = x$, pro $x < 0$: $\mu(x) = -x$

\item \textbf{Řešení pro $x > 0$}:
\[
\frac{d}{dx}[xy] = x \implies xy = \int x  dx = \frac{x^2}{2} + C
\]
\[
y(x) = \frac{x}{2} + \frac{C}{x}
\]

\item \textbf{Interval existence}: $(0, \infty)$

\item \textbf{Ověření}:
\[
\frac{dy}{dx} = \frac{1}{2} - \frac{C}{x^2}, \quad \frac{1}{x}y = \frac{1}{2} + \frac{C}{x^2}
\]
\[
\frac{dy}{dx} + \frac{1}{x}y = \frac{1}{2} - \frac{C}{x^2} + \frac{1}{2} + \frac{C}{x^2} = 1 \quad \checkmark
\end{enumerate}
\end{example}

\vspace{0.6\baselineskip}

\begin{example}[Střední příklad - racionální P(x)]
Řešte: $\frac{dy}{dx} + \frac{2x}{x^2 + 1}y = x$
\vspace{0.3\baselineskip}

\textbf{Řešení}: 
\begin{enumerate}
\item $P(x) = \frac{2x}{x^2 + 1}$, $Q(x) = x$

\item \textbf{Integrační faktor}:
\[
\int P(x)dx = \int \frac{2x}{x^2 + 1}dx = \ln(x^2 + 1)
\]
\[
\mu(x) = e^{\ln(x^2 + 1)} = x^2 + 1
\]

\item \textbf{Integrace}:
\[
\frac{d}{dx}[(x^2 + 1)y] = x(x^2 + 1) = x^3 + x
\]
\[
(x^2 + 1)y = \int (x^3 + x)dx = \frac{x^4}{4} + \frac{x^2}{2} + C
\]

\item \textbf{Výsledek}:
\[
y(x) = \frac{\frac{x^4}{4} + \frac{x^2}{2} + C}{x^2 + 1} = \frac{x^2(x^2 + 2)}{4(x^2 + 1)} + \frac{C}{x^2 + 1}
\]

\item \textbf{Interval existence}: $\mathbb{R}$
\end{enumerate}
\end{example}

\vspace{0.6\baselineskip}

\begin{example}[Složitý příklad - racionální P(x)]
Řešte: $\frac{dy}{dx} + \frac{1}{x-1}y = \frac{1}{x^2 - 1}$ s $y(0) = 2$
\vspace{0.3\baselineskip}

\textbf{Řešení}: 
\begin{enumerate}
\item $P(x) = \frac{1}{x-1}$, $Q(x) = \frac{1}{(x-1)(x+1)}$, $x \neq \pm 1$

\item \textbf{Integrační faktor}:
\[
\mu(x) = e^{\int \frac{1}{x-1}dx} = e^{\ln|x-1|} = |x-1|
\]
Pro $x < 1$: $\mu(x) = 1-x$, pro $x > 1$: $\mu(x) = x-1$

\item \textbf{Řešení pro $x < 1$} (kde $y(0) = 2$):
\[
\frac{d}{dx}[(1-x)y] = (1-x)\cdot\frac{1}{(x-1)(x+1)} = -\frac{1}{x+1}
\]
\[
(1-x)y = -\int \frac{1}{x+1}dx = -\ln|x+1| + C
\]

\item \textbf{Určení konstanty}:
\[
y(0) = 2 \implies (1-0)\cdot 2 = -\ln|1| + C \implies 2 = C
\]
\[
y(x) = \frac{-\ln|x+1| + 2}{1-x} = \frac{2 - \ln(x+1)}{1-x} \quad \text{pro } -1 < x < 1
\]

\item \textbf{Interval existence}: $(-1, 1)$
\end{enumerate}
\end{example}

\vspace{0.8\baselineskip}

\paragraph*{A4: Goniometrické P(x)}

\begin{example}[Lehký příklad - goniometrické P(x)]
Řešte: $\frac{dy}{dx} + (\tan x)y = \cos x$ pro $x \in (-\frac{\pi}{2}, \frac{\pi}{2})$
\vspace{0.3\baselineskip}

\textbf{Řešení}: 
\begin{enumerate}
\item $P(x) = \tan x$, $Q(x) = \cos x$

\item \textbf{Integrační faktor}:
\[
\int \tan x  dx = \int \frac{\sin x}{\cos x} dx = -\ln|\cos x|
\]
\[
\mu(x) = e^{-\ln|\cos x|} = \frac{1}{|\cos x|}
\]
Pro $x \in (-\frac{\pi}{2}, \frac{\pi}{2})$: $\cos x > 0$, tedy $\mu(x) = \frac{1}{\cos x} = \sec x$

\item \textbf{Integrace}:
\[
\frac{d}{dx}[\sec x \cdot y] = \sec x \cdot \cos x = 1
\]
\[
\sec x \cdot y = \int 1  dx = x + C
\]

\item \textbf{Výsledek}:
\[
y(x) = x \cos x + C \cos x
\]

\item \textbf{Interval existence}: $(-\frac{\pi}{2}, \frac{\pi}{2})$
\end{enumerate}
\end{example}

\vspace{0.6\baselineskip}

\begin{example}[Střední příklad - goniometrické P(x)]
Řešte: $\frac{dy}{dx} + (\sin x)y = e^{\cos x}$
\vspace{0.3\baselineskip}

\textbf{Řešení}: 
\begin{enumerate}
\item $P(x) = \sin x$, $Q(x) = e^{\cos x}$

\item \textbf{Integrační faktor}:
\[
\int \sin x  dx = -\cos x
\]
\[
\mu(x) = e^{-\cos x}
\]

\item \textbf{Integrace}:
\[
\frac{d}{dx}[e^{-\cos x}y] = e^{-\cos x} \cdot e^{\cos x} = 1
\]
\[
e^{-\cos x}y = \int 1  dx = x + C
\]

\item \textbf{Výsledek}:
\[
y(x) = (x + C)e^{\cos x}
\]

\item \textbf{Interval existence}: $\mathbb{R}$
\end{enumerate}
\end{example}

\vspace{0.8\baselineskip}

\subsubsection{Klasifikace Pravých Stran Q(x)}
\label{subsubsec:klasifikace-qx}

\paragraph*{B1: Polynomiální Q(x)}

\begin{remark}[Obecný postup pro polynomiální Q(x)]
Pro $Q(x) = a_nx^n + a_{n-1}x^{n-1} + \cdots + a_0$:
\begin{itemize}
\item Vypočítejte $\mu(x) = e^{\int P(x)dx}$
\item Spočítejte $\int \mu(x)Q(x)dx$ - typicky vede na integraci součinu
\item Výsledek může obsahovat neelementární integrály pro složitější P(x)
\end{itemize}
\end{remark}

\vspace{0.6\baselineskip}

\begin{example}[Konstantní Q(x)]
Řešte: $\frac{dy}{dx} + xy = 2$
\vspace{0.3\baselineskip}

\textbf{Řešení}: 
\begin{enumerate}
\item $P(x) = x$, $Q(x) = 2$ (konstantní)

\item \textbf{Integrační faktor}: $\mu(x) = e^{x^2/2}$

\item \textbf{Integrace}:
\[
\frac{d}{dx}[e^{x^2/2}y] = 2e^{x^2/2}
\]
\[
e^{x^2/2}y = 2\int e^{x^2/2}dx
\]

\item \textbf{Výsledek pomocí error funkce}:
\[
y(x) = 2e^{-x^2/2} \cdot \sqrt{\frac{\pi}{2}} \erf\left(\frac{x}{\sqrt{2}}\right) + Ce^{-x^2/2}
\]
\end{enumerate}
\end{example}

\vspace{0.6\baselineskip}

\paragraph*{B2: Exponenciální Q(x)}

\begin{remark}[Obecný postup pro exponenciální Q(x)]
Pro $Q(x) = Ae^{\alpha x}$:
\begin{itemize}
\item Pokud $\alpha \neq -P(x)$, řešení obsahuje exponenciální funkce
\item Speciální případ: rezonance když $\alpha = -P(x)$ pro konstantní P(x)
\item Integrace typicky vede na kombinace exponenciálních funkcí
\end{itemize}
\end{remark}

\vspace{0.6\baselineskip}

\begin{example}[Exponenciální Q(x) s konstantní P(x)]
Řešte: $\frac{dy}{dx} + 2y = 3e^{4x}$
\vspace{0.3\baselineskip}

\textbf{Řešení}: 
\begin{enumerate}
\item $P(x) = 2$, $Q(x) = 3e^{4x}$

\item \textbf{Integrační faktor}: $\mu(x) = e^{2x}$

\item \textbf{Integrace}:
\[
\frac{d}{dx}[e^{2x}y] = 3e^{2x}e^{4x} = 3e^{6x}
\]
\[
e^{2x}y = 3\int e^{6x}dx = \frac{1}{2}e^{6x} + C
\]

\item \textbf{Výsledek}:
\[
y(x) = \frac{1}{2}e^{4x} + Ce^{-2x}
\]
\end{enumerate}
\end{example}

\vspace{0.6\baselineskip}

\begin{example}[Rezonanční případ]
Řešte: $\frac{dy}{dx} + 2y = 3e^{-2x}$
\vspace{0.3\baselineskip}

\textbf{Řešení}: 
\begin{enumerate}
\item $P(x) = 2$, $Q(x) = 3e^{-2x}$ - rezonance protože $-P(x) = -2$

\item \textbf{Integrační faktor}: $\mu(x) = e^{2x}$

\item \textbf{Integrace}:
\[
\frac{d}{dx}[e^{2x}y] = 3e^{2x}e^{-2x} = 3
\]
\[
e^{2x}y = 3\int 1  dx = 3x + C
\]

\item \textbf{Výsledek}:
\[
y(x) = 3xe^{-2x} + Ce^{-2x}
\]
Všimněte si lineárního členu $x$ v partikulárním řešení - typické pro rezonanci.
\end{enumerate}
\end{example}

\vspace{0.8\baselineskip}

\paragraph*{B3: Goniometrické Q(x)}

\begin{remark}[Obecný postup pro goniometrické Q(x)]
Pro $Q(x) = A\sin(\omega x) + B\cos(\omega x)$:
\begin{itemize}
\item Řešení typicky obsahuje kombinace goniometrických funkcí
\item Pro konstantní P(x) lze použít metodu neurčitých koeficientů
\item Integrace může vyžadovat goniometrické identity
\end{itemize}
\end{remark}

\vspace{0.6\baselineskip}

\begin{example}[Goniometrické Q(x) s konstantní P(x)]
Řešte: $\frac{dy}{dx} + y = \sin x$
\vspace{0.3\baselineskip}

\textbf{Řešení}: 
\begin{enumerate}
\item $P(x) = 1$, $Q(x) = \sin x$

\item \textbf{Integrační faktor}: $\mu(x) = e^{x}$

\item \textbf{Integrace}:
\[
\frac{d}{dx}[e^{x}y] = e^{x}\sin x
\]
\[
e^{x}y = \int e^{x}\sin x  dx
\]

\item \textbf{Integrace per partes}:
\[
\int e^{x}\sin x  dx = \frac{e^{x}}{2}(\sin x - \cos x) + C
\]

\item \textbf{Výsledek}:
\[
y(x) = \frac{1}{2}(\sin x - \cos x) + Ce^{-x}
\]
\end{enumerate}
\end{example}

\vspace{0.8\baselineskip}

\subsubsection{Aplikace v Kvantitativních Vědách}
\label{subsubsec:aplikace-kvantitativni-linear}

\begin{application}[Spojité úročení s pravidelnými vklady]
Model: $\frac{dA}{dt} = rA + I(t)$, kde:
\begin{itemize}
\item $A(t)$: výše účtu v čase $t$
\item $r$: úroková míra
\item $I(t)$: okamžitá intenzita vkladů
\end{itemize}

\textbf{Řešení}: Jedná se o lineární rovnici s $P(t) = -r$, $Q(t) = I(t)$

\textbf{Integrační faktor}: $\mu(t) = e^{-rt}$

\textbf{Obecné řešení}:
\[
A(t) = e^{rt}\left[\int I(\tau)e^{-r\tau}d\tau + C\right]
\]

\textbf{Příklad}: Pro konstantní vklady $I(t) = I_0$:
\[
A(t) = e^{rt}\left[I_0\int e^{-r\tau}d\tau + C\right] = e^{rt}\left[-\frac{I_0}{r}e^{-r\tau} + C\right]
\]
\[
A(t) = -\frac{I_0}{r} + Ce^{rt}
\]
S počáteční podmínkou $A(0) = A_0$: $C = A_0 + \frac{I_0}{r}$
\[
A(t) = \left(A_0 + \frac{I_0}{r}\right)e^{rt} - \frac{I_0}{r}
\]
\end{application}

\vspace{0.6\baselineskip}

\begin{application}[RC obvod s proměnným napětím]
Model: $RC\frac{dV_C}{dt} + V_C = V_{in}(t)$, kde:
\begin{itemize}
\item $V_C(t)$: napětí na kondenzátoru
\item $V_{in}(t)$: vstupní napětí
\item $R$: odpor, $C$: kapacita
\end{itemize}

\textbf{Řešení}: Přepíšeme na standardní tvar:
\[
\frac{dV_C}{dt} + \frac{1}{RC}V_C = \frac{1}{RC}V_{in}(t)
\]

\textbf{Integrační faktor}: $\mu(t) = e^{t/RC}$

\textbf{Obecné řešení}:
\[
V_C(t) = e^{-t/RC}\left[\frac{1}{RC}\int V_{in}(\tau)e^{\tau/RC}d\tau + C\right]
\]

\textbf{Příklad}: Pro konstantní vstupní napětí $V_{in}(t) = V_0$:
\[
V_C(t) = e^{-t/RC}\left[\frac{V_0}{RC}\int e^{\tau/RC}d\tau + C\right] = e^{-t/RC}\left[V_0 e^{\tau/RC} + C\right]
\]
\[
V_C(t) = V_0 + Ce^{-t/RC}
\]
\end{application}

\vspace{0.6\baselineskip}

\begin{application}[Míchací problém s proměnnou koncentrací]
Model: $\frac{dS}{dt} = r_{in}c_{in}(t) - \frac{r_{out}}{V}S$, kde:
\begin{itemize}
\item $S(t)$: množství solutu v nádrži
\item $V$: objem nádrže
\item $r_{in}$, $r_{out}$: průtoky dovnitř/ven
\item $c_{in}(t)$: koncentrace v přítoku
\end{itemize}

\textbf{Řešení}: Standardní tvar:
\[
\frac{dS}{dt} + \frac{r_{out}}{V}S = r_{in}c_{in}(t)
\]

\textbf{Integrační faktor}: $\mu(t) = e^{r_{out}t/V}$

\textbf{Obecné řešení}:
\[
S(t) = e^{-r_{out}t/V}\left[r_{in}\int c_{in}(\tau)e^{r_{out}\tau/V}d\tau + C\right]
\]
\end{application}

\vspace{0.8\baselineskip}

\begin{transition}
V poslední části pokryjeme kombinované příklady, numerickou verifikaci a kompletní shrnutí metodologie lineárních rovnic 1. řádu.
\end{transition}

% !TEX root = ../main.tex
% ČÁST 3/3 - Dokončení 1.2 Lineární Rovnice

\subsubsection{Početní Sekce - Kategorie C: Kombinované Příklady}
\label{subsubsec:pocetni-kategorie-c-linear}

\paragraph*{C1: Jednoduché kombinace}

\begin{example}[Konstantní P(x) + Polynomiální Q(x)]
Řešte: $\frac{dy}{dx} + 2y = x^2 + 3x + 1$
\vspace{0.3\baselineskip}

\textbf{Řešení}: 
\begin{enumerate}
\item $P(x) = 2$, $Q(x) = x^2 + 3x + 1$

\item \textbf{Integrační faktor}: $\mu(x) = e^{2x}$

\item \textbf{Integrace}:
\[
\frac{d}{dx}[e^{2x}y] = e^{2x}(x^2 + 3x + 1)
\]
\[
e^{2x}y = \int e^{2x}(x^2 + 3x + 1)dx
\]

\item \textbf{Integrace per partes}:
\[
\int x^2 e^{2x}dx = \frac{e^{2x}}{4}(2x^2 - 2x + 1)
\]
\[
\int 3x e^{2x}dx = \frac{3e^{2x}}{4}(2x - 1)
\]
\[
\int e^{2x}dx = \frac{e^{2x}}{2}
\]

\item \textbf{Výsledek}:
\[
e^{2x}y = \frac{e^{2x}}{4}(2x^2 + 4x + 1) + C
\]
\[
y(x) = \frac{1}{4}(2x^2 + 4x + 1) + Ce^{-2x}
\]

\item \textbf{Ověření}:
\[
\frac{dy}{dx} = x + 1 - 2Ce^{-2x}, \quad 2y = \frac{1}{2}(2x^2 + 4x + 1) + 2Ce^{-2x}
\]
\[
\frac{dy}{dx} + 2y = x + 1 + x^2 + 2x + \frac{1}{2} = x^2 + 3x + \frac{3}{2}
\]
Oprava: $\frac{1}{4}(2x^2 + 4x + 1) = \frac{1}{2}x^2 + x + \frac{1}{4}$, tedy:
\[
\frac{dy}{dx} + 2y = x^2 + 3x + 1 \quad \checkmark
\]
\end{enumerate}
\end{example}

\vspace{0.6\baselineskip}

\begin{example}[Polynomiální P(x) + Exponenciální Q(x)]
Řešte: $\frac{dy}{dx} + xy = e^{-x^2/2}$
\vspace{0.3\baselineskip}

\textbf{Řešení}: 
\begin{enumerate}
\item $P(x) = x$, $Q(x) = e^{-x^2/2}$

\item \textbf{Integrační faktor}: $\mu(x) = e^{x^2/2}$

\item \textbf{Integrace}:
\[
\frac{d}{dx}[e^{x^2/2}y] = e^{x^2/2} \cdot e^{-x^2/2} = 1
\]
\[
e^{x^2/2}y = \int 1  dx = x + C
\]

\item \textbf{Výsledek}:
\[
y(x) = xe^{-x^2/2} + Ce^{-x^2/2}
\]

\item \textbf{Interval existence}: $\mathbb{R}$
\end{enumerate}
\end{example}

\vspace{0.8\baselineskip}

\paragraph*{C2: Střední kombinace}

\begin{example}[Racionální P(x) + Goniometrické Q(x)]
Řešte: $\frac{dy}{dx} + \frac{1}{x}y = \sin x$ pro $x > 0$
\vspace{0.3\baselineskip}

\textbf{Řešení}: 
\begin{enumerate}
\item $P(x) = \frac{1}{x}$, $Q(x) = \sin x$

\item \textbf{Integrační faktor}: $\mu(x) = e^{\int \frac{1}{x}dx} = x$

\item \textbf{Integrace}:
\[
\frac{d}{dx}[xy] = x\sin x
\]
\[
xy = \int x\sin x  dx
\]

\item \textbf{Integrace per partes}:
\[
\int x\sin x  dx = -x\cos x + \sin x + C
\]

\item \textbf{Výsledek}:
\[
y(x) = -\cos x + \frac{\sin x}{x} + \frac{C}{x}
\]

\item \textbf{Interval existence}: $(0, \infty)$
\end{enumerate}
\end{example}

\vspace{0.6\baselineskip}

\begin{example}[Goniometrické P(x) + Polynomiální Q(x)]
Řešte: $\frac{dy}{dx} + (\cos x)y = x$ s $y(0) = 1$
\vspace{0.3\baselineskip}

\textbf{Řešení}: 
\begin{enumerate}
\item $P(x) = \cos x$, $Q(x) = x$

\item \textbf{Integrační faktor}: $\mu(x) = e^{\int \cos x  dx} = e^{\sin x}$

\item \textbf{Integrace}:
\[
\frac{d}{dx}[e^{\sin x}y] = xe^{\sin x}
\]
\[
e^{\sin x}y = \int xe^{\sin x}dx
\]

\item \textbf{Integrál nelze vyjádřit elementárními funkcemi}:
\[
y(x) = e^{-\sin x}\left[\int xe^{\sin x}dx + C\right]
\]

\item \textbf{Určení konstanty}:
\[
y(0) = 1 \implies e^{-\sin 0}\left[\int_0^0 \tau e^{\sin \tau}d\tau + C\right] = 1 \implies C = 1
\]

\item \textbf{Výsledek}:
\[
y(x) = e^{-\sin x}\left[\int_0^x \tau e^{\sin \tau}d\tau + 1\right]
\]

\item \textbf{Interval existence}: $\mathbb{R}$
\end{enumerate}
\end{example}

\vspace{0.8\baselineskip}

\paragraph*{C3: Složité kombinace}

\begin{example}[Singulární P(x) + Speciální Q(x)]
Řešte: $\frac{dy}{dx} + \frac{1}{x-2}y = \frac{1}{(x-2)^2}$ s $y(1) = 0$
\vspace{0.3\baselineskip}

\textbf{Řešení}: 
\begin{enumerate}
\item $P(x) = \frac{1}{x-2}$, $Q(x) = \frac{1}{(x-2)^2}$, $x \neq 2$

\item \textbf{Integrační faktor}:
\[
\mu(x) = e^{\int \frac{1}{x-2}dx} = e^{\ln|x-2|} = |x-2|
\]
Pro $x < 2$: $\mu(x) = 2-x$, pro $x > 2$: $\mu(x) = x-2$

\item \textbf{Řešení pro $x < 2$} (kde $y(1) = 0$):
\[
\frac{d}{dx}[(2-x)y] = (2-x)\cdot\frac{1}{(x-2)^2} = -\frac{1}{x-2}
\]
\[
(2-x)y = -\int \frac{1}{x-2}dx = -\ln|2-x| + C
\]

\item \textbf{Určení konstanty}:
\[
y(1) = 0 \implies (2-1)\cdot 0 = -\ln|1| + C \implies C = 0
\]
\[
y(x) = -\frac{\ln(2-x)}{2-x} = \frac{\ln(2-x)}{x-2} \quad \text{pro } x < 2
\]

\item \textbf{Interval existence}: $(-\infty, 2)$
\end{enumerate}
\end{example}

\vspace{0.8\baselineskip}

\subsubsection{Početní Sekce - Kategorie D: Aplikované Příklady}
\label{subsubsec:pocetni-kategorie-d-linear}

\paragraph*{D1: Finanční modely}

\begin{example}[Spořicí účet s proměnnými vklady]
Mějme spořicí účet s úrokovou mírou 5\% p.a. spojitě. Vkládáme pravidelně s intenzitou $I(t) = 1000 + 100t$ Kč/rok. Najděte výši účtu po 10 letech, pokud počáteční vklad byl 5000 Kč.
\vspace{0.3\baselineskip}

\textbf{Řešení}: 
\begin{enumerate}
\item \textbf{Model}: $\frac{dA}{dt} = 0.05A + (1000 + 100t)$, $A(0) = 5000$

\item \textbf{Integrační faktor}: $\mu(t) = e^{-0.05t}$

\item \textbf{Integrace}:
\[
\frac{d}{dt}[e^{-0.05t}A] = e^{-0.05t}(1000 + 100t)
\]
\[
e^{-0.05t}A = \int e^{-0.05t}(1000 + 100t)dt
\]

\item \textbf{Integrace per partes}:
\[
\int 1000e^{-0.05t}dt = -20000e^{-0.05t}
\]
\[
\int 100te^{-0.05t}dt = -2000te^{-0.05t} - 40000e^{-0.05t}
\]

\item \textbf{Obecné řešení}:
\[
A(t) = e^{0.05t}\left[-20000e^{-0.05t} - 2000te^{-0.05t} - 40000e^{-0.05t} + C\right]
\]
\[
A(t) = -60000 - 2000t + Ce^{0.05t}
\]

\item \textbf{Určení konstanty}:
\[
A(0) = 5000 \implies -60000 + C = 5000 \implies C = 65000
\]

\item \textbf{Výsledek}:
\[
A(t) = -60000 - 2000t + 65000e^{0.05t}
\]
\[
A(10) = -60000 - 20000 + 65000e^{0.5} \approx -80000 + 65000 \cdot 1.6487 \approx 27,165 \text{ Kč}
\end{enumerate}
\end{example}

\vspace{0.8\baselineskip}

\paragraph*{D2: Fyzikální modely}

\begin{example}[RC obvod s proměnným napětím]
RC obvod s $R = 1000\ \Omega$, $C = 0.001\ \mathrm{F}$ je napájen napětím $V_{in}(t) = 10\sin(100t)\ \mathrm{V}$. Najděte napětí na kondenzátoru, pokud $V_C(0) = 0$.
\vspace{0.3\baselineskip}

\textbf{Řešení}: 
\begin{enumerate}
\item \textbf{Model}: $RC\frac{dV_C}{dt} + V_C = V_{in}(t)$
\[
0.001\frac{dV_C}{dt} + V_C = 10\sin(100t)
\]
\[
\frac{dV_C}{dt} + 1000V_C = 10000\sin(100t)
\]

\item \textbf{Integrační faktor}: $\mu(t) = e^{1000t}$

\item \textbf{Integrace}:
\[
\frac{d}{dt}[e^{1000t}V_C] = 10000e^{1000t}\sin(100t)
\]
\[
e^{1000t}V_C = 10000\int e^{1000t}\sin(100t)dt
\]

\item \textbf{Integrace per partes}:
\[
\int e^{1000t}\sin(100t)dt = \frac{e^{1000t}}{1000000 + 10000}(1000\sin(100t) - 100\cos(100t)) + K
\]
\[
= \frac{e^{1000t}}{1010000}(1000\sin(100t) - 100\cos(100t)) + K
\]

\item \textbf{Výsledek}:
\[
V_C(t) = \frac{10000}{1010000}(1000\sin(100t) - 100\cos(100t)) + Ke^{-1000t}
\]
\[
V_C(t) = \frac{100}{101}\sin(100t) - \frac{10}{101}\cos(100t) + Ke^{-1000t}
\]

\item \textbf{Určení konstanty}:
\[
V_C(0) = 0 \implies -\frac{10}{101} + K = 0 \implies K = \frac{10}{101}
\]

\item \textbf{Konečný výsledek}:
\[
V_C(t) = \frac{100}{101}\sin(100t) - \frac{10}{101}\cos(100t) + \frac{10}{101}e^{-1000t}
\]
\end{enumerate}
\end{example}

\vspace{0.8\baselineskip}

\subsubsection{Numerická Verifikace}
\label{subsubsec:numericka-verifikace}

\begin{method}[Eulerova metoda pro lineární rovnice]
\label{met:euler-linearni}
Pro rovnici $\frac{dy}{dx} + P(x)y = Q(x)$ s počáteční podmínkou $y(x_0) = y_0$:

\begin{enumerate}
\item Zvolte krok $h > 0$
\item Pro $n = 0, 1, 2, \dots$ počítáme:
\[
y_{n+1} = y_n + h[Q(x_n) - P(x_n)y_n]
\]
\[
x_{n+1} = x_n + h
\]
\item Chyba metody: $O(h)$
\end{enumerate}
\end{method}

\vspace{0.6\baselineskip}

\begin{example}[Numerická verifikace jednoduché rovnice]
Porovnejte analytické a numerické řešení: $\frac{dy}{dx} + 2y = 1$, $y(0) = 0$
\vspace{0.3\baselineskip}

\textbf{Analytické řešení}: $y(x) = \frac{1}{2}(1 - e^{-2x})$

\textbf{Numerické řešení} (krok $h = 0.1$):
\begin{center}
\begin{tabular}{c|c|c|c}
$n$ & $x_n$ & Analytické $y(x_n)$ & Numerické $y_n$ \\
\hline
0 & 0.0 & 0.0000 & 0.0000 \\
1 & 0.1 & 0.0906 & 0.1000 \\
2 & 0.2 & 0.1648 & 0.1800 \\
3 & 0.3 & 0.2260 & 0.2440 \\
4 & 0.4 & 0.2763 & 0.2952 \\
5 & 0.5 & 0.3161 & 0.3362 \\
\end{tabular}
\end{center}

\textbf{Analýza chyby}: Maximální chyba ≈ 0.0094 při $x=0.1$, klesá s klesajícím $h$.
\end{example}

\vspace{0.6\baselineskip}

\begin{example}[Numerická verifikace složité rovnice]
Ověřte numericky řešení: $\frac{dy}{dx} + \frac{1}{x}y = 1$, $y(1) = 0$ pro $x > 0$
\vspace{0.3\baselineskip}

\textbf{Analytické řešení}: $y(x) = \frac{x}{2} - \frac{1}{2x}$

\textbf{Numerické řešení} (krok $h = 0.1$):
\begin{center}
\begin{tabular}{c|c|c|c}
$x_n$ & Analytické & Numerické & Chyba \\
\hline
1.0 & 0.0000 & 0.0000 & 0.0000 \\
1.1 & 0.0955 & 0.1000 & 0.0045 \\
1.2 & 0.1833 & 0.1909 & 0.0076 \\
1.3 & 0.2654 & 0.2745 & 0.0091 \\
1.4 & 0.3429 & 0.3529 & 0.0100 \\
1.5 & 0.4167 & 0.4275 & 0.0108 \\
\end{tabular}
\end{center}

\textbf{Závěr}: Numerické řešení dobře aproximuje analytické s chybou $O(h)$.
\end{example}

\vspace{0.8\baselineskip}

\subsubsection{Shrnutí a Expertní Metodologie}
\label{subsubsec:shrnuti-metodologie}

\begin{method}[Decision tree pro řešení lineárních rovnic]
\label{met:decision-tree-linearni}
\begin{enumerate}
\item \textbf{Identifikace}: Je rovnice ve tvaru $\frac{dy}{dx} + P(x)y = Q(x)$?

\item \textbf{Výběr metody}:
\begin{itemize}
\item \textbf{Integrační faktor}: Vždy funguje, vhodný pro většinu případů
\item \textbf{Variace konstant}: Ekvivalentní, někdy intuitivnější
\item \textbf{Metoda neurčitých koeficientů}: Pouze pro konstantní P(x) a speciální Q(x)
\end{itemize}

\item \textbf{Analýza P(x)}:
\begin{itemize}
\item Konstantní: Jednoduchá exponenciální integrace
\item Polynomiální: Integrace polynomu
\item Racionální: Rozklad na parciální zlomky
\item Goniometrické: Trigonometrické identity
\end{itemize}

\item \textbf{Analýza Q(x)}:
\begin{itemize}
\item Polynomiální: Přímá integrace
\item Exponenciální: Pozor na rezonanci
\item Goniometrické: Goniometrické integrace
\end{itemize}

\item \textbf{Interval existence}: Určete maximální interval kde P(x) a Q(x) spojité

\item \textbf{Ověření}: Vždy dosaďte řešení do původní rovnice
\end{enumerate}
\end{method}

\vspace{0.8\baselineskip}

\begin{remark}[Časté chyby a jak se jim vyhnout]
\begin{itemize}
\item \textbf{Zapomnění na integrační konstantu}: Vždy přidejte $+C$ po integraci
\item \textbf{Špatný integrační faktor}: Ověřte že $\frac{d\mu}{dx} = P(x)\mu(x)$
\item \textbf{Nesprávný interval existence}: Vždy analyzujte definiční obory P(x) a Q(x)
\item \textbf{Chyba v integraci}: Ověřte integrál derivací
\item \textbf{Záměna metod}: Integrační faktor a variace konstant jsou ekvivalentní
\end{itemize}
\end{remark}

\vspace{0.6\baselineskip}

\begin{remark}[Optimalizace výpočtů]
\begin{itemize}
\item Pro konstantní P(x) a polynomiální Q(x): Metoda neurčitých koeficientů může být rychlejší
\item Pro složité P(x): Integrační faktor je robustnější
\item Pro numerické řešení: Eulerova metoda je jednoduchá, Runge-Kutta přesnější
\end{itemize}
\end{remark}

\vspace{0.8\baselineskip}

\begin{transition}
S kompletním zvládnutím lineárních rovnic 1. řádu jsme připraveni přejít k homogenním rovnicím (1.3), které představují speciální případ lineárních rovnic a otevírají cestu k hlubší geometrické interpretaci diferenciálních rovnic.
\end{transition}

\vspace{0.8\baselineskip}

\subsection*{Shrnutí Sekce 1.2}

Lineární rovnice 1. řádu představují systematický a mocný aparát pro modelování dynamických systémů s vnějšími vstupy. Klíčové poznatky:

\begin{itemize}
\item \textbf{Univerzální řešitelnost}: Každá lineární rovnice s spojitými koeficienty má explicitní řešení
\item \textbf{Dvě ekvivalentní metody}: Integrační faktor a variace konstant
\item \textbf{Struktura řešení}: Homogenní část + partikulární řešení
\item \textbf{Aplikace}: Finanční modelování, elektrické obvody, fyzikální systémy
\item \textbf{Numerická verifikace}: Důležitá pro ověření analytických výsledků
\end{itemize}

% !TEX root = ../main.tex
\subsection{Homogenní Rovnice - Geometrický Přístup a Substituční Metody}
\label{subsec:homogenni-rovnice}

\subsubsection{Teoretický Fundament Homogenity}
\label{subsubsec:teoreticky-fundament-homogenni}

\begin{definition}[Homogenní diferenciální rovnice 1. řádu]
Rovnice je \emph{homogenní}, jestliže ji lze zapsat ve tvaru:
\[
\frac{dy}{dx} = f\left(\frac{y}{x}\right)
\]
kde $f$ je funkce jedné proměnné $v = \frac{y}{x}$.
\end{definition}

\vspace{0.6\baselineskip}

\begin{theorem}[Test homogenity]
Funkce $F(x, y)$ je homogenní stupně $n$, jestliže pro všechna $\lambda > 0$ platí:
\[
F(\lambda x, \lambda y) = \lambda^n F(x, y)
\]
Diferenciální rovnice $dy/dx = F(x, y)$ je homogenní, jestliže $F(x, y)$ je homogenní stupně 0.
\end{theorem}

\vspace{0.4\baselineskip}

\begin{proof}
Je-li $F(x, y)$ homogenní stupně 0, pak:
\[
F(\lambda x, \lambda y) = \lambda^0 F(x, y) = F(x, y)
\]
Speciálně pro $\lambda = 1/x$ (pro $x > 0$):
\[
F(x, y) = F\left(1, \frac{y}{x}\right) = f\left(\frac{y}{x}\right)
\]
\end{proof}

\vspace{0.6\baselineskip}

\begin{theorem}[Existence a jednoznačnost]
Nechť $f(v)$ je spojitá na intervalu $J \subseteq \mathbb{R}$ a Lipschitzovská na kompaktech. Pak pro libovolný bod $(x_0, y_0)$ s $x_0 \neq 0$ a $y_0/x_0 \in J$ existuje právě jedno řešení homogenní rovnice definované na okolí $x_0$.
\end{theorem}

\vspace{0.8\baselineskip}

\subsubsection{Substituční Metodologie - Kompletní Analýza}
\label{subsubsec:substituci-metodologie}

\begin{method}[Základní substituce $v = y/x$]
\label{met:zakladni-substituce}
\begin{enumerate}
\item \textbf{Substituce}: Položíme $v = \frac{y}{x} \implies y = vx$

\item \textbf{Derivace}:
\[
\frac{dy}{dx} = v + x\frac{dv}{dx}
\]

\item \textbf{Dosazení do rovnice}:
\[
v + x\frac{dv}{dx} = f(v)
\]

\item \textbf{Separace proměnných}:
\[
x\frac{dv}{dx} = f(v) - v \implies \frac{dv}{f(v) - v} = \frac{dx}{x}
\]

\item \textbf{Integrace}:
\[
\int \frac{dv}{f(v) - v} = \int \frac{dx}{x} = \ln|x| + C
\]

\item \textbf{Zpětná substituce}: Po integraci dosadíme $v = y/x$

\item \textbf{Analýza singulárních bodů}: Body kde $f(v) - v = 0$
\end{enumerate}
\end{method}

\vspace{0.8\baselineskip}

\begin{example}[Odvození metody]
Mějme homogenní rovnici $\frac{dy}{dx} = f\left(\frac{y}{x}\right)$. Substituce $y = vx$ dává:
\[
\frac{d}{dx}[vx] = v + x\frac{dv}{dx} = f(v)
\]
Tedy:
\[
x\frac{dv}{dx} = f(v) - v
\]
Tato rovnice je separabilní v proměnných $v$ a $x$.
\end{example}

\vspace{0.6\baselineskip}

\begin{method}[Alternativní substituce]
\label{met:alternativni-substituce}
\begin{itemize}
\item \textbf{Substituce $u = \ln|x|$}: Pro rovnice s logaritmickou strukturou

\item \textbf{Substituce $y = x^m v$}: Pro zobecněnou homogenitu

\item \textbf{Polární souřadnice}: $x = r\cos\theta$, $y = r\sin\theta$ pro geometrické interpretace
\end{itemize}
\end{method}

\vspace{0.8\baselineskip}

\subsubsection{Klasifikace Typů f(y/x)}
\label{subsubsec:klasifikace-funkci}

\begin{remark}[Racionální funkce]
Pro $f(v) = \frac{P(v)}{Q(v)}$ kde $P$, $Q$ jsou polynomy:
\[
\frac{dy}{dx} = \frac{P(y/x)}{Q(y/x)}
\]
Substituce vede na integraci racionální funkce.
\end{remark}

\vspace{0.6\baselineskip}

\begin{remark}[Algebraické funkce]
Pro $f(v)$ obsahující odmocniny:
\[
\frac{dy}{dx} = \sqrt{\frac{y}{x} + 1} \quad \text{nebo} \quad \frac{dy}{dx} = \left(\frac{y}{x}\right)^{1/3}
\]
Často vyžaduje speciální substituce nebo umocňování.
\end{remark}

\vspace{0.6\baselineskip}

\begin{remark}[Goniometrické funkce]
Pro $f(v) = \sin(v)$, $\cos(v)$, $\tan(v)$, atd.:
\[
\frac{dy}{dx} = \sin\left(\frac{y}{x}\right)
\]
Substituce vede na rovnici se separovanými proměnnými, ale integrace může vyžadovat goniometrické identity.
\end{remark}

\vspace{0.8\baselineskip}

\subsubsection{Početní Sekce - Kategorie A: Podle typu f(y/x)}
\label{subsubsec:pocetni-kategorie-a-fv}

\paragraph*{A1: Lineární f(v)}

\begin{example}[Lehký příklad - lineární f(v)]
Řešte: $\frac{dy}{dx} = \frac{y}{x} + 1$
\vspace{0.3\baselineskip}

\textbf{Řešení}: 
\begin{enumerate}
\item $f(v) = v + 1$

\item \textbf{Substituce}: $y = vx \implies \frac{dy}{dx} = v + x\frac{dv}{dx}$

\item \textbf{Dosazení}:
\[
v + x\frac{dv}{dx} = v + 1 \implies x\frac{dv}{dx} = 1
\]

\item \textbf{Separace}:
\[
dv = \frac{dx}{x} \implies v = \ln|x| + C
\]

\item \textbf{Zpětná substituce}:
\[
\frac{y}{x} = \ln|x| + C \implies y = x\ln|x| + Cx
\]

\item \textbf{Interval existence}: $(-\infty, 0)$ nebo $(0, \infty)$

\item \textbf{Ověření}:
\[
\frac{dy}{dx} = \ln|x| + 1 + C = \frac{y}{x} + 1 \quad \checkmark
\end{enumerate}
\end{example}

\vspace{0.6\baselineskip}

\begin{example}[Střední příklad - lineární f(v)]
Řešte: $\frac{dy}{dx} = \frac{2y - x}{y + x}$
\vspace{0.3\baselineskip}

\textbf{Řešení}: 
\begin{enumerate}
\item \textbf{Test homogenity}:
\[
F(\lambda x, \lambda y) = \frac{2\lambda y - \lambda x}{\lambda y + \lambda x} = \frac{2y - x}{y + x} = F(x, y)
\]
Homogenní stupně 0.

\item \textbf{Úprava}:
\[
\frac{dy}{dx} = \frac{2(y/x) - 1}{(y/x) + 1} = f(v) \quad \text{kde } v = \frac{y}{x}
\]

\item \textbf{Substituce}: $y = vx \implies \frac{dy}{dx} = v + x\frac{dv}{dx}$

\item \textbf{Dosazení}:
\[
v + x\frac{dv}{dx} = \frac{2v - 1}{v + 1}
\]
\[
x\frac{dv}{dx} = \frac{2v - 1}{v + 1} - v = \frac{2v - 1 - v^2 - v}{v + 1} = \frac{-v^2 + v - 1}{v + 1}
\]

\item \textbf{Separace}:
\[
\frac{v + 1}{-v^2 + v - 1}dv = \frac{dx}{x}
\]

\item \textbf{Integrace}: Nejprve upravíme jmenovatele:
\[
-v^2 + v - 1 = -(v^2 - v + 1) = -\left[(v - \frac{1}{2})^2 + \frac{3}{4}\right]
\]
\[
\int \frac{v + 1}{-v^2 + v - 1}dv = -\int \frac{v + 1}{(v - \frac{1}{2})^2 + \frac{3}{4}}dv
\]

\item \textbf{Substituce}: $u = v - \frac{1}{2} \implies v = u + \frac{1}{2}$, $dv = du$
\[
-\int \frac{u + \frac{3}{2}}{u^2 + \frac{3}{4}}du = -\int \frac{u}{u^2 + \frac{3}{4}}du - \frac{3}{2}\int \frac{1}{u^2 + \frac{3}{4}}du
\]
\[
= -\frac{1}{2}\ln|u^2 + \frac{3}{4}| - \frac{3}{2} \cdot \frac{2}{\sqrt{3}}\arctan\left(\frac{2u}{\sqrt{3}}\right) + K
\]

\item \textbf{Výsledek}:
\[
-\frac{1}{2}\ln\left|\left(v - \frac{1}{2}\right)^2 + \frac{3}{4}\right| - \sqrt{3}\arctan\left(\frac{2v - 1}{\sqrt{3}}\right) = \ln|x| + C
\]

\item \textbf{Zpětná substituce}: $v = y/x$
\[
-\frac{1}{2}\ln\left|\left(\frac{y}{x} - \frac{1}{2}\right)^2 + \frac{3}{4}\right| - \sqrt{3}\arctan\left(\frac{2y/x - 1}{\sqrt{3}}\right) = \ln|x| + C
\]
\end{enumerate}
\end{example}

\vspace{0.8\baselineskip}

\paragraph*{A2: Kvadratické f(v)}

\begin{example}[Lehký příklad - kvadratické f(v)]
Řešte: $\frac{dy}{dx} = \frac{y^2}{x^2}$
\vspace{0.3\baselineskip}

\textbf{Řešení}: 
\begin{enumerate}
\item $f(v) = v^2$

\item \textbf{Substituce}: $y = vx \implies \frac{dy}{dx} = v + x\frac{dv}{dx}$

\item \textbf{Dosazení}:
\[
v + x\frac{dv}{dx} = v^2 \implies x\frac{dv}{dx} = v^2 - v
\]

\item \textbf{Separace}:
\[
\frac{dv}{v(v - 1)} = \frac{dx}{x}
\]

\item \textbf{Rozklad na parciální zlomky}:
\[
\frac{1}{v(v - 1)} = \frac{1}{v - 1} - \frac{1}{v}
\]

\item \textbf{Integrace}:
\[
\int \left(\frac{1}{v - 1} - \frac{1}{v}\right)dv = \int \frac{dx}{x}
\]
\[
\ln|v - 1| - \ln|v| = \ln|x| + C
\]
\[
\ln\left|\frac{v - 1}{v}\right| = \ln|x| + C \implies \frac{v - 1}{v} = Ax \quad (A = \pm e^C)
\]

\item \textbf{Zpětná substituce}:
\[
\frac{y/x - 1}{y/x} = Ax \implies 1 - \frac{x}{y} = Ax \implies \frac{x}{y} = 1 - Ax
\]
\[
y = \frac{x}{1 - Ax}
\]

\item \textbf{Singulární řešení}: $v = 0$ a $v = 1$ $\implies$ $y = 0$ a $y = x$

\item \textbf{Intervaly existence}: Závisí na konstantě $A$
\end{enumerate}
\end{example}

\vspace{0.6\baselineskip}

\begin{example}[Střední příklad - kvadratické f(v)]
Řešte: $\frac{dy}{dx} = \frac{x^2 + y^2}{xy}$
\vspace{0.3\baselineskip}

\textbf{Řešení}: 
\begin{enumerate}
\item \textbf{Úprava}:
\[
\frac{dy}{dx} = \frac{x^2 + y^2}{xy} = \frac{x}{y} + \frac{y}{x} = \frac{1}{v} + v \quad \text{kde } v = \frac{y}{x}
\]

\item \textbf{Substituce}: $y = vx \implies \frac{dy}{dx} = v + x\frac{dv}{dx}$

\item \textbf{Dosazení}:
\[
v + x\frac{dv}{dx} = \frac{1}{v} + v \implies x\frac{dv}{dx} = \frac{1}{v}
\]

\item \textbf{Separace}:
\[
v  dv = \frac{dx}{x} \implies \frac{v^2}{2} = \ln|x| + C
\]

\item \textbf{Zpětná substituce}:
\[
\frac{y^2}{2x^2} = \ln|x| + C \implies y^2 = 2x^2(\ln|x| + C)
\]

\item \textbf{Interval existence}: $(-\infty, 0)$ nebo $(0, \infty)$
\end{enumerate}
\end{example}

\vspace{0.8\baselineskip}

\paragraph*{A3: Racionální f(v)}

\begin{example}[Lehký příklad - racionální f(v)]
Řešte: $\frac{dy}{dx} = \frac{x + y}{x - y}$
\vspace{0.3\baselineskip}

\textbf{Řešení}: 
\begin{enumerate}
\item \textbf{Úprava}:
\[
\frac{dy}{dx} = \frac{1 + y/x}{1 - y/x} = \frac{1 + v}{1 - v} \quad \text{kde } v = \frac{y}{x}
\]

\item \textbf{Substituce}: $y = vx \implies \frac{dy}{dx} = v + x\frac{dv}{dx}$

\item \textbf{Dosazení}:
\[
v + x\frac{dv}{dx} = \frac{1 + v}{1 - v} \implies x\frac{dv}{dx} = \frac{1 + v}{1 - v} - v = \frac{1 + v^2}{1 - v}
\]

\item \textbf{Separace}:
\[
\frac{1 - v}{1 + v^2}dv = \frac{dx}{x}
\]

\item \textbf{Integrace}:
\[
\int \frac{1 - v}{1 + v^2}dv = \int \frac{1}{1 + v^2}dv - \int \frac{v}{1 + v^2}dv = \arctan v - \frac{1}{2}\ln(1 + v^2)
\]

\item \textbf{Výsledek}:
\[
\arctan v - \frac{1}{2}\ln(1 + v^2) = \ln|x| + C
\]

\item \textbf{Zpětná substituce}:
\[
\arctan\left(\frac{y}{x}\right) - \frac{1}{2}\ln\left(1 + \frac{y^2}{x^2}\right) = \ln|x| + C
\]
\[
\arctan\left(\frac{y}{x}\right) - \frac{1}{2}\ln(x^2 + y^2) + \ln|x| = \ln|x| + C
\]
\[
\arctan\left(\frac{y}{x}\right) - \frac{1}{2}\ln(x^2 + y^2) = C
\]

\item \textbf{Geometrická interpretace}: Rovnice kružnic v polárních souřadnicích
\end{enumerate}
\end{example}

\vspace{0.8\baselineskip}





% !TEX root = ../main.tex
% ČÁST 2/3 - Pokračování 1.3 Homogenní Rovnice

\subsubsection{Početní Sekce - Kategorie A (pokračování)}
\label{subsubsec:pocetni-kategorie-a-fv-pokracovani}

\paragraph*{A4: Algebraické f(v)}

\begin{example}[Lehký příklad - algebraické f(v)]
Řešte: $\frac{dy}{dx} = \sqrt{\frac{y}{x}}$
\vspace{0.3\baselineskip}

\textbf{Řešení}: 
\begin{enumerate}
\item $f(v) = \sqrt{v}$, předpokládáme $y/x \geq 0$

\item \textbf{Substituce}: $y = vx \implies \frac{dy}{dx} = v + x\frac{dv}{dx}$

\item \textbf{Dosazení}:
\[
v + x\frac{dv}{dx} = \sqrt{v} \implies x\frac{dv}{dx} = \sqrt{v} - v
\]

\item \textbf{Separace}:
\[
\frac{dv}{\sqrt{v} - v} = \frac{dx}{x}
\]

\item \textbf{Úprava integrandu}:
\[
\frac{1}{\sqrt{v} - v} = \frac{1}{\sqrt{v}(1 - \sqrt{v})}
\]
Substituce $u = \sqrt{v} \implies v = u^2$, $dv = 2u  du$
\[
\int \frac{2u  du}{u(1 - u)} = 2\int \frac{du}{1 - u} = -2\ln|1 - u| + C
\]

\item \textbf{Výsledek}:
\[
-2\ln|1 - \sqrt{v}| = \ln|x| + C \implies \ln|1 - \sqrt{v}| = -\frac{1}{2}\ln|x| + K
\]

\item \textbf{Zpětná substituce}:
\[
1 - \sqrt{\frac{y}{x}} = \frac{A}{\sqrt{x}} \quad (A = \pm e^K)
\]
\[
\sqrt{\frac{y}{x}} = 1 - \frac{A}{\sqrt{x}} \implies \frac{y}{x} = \left(1 - \frac{A}{\sqrt{x}}\right)^2
\]
\[
y = x\left(1 - \frac{2A}{\sqrt{x}} + \frac{A^2}{x}\right) = x - 2A\sqrt{x} + A^2
\]

\item \textbf{Singulární řešení}: $v = 0$ a $v = 1$ $\implies$ $y = 0$ a $y = x$

\item \textbf{Interval existence}: $x > 0$, $y \geq 0$
\end{enumerate}
\end{example}

\vspace{0.6\baselineskip}

\begin{example}[Střední příklad - algebraické f(v)]
Řešte: $\frac{dy}{dx} = \frac{\sqrt{x^2 + y^2}}{x}$ pro $x > 0$
\vspace{0.3\baselineskip}

\textbf{Řešení}: 
\begin{enumerate}
\item \textbf{Úprava}:
\[
\frac{dy}{dx} = \frac{\sqrt{x^2 + y^2}}{x} = \sqrt{1 + \left(\frac{y}{x}\right)^2} = \sqrt{1 + v^2}
\]

\item \textbf{Substituce}: $y = vx \implies \frac{dy}{dx} = v + x\frac{dv}{dx}$

\item \textbf{Dosazení}:
\[
v + x\frac{dv}{dx} = \sqrt{1 + v^2} \implies x\frac{dv}{dx} = \sqrt{1 + v^2} - v
\]

\item \textbf{Separace}:
\[
\frac{dv}{\sqrt{1 + v^2} - v} = \frac{dx}{x}
\]

\item \textbf{Racionalizace jmenovatele}:
\[
\frac{1}{\sqrt{1 + v^2} - v} \cdot \frac{\sqrt{1 + v^2} + v}{\sqrt{1 + v^2} + v} = \frac{\sqrt{1 + v^2} + v}{1}
\]

\item \textbf{Integrace}:
\[
\int (\sqrt{1 + v^2} + v)dv = \int \frac{dx}{x}
\]
\[
\frac{v}{2}\sqrt{1 + v^2} + \frac{1}{2}\ln|v + \sqrt{1 + v^2}| + \frac{v^2}{2} = \ln|x| + C
\]

\item \textbf{Zpětná substituce}: $v = y/x$
\[
\frac{y}{2x}\sqrt{1 + \frac{y^2}{x^2}} + \frac{1}{2}\ln\left|\frac{y}{x} + \sqrt{1 + \frac{y^2}{x^2}}\right| + \frac{y^2}{2x^2} = \ln|x| + C
\]

\item \textbf{Zjednodušení}:
\[
\frac{y}{2x^2}\sqrt{x^2 + y^2} + \frac{1}{2}\ln\left|\frac{y + \sqrt{x^2 + y^2}}{x}\right| + \frac{y^2}{2x^2} = \ln|x| + C
\]
\end{enumerate}
\end{example}

\vspace{0.8\baselineskip}

\paragraph*{A5: Goniometrické f(v)}

\begin{example}[Lehký příklad - goniometrické f(v)]
Řešte: $\frac{dy}{dx} = \tan\left(\frac{y}{x}\right)$
\vspace{0.3\baselineskip}

\textbf{Řešení}: 
\begin{enumerate}
\item $f(v) = \tan v$

\item \textbf{Substituce}: $y = vx \implies \frac{dy}{dx} = v + x\frac{dv}{dx}$

\item \textbf{Dosazení}:
\[
v + x\frac{dv}{dx} = \tan v \implies x\frac{dv}{dx} = \tan v - v
\]

\item \textbf{Separace}:
\[
\frac{dv}{\tan v - v} = \frac{dx}{x}
\]

\item \textbf{Integrace}: Integrál nelze vyjádřit elementárními funkcemi
\[
\int \frac{dv}{\tan v - v} = \ln|x| + C
\]

\item \textbf{Implicitní řešení}:
\[
\int \frac{dv}{\tan v - v} = \ln|x| + C \quad \text{kde } v = \frac{y}{x}
\]

\item \textbf{Singulární řešení}: $\tan v - v = 0 \implies v = 0 \implies y = 0$
\end{enumerate}
\end{example}

\vspace{0.6\baselineskip}

\begin{example}[Střední příklad - goniometrické f(v)]
Řešte: $\frac{dy}{dx} = \sin\left(\frac{y}{x}\right) + \frac{y}{x}$
\vspace{0.3\baselineskip}

\textbf{Řešení}: 
\begin{enumerate}
\item $f(v) = \sin v + v$

\item \textbf{Substituce}: $y = vx \implies \frac{dy}{dx} = v + x\frac{dv}{dx}$

\item \textbf{Dosazení}:
\[
v + x\frac{dv}{dx} = \sin v + v \implies x\frac{dv}{dx} = \sin v
\]

\item \textbf{Separace}:
\[
\frac{dv}{\sin v} = \frac{dx}{x} \implies \csc v  dv = \frac{dx}{x}
\]

\item \textbf{Integrace}:
\[
\int \csc v  dv = \ln|\csc v - cot v| = \ln\left|\frac{1 - \cos v}{\sin v}\right| = \ln\left|\tan\frac{v}{2}\right|
\]
\[
\ln\left|\tan\frac{v}{2}\right| = \ln|x| + C
\]

\item \textbf{Výsledek}:
\[
\tan\frac{v}{2} = Ax \quad (A = \pm e^C)
\]

\item \textbf{Zpětná substituce}:
\[
\tan\left(\frac{y}{2x}\right) = Ax \implies \frac{y}{2x} = \arctan(Ax) \implies y = 2x \arctan(Ax)
\]

\item \textbf{Interval existence}: Závisí na konstantě $A$
\end{enumerate}
\end{example}

\vspace{0.8\baselineskip}

\subsubsection{Geometrická Interpretace a Fázová Analýza}
\label{subsubsec:geometricka-interpretace}

\begin{theorem}[Geometrická vlastnost homogenních rovnic]
Trajektorie řešení homogenní rovnice $\frac{dy}{dx} = f\left(\frac{y}{x}\right)$ jsou invariantní vůči dilatacím. To znamená, že pokud $y = \phi(x)$ je řešení, pak pro libovolné $\lambda > 0$ je $y = \lambda\phi\left(\frac{x}{\lambda}\right)$ také řešení.
\end{theorem}

\vspace{0.4\baselineskip}

\begin{proof}
Nechť $y = \phi(x)$ je řešení. Pak:
\[
\frac{d}{dx}[\lambda\phi(x/\lambda)] = \lambda\cdot\frac{1}{\lambda}\phi'(x/\lambda) = \phi'(x/\lambda)
\]
a
\[
f\left(\frac{\lambda\phi(x/\lambda)}{x}\right) = f\left(\frac{\phi(x/\lambda)}{x/\lambda}\right) = \phi'(x/\lambda)
\]
Tedy $y = \lambda\phi(x/\lambda)$ je také řešení.
\end{proof}

\vspace{0.6\baselineskip}

\begin{method}[Konstrukce směrového pole]
\label{met:smerove-pole-homogenni}
Pro homogenní rovnici $\frac{dy}{dx} = f\left(\frac{y}{x}\right)$:
\begin{enumerate}
\item Směrové pole závisí pouze na poměru $y/x$
\item Přímky $y = mx$ jsou izokliny
\item Na každé přímce $y = mx$ je sklen řešení konstantní: $f(m)$
\item Celé směrové pole lze získat z jedné přímky dilatacemi
\end{enumerate}
\end{method}

\vspace{0.6\baselineskip}

\begin{example}[Směrové pole pro $\frac{dy}{dx} = \frac{y}{x}$]
\label{ex:smerove-pole-zaklad}
Pro rovnici $\frac{dy}{dx} = \frac{y}{x}$:
\begin{itemize}
\item Na přímce $y = mx$: sklon $m$
\item Přímky procházející počátkem jsou řešeními
\item Směrové pole je radiální
\end{itemize}
\end{example}

\vspace{0.6\baselineskip}

\begin{example}[Fázový portrét homogenní rovnice]
Analyzujte rovnici: $\frac{dy}{dx} = \frac{y^2 - x^2}{2xy}$
\vspace{0.3\baselineskip}

\textbf{Řešení}: 
\begin{enumerate}
\item \textbf{Úprava}:
\[
\frac{dy}{dx} = \frac{(y/x)^2 - 1}{2(y/x)} = \frac{v^2 - 1}{2v}
\]

\item \textbf{Rovnovážné body}: $f(v) - v = 0$
\[
\frac{v^2 - 1}{2v} - v = \frac{v^2 - 1 - 2v^2}{2v} = \frac{-v^2 - 1}{2v} = 0
\]
Žádné reálné rovnovážné body.

\item \textbf{Substituce}: $y = vx$
\[
v + x\frac{dv}{dx} = \frac{v^2 - 1}{2v} \implies x\frac{dv}{dx} = \frac{v^2 - 1}{2v} - v = \frac{-v^2 - 1}{2v}
\]

\item \textbf{Separace}:
\[
\frac{2v}{-v^2 - 1}dv = \frac{dx}{x} \implies -\int \frac{2v}{v^2 + 1}dv = \int \frac{dx}{x}
\]
\[
-\ln|v^2 + 1| = \ln|x| + C \implies \frac{1}{v^2 + 1} = Ax
\]

\item \textbf{Zpětná substituce}:
\[
\frac{1}{(y/x)^2 + 1} = Ax \implies \frac{x^2}{x^2 + y^2} = Ax
\]
\[
x^2 + y^2 = \frac{x}{A} \implies x^2 + y^2 - \frac{x}{A} = 0
\]

\item \textbf{Geometrická interpretace}: Kružnice procházející počátkem
\end{enumerate}
\end{example}

\vspace{0.8\baselineskip}

\subsubsection{Početní Sekce - Kategorie B: Podle složitosti substituce}
\label{subsubsec:pocetni-kategorie-b}

\paragraph*{B1: Přímá substituce v = y/x}

\begin{example}[Jednoduchá integrace]
Řešte: $\frac{dy}{dx} = \frac{y}{x} + \left(\frac{y}{x}\right)^2$
\vspace{0.3\baselineskip}

\textbf{Řešení}: 
\begin{enumerate}
\item $f(v) = v + v^2$

\item \textbf{Substituce}: $y = vx$
\[
v + x\frac{dv}{dx} = v + v^2 \implies x\frac{dv}{dx} = v^2
\]

\item \textbf{Separace}:
\[
\frac{dv}{v^2} = \frac{dx}{x} \implies -\frac{1}{v} = \ln|x| + C
\]

\item \textbf{Výsledek}:
\[
-\frac{x}{y} = \ln|x| + C \implies y = -\frac{x}{\ln|x| + C}
\]

\item \textbf{Singulární řešení}: $v = 0 \implies y = 0$
\end{enumerate}
\end{example}

\vspace{0.8\baselineskip}

\paragraph*{B2: Vícenásobná substituce}

\begin{example}[Řetězová substituce]
Řešte: $\frac{dy}{dx} = \sqrt{\frac{y}{x} + \sqrt{\frac{y}{x}}}$
\vspace{0.3\baselineskip}

\textbf{Řešení}: 
\begin{enumerate}
\item \textbf{První substituce}: $v = \frac{y}{x} \implies y = vx$
\[
v + x\frac{dv}{dx} = \sqrt{v + \sqrt{v}}
\]

\item \textbf{Druhá substituce}: $u = \sqrt{v} \implies v = u^2$, $dv = 2u  du$
\[
u^2 + x\cdot 2u\frac{du}{dx} = \sqrt{u^2 + u} = \sqrt{u(u + 1)}
\]

\item \textbf{Separace}:
\[
2xu\frac{du}{dx} = \sqrt{u(u + 1)} - u^2
\]
\[
\frac{2u  du}{\sqrt{u(u + 1)} - u^2} = \frac{dx}{x}
\]

\item \textbf{Integrace vyžaduje numerické metody nebo speciální funkce}
\end{enumerate}
\end{example}

\vspace{0.8\baselineskip}

\subsubsection{Aplikace v Kvantitativních Vědách}
\label{subsubsec:aplikace-kvantitativni-homog}

\begin{application}[Cobb-Douglasova produkční funkce]
\label{app:cobb-douglas}
Produkční funkce $Y = AK^\alpha L^{1-\alpha}$ je homogenní stupně 1. Rovnice popisující izokvanty:
\[
\frac{dK}{dL} = -\frac{\partial Y/\partial L}{\partial Y/\partial K} = -\frac{(1-\alpha)AK^\alpha L^{-\alpha}}{\alpha AK^{\alpha-1}L^{1-\alpha}} = -\frac{1-\alpha}{\alpha}\cdot\frac{K}{L}
\]

\textbf{Řešení}: Homogenní rovnice s $f(v) = -\frac{1-\alpha}{\alpha}v$
\[
\frac{dK}{dL} = -\frac{1-\alpha}{\alpha}\cdot\frac{K}{L}
\]
Substituce $v = K/L$:
\[
v + L\frac{dv}{dL} = -\frac{1-\alpha}{\alpha}v \implies L\frac{dv}{dL} = -\left(\frac{1-\alpha}{\alpha} + 1\right)v = -\frac{1}{\alpha}v
\]
\[
\frac{dv}{v} = -\frac{1}{\alpha}\frac{dL}{L} \implies \ln|v| = -\frac{1}{\alpha}\ln|L| + C
\]
\[
v = AL^{-1/\alpha} \implies \frac{K}{L} = AL^{-1/\alpha} \implies K = AL^{1-1/\alpha}
\]
\end{application}

\vspace{0.6\baselineskip}

\begin{application}[Model s konstantními výnosy z rozsahu]
\label{app:konstantni-vynosy}
Uvažujme firmu s nákladovou funkcí $C(Y)$ homogenní stupně 1. Pak mezní náklady závisí pouze na poměru vstupů:
\[
\frac{dC}{dY} = f\left(\frac{K}{L}\right)
\]
Homogenní rovnice popisuje optimální kombinaci vstupů.
\end{application}

\vspace{0.6\baselineskip}

\begin{application}[Podobnostní zákony v mechanice tekutin]
\label{app:podobnostni-zakony}
Reynoldsova rovnice pro proudění:
\[
\frac{dv}{dr} = f\left(\frac{v}{r}\right)
\]
kde $v$ je rychlost, $r$ poloměr. Homogenita zajišťuje podobnostní zákony.
\end{application}

\vspace{0.8\baselineskip}

% !TEX root = ../main.tex
% ČÁST 3/3 - Dokončení 1.3 Homogenní Rovnice

\subsubsection{Početní Sekce - Kategorie C: Aplikované Příklady}
\label{subsubsec:pocetni-kategorie-c-homog}

\paragraph*{C1: Ekonomické modely}

\begin{example}[Optimalizace produkční funkce]
Firma má produkční funkci $Q = K^{0.6}L^{0.4}$ a nákladovou funkci $C = 2K + 3L$. Najděte trajektorii optimálního růstu při konstantních výnosech z rozsahu.
\vspace{0.3\baselineskip}

\textbf{Řešení}: 
\begin{enumerate}
\item \textbf{Mezní produktivity}:
\[
\frac{\partial Q}{\partial K} = 0.6K^{-0.4}L^{0.4}, \quad \frac{\partial Q}{\partial L} = 0.4K^{0.6}L^{-0.6}
\]

\item \textbf{Podmínka optima}: $\frac{\partial Q/\partial K}{\partial Q/\partial L} = \frac{\text{cena K}}{\text{cena L}}$
\[
\frac{0.6K^{-0.4}L^{0.4}}{0.4K^{0.6}L^{-0.6}} = \frac{2}{3} \implies \frac{3}{2}\cdot\frac{L}{K} = \frac{2}{3}
\]
\[
\frac{L}{K} = \frac{4}{9} \implies K = \frac{9}{4}L
\]

\item \textbf{Růstová trajektorie}: $\frac{dK}{dL} = \frac{9}{4}$ - konstantní poměr

\item \textbf{Homogenní formulace}: 
\[
\frac{dK}{dL} = f\left(\frac{K}{L}\right) = \frac{9}{4}
\]
Řešení: $K = \frac{9}{4}L + C$, ale pro homogenní případ $C = 0$
\end{enumerate}
\end{example}

\vspace{0.8\baselineskip}

\paragraph*{C2: Fyzikální modely}

\begin{example}[Podobnostní zákony v dynamice]
Analyzujte rovnici popisující podobnostní zákony: $\frac{dr}{dt} = k\sqrt{\frac{r}{t}}$
\vspace{0.3\baselineskip}

\textbf{Řešení}: 
\begin{enumerate}
\item \textbf{Test homogenity}: $F(\lambda r, \lambda t) = k\sqrt{\frac{\lambda r}{\lambda t}} = k\sqrt{\frac{r}{t}} = F(r,t)$

\item \textbf{Substituce}: $v = \frac{r}{t} \implies r = vt \implies \frac{dr}{dt} = v + t\frac{dv}{dt}$

\item \textbf{Dosazení}:
\[
v + t\frac{dv}{dt} = k\sqrt{v} \implies t\frac{dv}{dt} = k\sqrt{v} - v
\]

\item \textbf{Separace}:
\[
\frac{dv}{k\sqrt{v} - v} = \frac{dt}{t}
\]

\item \textbf{Substituce}: $u = \sqrt{v} \implies v = u^2$, $dv = 2u  du$
\[
\int \frac{2u  du}{ku - u^2} = \int \frac{dt}{t} \implies 2\int \frac{du}{k - u} = \ln|t| + C
\]
\[
-2\ln|k - u| = \ln|t| + C \implies (k - u)^{-2} = At
\]

\item \textbf{Výsledek}:
\[
k - \sqrt{\frac{r}{t}} = \frac{B}{\sqrt{t}} \implies r(t) = t\left(k - \frac{B}{\sqrt{t}}\right)^2 = k^2t - 2kB\sqrt{t} + B^2
\]

\item \textbf{Fyzikální interpretace}: Růst s různými režimy v závislosti na čase
\end{enumerate}
\end{example}

\vspace{0.8\baselineskip}

\subsubsection{Početní Sekce - Kategorie D: Speciální Případy}
\label{subsubsec:pocetni-kategorie-d-homog}

\paragraph*{D1: Rovnice s parametry}

\begin{example}[Bifurkační analýza]
Analyzujte rovnici: $\frac{dy}{dx} = \frac{y}{x} + \alpha\left(\frac{y}{x}\right)^2$ v závislosti na parametru $\alpha$
\vspace{0.3\baselineskip}

\textbf{Řešení}: 
\begin{enumerate}
\item $f(v) = v + \alpha v^2$

\item \textbf{Substituce}: $y = vx$
\[
v + x\frac{dv}{dx} = v + \alpha v^2 \implies x\frac{dv}{dx} = \alpha v^2
\]

\item \textbf{Rovnovážné body}: $f(v) - v = \alpha v^2 = 0 \implies v = 0$

\item \textbf{Stabilita}: 
\begin{itemize}
\item Pro $\alpha > 0$: $v = 0$ nestabilní
\item Pro $\alpha < 0$: $v = 0$ stabilní
\item Pro $\alpha = 0$: všechny body na přímkách řešeními
\end{itemize}

\item \textbf{Obecné řešení}:
\[
\frac{dv}{v^2} = \alpha\frac{dx}{x} \implies -\frac{1}{v} = \alpha\ln|x| + C
\]
\[
v = -\frac{1}{\alpha\ln|x| + C} \implies y = -\frac{x}{\alpha\ln|x| + C}
\]

\item \textbf{Bifurkační diagram}: Transkritická bifurkace při $\alpha = 0$
\end{enumerate}
\end{example}

\vspace{0.8\baselineskip}

\paragraph*{D2: Singulární řešení}

\begin{example}[Obálka rodiny řešení]
Najděte singulární řešení rovnice: $\frac{dy}{dx} = \sqrt{\frac{y}{x}}$
\vspace{0.3\baselineskip}

\textbf{Řešení}: 
\begin{enumerate}
\item \textbf{Obecné řešení} (z příkladu A4): $y = x - 2A\sqrt{x} + A^2$

\item \textbf{Diskriminantní křivka}: Derivujeme podle parametru $A$:
\[
\frac{\partial y}{\partial A} = -2\sqrt{x} + 2A = 0 \implies A = \sqrt{x}
\]

\item \textbf{Dosazení do obecného řešení}:
\[
y = x - 2\sqrt{x}\cdot\sqrt{x} + (\sqrt{x})^2 = x - 2x + x = 0
\]

\item \textbf{Singulární řešení}: $y = 0$

\item \textbf{Ověření}: $y = 0$ splňuje původní rovnici
\[
\frac{d}{dx}[0] = 0 = \sqrt{\frac{0}{x}} \quad \checkmark
\]

\item \textbf{Geometrická interpretace}: $y = 0$ je obálkou rodiny řešení
\end{enumerate}
\end{example}

\vspace{0.8\baselineskip}

\subsubsection{Numerická Verifikace a Analýza}
\label{subsubsec:numericka-verifikace}

\begin{method}[Eulerova metoda pro homogenní rovnice]
\label{met:euler-homogenni}
Pro rovnici $\frac{dy}{dx} = f\left(\frac{y}{x}\right)$ s počáteční podmínkou $y(x_0) = y_0$:

\begin{enumerate}
\item Zvolte krok $h > 0$
\item Pro $n = 0, 1, 2, \dots$ počítáme:
\[
y_{n+1} = y_n + h\cdot f\left(\frac{y_n}{x_n}\right)
\]
\[
x_{n+1} = x_n + h
\]
\item Speciální vlastnost: Metoda zachovává homogenitu diskrétně
\end{enumerate}
\end{method}

\vspace{0.6\baselineskip}

\begin{example}[Numerická verifikace]
Porovnejte analytické a numerické řešení: $\frac{dy}{dx} = \frac{y}{x} + 1$, $y(1) = 2$
\vspace{0.3\baselineskip}

\textbf{Analytické řešení}: $y = x\ln|x| + 2x$

\textbf{Numerické řešení} (krok $h = 0.1$):
\begin{center}
\begin{tabular}{c|c|c|c}
$x_n$ & Analytické $y(x_n)$ & Numerické $y_n$ & Chyba \\
\hline
1.0 & 2.0000 & 2.0000 & 0.0000 \\
1.1 & 2.3046 & 2.3000 & 0.0046 \\
1.2 & 2.6194 & 2.6091 & 0.0103 \\
1.3 & 2.9443 & 2.9273 & 0.0170 \\
1.4 & 3.2792 & 3.2545 & 0.0247 \\
1.5 & 3.6240 & 3.5909 & 0.0331 \\
\end{tabular}
\end{center}

\textbf{Analýza}: Chyba roste s $x$, ale relativní chyba zůstává kontrolovaná.
\end{example}

\vspace{0.6\baselineskip}

\begin{example}[Numerická analýza singularity]
Analyzujte numericky chování řešení: $\frac{dy}{dx} = \frac{1}{y/x - 1}$, $y(2) = 1$
\vspace{0.3\baselineskip}

\textbf{Analytické řešení}: Singularita při $y = x$

\textbf{Numerická simulace}:
\begin{itemize}
\item Pro $x < 2$: řešení konverguje k $y = x$
\item Pro $x > 2$: řešení diverguje od $y = x$
\item Numerická metoda detekuje singularitu nestabilitou
\end{itemize}

\textbf{Závěr}: Numerické metody pomáhají identifikovat singularitu.
\end{example}

\vspace{0.8\baselineskip}

\subsubsection{Vztah k Dalším Typům Rovnic}
\label{subsubsec:vztah-k-dalsim-rovnicim}

\begin{theorem}[Vztah k lineárním rovnicím]
Každá homogenní lineární rovnice $\frac{dy}{dx} + P(x)y = 0$ je také homogenní v geometrickém smyslu.
\end{theorem}

\vspace{0.4\baselineskip}

\begin{proof}
Lineární homogenní rovnici lze zapsat jako:
\[
\frac{dy}{dx} = -P(x)y
\]
Tato rovnice je homogenní právě tehdy, když $P(x)$ je homogenní stupně $-1$, tj. $P(\lambda x) = \frac{1}{\lambda}P(x)$.
\end{proof}

\vspace{0.6\baselineskip}

\begin{theorem}[Vztah k exaktním rovnicím]
Homogenní rovnice mohou být často převedeny na exaktní rovnice vhodnou substitucí.
\end{theorem}

\vspace{0.4\baselineskip}

\begin{proof}
Uvažujme homogenní rovnici $M(x,y)dx + N(x,y)dy = 0$ kde $M$, $N$ jsou homogenní stejného stupně. Substituce $y = vx$ vede na separabilní rovnici, která je speciálním případem exaktní rovnice.
\end{proof}

\vspace{0.8\baselineskip}

\subsubsection{Shrnutí a Expertní Metodologie}
\label{subsubsec:shrnuti-metodologie}

\begin{method}[Decision tree pro homogenní rovnice]
\label{met:decision-tree-homogenni}
\begin{enumerate}
\item \textbf{Identifikace}: Lze rovnici zapsat jako $\frac{dy}{dx} = f\left(\frac{y}{x}\right)$?

\item \textbf{Test homogenity}: Platí $F(\lambda x, \lambda y) = F(x, y)$?

\item \textbf{Základní substituce}: $v = \frac{y}{x} \implies y = vx$

\item \textbf{Analýza singulárních bodů}: Najdi $v$ taková, že $f(v) - v = 0$

\item \textbf{Integrace}:
\[
\int \frac{dv}{f(v) - v} = \int \frac{dx}{x} = \ln|x| + C
\]

\item \textbf{Zpětná substituce}: $v = \frac{y}{x}$

\item \textbf{Analýza řešení}:
\begin{itemize}
\item Urči maximální intervaly existence
\item Najdi singulární řešení
\item Analyzuj asymptotické chování
\end{itemize}
\end{enumerate}
\end{method}

\vspace{0.8\baselineskip}

\begin{remark}[Časté chyby a jak se jim vyhnout]
\begin{itemize}
\item \textbf{Zapomnění na absolutní hodnotu}: $\int \frac{dx}{x} = \ln|x| + C$, ne $\ln x + C$
\item \textbf{Špatná derivace}: $\frac{d}{dx}[vx] = v + x\frac{dv}{dx}$, ne $x\frac{dv}{dx}$
\item \textbf{Ignorování singularit}: Vždy analyzuj body kde $f(v) - v = 0$
\item \textbf{Chybný definiční obor}: Homogenní rovnice typicky nejsou definované v $x = 0$
\item \textbf{Záměna s linearitou}: Homogenní ≠ lineární homogenní
\end{itemize}
\end{remark}

\vspace{0.6\baselineskip}

\begin{remark}[Optimalizace řešení]
\begin{itemize}
\item Pro racionální $f(v)$: použij rozklad na parciální zlomky
\item Pro algebraické $f(v)$: vhodné substituce pro odstranění odmocnin
\item Pro goniometrické $f(v)$: goniometrické identity a substituce
\item Pro složité integrály: numerické metody nebo speciální funkce
\end{itemize}
\end{remark}

\vspace{0.8\baselineskip}

\begin{transition}
S kompletním zvládnutím homogenních rovnic jsme připraveni přejít k exaktním rovnicím (1.4), které představují další mocnou třídu řešitelných diferenciálních rovnic 1. řádu a otevírají cestu k teorii diferenciálních forem.
\end{transition}

\vspace{0.8\baselineskip}

\subsection*{Shrnutí Sekce 1.3}

Homogenní rovnice představují elegantní třídu diferenciálních rovnic s hlubokou geometrickou interpretací. Klíčové poznatky:

\begin{itemize}
\item \textbf{Geometrická podstata}: Invariance vůči dilatacím, radiální struktura řešení
\item \textbf{Univerzální řešicí metoda}: Substituce $v = y/x$ převádí na separabilní rovnici
\item \textbf{Aplikace}: Ekonomické modely s konstantními výnosy z rozsahu, podobnostní zákony ve fyzice
\item \textbf{Speciální vlastnosti}: Singulární řešení jako obálky, bifurkační chování
\item \textbf{Numerická stabilita}: Eulerova metoda dobře funguje pro homogenní rovnice
\end{itemize}

Zvládnutí homogenních rovnic je zásadní pro rozvoj geometrické intuice v teorii diferenciálních rovnic a připravuje půdu pro pokročilejší témata v následujících úrovních.


\subsection{Exaktní Rovnice - Teorie Potenciálů a Konzervativní Pole}
\label{subsec:exaktni-rovnice}

\subsubsection{Teoretický Fundament Exaktnosti}
\label{subsubsec:teoreticky-fundament-exaktni}

\begin{definition}[Exaktní diferenciální rovnice]
Rovnice tvaru
\[
M(x, y)dx + N(x, y)dy = 0
\]
je \emph{exaktní}, jestliže existuje funkce $F(x, y)$ taková, že:
\[
\frac{\partial F}{\partial x} = M(x, y) \quad \text{a} \quad \frac{\partial F}{\partial y} = N(x, y)
\]
na nějaké oblasti $D \subseteq \mathbb{R}^2$.
\end{definition}

\vspace{0.6\baselineskip}

\begin{theorem}[Nutná a postačující podmínka exaktnosti]
Nechť $M(x, y)$ a $N(x, y)$ mají spojité parciální derivace prvního řádu na jednoduše souvislé oblasti $D$. Pak rovnice $M dx + N dy = 0$ je exaktní právě tehdy, když:
\[
\frac{\partial M}{\partial y} = \frac{\partial N}{\partial x} \quad \text{na } D
\]
\end{theorem}

\vspace{0.4\baselineskip}

\begin{proof}
\textbf{(⇒)} Je-li rovnice exaktní, existuje $F$ taková, že $F_x = M$, $F_y = N$. Ze Schwarzovy věty:
\[
\frac{\partial M}{\partial y} = \frac{\partial^2 F}{\partial y \partial x} = \frac{\partial^2 F}{\partial x \partial y} = \frac{\partial N}{\partial x}
\]

\textbf{(⇐)} Konstruujeme $F(x, y)$ přímo. Zafixujeme $(x_0, y_0) \in D$ a definujeme:
\[
F(x, y) = \int_{x_0}^x M(t, y)dt + \int_{y_0}^y N(x_0, s)ds
\]
Pak $\frac{\partial F}{\partial x} = M(x, y)$ a pomocí podmínky $\frac{\partial M}{\partial y} = \frac{\partial N}{\partial x}$ dostaneme $\frac{\partial F}{\partial y} = N(x, y)$.
\end{proof}

\vspace{0.6\baselineskip}

\begin{theorem}[Věta o nezávislosti na cestě]
Nechť $M dx + N dy$ je exaktní diferenciální forma na jednoduše souvislé oblasti $D$. Pak integrál
\[
\int_C M dx + N dy
\]
závisí pouze na počátečním a koncovém bodě křivky $C$, nikoli na její konkrétní trajektorii.
\end{theorem}

\vspace{0.8\baselineskip}

\subsubsection{Metoda Hledání Potenciálu - Kompletní Analýza}
\label{subsubsec:metoda-hledani-potencialu}

\begin{method}[Přímá integrační metoda]
\label{met:primaintegrace-potencial}
\begin{enumerate}
\item \textbf{Ověření exaktnosti}: Zkontrolujte že $\frac{\partial M}{\partial y} = \frac{\partial N}{\partial x}$

\item \textbf{Integrace podle x}:
\[
F(x, y) = \int M(x, y)dx + \phi(y)
\]
kde $\phi(y)$ je "integrační konstanta" závislá na $y$

\item \textbf{Určení $\phi(y)$}: Derivujte podle $y$ a porovnejte s $N(x, y)$:
\[
\frac{\partial F}{\partial y} = \frac{\partial}{\partial y}\left[\int M(x, y)dx\right] + \phi'(y) = N(x, y)
\]

\item \textbf{Integrace $\phi(y)$}:
\[
\phi(y) = \int \left[N(x, y) - \frac{\partial}{\partial y}\left(\int M(x, y)dx\right)\right] dy
\]

\item \textbf{Konečný tvar}:
\[
F(x, y) = \int M(x, y)dx + \int \left[N(x, y) - \frac{\partial}{\partial y}\left(\int M(x, y)dx\right)\right] dy
\]

\item \textbf{Řešení rovnice}: $F(x, y) = C$
\end{enumerate}
\end{method}

\vspace{0.8\baselineskip}

\begin{method}[Alternativní metoda - integrace podle y]
\label{met:alternativni-integracce}
\begin{enumerate}
\item \textbf{Integrace podle y}:
\[
F(x, y) = \int N(x, y)dy + \psi(x)
\]

\item \textbf{Určení $\psi(x)$}:
\[
\frac{\partial F}{\partial x} = \frac{\partial}{\partial x}\left[\int N(x, y)dy\right] + \psi'(x) = M(x, y)
\]

\item \textbf{Výběr metody}: Volíme tu, která vede k jednodušší integraci
\end{enumerate}
\end{method}

\vspace{0.6\baselineskip}

\begin{example}[Kompletní postup přímé integrace]
Řešte: $(2x + y)dx + (x + 2y)dy = 0$
\vspace{0.3\baselineskip}

\textbf{Řešení}: 
\begin{enumerate}
\item \textbf{Ověření exaktnosti}:
\[
M(x, y) = 2x + y, \quad N(x, y) = x + 2y
\]
\[
\frac{\partial M}{\partial y} = 1, \quad \frac{\partial N}{\partial x} = 1 \quad \checkmark
\]

\item \textbf{Integrace M podle x}:
\[
\int (2x + y)dx = x^2 + xy + \phi(y)
\]

\item \textbf{Derivace podle y}:
\[
\frac{\partial F}{\partial y} = x + \phi'(y)
\]

\item \textbf{Porovnání s N}:
\[
x + \phi'(y) = x + 2y \implies \phi'(y) = 2y
\]

\item \textbf{Integrace $\phi(y)$}:
\[
\phi(y) = \int 2y  dy = y^2 + C_1
\]

\item \textbf{Potenciál}:
\[
F(x, y) = x^2 + xy + y^2 + C_1
\]

\item \textbf{Řešení}:
\[
x^2 + xy + y^2 = C \quad (C = -C_1)
\]

\item \textbf{Ověření}:
\[
d(x^2 + xy + y^2) = (2x + y)dx + (x + 2y)dy \quad \checkmark
\end{enumerate}
\end{example}

\vspace{0.8\baselineskip}

\subsubsection{Početní Sekce - Kategorie A: Přímé Exaktní Rovnice}
\label{subsubsec:pocetni-kategorie-a}

\paragraph*{A1: Jednoduché polynomy}

\begin{example}[Lineární polynomy]
Řešte: $(3x^2 + 2y)dx + (2x + 4y)dy = 0$
\vspace{0.3\baselineskip}

\textbf{Řešení}: 
\begin{enumerate}
\item \textbf{Ověření exaktnosti}:
\[
\frac{\partial}{\partial y}(3x^2 + 2y) = 2, \quad \frac{\partial}{\partial x}(2x + 4y) = 2 \quad \checkmark
\]

\item \textbf{Integrace M podle x}:
\[
\int (3x^2 + 2y)dx = x^3 + 2xy + \phi(y)
\]

\item \textbf{Derivace a porovnání}:
\[
\frac{\partial F}{\partial y} = 2x + \phi'(y) = 2x + 4y \implies \phi'(y) = 4y
\]

\item \textbf{Potenciál}:
\[
F(x, y) = x^3 + 2xy + 2y^2
\]

\item \textbf{Řešení}:
\[
x^3 + 2xy + 2y^2 = C
\end{enumerate}
\end{example}

\vspace{0.6\baselineskip}

\begin{example}[Kvadratické polynomy]
Řešte: $(2xy + y^2)dx + (x^2 + 2xy)dy = 0$
\vspace{0.3\baselineskip}

\textbf{Řešení}: 
\begin{enumerate}
\item \textbf{Ověření}:
\[
\frac{\partial}{\partial y}(2xy + y^2) = 2x + 2y, \quad \frac{\partial}{\partial x}(x^2 + 2xy) = 2x + 2y \quad \checkmark
\]

\item \textbf{Integrace M podle x}:
\[
\int (2xy + y^2)dx = x^2y + xy^2 + \phi(y)
\]

\item \textbf{Derivace a porovnání}:
\[
\frac{\partial F}{\partial y} = x^2 + 2xy + \phi'(y) = x^2 + 2xy \implies \phi'(y) = 0
\]

\item \textbf{Potenciál}:
\[
F(x, y) = x^2y + xy^2
\]

\item \textbf{Řešení}:
\[
x^2y + xy^2 = C \quad \text{nebo} \quad xy(x + y) = C
\end{enumerate}
\end{example}

\vspace{0.8\baselineskip}

\paragraph*{A2: Racionální funkce}

\begin{example}[Jednoduché racionální funkce]
Řešte: $\left(\frac{1}{y} - \frac{y}{x^2}\right)dx + \left(\frac{1}{x} - \frac{x}{y^2}\right)dy = 0$
\vspace{0.3\baselineskip}

\textbf{Řešení}: 
\begin{enumerate}
\item \textbf{Ověření exaktnosti}:
\[
M = \frac{1}{y} - \frac{y}{x^2}, \quad N = \frac{1}{x} - \frac{x}{y^2}
\]
\[
\frac{\partial M}{\partial y} = -\frac{1}{y^2} - \frac{1}{x^2}, \quad \frac{\partial N}{\partial x} = -\frac{1}{x^2} - \frac{1}{y^2} \quad \checkmark
\]

\item \textbf{Integrace M podle x}:
\[
\int \left(\frac{1}{y} - \frac{y}{x^2}\right)dx = \frac{x}{y} + \frac{y}{x} + \phi(y)
\]

\item \textbf{Derivace a porovnání}:
\[
\frac{\partial F}{\partial y} = -\frac{x}{y^2} + \frac{1}{x} + \phi'(y) = \frac{1}{x} - \frac{x}{y^2}
\]
\[
\phi'(y) = 0
\]

\item \textbf{Potenciál}:
\[
F(x, y) = \frac{x}{y} + \frac{y}{x}
\]

\item \textbf{Řešení}:
\[
\frac{x}{y} + \frac{y}{x} = C
\end{enumerate}
\end{example}

\vspace{0.8\baselineskip}

\begin{transition}
V další části pokryjeme goniometrické a exponenciální funkce, integrační faktory a jejich systematickou klasifikaci.
\end{transition}

\subsubsection{Početní Sekce - Kategorie A: Přímé Exaktní Rovnice (pokračování)}
\label{subsubsec:pocetni-kategorie-a-pokracovani}

\paragraph*{A3: Goniometrické funkce}

\begin{example}[Základní goniometrické funkce]
Řešte: $(\sin y + y\cos x)dx + (\sin x + x\cos y)dy = 0$
\vspace{0.3\baselineskip}

\textbf{Řešení}: 
\begin{enumerate}
\item \textbf{Ověření exaktnosti}:
\[
M(x, y) = \sin y + y\cos x, \quad N(x, y) = \sin x + x\cos y
\]
\[
\frac{\partial M}{\partial y} = \cos y + \cos x, \quad \frac{\partial N}{\partial x} = \cos x + \cos y \quad \checkmark
\]

\item \textbf{Integrace M podle x}:
\[
\int (\sin y + y\cos x)dx = x\sin y + y\sin x + \phi(y)
\]

\item \textbf{Derivace a porovnání}:
\[
\frac{\partial F}{\partial y} = x\cos y + \sin x + \phi'(y) = \sin x + x\cos y
\]
\[
\phi'(y) = 0
\]

\item \textbf{Potenciál}:
\[
F(x, y) = x\sin y + y\sin x
\]

\item \textbf{Řešení}:
\[
x\sin y + y\sin x = C
\]

\item \textbf{Geometrická interpretace}: Rovnice popisuje rodinu křivek v potenciálovém poli s periodickou strukturou.
\end{enumerate}
\end{example}

\vspace{0.6\baselineskip}

\begin{example}[Kombinace goniometrických funkcí]
Řešte: $(2x\sin y + y^2\cos x)dx + (x^2\cos y + 2y\sin x)dy = 0$
\vspace{0.3\baselineskip}

\textbf{Řešení}: 
\begin{enumerate}
\item \textbf{Ověření exaktnosti}:
\[
\frac{\partial M}{\partial y} = 2x\cos y + 2y\cos x, \quad \frac{\partial N}{\partial x} = 2x\cos y + 2y\cos x \quad \checkmark
\]

\item \textbf{Integrace M podle x}:
\[
\int (2x\sin y + y^2\cos x)dx = x^2\sin y + y^2\sin x + \phi(y)
\]

\item \textbf{Derivace a porovnání}:
\[
\frac{\partial F}{\partial y} = x^2\cos y + 2y\sin x + \phi'(y) = x^2\cos y + 2y\sin x
\]
\[
\phi'(y) = 0
\]

\item \textbf{Potenciál}:
\[
F(x, y) = x^2\sin y + y^2\sin x
\]

\item \textbf{Řešení}:
\[
x^2\sin y + y^2\sin x = C
\end{enumerate}
\end{example}

\vspace{0.8\baselineskip}

\paragraph*{A4: Exponenciální a logaritmické funkce}

\begin{example}[Jednoduché exponenciály]
Řešte: $(e^x \ln y + \frac{y}{x})dx + (\frac{e^x}{y} + \ln x)dy = 0$
\vspace{0.3\baselineskip}

\textbf{Řešení}: 
\begin{enumerate}
\item \textbf{Ověření exaktnosti}:
\[
M = e^x \ln y + \frac{y}{x}, \quad N = \frac{e^x}{y} + \ln x
\]
\[
\frac{\partial M}{\partial y} = \frac{e^x}{y} + \frac{1}{x}, \quad \frac{\partial N}{\partial x} = \frac{e^x}{y} + \frac{1}{x} \quad \checkmark
\]

\item \textbf{Integrace M podle x}:
\[
\int \left(e^x \ln y + \frac{y}{x}\right)dx = e^x \ln y + y\ln x + \phi(y)
\]

\item \textbf{Derivace a porovnání}:
\[
\frac{\partial F}{\partial y} = \frac{e^x}{y} + \ln x + \phi'(y) = \frac{e^x}{y} + \ln x
\]
\[
\phi'(y) = 0
\]

\item \textbf{Potenciál}:
\[
F(x, y) = e^x \ln y + y\ln x
\]

\item \textbf{Řešení}:
\[
e^x \ln y + y\ln x = C
\end{enumerate}
\end{example}

\vspace{0.6\baselineskip}

\begin{example}[Složené exponenciálně-logaritmické výrazy]
Řešte: $(y^x \ln y + e^{x+y})dx + (xy^{x-1} + e^{x+y})dy = 0$
\vspace{0.3\baselineskip}

\textbf{Řešení}: 
\begin{enumerate}
\item \textbf{Ověření exaktnosti}:
\[
M = y^x \ln y + e^{x+y}, \quad N = xy^{x-1} + e^{x+y}
\]
\[
\frac{\partial M}{\partial y} = xy^{x-1} \ln y + y^x \cdot \frac{1}{y} + e^{x+y} = xy^{x-1} \ln y + y^{x-1} + e^{x+y}
\]
\[
\frac{\partial N}{\partial x} = y^{x-1} + xy^{x-1} \ln y + e^{x+y} \quad \checkmark
\]

\item \textbf{Integrace M podle x}:
\[
\int (y^x \ln y + e^{x+y})dx = y^x + e^{x+y} + \phi(y)
\]

\item \textbf{Derivace a porovnání}:
\[
\frac{\partial F}{\partial y} = xy^{x-1} + e^{x+y} + \phi'(y) = xy^{x-1} + e^{x+y}
\]
\[
\phi'(y) = 0
\]

\item \textbf{Potenciál}:
\[
F(x, y) = y^x + e^{x+y}
\]

\item \textbf{Řešení}:
\[
y^x + e^{x+y} = C
\end{enumerate}
\end{example}

\vspace{0.8\baselineskip}

\subsubsection{Integrační Faktory pro Neexaktní Rovnice}
\label{subsubsec:integracni-faktory}

\begin{definition}[Integrační faktor]
Funkce $\mu(x, y)$ je \emph{integrační faktor} pro rovnici $M dx + N dy = 0$, jestliže rovnice
\[
\mu M dx + \mu N dy = 0
\]
je exaktní.
\end{definition}

\vspace{0.6\baselineskip}

\begin{theorem}[Rovnice pro integrační faktor]
Funkce $\mu(x, y)$ je integrační faktor právě tehdy, když splňuje parciální diferenciální rovnici:
\[
M\frac{\partial \mu}{\partial y} - N\frac{\partial \mu}{\partial x} + \mu\left(\frac{\partial M}{\partial y} - \frac{\partial N}{\partial x}\right) = 0
\]
\end{theorem}

\vspace{0.4\baselineskip}

\begin{proof}
Podmínka exaktnosti pro $\mu M dx + \mu N dy = 0$ je:
\[
\frac{\partial (\mu M)}{\partial y} = \frac{\partial (\mu N)}{\partial x}
\]
\[
M\frac{\partial \mu}{\partial y} + \mu\frac{\partial M}{\partial y} = N\frac{\partial \mu}{\partial x} + \mu\frac{\partial N}{\partial x}
\]
Po úpravě dostaneme uvedenou rovnici.
\end{proof}

\vspace{0.8\baselineskip}

\begin{method}[Hledání integračních faktorů speciálního tvaru]
\label{met:integracni-faktory-speciální}
\begin{enumerate}
\item \textbf{Předpoklad $\mu = \mu(x)$}: Pak $\frac{\partial \mu}{\partial y} = 0$ a rovnice se zjednoduší na:
\[
\frac{d\mu}{dx} = \frac{\frac{\partial M}{\partial y} - \frac{\partial N}{\partial x}}{N} \mu
\]
Tento předpoklad je rozumný, pokud výraz $\frac{\frac{\partial M}{\partial y} - \frac{\partial N}{\partial x}}{N}$ závisí pouze na $x$.

\item \textbf{Předpoklad $\mu = \mu(y)$}: Analogicky:
\[
\frac{d\mu}{dy} = \frac{\frac{\partial N}{\partial x} - \frac{\partial M}{\partial y}}{M} \mu
\]
Rozumný, pokud výraz závisí pouze na $y$.

\item \textbf{Řešení separované rovnice}:
\[
\mu(x) = \exp\left(\int \frac{\frac{\partial M}{\partial y} - \frac{\partial N}{\partial x}}{N} dx\right)
\]
\[
\mu(y) = \exp\left(\int \frac{\frac{\partial N}{\partial x} - \frac{\partial M}{\partial y}}{M} dy\right)
\]
\end{enumerate}
\end{method}

\vspace{0.8\baselineskip}

\begin{example}[Integrační faktor $\mu(x)$]
Řešte: $(3xy + y^2)dx + (x^2 + xy)dy = 0$
\vspace{0.3\baselineskip}

\textbf{Řešení}: 
\begin{enumerate}
\item \textbf{Ověření exaktnosti}:
\[
\frac{\partial M}{\partial y} = 3x + 2y, \quad \frac{\partial N}{\partial x} = 2x + y
\]
Rovnice není exaktní.

\item \textbf{Hledání $\mu(x)$}:
\[
\frac{\frac{\partial M}{\partial y} - \frac{\partial N}{\partial x}}{N} = \frac{(3x + 2y) - (2x + y)}{x^2 + xy} = \frac{x + y}{x(x + y)} = \frac{1}{x}
\]
Závisí pouze na $x$ → předpoklad je platný.

\item \textbf{Výpočet $\mu(x)$}:
\[
\mu(x) = \exp\left(\int \frac{1}{x} dx\right) = \exp(\ln|x|) = |x| \quad (\text{volíme } x > 0 \Rightarrow \mu = x)
\]

\item \textbf{Násobení rovnice}:
\[
x[(3xy + y^2)dx + (x^2 + xy)dy] = 0
\]
\[
(3x^2y + xy^2)dx + (x^3 + x^2y)dy = 0
\]

\item \textbf{Ověření exaktnosti nové rovnice}:
\[
\frac{\partial}{\partial y}(3x^2y + xy^2) = 3x^2 + 2xy
\]
\[
\frac{\partial}{\partial x}(x^3 + x^2y) = 3x^2 + 2xy \quad \checkmark
\]

\item \textbf{Řešení exaktní rovnice}:
\[
F(x, y) = \int (3x^2y + xy^2)dx = x^3y + \frac{1}{2}x^2y^2 + \phi(y)
\]
\[
\frac{\partial F}{\partial y} = x^3 + x^2y + \phi'(y) = x^3 + x^2y \Rightarrow \phi'(y) = 0
\]
\[
F(x, y) = x^3y + \frac{1}{2}x^2y^2
\]

\item \textbf{Konečné řešení}:
\[
x^3y + \frac{1}{2}x^2y^2 = C \quad \text{nebo} \quad x^2y(2x + y) = 2C
\end{enumerate}
\end{example}

\vspace{0.8\baselineskip}

\begin{transition}
V další části pokryjeme integrační faktory tvaru $\mu(xy)$, $\mu(x/y)$ a jejich aplikace na symetrické rovnice, včetně geometrické interpretace a pokročilých technik.
\end{transition}


% !TEX root = ../main.tex
\subsubsection{Integrační Faktory Speciálních Tvarů}
\label{subsubsec:integracni-faktory-specialni}

\begin{method}[Integrační faktor $\mu = \mu(xy)$]
\label{met:integracni-faktor-xy}
Předpokládáme, že $\mu = \mu(z)$ kde $z = xy$. Pak:
\[
\frac{\partial \mu}{\partial x} = \mu'(z)y, \quad \frac{\partial \mu}{\partial y} = \mu'(z)x
\]
Rovnice pro integrační faktor se zjednoduší na:
\[
\frac{d\mu}{dz} = \frac{\frac{\partial M}{\partial y} - \frac{\partial N}{\partial x}}{Ny - Mx} \mu
\]
Tento předpoklad je rozumný, pokud výraz $\frac{\frac{\partial M}{\partial y} - \frac{\partial N}{\partial x}}{Ny - Mx}$ závisí pouze na $z = xy$.
\end{method}

\vspace{0.6\baselineskip}

\begin{example}[Integrační faktor $\mu(xy)$]
Řešte: $(y + xy^2)dx + (x - x^2y)dy = 0$
\vspace{0.3\baselineskip}

\textbf{Řešení}:
\begin{enumerate}
\item \textbf{Ověření exaktnosti}:
\[
\frac{\partial M}{\partial y} = 1 + 2xy, \quad \frac{\partial N}{\partial x} = 1 - 2xy \quad \text{NEJSOU STEJNÉ}
\]

\item \textbf{Test pro $\mu(xy)$}:
\[
Ny - Mx = (x - x^2y)y - (y + xy^2)x = xy - x^2y^2 - xy - x^2y^2 = -2x^2y^2
\]
\[
\frac{\frac{\partial M}{\partial y} - \frac{\partial N}{\partial x}}{Ny - Mx} = \frac{(1 + 2xy) - (1 - 2xy)}{-2x^2y^2} = \frac{4xy}{-2x^2y^2} = -\frac{2}{xy}
\]
Závisí pouze na $z = xy$ → předpoklad platný.

\item \textbf{Výpočet $\mu(z)$}:
\[
\frac{d\mu}{dz} = -\frac{2}{z} \mu \Rightarrow \frac{d\mu}{\mu} = -\frac{2}{z} dz
\]
\[
\ln|\mu| = -2\ln|z| + C \Rightarrow \mu = \frac{1}{z^2} = \frac{1}{x^2y^2}
\]

\item \textbf{Násobení a řešení}:
\[
\frac{1}{x^2y^2}[(y + xy^2)dx + (x - x^2y)dy] = 0
\]
\[
\left(\frac{1}{x^2y} + \frac{1}{x}\right)dx + \left(\frac{1}{xy^2} - \frac{1}{y}\right)dy = 0
\]
Tato rovnice je již exaktní. Řešením dostaneme:
\[
F(x, y) = -\frac{1}{xy} + \ln\left|\frac{x}{y}\right| = C
\end{enumerate}
\end{example}

\vspace{0.8\baselineskip}

\begin{method}[Integrační faktor $\mu = \mu(x/y)$]
\label{met:integracni-faktor-x-y}
Předpokládáme, že $\mu = \mu(z)$ kde $z = x/y$. Pak:
\[
\frac{\partial \mu}{\partial x} = \frac{\mu'(z)}{y}, \quad \frac{\partial \mu}{\partial y} = -\frac{x\mu'(z)}{y^2}
\]
Rovnice pro integrační faktor:
\[
\frac{d\mu}{dz} = \frac{\frac{\partial M}{\partial y} - \frac{\partial N}{\partial x}}{\frac{N}{y} + \frac{Mx}{y^2}} \mu
\]
Předpoklad platný, pokud výraz vpravo závisí pouze na $z = x/y$.
\end{method}

\vspace{0.6\baselineskip}

\begin{example}[Integrační faktor $\mu(x/y)$]
Řešte: $(x^2 + y^2)dx + (x^2 - xy)dy = 0$
\vspace{0.3\baselineskip}

\textbf{Řešení}:
\begin{enumerate}
\item \textbf{Ověření exaktnosti}:
\[
\frac{\partial M}{\partial y} = 2y, \quad \frac{\partial N}{\partial x} = 2x - y \quad \text{NEJSOU STEJNÉ}
\]

\item \textbf{Test pro $\mu(x/y)$}:
\[
\frac{N}{y} + \frac{Mx}{y^2} = \frac{x^2 - xy}{y} + \frac{(x^2 + y^2)x}{y^2} = \frac{x^2}{y} - x + \frac{x^3}{y^2} + \frac{x}{y}
\]
\[
\frac{\frac{\partial M}{\partial y} - \frac{\partial N}{\partial x}}{\frac{N}{y} + \frac{Mx}{y^2}} = \frac{2y - (2x - y)}{\frac{x^2}{y} - x + \frac{x^3}{y^2} + \frac{x}{y}} = \frac{3y - 2x}{\frac{x^2}{y} - x + \frac{x^3}{y^2} + \frac{x}{y}}
\]
Vynásobením čitatele i jmenovatele $y^2$:
\[
= \frac{(3y - 2x)y^2}{x^2y - xy^2 + x^3 + xy} = \frac{y^2(3y - 2x)}{x(x^2 + xy + y^2 - y^2)}
\]
Po zjednodušení zjistíme, že výraz závisí pouze na $x/y$.

\item \textbf{Výpočet $\mu(z)$} a řešení vede na:
\[
\mu = \frac{1}{x^2} \quad \text{a řešení} \quad \ln|x| - \frac{y}{x} + \frac{1}{2}\left(\frac{y}{x}\right)^2 = C
\end{enumerate}
\end{example}

\vspace{0.8\baselineskip}

\subsubsection{Geometrická Interpretace a Fázová Analýza}
\label{subsubsec:geometricka-interpretace}

\begin{definition}[Ekvipotenciální křivky]
Pro exaktní rovnici $M dx + N dy = 0$ s potenciálem $F(x, y)$ jsou \emph{ekvipotenciální křivky} definovány jako:
\[
F(x, y) = C, \quad C \in \mathbb{R}
\]
Tyto křivky tvoří řešení diferenciální rovnice.
\end{definition}

\vspace{0.6\baselineskip}

\begin{theorem}[Ortogonální trajektorie]
Nechť $F(x, y) = C$ je rodina ekvipotenciálních křivek. Pak rodina křivek $G(x, y) = K$ splňující:
\[
\frac{\partial F}{\partial x} \frac{\partial G}{\partial x} + \frac{\partial F}{\partial y} \frac{\partial G}{\partial y} = 0
\]
jsou \emph{ortogonální trajektorie} k původní rodině.
\end{theorem}

\vspace{0.4\baselineskip}

\begin{proof}
Gradient $\nabla F$ je normálový vektor k ekvipotenciálním křivkám. Podmínka ortogonality vyžaduje, aby $\nabla F \cdot \nabla G = 0$.
\end{proof}

\vspace{0.6\baselineskip}

\begin{example}[Ekvipotenciály a ortogonální trajektorie]
Najděte ortogonální trajektorie k rodině: $x^2 + y^2 = C$
\vspace{0.3\baselineskip}

\textbf{Řešení}:
\begin{enumerate}
\item \textbf{Původní rodina}: $F(x, y) = x^2 + y^2 = C$
\[
\nabla F = (2x, 2y)
\]

\item \textbf{Diferenciální rovnice ortogonálních trajektorií}:
\[
\frac{dy}{dx} = \frac{-F_x}{F_y} = \frac{-2x}{2y} = -\frac{x}{y}
\]

\item \textbf{Řešení}:
\[
\frac{dy}{dx} = -\frac{x}{y} \Rightarrow y dy = -x dx
\]
\[
\frac{1}{2}y^2 = -\frac{1}{2}x^2 + K \Rightarrow x^2 + y^2 = 2K
\]

\item \textbf{Interpretace}: Ortogonální trajektorie jsou opět kružnice se středem v počátku.
\end{enumerate}
\end{example}

\vspace{0.8\baselineskip}

\begin{definition}[Singulární body]
Bod $(x_0, y_0)$ je \emph{singulárním bodem} rovnice $M dx + N dy = 0$, jestliže:
\[
M(x_0, y_0) = N(x_0, y_0) = 0
\]
V těchto bodech není směrové pole definováno.
\end{definition}

\vspace{0.6\baselineskip}

\begin{example}[Analýza singulárních bodů]
Analyzujte singulární body rovnice: $(x^2 - y^2)dx + 2xydy = 0$
\vspace{0.3\baselineskip}

\textbf{Řešení}:
\begin{enumerate}
\item \textbf{Singulární body}: 
\[
M = x^2 - y^2 = 0, \quad N = 2xy = 0
\]
Jediný singulární bod je $(0, 0)$.

\item \textbf{Chvání v okolí singularity}:
\[
\frac{dy}{dx} = -\frac{M}{N} = -\frac{x^2 - y^2}{2xy} = \frac{y^2 - x^2}{2xy}
\]
V polárních souřadnicích $x = r\cos\theta$, $y = r\sin\theta$:
\[
\frac{dy}{dx} = \frac{\sin^2\theta - \cos^2\theta}{2\sin\theta\cos\theta} = -\frac{\cos 2\theta}{\sin 2\theta} = -\cot 2\theta
\]

\item \textbf{Interpretace}: Směrové pole závisí pouze na úhlu $\theta$, ne na poloměru $r$. Jedná se o \emph{radiálně homogenní} pole.
\end{enumerate}
\end{example}

\vspace{0.8\baselineskip}

\subsubsection{Početní Sekce - Kategorie B: Rovnice s Integračními Faktory}
\label{subsubsec:pocetni-kategorie-b}

\paragraph*{B1: Jednoduché integrační faktory $\mu(x)$ a $\mu(y)$}

\begin{example}[$\mu(x)$ - fyzikální aplikace]
Řešte rovnici popisující izotermický děj: $(PV - a)dP + P^2 dV = 0$
\vspace{0.3\baselineskip}

\textbf{Řešení}:
\begin{enumerate}
\item \textbf{Ověření exaktnosti}:
\[
\frac{\partial}{\partial V}(PV - a) = P, \quad \frac{\partial}{\partial P}(P^2) = 2P \quad \text{NEJSOU STEJNÉ}
\]

\item \textbf{Hledání $\mu(P)$}:
\[
\frac{\frac{\partial M}{\partial V} - \frac{\partial N}{\partial P}}{N} = \frac{P - 2P}{P^2} = -\frac{1}{P}
\]
\[
\mu(P) = \exp\left(-\int \frac{1}{P} dP\right) = \frac{1}{P}
\]

\item \textbf{Řešení}:
\[
\frac{1}{P}[(PV - a)dP + P^2 dV] = 0 \Rightarrow (V - \frac{a}{P})dP + P dV = 0
\]
Tato rovnice je exaktní. Potenciál:
\[
F(P, V) = PV - a\ln|P| = C
\]
\end{enumerate}
\end{example}

\vspace{0.6\baselineskip}

\begin{example}[$\mu(y)$ - ekonomická aplikace]
Řešte rovnici indiferenční křivky: $(2xy + y^2)dx + (x^2 + 2xy - 1)dy = 0$
\vspace{0.3\baselineskip}

\textbf{Řešení}:
\begin{enumerate}
\item \textbf{Ověření exaktnosti}:
\[
\frac{\partial M}{\partial y} = 2x + 2y, \quad \frac{\partial N}{\partial x} = 2x + 2y \quad \checkmark
\]
Poznámka: Tato rovnice je již exaktní, slouží jako ilustrace.

\item \textbf{Potenciál}:
\[
F(x, y) = \int (2xy + y^2)dx = x^2y + xy^2 + \phi(y)
\]
\[
\frac{\partial F}{\partial y} = x^2 + 2xy + \phi'(y) = x^2 + 2xy - 1 \Rightarrow \phi'(y) = -1
\]
\[
F(x, y) = x^2y + xy^2 - y
\]

\item \textbf{Řešení}:
\[
x^2y + xy^2 - y = C \quad \text{nebo} \quad y(x^2 + xy - 1) = C
\]

\item \textbf{Ekonomická interpretace}: Jedná se o indiferenční křivku utility funkce $U(x, y) = x^2y + xy^2 - y$.
\end{enumerate}
\end{example}

\vspace{0.8\baselineskip}

\paragraph*{B2: Pokročilé integrační faktory}

\begin{example}[$\mu(x^2 + y^2)$ - radiálně symetrické pole]
Řešte: $(x^2 + y^2 + y)dx + (x^2 + y^2 - x)dy = 0$
\vspace{0.3\baselineskip}

\textbf{Řešení}:
\begin{enumerate}
\item \textbf{Ověření exaktnosti}:
\[
\frac{\partial M}{\partial y} = 2y + 1, \quad \frac{\partial N}{\partial x} = 2x - 1 \quad \text{NEJSOU STEJNÉ}
\]

\item \textbf{Test pro $\mu(r^2)$ kde $r^2 = x^2 + y^2$}:
Výpočtem zjistíme, že vhodný integrační faktor je $\mu = \frac{1}{x^2 + y^2}$.

\item \textbf{Násobení a řešení}:
\[
\left(1 + \frac{y}{x^2 + y^2}\right)dx + \left(1 - \frac{x}{x^2 + y^2}\right)dy = 0
\]
Řešením dostaneme:
\[
x + y + \arctan\left(\frac{y}{x}\right) = C
\end{enumerate}
\end{example}

\vspace{0.8\baselineskip}

\begin{transition}
V závěrečné části pokryjeme aplikace ve fyzikálních a ekonomických modelech, numerické metody a přípravu na pokročilejší témata v diferenciální geometrii.
\end{transition}

% !TEX root = ../main.tex
\subsubsection{Aplikace v Kvantitativních Vědách - Kategorie C}
\label{subsubsec:aplikace-kategorie-c}

\paragraph*{C1: Fyzikální modely}

\begin{example}[Gravitační pole]
Odvoďte rovnice pro gravitační pole hmotného bodu a najděte ekvipotenciální plochy.
\vspace{0.3\baselineskip}

\textbf{Řešení}:
\begin{enumerate}
\item \textbf{Gravitační síla}: Pro hmotný bod $M$ v počátku je síla na jednotkovou hmotnost:
\[
\vec{F} = -\frac{GM}{r^2} \hat{r} = -\frac{GM}{(x^2 + y^2)^{3/2}}(x, y)
\]

\item \textbf{Diferenciální forma}:
\[
-\frac{GMx}{(x^2 + y^2)^{3/2}}dx - \frac{GMy}{(x^2 + y^2)^{3/2}}dy = 0
\]

\item \textbf{Ověření exaktnosti}:
\[
\frac{\partial M}{\partial y} = GMx \cdot \frac{3}{2}(x^2 + y^2)^{-5/2} \cdot 2y = \frac{3GMxy}{(x^2 + y^2)^{5/2}}
\]
\[
\frac{\partial N}{\partial x} = \frac{3GMxy}{(x^2 + y^2)^{5/2}} \quad \checkmark
\]

\item \textbf{Potenciál}:
\[
F(x, y) = \int -\frac{GMx}{(x^2 + y^2)^{3/2}}dx = \frac{GM}{\sqrt{x^2 + y^2}} + \phi(y)
\]
\[
\frac{\partial F}{\partial y} = -\frac{GMy}{(x^2 + y^2)^{3/2}} + \phi'(y) = -\frac{GMy}{(x^2 + y^2)^{3/2}} \Rightarrow \phi'(y) = 0
\]

\item \textbf{Gravitační potenciál}:
\[
F(x, y) = \frac{GM}{\sqrt{x^2 + y^2}} = \frac{GM}{r}
\]

\item \textbf{Ekvipotenciální křivky}: $r = \text{konst}$ - kružnice se středem v počátku.
\end{enumerate}
\end{example}

\vspace{0.6\baselineskip}

\begin{example}[Termodynamika - stavová rovnice ideálního plynu]
Odvoďte vztah mezi tlakem, objemem a teplotou pro adiabatický děj.
\vspace{0.3\baselineskip}

\textbf{Řešení}:
\begin{enumerate}
\item \textbf{První termodynamický zákon}: $dU = \delta Q - \delta W$
\item \textbf{Adiabatický děj}: $\delta Q = 0$, $dU = c_V dT$, $\delta W = P dV$
\item \textbf{Rovnice}: $c_V dT + P dV = 0$
\item \textbf{Stavová rovnice}: $PV = nRT \Rightarrow P = \frac{nRT}{V}$
\item \textbf{Diferenciální rovnice}:
\[
c_V dT + \frac{nRT}{V} dV = 0
\]

\item \textbf{Integrační faktor}: $\mu = \frac{1}{T}$
\[
\frac{c_V}{T} dT + \frac{nR}{V} dV = 0
\]

\item \textbf{Řešení}:
\[
c_V \ln T + nR \ln V = \text{konst} \Rightarrow TV^{\frac{nR}{c_V}} = \text{konst}
\]

\item \textbf{Poměr měrných tepel}: $\gamma = \frac{c_P}{c_V}$, $c_P - c_V = nR$
\[
TV^{\gamma-1} = \text{konst} \quad \text{nebo} \quad PV^\gamma = \text{konst}
\end{enumerate}
\end{example}

\vspace{0.8\baselineskip}

\paragraph*{C2: Ekonomické aplikace}

\begin{example}[Utility funkce a indiferenční křivky]
Mějme Cobb-Douglasovu utility funkci $U(x, y) = x^\alpha y^\beta$. Najděte indiferenční křivky.
\vspace{0.3\baselineskip}

\textbf{Řešení}:
\begin{enumerate}
\item \textbf{Totální diferenciál utility}:
\[
dU = \frac{\partial U}{\partial x}dx + \frac{\partial U}{\partial y}dy = \alpha x^{\alpha-1}y^\beta dx + \beta x^\alpha y^{\beta-1} dy
\]

\item \textbf{Podél indiferenční křivky}: $dU = 0$
\[
\alpha x^{\alpha-1}y^\beta dx + \beta x^\alpha y^{\beta-1} dy = 0
\]

\item \textbf{Diferenciální rovnice}:
\[
\alpha y dx + \beta x dy = 0
\]

\item \textbf{Ověření exaktnosti}:
\[
\frac{\partial}{\partial y}(\alpha y) = \alpha, \quad \frac{\partial}{\partial x}(\beta x) = \beta
\]
Exaktní pouze pokud $\alpha = \beta$.

\item \textbf{Integrační faktor pro $\alpha \neq \beta$}: $\mu = \frac{1}{xy}$
\[
\frac{\alpha}{x} dx + \frac{\beta}{y} dy = 0
\]

\item \textbf{Řešení}:
\[
\alpha \ln|x| + \beta \ln|y| = C \Rightarrow x^\alpha y^\beta = e^C = K
\]

\item \textbf{Interpretace}: Indiferenční křivky jsou dány $U(x, y) = \text{konst}$.
\end{enumerate}
\end{example}

\vspace{0.6\baselineskip}

\begin{example}[Edgeworthův box - směna mezi dvěma spotřebiteli]
Analyzujte rovnováhu v modelu čisté směny.
\vspace{0.3\baselineskip}

\textbf{Řešení}:
\begin{enumerate}
\item \textbf{Spotřebitel A}: Utility $U_A(x_A, y_A) = x_A^\alpha y_A^{1-\alpha}$
\item \textbf{Spotřebitel B}: Utility $U_B(x_B, y_B) = x_B^\beta y_B^{1-\beta}$
\item \textbf{Prostředkové omezení}: $x_A + x_B = \bar{x}$, $y_A + y_B = \bar{y}$
\item \textbf{Podmínka Pareto optimality}:
\[
\frac{\partial U_A/\partial x_A}{\partial U_A/\partial y_A} = \frac{\partial U_B/\partial x_B}{\partial U_B/\partial y_B}
\]
\[
\frac{\alpha y_A}{(1-\alpha)x_A} = \frac{\beta y_B}{(1-\beta)x_B}
\]

\item \textbf{Diferenciální forma}:
\[
[\alpha(1-\beta)y_A x_B - \beta(1-\alpha)x_A y_B] = 0
\]
S využitím $x_B = \bar{x} - x_A$, $y_B = \bar{y} - y_A$ dostaneme exaktní rovnici.
\end{enumerate}
\end{example}

\vspace{0.8\baselineskip}

\subsubsection{Pokročilé Techniky - Kategorie D}
\label{subsubsec:pokrocile-techniky}

\paragraph*{D1: Vícerozměrné integrační faktory}

\begin{method}[Soustava integračních faktorů]
Pro rovnici $M dx + N dy = 0$ může existovat více integračních faktorů. Pokud $\mu_1$ a $\mu_2$ jsou integrační faktory, pak:
\[
\frac{\mu_1}{\mu_2} = \Phi(F(x, y))
\]
kde $F(x, y)$ je potenciál.
\end{method}

\vspace{0.4\baselineskip}

\begin{proof}
Nechť $\mu_1 M dx + \mu_1 N dy = dF_1$ a $\mu_2 M dx + \mu_2 N dy = dF_2$. Pak:
\[
\frac{\mu_1}{\mu_2} = \frac{dF_1}{dF_2} = \Phi(F_1) \quad \text{nebo} \quad \frac{\mu_1}{\mu_2} = \Psi(F_2)
\]
\end{proof}

\vspace{0.6\baselineskip}

\begin{example}[Více integračních faktorů]
Pro rovnici $y dx - x dy = 0$ najděte všechny integrační faktory tvaru $\mu = x^m y^n$.
\vspace{0.3\baselineskip}

\textbf{Řešení}:
\begin{enumerate}
\item \textbf{Původní rovnice}: $M = y$, $N = -x$
\item \textbf{Podmínka exaktnosti pro $\mu M dx + \mu N dy$}:
\[
\frac{\partial (\mu y)}{\partial y} = \frac{\partial (-\mu x)}{\partial x}
\]
\[
\mu + y\frac{\partial \mu}{\partial y} = -\mu - x\frac{\partial \mu}{\partial x}
\]

\item \textbf{Pro $\mu = x^m y^n$}:
\[
\frac{\partial \mu}{\partial x} = m x^{m-1} y^n, \quad \frac{\partial \mu}{\partial y} = n x^m y^{n-1}
\]
\[
x^m y^n + y \cdot n x^m y^{n-1} = -x^m y^n - x \cdot m x^{m-1} y^n
\]
\[
x^m y^n + n x^m y^n = -x^m y^n - m x^m y^n
\]
\[
(1 + n) = -(1 + m) \Rightarrow m + n = -2
\]

\item \textbf{Obecné řešení}: $\mu = x^m y^{-2-m}$ pro libovolné $m \in \mathbb{R}$
\end{enumerate}
\end{example}

\vspace{0.8\baselineskip}

\paragraph*{D2: Numerická analýza a verifikace}

\begin{method}[Numerický výpočet potenciálu]
Pro komplikované rovnice můžeme potenciál počítat numericky:
\[
F(x, y) \approx \int_{\gamma} M dx + N dy
\]
kde $\gamma$ je vhodná integrační cesta.
\end{method}

\vspace{0.6\baselineskip}

\begin{example}[Numerická verifikace exaktnosti]
Ověřte exaktnost rovnice $(x^2 + \sin y)dx + (x\cos y + e^y)dy = 0$ numericky.
\vspace{0.3\baselineskip}

\textbf{Řešení}:
\begin{enumerate}
\item \textbf{Analytická podmínka}:
\[
\frac{\partial M}{\partial y} = \cos y, \quad \frac{\partial N}{\partial x} = \cos y \quad \checkmark
\]

\item \textbf{Numerická verifikace}:
\begin{verbatim}
import numpy as np
from scipy import integrate

def M(x, y): return x**2 + np.sin(y)
def N(x, y): return x*np.cos(y) + np.exp(y)

# Test na mřížce
x_vals = np.linspace(0.1, 2, 10)
y_vals = np.linspace(0.1, 2, 10)

for x in x_vals:
    for y in y_vals:
        dM_dy = (M(x, y+1e-6) - M(x, y))/1e-6
        dN_dx = (N(x+1e-6, y) - N(x, y))/1e-6
        assert abs(dM_dy - dN_dx) < 1e-8
\end{verbatim}

\item \textbf{Numerický výpočet potenciálu}:
\[
F(x, y) = \int_0^x M(t, 0)dt + \int_0^y N(x, s)ds
\]
\end{enumerate}
\end{example}

\vspace{0.8\baselineskip}

\subsubsection{Shrnutí a Expertní Metodologie}
\label{subsubsec:shrnutí-metodologie}

\begin{method}[Decision tree pro řešení rovnic 1. řádu]
\label{met:decision-tree}
\begin{enumerate}
\item \textbf{Krok 1: Test exaktnosti}
\[
\text{Vypočítej } \frac{\partial M}{\partial y} - \frac{\partial N}{\partial x}
\]
\begin{itemize}
\item Pokud $= 0$ → přejdi na Krok 2A
\item Pokud $\neq 0$ → přejdi na Krok 2B
\end{itemize}

\item \textbf{Krok 2A: Řešení exaktní rovnice}
\begin{itemize}
\item Integruj $M$ podle $x$: $F(x, y) = \int M dx + \phi(y)$
\item Derivuj podle $y$: $\frac{\partial F}{\partial y} = \frac{\partial}{\partial y}(\int M dx) + \phi'(y)$
\item Porovnej s $N$: $\phi'(y) = N - \frac{\partial}{\partial y}(\int M dx)$
\item Integruj $\phi(y)$
\end{itemize}

\item \textbf{Krok 2B: Hledání integračního faktoru}
\begin{enumerate}
\item \textbf{Test $\mu(x)$}: $\frac{\frac{\partial M}{\partial y} - \frac{\partial N}{\partial x}}{N}$ závisí pouze na $x$?
\item \textbf{Test $\mu(y)$}: $\frac{\frac{\partial N}{\partial x} - \frac{\partial M}{\partial y}}{M}$ závisí pouze na $y$?
\item \textbf{Test $\mu(xy)$}: $\frac{\frac{\partial M}{\partial y} - \frac{\partial N}{\partial x}}{Ny - Mx}$ závisí pouze na $xy$?
\item \textbf{Test $\mu(x/y)$}: $\frac{\frac{\partial M}{\partial y} - \frac{\partial N}{\partial x}}{\frac{N}{y} + \frac{Mx}{y^2}}$ závisí pouze na $x/y$?
\item \textbf{Obecný případ}: Řeš rovnici $M\frac{\partial \mu}{\partial y} - N\frac{\partial \mu}{\partial x} + \mu(\frac{\partial M}{\partial y} - \frac{\partial N}{\partial x}) = 0$
\end{enumerate}

\item \textbf{Krok 3: Analýza řešení}
\begin{itemize}
\item Urči definiční obor
\item Identifikuj singulární body
\item Analyzuj asymptotické chování
\item Ověř řešení derivací
\end{itemize}
\end{enumerate}
\end{method}

\vspace{0.8\baselineskip}

\begin{table}[h]
\centering
\caption{Přehled metod pro exaktní rovnice}
\label{tab:prehled-metod}
\begin{tabular}{p{0.2\textwidth}p{0.35\textwidth}p{0.35\textwidth}}
\toprule
\textbf{Typ rovnice} & \textbf{Metoda řešení} & \textbf{Kontrolní podmínka} \\
\midrule
Přímá exaktní & Přímá integrace & $\frac{\partial M}{\partial y} = \frac{\partial N}{\partial x}$ \\
$\mu = \mu(x)$ & $\mu(x) = \exp\left(\int \frac{M_y - N_x}{N} dx\right)$ & $\frac{M_y - N_x}{N}$ závisí pouze na $x$ \\
$\mu = \mu(y)$ & $\mu(y) = \exp\left(\int \frac{N_x - M_y}{M} dy\right)$ & $\frac{N_x - M_y}{M}$ závisí pouze na $y$ \\
$\mu = \mu(xy)$ & Řeš $\frac{d\mu}{dz} = \frac{M_y - N_x}{Ny - Mx} \mu$ & $\frac{M_y - N_x}{Ny - Mx}$ závisí pouze na $xy$ \\
Obecný případ & Řeš $M\mu_y - N\mu_x + (M_y - N_x)\mu = 0$ & - \\
\bottomrule
\end{tabular}
\end{table}

\vspace{0.8\baselineskip}

\subsubsection{Příprava na Pokročilejší Témata}
\label{subsubsec:priprava-pokrocila-temata}

\begin{transition}[Spojitost s parciálními diferenciálními rovnicemi]
Teorie exaktních rovnice přímo navazuje na řešení parciálních diferenciálních rovnic 1. řádu. Rovnice:
\[
M(x, y) \frac{\partial F}{\partial x} + N(x, y) \frac{\partial F}{\partial y} = 0
\]
je řešitelná metodou charakteristik, která zobecňuje metodu integračních faktorů.
\end{transition}

\vspace{0.6\baselineskip}

\begin{transition}[Diferenciální geometrie a diferenciální formy]
V diferenciální geometrii se exaktní rovnice zapisují jako:
\[
\omega = M dx + N dy = 0
\]
kde $\omega$ je 1-forma. Podmínka exaktnosti $d\omega = 0$ odpovídá uzavřenosti diferenciální formy. Věta o nezávislosti na cestě je speciálním případem Stokesovy věty.
\end{transition}

\vspace{0.6\baselineskip}

\begin{transition}[Příprava na systémy rovnic]
Metody pro exaktní rovnice se zobecňují pro systémy:
\[
M_1 dx + M_2 dy + M_3 dz = 0
\]
s podmínkami uzavřenosti:
\[
\frac{\partial M_1}{\partial y} = \frac{\partial M_2}{\partial x}, \quad
\frac{\partial M_1}{\partial z} = \frac{\partial M_3}{\partial x}, \quad
\frac{\partial M_2}{\partial z} = \frac{\partial M_3}{\partial y}
\]
\end{transition}

\vspace{0.8\baselineskip}

\begin{conclusion}
Exaktní diferenciální rovnice představují fundamentální nástroj pro modelování konzervativních systémů v přírodních, technických i ekonomických vědách. Zvládnutí jejich teorie a řešitelských metod poskytuje:
\begin{itemize}
\item \textbf{Hluboké porozumění} geometrické struktuce diferenciálních rovnic
\item \textbf{Efektivní nástroje} pro řešení široké třídy fyzikálních a ekonomických modelů
\item \textbf{Pevný základ} pro studium pokročilejších témat v parciálních diferenciálních rovnicích a diferenciální geometrii
\item \textbf{Systematický přístup} k analýze a verifikaci řešení
\end{itemize}

Metody prezentované v této kapitole tvoří esenciální součást instrumentáře každého kvantitativního experta.
\end{conclusion}

% !TEX root = ../main.tex
\subsection{Závěr: Exaktní Rovnice - Syntéza Metod}
\label{subsec:zaver-exaktni}

\subsubsection{Klíčové Poznatky a Metodologické Shrnutí}
\label{subsubsec:klicove-poznatky-exaktni}

\begin{summary}[Kompletní metodologie exaktních rovnic]
\label{sum:kompletni-metodologie}
Po prostudování této kapitoly by měl kvantitativní expert ovládat:

\begin{enumerate}
\item \textbf{Teoretický fundament}:
\begin{itemize}
\item Definice exaktní rovnice a geometrická interpretace
\item Nutná a postačující podmínka exaktnosti
\item Věta o nezávislosti na cestě integrace
\end{itemize}

\item \textbf{Řešitelské metody}:
\begin{itemize}
\item Přímá integrační metoda s křížovou verifikací
\item Alternativní integrace podle různých proměnných
\item Systematické hledání integračních faktorů
\end{itemize}

\item \textbf{Klasifikační schopnosti}:
\begin{itemize}
\item Rozpoznání typu $M(x,y)$ a $N(x,y)$ (polynomiální, racionální, goniometrické, exponenciální)
\item Identifikace vhodného typu integračního faktoru
\item Analýza singularit a definičních oborů
\end{itemize}

\item \textbf{Aplikační kompetence}:
\begin{itemize}
\item Modelování konzervativních fyzikálních systémů
\item Analýza ekonomických modelů s potenciálem
\item Numerická verifikace řešení
\end{itemize}
\end{enumerate}
\end{summary}

\vspace{0.8\baselineskip}

\begin{method}[Rozhodovací strom pro exaktní rovnice - EXPERTNÍ VERZE]
\label{met:rozhodovaci-strom-expert}
\begin{enumerate}
\item \textbf{KROK 0: Příprava}
\begin{itemize}
\item Normalizace rovnice do tvaru $M dx + N dy = 0$
\item Kontrola spojitosti $M$, $N$ a jejich parciálních derivací
\item Identifikace definičního oboru a singularit
\end{itemize}

\item \textbf{KROK 1: Test exaktnosti}
\[
\Delta = \frac{\partial M}{\partial y} - \frac{\partial N}{\partial x}
\]
\begin{itemize}
\item \textbf{ANO} ($\Delta = 0$): → KROK 2A
\item \textbf{NE} ($\Delta \neq 0$): → KROK 2B
\end{itemize}

\item \textbf{KROK 2A: Řešení exaktní rovnice}
\begin{itemize}
\item Volba integrační strategie (podle $x$ nebo $y$)
\item Integrace a určení "integrační funkce"
\item Křížová verifikace řešení
\item Analýza řešení: singularita, asymptotické chování
\end{itemize}

\item \textbf{KROK 2B: Hledání integračního faktoru}
\begin{enumerate}
\item \textbf{Test $\mu(x)$}: $\frac{\Delta}{N}$ závisí pouze na $x$?
\item \textbf{Test $\mu(y)$}: $\frac{-\Delta}{M}$ závisí pouze na $y$?
\item \textbf{Test $\mu(xy)$}: $\frac{\Delta}{Ny - Mx}$ závisí pouze na $xy$?
\item \textbf{Test $\mu(x/y)$}: $\frac{\Delta}{\frac{N}{y} + \frac{Mx}{y^2}}$ závisí pouze na $x/y$?
\item \textbf{Obecný případ}: Řešení PDE pro $\mu(x,y)$
\end{enumerate}

\item \textbf{KROK 3: Validace řešení}
\begin{itemize}
\item Dosazení do původní rovnice
\item Kontinuita a diferencovatelnost v definičním oboru
\item Analýza singularit a hraničního chování
\end{itemize}
\end{enumerate}
\end{method}

\vspace{0.8\baselineskip}

\begin{table}[h]
\centering
\caption{Srovnání metod pro různé typy funkcí $M(x,y)$, $N(x,y)$}
\label{tab:srovnani-metod-exaktni}
\begin{tabular}{p{0.25\textwidth}p{0.3\textwidth}p{0.35\textwidth}}
\toprule
\textbf{Typ funkcí} & \textbf{Doporučená metoda} & \textbf{Kritické body} \\
\midrule
\textbf{Polynomiální} & Přímá integrace & Kontrola stupňů polynomů, identifikace singularit \\
\textbf{Racionální} & Integrační faktory $\mu(x)$, $\mu(y)$ & Analýza nulových bodů jmenovatelů \\
\textbf{Goniometrické} & Substituce před integrací & Periodicita, definiční obor \\
\textbf{Exponenciální} & Přímá integrace & Růstové chování, asymptotiky \\
\textbf{Smíšené typy} & Systematické testování $\mu$ & Volba optimální strategie \\
\bottomrule
\end{tabular}
\end{table}

\vspace{0.8\baselineskip}

\subsubsection{Příprava na Pokročilejší Témata}
\label{subsubsec:priprava-pokrocila-exaktni}

\begin{transition}[Spojitost s parciálními diferenciálními rovnicemi]
Teorie exaktních rovnic tvoří fundament pro:
\[
M(x,y) \frac{\partial F}{\partial x} + N(x,y) \frac{\partial F}{\partial y} = 0
\]
která se řeší metodou charakteristik. Exaktní rovnice představují speciální případ, kdy charakteristiky jsou ekvipotenciální křivky.
\end{transition}

\vspace{0.6\baselineskip}

\begin{transition}[Diferenciální formy a variabilní počet]
V pokročilejší matematice se exaktní rovnice zapisují jako:
\[
\omega = M dx + N dy = 0
\]
kde $\omega$ je diferenciální 1-forma. Podmínka $d\omega = 0$ odpovídá uzavřenosti formy a zobecňuje se ve Stokesově větě.
\end{transition}

\vspace{0.6\baselineskip}

\begin{conclusion}[Exaktní rovnice v kvantitativních vědách]
Exaktní rovnice poskytují mocný nástroj pro:
\begin{itemize}
\item \textbf{Modelování konzervativních systémů} v mechanice a elektrodynamice
\item \textbf{Analýzu potenciálů} v ekonomických a finančních modelech  
\item \textbf{Geometrickou interpretaci} řešení pomocí ekvipotenciálních křivek
\item \textbf{Numerickou verifikaci} pomocí nezávislosti na integrační cestě
\end{itemize}
Zvládnutí této problematiky tvoří pevný základ pro přechod k nelineárním rovnicím v Level 2.
\end{conclusion}

\subsection{Závěr Level 1: Základní ODE 1. Řádu - Kompletní Syntéza}
\label{sec:zaver-level1}

\subsubsection{Systematický Přehled Metod Řešení}
\label{subsec:systematicky-prehled}

\begin{overview}[Kompletní metodologie ODE 1. řádu]
\label{over:kompletni-metodologie-level1}
Po absolvování Level 1 disponuje kvantitativní expert následujícími kompetencemi:

\begin{table}[h]
\centering
\caption{Srovnání všech metod ODE 1. řádu}
\label{tab:srovnani-vsech-metod}
\begin{tabular}{p{0.2\textwidth}p{0.25\textwidth}p{0.2\textwidth}p{0.25\textwidth}}
\toprule
\textbf{Typ rovnice} & \textbf{Standardní tvar} & \textbf{Klíčová metoda} & \textbf{Kontrolní podmínka} \\
\midrule
\textbf{Separabilní} & $y' = f(x)g(y)$ & Separace proměnných & Rozdělitelnost na $f(x)$ a $g(y)$ \\
\textbf{Lineární} & $y' + p(x)y = q(x)$ & Integrační faktor & Linearita v $y$ a $y'$ \\
\textbf{Homogenní} & $y' = f(y/x)$ & Substituce $v = y/x$ & Homogenita stupně 0 \\
\textbf{Exaktní} & $M dx + N dy = 0$ & Hledání potenciálu & $\frac{\partial M}{\partial y} = \frac{\partial N}{\partial x}$ \\
\bottomrule
\end{tabular}
\end{table}
\end{overview}

\vspace{0.8\baselineskip}

\begin{method}[Univerzální rozhodovací algoritmus pro ODE 1. řádu]
\label{met:univerzalni-algoritmus-level1}
\begin{enumerate}
\item \textbf{KROK 1: Identifikace typu}
\begin{itemize}
\item Test separability: Lze psát jako $y' = f(x)g(y)$?
\item Test linearity: Je tvaru $y' + p(x)y = q(x)$?
\item Test homogenity: Platí $f(tx, ty) = f(x, y)$?
\item Test exaktnosti: Platí $M_y = N_x$?
\end{itemize}

\item \textbf{KROK 2: Aplikace specifické metody}
\begin{itemize}
\item \textbf{Separabilní}: $\int \frac{dy}{g(y)} = \int f(x) dx$
\item \textbf{Lineární}: $\mu(x) = e^{\int p(x)dx}$, pak $y = \frac{1}{\mu}\int \mu q dx$
\item \textbf{Homogenní}: $v = y/x$, řeš $v' = \frac{f(v)-v}{x}$
\item \textbf{Exaktní}: Najdi $F(x,y)$ tak, že $F_x = M$, $F_y = N$
\end{itemize}

\item \textbf{KROK 3: Transformace mezi typy}
\begin{itemize}
\item Některé rovnice lze řešit více způsoby
\item Homogenní rovnice mohou být exaktní
\item Lineární rovnice mohou mít separabilní homogenní část
\end{itemize}

\item \textbf{KROK 4: Validace a interpretace}
\begin{itemize}
\item Dosazení do původní rovnice
\item Analýza definičního oboru
\item Fyzikální/ekonomická interpretace
\end{itemize}
\end{enumerate}
\end{method}

\vspace{0.8\baselineskip}

\subsubsection{Aplikační Syntéza v Kvantitativních Vědách}
\label{subsec:aplikacni-syntéza}

\begin{application}[Komparativní analýza růstových modelů]
\label{app:komparativni-rustove-modely}
\begin{table}[h]
\centering
\caption{Srovnání růstových modelů řešitelných metodami Level 1}
\label{tab:srovnani-rustovych-modelu}
\begin{tabular}{p{0.25\textwidth}p{0.25\textwidth}p{0.2\textwidth}p{0.2\textwidth}}
\toprule
\textbf{Model} & \textbf{Rovnice} & \textbf{Typ} & \textbf{Řešení} \\
\midrule
Exponenciální růst & $\frac{dP}{dt} = rP$ & Separabilní & $P(t) = P_0 e^{rt}$ \\
Logistický růst & $\frac{dP}{dt} = rP(1-\frac{P}{L})$ & Separabilní & $P(t) = \frac{L}{1+Ae^{-rt}}$ \\
Gompertzův růst & $\frac{dP}{dt} = rP\ln(\frac{L}{P})$ & Separabilní & $P(t) = L e^{\ln(P_0/L)e^{-rt}}$ \\
Lineární s forcing & $\frac{dP}{dt} = aP + b$ & Lineární & $P(t) = Ce^{at} - \frac{b}{a}$ \\
\bottomrule
\end{tabular}
\end{table}
\end{application}

\vspace{0.6\baselineskip}

\begin{application}[Finanční modely - spojité procesy]
\label{app:financni-modely-level1}
\begin{itemize}
\item \textbf{Spojité úročení}: $\frac{dK}{dt} = rK$ (separabilní)
\item \textbf{Dluhová dynamika}: $\frac{dD}{dt} = rD - A$ (lineární)  
\item \textbf{Optimalizace spotřeby}: $\frac{dC}{dt} = f(C, W)$ (různé typy)
\item \textbf{Tržní difuze}: $\frac{dS}{dt} = kS(M-S)$ (logistický růst)
\end{itemize}
\end{application}

\vspace{0.6\baselineskip}

\begin{application}[Fyzikální systémy - konzervativní modely]
\label{app:fyzikalni-systemy-level1}
\begin{itemize}
\item \textbf{Radioaktivní rozpad}: $\frac{dN}{dt} = -\lambda N$ (separabilní)
\item \textbf{Newtonův zákon ochlazování}: $\frac{dT}{dt} = k(T - T_{env})$ (lineární)
\item \textbf{Konzervativní silová pole}: $F_x dx + F_y dy = 0$ (exaktní)
\item \textbf{Kapacitní vybíjení}: $\frac{dQ}{dt} = -\frac{Q}{RC}$ (separabilní)
\end{itemize}
\end{application}

\vspace{0.8\baselineskip}

\subsubsection{Přechod k Level 2: Nelineární Rovnice 1. Řádu}
\label{subsec:prechod-level2}

\begin{transition}[Od lineárních k nelineárním systémům]
\label{trans:linearni-nelinearni}
Level 1 poskytuje kompletní nástroje pro \textbf{lineární a kvazilineární} systémy. Level 2 rozšiřuje tuto schopnost na:

\begin{itemize}
\item \textbf{Bernoulliho rovnice}: $y' + p(x)y = q(x)y^n$ - nelineární zobecnění lineárních rovnic
\item \textbf{Riccatiho rovnice}: $y' = p(x)y^2 + q(x)y + r(x)$ - kvadratická nelinearita
\item \textbf{Clairautovy rovnice}: $y = xy' + f(y')$ - singulární řešení a obálky
\item \textbf{Lagrangeovy rovnice}: $y = xf(y') + g(y')$ - parametrické řešení
\item \textbf{Abelovy rovnice}: $y' = f(x)y^3 + g(x)y^2 + h(x)y$ - kubická nelinearita
\end{itemize}
\end{transition}

\vspace{0.6\baselineskip}

\begin{transition}[Rozšíření aplikačního spektra]
\label{trans:rozsireni-aplikaci}
Zatímco Level 1 pokrývá základní růstové a dynamické modely, Level 2 umožní modelování:

\begin{itemize}
\item \textbf{Nelineární oscilace} v ekonomických cyklech
\item \textbf{Saturační efekty} v tržní penetraci
\item \textbf{Competitivní dynamiku} ve víceagentových systémech
\item \textbf{Singulární chování} v kritických bodech systémů
\end{itemize}
\end{transition}

\vspace{0.6\baselineskip}

\begin{conclusion}[Level 1 - Kvantitativní Fundament]
\label{con:level1-kvantitativni-fundament}
Level 1 představuje \textbf{nezbytný fundament} pro každého kvantitativního experta:

\begin{itemize}
\item \textbf{Teoretická hloubka}: Důkladné pochopení čtyř základních typů ODE 1. řádu
\item \textbf{Praktické dovednosti}: Systematická metodologie identifikace a řešení
\item \textbf{Aplikační kompetence}: Modelování reálných systémů v ekonomii, financích a přírodních vědách
\item \textbf{Příprava na pokročilá témata}: Solidní základ pro nelineární rovnice a systémy vyšších řádů
\end{itemize}

\textbf{Zvládnutí Level 1 garantuje schopnost analyzovat a řešit převážnou většinu základních dynamických modelů v kvantitativních vědách.}
\end{conclusion}

\vspace{0.8\baselineskip}

\begin{remark}[Certifikační úroveň]
\label{rem:certifikacni-uroven}
Úspěšné zvládnutí Level 1 odpovídá úrovni \textbf{Junior Quantitative Analyst} a je předpokladem pro přechod k pokročilejším tématům v Level 2 a vyšších.
\end{remark}

\vspace{0.8\baselineskip}






% --- (Volitelně) další sekce Calculusu ---
% \input{chapters/20-limity}
% \input{chapters/21-derivace}
% \input{chapters/22-integraly}
% \input{chapters/23-mv-calculus}

% ---- Přílohy (pokud chceš) ----
% \appendix
% \section{Dodatek A: Notace a zkratky}
% \label{app:notace}
% Stručný souhrn používané notace. (Můžeš také držet v macros.tex)

\end{document}
