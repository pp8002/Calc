% !TEX root = main.tex
\documentclass[a4paper,12pt]{article}

% --- jazyk a kódování ---
\usepackage[main=czech]{babel}
\usepackage[utf8]{inputenc}   % pro pdfLaTeX
\usepackage[T1]{fontenc}
\usepackage{lmodern}

% --- matematika ---
\usepackage{amsmath,amssymb,amsthm}
\allowdisplaybreaks[3] % dovol rozdělování vícerovnic přes stránky

% --- okraje ---
\usepackage{geometry}
\geometry{margin=2.5cm}

% --- čitelnost ---
\linespread{1.1}

% --- jednoduché prostředí s římským číslováním (bez potřeby enumitem) ---
\newenvironment{romanenum}%
  {\begin{enumerate}\renewcommand\labelenumi{\roman{enumi})}}%
  {\end{enumerate}}

% --- hyperref vždy; cleveref fallback (když není nainstalován) ---
\usepackage{hyperref}
\hypersetup{
  colorlinks=true,
  linkcolor=black,
  citecolor=black,
  urlcolor=black
}

\makeatletter
\@ifundefined{ver@cleveref.sty}{%
  % fallback: základní aliasy, pokud cleveref není k dispozici
  \newcommand{\cref}[1]{\ref{#1}}
  \newcommand{\Cref}[1]{\ref{#1}}
}{%
  \usepackage[capitalize,nameinlink]{cleveref}
}
\makeatother

% --- mikrotypografie ---
\usepackage{microtype}
\emergencystretch=1.5em

% --- theoremstyle: vlastní nastavení s mezerami a nadpisem na novém řádku ---
\newtheoremstyle{mytheoremstyle}%
  {12pt}{12pt}% space above/below
  {\normalfont}{}%
  {\bfseries}{}%
  {\newline}{} % head–body separator = new line

\theoremstyle{mytheoremstyle}

% --- prostředí, každé má vlastní číslování ---
\newtheorem{definition}{Definice}[section]
\newtheorem{example}{Příklad}[section]
\newtheorem{remark}{Poznámka}[section]
\newtheorem{theorem}{Věta}[section]
\newtheorem{lemma}{Lemma}[section]
\newtheorem{proposition}{Tvrzení}[section]
\newtheorem{corollary}{Důsledek}[section]

% --- důkazy ---
\renewcommand{\proofname}{Důkaz}

% --- číslování rovnic, obrázků a tabulek podle sekcí ---
\numberwithin{equation}{section}
\numberwithin{figure}{section}
\numberwithin{table}{section}

% --- vlastní příkazy ---
\providecommand{\diff}{\mathrm{d}}

% --- pomocné blokové nadpisy a jemné svislé mezery ---
\newcommand{\blocktitle}[1]{\vspace{0.6\baselineskip}\noindent\textbf{#1}\par\vspace{0.25\baselineskip}}
\newcommand{\spc}{\vspace{0.6\baselineskip}}

\usepackage{tikz}
\usetikzlibrary{arrows.meta} % volitelné, jen pro hezčí šipky