% ===========================
% macros.tex — vlastní příkazy
% (bez \usepackage — vše řeší preamble)
% ===========================

% ---------- Základní množiny ----------
\newcommand{\N}{\mathbb{N}}
\newcommand{\Z}{\mathbb{Z}}
\newcommand{\Q}{\mathbb{Q}}
\newcommand{\R}{\mathbb{R}}
\newcommand{\C}{\mathbb{C}}

% ---------- Základní konstanty a symboly ----------
\newcommand{\e}{\mathrm{e}}           % Eulerovo číslo
\newcommand{\ii}{\mathrm{i}}          % imaginární jednotka
\newcommand{\dd}{\mathrm{d}}          % diferenciál (pro integrály)
\newcommand{\Dif}[1][]{\,\mathrm{d}#1} % alternativní diferenciál s mezerou: \Dif{x}

% ---------- Derivace ----------
% Skalární
\newcommand{\deriv}[3][]{\frac{\mathrm{d}^{#1} #2}{\mathrm{d} #3^{#1}}}     % \deriv[k]{y}{x}
\newcommand{\pderiv}[3][]{\frac{\partial^{#1} #2}{\partial #3^{#1}}}        % \pderiv[2]{u}{x}
\newcommand{\mderiv}[3][]{\frac{\mathrm{D}^{#1} #2}{\mathrm{D} #3^{#1}}}    % materiální/úplná derivace
% Evaluace na bodě
\newcommand{\at}[2]{\left.#1\right|_{#2}}                                   % \at{f'(x)}{x=x_0}

% ---------- Vektory, matice, normy ----------
\newcommand{\vect}[1]{\boldsymbol{#1}}                   % tučný symbol (vektor)
\newcommand{\mat}[1]{\mathbf{#1}}                        % tučné písmo pro matice
\newcommand{\abs}[1]{\left\lvert #1 \right\rvert}
\newcommand{\norm}[1]{\left\lVert #1 \right\rVert}
\newcommand{\ip}[2]{\left\langle #1,\, #2 \right\rangle} % skalární součin

% ---------- Závorky a pomocné ----------
\newcommand{\paren}[1]{\left( #1 \right)}
\newcommand{\bracket}[1]{\left[ #1 \right]}
\newcommand{\bracec}[1]{\left\{ #1 \right\}}
\newcommand{\floor}[1]{\left\lfloor #1 \right\rfloor}
\newcommand{\ceil}[1]{\left\lceil #1 \right\rceil}

% ---------- Operátory (deklarované správně) ----------
\DeclareMathOperator{\sgn}{sgn}
\DeclareMathOperator{\Span}{span}
\DeclareMathOperator{\rank}{rank}
\DeclareMathOperator{\tr}{tr}
\DeclareMathOperator{\diag}{diag}
\DeclareMathOperator{\diver}{div}
\DeclareMathOperator{\grad}{grad}
\DeclareMathOperator{\rot}{rot}        % alias curl
\DeclareMathOperator{\curl}{curl}
\DeclareMathOperator{\supp}{supp}
\DeclareMathOperator{\Res}{Res}
\DeclareMathOperator{\Real}{Re}
\DeclareMathOperator{\Imag}{Im}

% Vytažené operátory s „*“ (správné umístění limitů pod/nahoru)
\DeclareMathOperator*{\argmin}{arg\,min}
\DeclareMathOperator*{\argmax}{arg\,max}

% ---------- Pravděpodobnost a statistika (pokud se ti hodí) ----------
\newcommand{\Prob}{\mathbb{P}}
\newcommand{\E}{\mathbb{E}}
\newcommand{\Var}{\mathrm{Var}}
\newcommand{\Cov}{\mathrm{Cov}}
\newcommand{\1}{\mathbf{1}}            % indikační funkce

% ---------- Diferenciální rovnice – specifické ----------
\newcommand{\Lcal}{\mathcal{L}}         % lineární operátor
\newcommand{\Fcal}{\mathcal{F}}         % Fourierova transformace (pokud použiješ)
\newcommand{\Gcal}{\mathcal{G}}         % Greenova funkce / operátor
\newcommand{\Wr}{\mathrm{W}}            % Wronskián (příp. doplníš definici)
\newcommand{\Heaviside}{\mathrm{H}}     % Heaviside
\newcommand{\dirac}{\delta}             % Dirac delta

% Notace pro ODE/PDE
\newcommand{\ode}[2]{#1'\!(#2)}         % např. \ode{y}{x} = y'(x)
\newcommand{\odeN}[3]{#1^{(#2)}\!(#3)}  % \odeN{y}{n}{x} = y^{(n)}(x)
\newcommand{\pde}[3]{\frac{\partial #1}{\partial #2}\!(#3)}
\newcommand{\pdeN}[4]{\frac{\partial^{#2} #1}{\partial #3^{#2}}\!(#4)}

% Standardní tvary
\newcommand{\odeStd}{y' = f(x,y)}                                   % rychlé vložení obecné 1. ř.
\newcommand{\odeLin}{y' + p(x)\,y = q(x)}                           % lineární 1. řádu
\newcommand{\odeSep}{y' = f(x)\,g(y)}                               % separabilní
\newcommand{\pdeHeat}{u_t = \kappa\,u_{xx}}                         % rovnice vedení tepla
\newcommand{\pdeWave}{u_{tt} = c^2 u_{xx}}                          % vlnová rovnice
\newcommand{\pdeLaplace}{\Delta u = 0}                              % Laplaceova rovnice

% Integrující faktor
\newcommand{\IF}{\mu(x)}                                            % \IF = e^{\int p(x)\,\dd x} typicky
% Wronskián dvou funkcí
\newcommand{\Wronskian}[2]{%
  \begin{vmatrix}
    #1 & #2 \\
    #1' & #2' \\
  \end{vmatrix}
}

% Laplaceův/Poissonův operátor
\newcommand{\Lapl}{\Delta}                                          % \Lapl u = u_{xx}+u_{yy}(+...)

% Okrajové a počáteční podmínky (rychlá notace)
\newcommand{\IC}[1]{\text{IC: } #1}                                 % \IC{y(0)=y_0}
\newcommand{\BC}[1]{\text{BC: } #1}                                 % \BC{u(0,t)=0,\;u(L,t)=0}

% Greenova funkce – symbolická notace
\newcommand{\Green}[2]{G\!\paren{#1\,;\,#2}}                        % G(x;\xi)

% ---------- Fourier / konvoluce ----------
\newcommand{\FT}[1]{\widehat{#1}}                                   % Fourierův obraz
\newcommand{\IFT}[1]{\check{#1}}                                    % inverzní FT (notace dle zvyku)
\newcommand{\conv}{\ast}                                            % konvoluce

% ---------- Intervals a set-builder ----------
\newcommand{\set}[1]{\left\{\, #1 \,\right\}}
\newcommand{\openInterval}[2]{\left(#1,\,#2\right)}
\newcommand{\closedInterval}[2]{\left[#1,\,#2\right]}
\newcommand{\halfOpenInterval}[2]{\left[#1,\,#2\right)}
\newcommand{\halfClosedInterval}[2]{\left(#1,\,#2\right]}

% ---------- Zkrácené odkazy a eqref ----------
\newcommand{\eqn}[1]{\eqref{#1}}                                    % zkrácený odkaz na rovnici
\newcommand{\crefp}[1]{(\cref{#1})}                                 % "viz (1.2)" s cleveref

% ---------- Malé užitečnosti ----------
\newcommand{\ddx}{\frac{\mathrm{d}}{\mathrm{d}x}}
\newcommand{\ddy}{\frac{\mathrm{d}}{\mathrm{d}y}}
\newcommand{\ppt}{\frac{\partial}{\partial t}}
\newcommand{\ppx}{\frac{\partial}{\partial x}}
\newcommand{\ppy}{\frac{\partial}{\partial y}}
\newcommand{\ppxx}{\frac{\partial^2}{\partial x^2}}
\newcommand{\ppyy}{\frac{\partial^2}{\partial y^2}}
\newcommand{\ppxy}{\frac{\partial^2}{\partial x\,\partial y}}

% ---------- Textové zvýraznění v matematice ----------
\newcommand{\textemph}[1]{\textbf{\textit{#1}}}                     % rychlé zdůraznění v odstavcích

% ---------- Krátké značky pro kolonky/souhrny ----------
\newcommand{\Def}[1]{\textbf{Def.:} #1}
\newcommand{\Thm}[1]{\textbf{Věta:} #1}
\newcommand{\Rem}[1]{\textbf{Pozn.:} #1}
\newcommand{\Exa}[1]{\textbf{Př.:} #1}
