% !TEX root = main.tex

% --- NASTAVENÍ ZÁ HLAVÍ A ZÁPATÍ ---
\pagestyle{fancy}
\fancyhf{}
\fancyhead[R]{\thepage}
\fancyhead[L]{\leftmark}
\renewcommand{\headrulewidth}{0.4pt}

% --- DEFINICE BAREV PRO KVANTITATIVNÍ APLIKACE ---
\definecolor{quantblue}{RGB}{0, 82, 147}
\definecolor{quantgreen}{RGB}{0, 128, 0}
\definecolor{quantred}{RGB}{192, 0, 0}
\definecolor{quantgray}{RGB}{240, 240, 240}

% --- TCOROLORBOX PROSTŘEDÍ PRO SPECIÁLNÍ BLOKY ---
\newtcolorbox{keyinsight}[1][]{
  breakable,
  enhanced,
  colback=quantgray,
  colframe=quantblue,
  title=#1,
  fonttitle=\bfseries,
  attach boxed title to top left={yshift=-2mm, xshift=5mm},
  boxed title style={colback=quantblue, colframe=quantblue}
}

\newtcolorbox{application}[1][]{
  breakable,
  enhanced,
  colback=white,
  colframe=quantgreen,
  title=#1,
  fonttitle=\bfseries,
  attach boxed title to top left={yshift=-2mm, xshift=5mm},
  boxed title style={colback=quantgreen, colframe=quantgreen}
}

\newtcolorbox{expertnote}[1][]{
  breakable,
  enhanced,
  colback=white,
  colframe=quantred,
  title=#1,
  fonttitle=\bfseries,
  attach boxed title to top left={yshift=-2mm, xshift=5mm},
  boxed title style={colback=quantred, colframe=quantred}
}

\newtcolorbox{transition}{
  breakable,
  enhanced,
  colback=white,
  colframe=black,
  arc=0pt,
  boxrule=1pt,
  left=5mm,
  right=5mm
}

\newtcolorbox{researcharea}[1][]{
  breakable,
  enhanced,
  colback=quantgray!20,
  colframe=quantblue!80,
  title=#1,
  fonttitle=\bfseries,
  attach boxed title to top left={yshift=-2mm, xshift=5mm}
}

% --- ROADMAP BOX ---
\newcommand{\roadmap}[1]{
  \begin{tcolorbox}[
    title=Roadmap, 
    breakable, 
    enhanced, 
    colback=quantgray!30, 
    colframe=quantblue
  ] 
  #1 
  \end{tcolorbox}
}

% --- THEOREM STYLES A PROSTŘEDÍ ---
\newtheoremstyle{mytheoremstyle}%
  {12pt}{12pt}%
  {\normalfont}{}%
  {\bfseries}{}%
  {\newline}{}

\theoremstyle{mytheoremstyle}

\newtheorem{definition}{Definice}[section]
\newtheorem{example}{Příklad}[section]
\newtheorem{remark}{Poznámka}[section]
\newtheorem{theorem}{Věta}[section]
\newtheorem{lemma}{Lemma}[section]
\newtheorem{proposition}{Tvrzení}[section]
\newtheorem{corollary}{Důsledek}[section]
\newtheorem{principle}{Princip}[section]
\newtheorem{innovation}{Inovace}[section]
\newtheorem{researchareaenv}{Výzkumná Oblast}[section]
\newtheorem{assessment}{Hodnocení}[section]

% --- DŮKAZY ---
\renewcommand{\proofname}{Důkaz}

% --- ČÍSLOVÁNÍ PODLE SEKCI ---
\numberwithin{equation}{section}
\numberwithin{figure}{section}
\numberwithin{table}{section}
\numberwithin{algorithm}{section}

% --- ALGORITMY ---
\floatname{algorithm}{Algoritmus}
\renewcommand{\algorithmicrequire}{\textbf{Vstup:}}
\renewcommand{\algorithmicensure}{\textbf{Výstup:}}
\crefname{algorithm}{algoritmus}{algoritmy}
\Crefname{algorithm}{Algoritmus}{Algoritmy}

% --- NOVÉ SLOUPCE PRO TABULKY ---
\newcolumntype{L}{>{\raggedright\arraybackslash}p{0.2\textwidth}}
\newcolumntype{M}{>{\centering\arraybackslash}p{0.3\textwidth}}
\newcolumntype{R}{>{\raggedleft\arraybackslash}p{0.4\textwidth}}

% --- LISTINGS STYLES ---
\lstset{
  basicstyle=\ttfamily\small,
  breaklines=true,
  frame=single,
  numbers=left,
  numberstyle=\tiny\color{gray},
  showstringspaces=false,
  commentstyle=\color{quantgreen},
  keywordstyle=\color{quantblue},
  stringstyle=\color{quantred},
  backgroundcolor=\color{quantgray!20}
}

\lstdefinestyle{python}{
  language=Python,
  morekeywords={import, def, class, return, yield, lambda, with, async, await}
}

\lstdefinestyle{matlab}{
  language=Matlab,
  morekeywords={function, end, if, else, for, while}
}

% --- MATEMATICKÉ PŘÍKAZY ---
\providecommand{\diff}{\mathrm{d}}
\newcommand{\bmm}[1]{\bm{#1}}

% --- ČÍSELNÉ MNOŽINY ---
\newcommand{\RR}{\mathbb{R}}
\newcommand{\CC}{\mathbb{C}}
\newcommand{\NN}{\mathbb{N}}
\newcommand{\ZZ}{\mathbb{Z}}
\newcommand{\QQ}{\mathbb{Q}}

% --- PRAVDĚPODOBNOSTNÍ SYMBOLY ---
\newcommand{\EE}{\mathbb{E}}
\newcommand{\PP}{\mathbb{P}}
\newcommand{\Var}{\mathrm{Var}}
\newcommand{\Cov}{\mathrm{Cov}}
\newcommand{\Corr}{\mathrm{Corr}}

% --- DALŠÍ MATEMATICKÉ SYMBOLY ---
\newcommand{\supp}{\mathrm{supp}}
\newcommand{\sgn}{\mathrm{sgn}}
\newcommand{\id}{\mathrm{id}}

% --- OPERÁTORY PRO FUNKCIONÁLNÍ ANALÝZU ---
\DeclareMathOperator{\diag}{diag}
\DeclareMathOperator{\tr}{tr}
\DeclareMathOperator{\Span}{span}
\DeclareMathOperator{\grad}{grad}
\DeclareMathOperator{\curl}{curl}
\DeclareMathOperator{\divg}{div}
\DeclareMathOperator{\Hess}{Hess}
\DeclareMathOperator{\Jac}{Jac}

% --- POMOCNÉ BLOKOVÉ NADPISY ---
\newcommand{\blocktitle}[1]{\vspace{0.6\baselineskip}\noindent\textbf{#1}\par\vspace{0.25\baselineskip}}
\newcommand{\spc}{\vspace{0.6\baselineskip}}

% --- HYPERREF NASTAVENÍ ---
\hypersetup{
  colorlinks=true,
  linkcolor=quantblue,
  citecolor=quantgreen,
  urlcolor=quantred,
  pdftitle={Pokročilé Diferenciální Rovnice pro Kvantitativní Odborníky},
  pdfauthor={Quant Expert Team},
  pdfsubject={Matematika, Finanční Modelování},
  pdfkeywords={diferenciální rovnice, kvantitativní finance, matematické modelování}
}

% --- MIKROTYPOGRAFIE ---
\microtypesetup{expansion=true}
\emergencystretch=1.5em

% --- ČESKÉ DĚLENÍ SLOV ---
